%!TEX root = ./main.tex
\chapter{Notes on \texorpdfstring{$NK_0$}{NK0} and \texorpdfstring{$NK_1$}{NK1} of the groups \texorpdfstring{$C_4$}{C4} and \texorpdfstring{$D_4$}{D4}}
This note is based on the paper \cite{weibel2009nk0}.
\section{Outline}
\begin{definition}[Bass $Nil$-groups]
$NK_n(\mathbb{Z}G)=\ker(K_n(\mathbb{Z}G[x])\overset{x\mapsto 0}\longrightarrow K_n(\mathbb{Z}G))$
\end{definition}
\begin{equation*}
	\begin{array}{c|c|c|c}
	G& NK_0(\mathbb{Z}G) & NK_1(\mathbb{Z}G) &NK_2(\mathbb{Z}G) \\
	\hline
	C_2 & 0 & 0&V \\
	\hline
	D_2=C_2\times C_2&V&\Omega_{\mathbb{F}_2[x]} & \\
	\hline
	C_4 & V&\Omega_{\mathbb{F}_2[x]} & \\
	\hline
	D_4=C_4\rtimes C_2& \\
	\end{array}
\end{equation*}

Note that $D_4=\langle \sigma, \tau|\sigma^4=1,\tau^2=1,\tau \sigma \tau=\sigma^{-1} \rangle$.\\
$V=x \mathbb{F}_2[x]=\oplus_{i=1}^\infty \mathbb{F}_2 x^i = \oplus_{i=1}^\infty \mathbb{Z}/2 x^i$: continuous $W(\mathbb{F}_2)$-module. As an abelian group, it is countable direct sum of copies of $\mathbb{F}_2=\mathbb{Z}/2$ on generators $x^i,i>0$.\\
$\Omega_{\mathbb{F}_2[x]}= \mathbb{F}_2[x]\,dx = \oplus_{i=1}^\infty \mathbb{F}_2 e^i $, often write $e^i$ stands for $x^{i-1}\, dx$. As an abelian group, $\Omega_{\mathbb{F}_2[x]}\cong V$. But it has a different $W(\mathbb{F}_2)$-module structure.

\section{Preliminaries}
\subsection{Regular rings} % (fold)
\label{sub:regular_rings}
We list some useful notations here:

$R$: ring with unit (usually commutative in this chapter)\\
$R$-mod: the category of $R$-modules,\\
$\mathbf{M}(R)$: the subcategory of finitely generated $R$-modules,\\
$\mathbf{P}(R)$: the subcategory of finitely generated projective $R$-modules.\\


Let $\mathbf{H}(R)\subset \mbox{$R$-mod} $ be the full subcategory contains all $M$ which has finte $\mathbf{P}(R)$-resolutions. $R$ is called {\em{regular}}\index{regular ring} if $\mathbf{M}(R)=\mathbf{P}(R)$.

\begin{prop}
	Let $R$ be a commutative ring with unit, $A$ an $R$-algebra and $S\subset R$ a multiplicative set, if $A$ is regular, then $S^{-1}A$ is also regular.
\end{prop}

% subsection regular_rings (end)
\subsection{The ring of Witt vectors} % (fold)
\label{sub:the_ring_of_witt_vectors}
As additive group $W(\mathbb{Z})=(1+x \mathbb{Z}[[x]])^{\times}$, it is a module over the Cartier algebra consisting of row-and-column finite sums $\sum V_m [a_{mn}]F_n$, where $[a]$ are homothety operators for $a\in \mathbb{Z}$.

\paragraph{additional structure}
Verschiebung operators $V_m$, Frobenius operators $F_m$ (ring endomorphism), homothety operators $[a]$.
\begin{align*}
[a]\colon & \alpha(x) \mapsto \alpha(ax)\\
V_m\colon & \alpha(x) \mapsto \alpha(x^m)\\
F_m\colon & \alpha(x) \mapsto \sum_{\zeta^m=1} \alpha(\zeta x^{\frac{1}{m}})\\
F_m\colon & 1-rx \mapsto 1-r^mx
\end{align*}
\begin{remark}
	$W(R)\subset Cart(R)$, $\prod_{m=1}^\infty(1-r_mx^m)=\sum_{m=1}^\infty V_m[a_m]F_m$. See \cite{MR96j:16008}.
\end{remark}
\begin{prop}
	$[1]=V_1=F_1$: multiplicative identity. There are some identities:
	\begin{align*}
	 V_mV_n&=V_{mn}\\
	 F_mF_n&=F_{mn}\\
	 F_mV_n&=m\\
	 [a]V_m&=V_{m}[a^m]\\
	 F_m[a]&=[a^m] F_m\\
	 [a][b]&=[ab]\\
	 V_mF_k=F_kV_m,&\mbox{ if } (k,m)=1 \\
	\end{align*}
	
\end{prop}

We call a $W(R)$-module $M$ continuous if $\forall v \in M$, $\textrm{ann}_{W(R)}(v)$ is an open ideal in $W(R)$, that is $\exists k$ s.t. $(1-rx)^m *v =0$ for all $r\in R$ and $m\geqslant k$. Note that if $A$ is an $R$-module, $xA[x]$ is a continuous $W(R)$-module but that $xA[[x]]$ is not.
% subsection the_ring_of_witt_vectors (end)

\subsection{Dennis-Stein symbols} % (fold)
\label{sub:dennis_stein_symbols}
\paragraph{Steinberg symbol} % (fold)
\label{par:steinberg_symbol}


Let $R$ be a commutative ring, $u,v\in R^*$. First we construct Steinberg symbol $\{u,v\}\in K_2(R)$ as follows:
\[\{u,v\}=h_{12}(uv)h_{12}(u)^{-1}h_12(v)^{-1}\]
where $h_{ij}(u)=w_{ij}(u)w_{ij}(-1)$ and $w_{ij}(u)=x_{ij}(u)x_{ji}(-u^{-1})x_{ij}(u)$.

These symbols satisfy
\begin{itemize}
	\item[(a)] $\{u_1u_2,v\}=\{u_1,v\}\{u_2,v\}$ for $u_1,u_2,v\in R^*$. [Bilinear]\\
	\item[(b)] $\{u,v\}\{v,u\}=1$ for $u,v \in R^*$. [Skew-symmetric]\\
	\item[(c)] $\{u,1-u\}=1$ for $u,1-u\in R^*$.
\end{itemize}
\begin{theorem}
	If $R$ is a field, division ring, local ring or even a commutative semilocal ring, $K_2(R)$ is generated by Steinberg symbols $\{r,s\}$.
\end{theorem}
% paragraph steinberg_symbol (end)
\paragraph{Dennis-Stein symbol {\color{green}version 1}} % (fold)
\label{par:dennis_stein_symbol_greenversion_1}
If $a,b\in R$ with $1+ab \in R^*$, Dennis-Stein symbol $\langle a,b \rangle \in K_2(R)$ is defined by 
\[\langle a,b \rangle = x_{21}(-\frac{b}{1+ab})x_{12}(a)x_{21}(b)x_{12}(-\frac{a}{1+ab})h_{12}(1+ab)^{-1}.\]
Note that 
\[\langle a,b \rangle = \begin{cases}
	\{-a,1+ab\},\quad \mbox{if $a\in R^*$}\\
	\{1+ab,b\},\quad \mbox{if $b\in R^*$}\\
\end{cases}\]
and if $u,v\in R^*-\{1\}$, $\{u,v\}=\langle -u, \frac{1-v}{u} \rangle = \langle \frac{u-1}{v},v\rangle$, thus Steinberg symbol is also a Dennis-Stein symbol. See Dennis, Stein \emph{The functor $K_2$: a survey of computational problem}.

Maazen and Stienstra define the group $D(R)$ as follows:\\
take a generator $\langle a,b \rangle$ for each pair $a,b\in R$ with $1+ab\in R^*$,\\
defining relations:
\begin{itemize}
	\item[(D1)] $\langle a,b\rangle \langle -b,-a \rangle=1$,\\
	\item[(D2)] $\langle a,b\rangle \langle a,c \rangle=\langle a,b+c+abc \rangle$, \\
	\item[(D3)] $\langle a,bc\rangle =\langle ab,c\rangle \langle ac,b\rangle$.
\end{itemize}
If $I\subset R$ is an ideal, $a\in I$ or $b\in I$, we can consider $\langle a,b\rangle \in K_2(R,I)$ satisfy following relations
\begin{itemize}
	\item[(D1)] $\langle a,b\rangle \langle -b,-a \rangle=1$,\\
	\item[(D2)] $\langle a,b\rangle \langle a,c \rangle=\langle a,b+c+abc \rangle$, \\
	\item[(D3)] $\langle a,bc\rangle =\langle ab,c\rangle \langle ac,b\rangle$ if any of $a,b,c$ are in $I$.
\end{itemize}
\begin{theorem}
	\begin{enumerate}
		\item If $R$ is a {\color{green} commutative local ring}, then $D(R)\iso K_2(R)$ is isomorphic. (Maazen-Stienstra, Dennis-Stein, van der Kallen)\\
		\item Let $R$ be a commutative ring. If $I \subset \mathrm{Rad}(R)$ (ideal $I$ is contained in the Jacobson radical), $D(R,I)\iso K_2(R,I)$.
	\end{enumerate}
	
\end{theorem}
% paragraph dennis_stein_symbol_greenversion_1 (end)


\paragraph{Dennis-Stein symbol {\color{green}version 2}} % (fold)
\label{par:dennis_stein_symbol_greenversion_2_}
In 1980s, things have changed. Dennis-Stein symbol is defined as follows\\
$r,s\in R$ commute and $1-rs$ is a unit, that is $rs=sr$ and $1-rs\in R^*$,
\[\langle r,s \rangle = x_{ji}(-s(1-rs)^{-1})x_{ij}(-r)x_{ji}(s)x_{ij}((1-rs)^{-1}r)h_{ij}(1-rs)^{-1}.\]
Note that if $r\in R^*$, $\langle r,s \rangle =\{r,1-rs\}$. If $I\subset R$ is an ideal, $r\in I$ or $\in I$, we can even consider $\langle r,s\rangle \in K_2(R,I)$

\begin{itemize}
	\item[(D1)] $\langle r,s\rangle \langle s,r \rangle=1$,\\
	\item[(D2)] $\langle r,s\rangle \langle r,t \rangle=\langle r,s+t-rst \rangle$, \\
	\item[(D3)] $\langle r,st\rangle =\langle rs,t\rangle \langle tr,s\rangle$ (this holds in $K_2(R,I)$ if any of $r,s,t$ are in $I$).
\end{itemize}

Note that $\langle r,1\rangle=0$ for any $r\in R$ and $\langle r,s \rangle_{version 2}=\langle -r,s \rangle_{version 1}$.
\begin{theorem}
	\begin{enumerate}
		\item If $R$ is a {\color{green} commutative local ring or a field}, then $K_2(R)$ is generated by $\langle r,s \rangle$ satisfying D1, D2, D3. \\
		\item Let $R$ be a commutative ring. If $I \subset \mathrm{Rad}(R)$ (ideal $I$ is contained in the Jacobson radical), $K_2(R,I)$ is generated by $\langle r,s \rangle$ (either $r\in R$ and $s\in I$ or $r\in I$ and $s\in R$) satisfying D1, D2, D3.
	\end{enumerate}
	
\end{theorem}
% paragraph dennis_stein_symbol_greenversion_2_ (end)











% subsection dennis_stein_symbols (end)








\subsection{Relative group and double relative group} % (fold)
\label{sub:relative_group}
You can skip this subsection for first reading. We will use the results in \ref{sec:C2}.

\paragraph{Relative groups} % (fold)
\label{par:relative_groups}
Let $R$ be a ring (not necessarily commutative), $I\subset R$ a two-sided ideal, by definition $K_i(R)=\pi_i(BGL(R)^+)$, $i\geq 1$, there exists a map
\[BGL(R)^+ \longrightarrow BGL(R/I)^+\]
\begin{definition}
	$K(R,I)$ is the homotopy fibre of the map $BGL(R)^+ \longrightarrow BGL(R/I)^+$. $K_i(R,I):=\pi_i(K(R,I)), i\geq 1$.
\end{definition}

By long exact sequences of homotopy groups of a homotopy fibre, there is an exact sequence
\[\cdots \longrightarrow K_{i+1}(R)\longrightarrow K_{i+1}(R/I) \longrightarrow K_i(R,I)\longrightarrow K_i(R)\longrightarrow K_i(R/I)\longrightarrow \cdots.\]
In particular,
\begin{align*}
&K_3(R,I)\longrightarrow K_3(R)\longrightarrow K_3(R/I) \longrightarrow K_2(R,I)\longrightarrow K_2(R)\longrightarrow K_2(R/I)\longrightarrow\\
\longrightarrow & K_1(R,I)\longrightarrow K_1(R)\longrightarrow K_1(R/I)
\end{align*}


% paragraph relative_groups (end)
\paragraph{Double relative groups} % (fold)
\label{par:double_relative_groups}

Let $R$ be any ring (not necessarily commutative), if $I,J\subset R$ are two-sided ideals, there is a map
\[K(R,I)\longrightarrow K(R/J,I+J/J).\]

\begin{definition}
	$K(R;I,J)$ is the homotopy fibre of the map $K(R,I)\longrightarrow K(R/J,I+J/J)$. $K_i(R,I,J):=\pi_i(K(R;I,J)), i\geq 1$.
\end{definition}
\[
\begin{tikzcd}
	K(R;I,J) \ar[d] & & \\
	K(R,I) \ar[r] \ar[d] & BGL(R)^+ \ar[r] \ar[d] &BGL(R/I)^+ \ar[d] \\
	K(R/J,I+J/J) \ar[r]  & BGL(R/J)^+ \ar[r]  &BGL(R/I+J)^+ 
\end{tikzcd}\]
\begin{remark}
	$K_i(R;I,J)\cong K_i(R;J,I)$, $K_i(R;I,I)=K_i(R,I)$.
\end{remark}


Let $R$ be any ring (not necessarily commutative), if $I,J\subset R$ are two-sided ideals such that $I\cap J =0$, then there is an exact sequence
\[K_3(R,I) \longrightarrow K_3(R/I,I+J/J)\longrightarrow I/I^2\otimes_{R^e}J/J^2\overset{\psi}{\longrightarrow}K_2(R,I)\longrightarrow K_2(R/I,I+J/J)\longrightarrow 0\]
where $R^e= R\otimes_{\mathbb{Z}}R^{op}$, $\psi([a]\otimes [b])=\langle a, b\rangle$, see \cite{weibel2013k} 3.5.10 or \cite{friedlander1981algebraic} p.\,195.

In the case $I\cap J =0$, $K_2(R;I,J)\cong I/I^2\otimes_{R^e}J/J^2$.

我的疑问: if $R$ is commutative, whether $K_2(R;I,J)=I\otimes_R J$ or not?

\begin{lemma}
	Let $(R;I,J)$ satisfy the following Cartesian square
		\[\begin{tikzcd}
			R \ar[r] \ar[d]& R/I\ar[d]\\
			R/J \ar[r] & R/I+J\\
		\end{tikzcd}\]
	suppose $f\colon (R,I)\longrightarrow (R/J,I+J/J)$ has a section $g$, then
	\[0 \longrightarrow I/I^2\otimes_{R^e}J/J^2\longrightarrow K_2(R,I)\longrightarrow K_2(R/I,I+J/J)\longrightarrow 0\]
	is split exact.
\end{lemma}


% paragraph double_relative_groups (end)
% subsection relative_group (end)
\section{\texorpdfstring{$W(R)$}{W(R)}-module structure} % (fold)
\label{sec:module_structure}


\paragraph{$W(\mathbb{F}_2)$-module structure on $V=x \mathbb{F}_2[x]$} See Dayton\& Weibel \cite{MR96j:16008} example 2.6, 2.9.

\begin{align*}
 V_m(x^n)&=x^{mn}\\
 F_d(x^n)&=\begin{cases}
 	dx^{n/d},& \mbox{ if $d|n$}\\
 	0,& \mbox{otherwise}
 \end{cases}\\
 [a]x^n&=a^nx^n
 \end{align*}
\paragraph{$W(\mathbb{F}_2)$-module structure on $\Omega_{\mathbb{F}_2[x]}=\mathbb{F}_2[x]\,dx = \oplus_{i=1}^\infty \mathbb{F}_2 e^i$} Dayton\& Weibel \cite{MR96j:16008}example 2.10
\begin{align*}
 V_m(x^{n-1}\,dx)&=mx^{mn-1}\,dx\\
 F_d(x^{n-1}\,dx)&=\begin{cases}
 	x^{n/d-1}\,dx,& \mbox{ if $d|n$}\\
 	0,& \mbox{otherwise}
 \end{cases}\\
 [a]x^{n-1}\,dx&=a^nx^{n-1}\,dx
 \end{align*}
 \begin{remark}
 	$\Omega_{\mathbb{F}_2[x]}$ is {\color{red}not} finitely generated as a module over the $\mathbb{F}_2$-Cartier algebra or over the subalgebra $W(\mathbb{F}_2)$.
 \end{remark}

 In general, for any map $R\longrightarrow S$ of communicative rings, the $S$-module $\Omega_{S/R}^1$(relative K{\"{a}}hler differential module $\Omega_{S/R}$) is defined by \\
 generators: $ds$, $s\in S$,\\
 relations: $d(s+s')=ds+ds'$, $d(ss')=sds'+s'ds$, and if $r\in R$, $dr=0$.\\

\begin{remark}
	If $R=\mathbb{Z}$, we often omit it. In the previous section, $\Omega_{\mathbb{F}_2[x]}=\Omega_{\mathbb{F}_2[x]/\mathbb{Z}}^1$.
\end{remark}

As abelian groups, $x \mathbb{F}_2[x] \overset{\sim}\longrightarrow \Omega_{\mathbb{F}_2[x]}$, $x^i \mapsto x^{i-1}dx$. However, as $W(\mathbb{F}_2)$-modules,
\begin{align*}
V_m(x^i)&=x^{im},\\
V_m(x^{i-1}dx)&=mx^{im-1}dx
\end{align*}
$x^{im}$ is corresponding to $x^{im-1}dx$ but not to $mx^{im-1}dx$. So they have different $W(\mathbb{F}_2)$-module structure.

\begin{remark}
	一个不知道有没有用的结论, see \cite{weibel2009nk0} 

	There is a $W(\mathbb{F}_2)$-module homomorphism called de Rham differential
	\begin{align*}
	D\colon x \mathbb{F}_2[x] &\longrightarrow \Omega_{\mathbb{F}_2[x]}\\
	x^i &\mapsto ix^{i-1}dx
	\end{align*}
	Then $\ker D = H_{dR}^0(\mathbb{F}_2[x]/\mathbb{F}_2)$ is the de Rham cohomology group and $\coker D = HC_1^{\mathbb{F}_2}(\mathbb{F}_2[x])$ is the cyclic homology group. Note that $HC_1(\mathbb{F}_2[x])= \sum_{l=1}^{\infty}\mathbb{F}_2 e_{2l}$ where $e_{2l}=x^{2l-1}dx$, and $H_{dR}^0(\mathbb{F}_2[x])=x^2 \mathbb{F}_2[x^2]$.
\end{remark}
% section module_structure (end)
\section{\texorpdfstring{$NK_i$}{NKi} of the group \texorpdfstring{$C_2$}{C2}}
\label{sec:C2}
First, consider the simplest example $G=C_2=\langle \sigma \rangle=\{1,\sigma\}$. There is a Rim square
	\[\begin{tikzcd}
		\mathbb{Z}[C_2] \ar[r,"\sigma\mapsto 1"] \ar[d,"\sigma\mapsto -1"']& \mathbb{Z}\ar[d,"q"]\\
		 \mathbb{Z}\ar[r,"q"] & \mathbb{F}_2\\
	\end{tikzcd}\]
Since $\mathbb{F}_2$ (field) and $\mathbb{Z}$ (PID) are regular rings, $NK_i(\mathbb{F}_2)=0=NK_i(\mathbb{Z})$ for all $i$.

By Mayer–Vietoris sequence, one can get $NK_1(\mathbb{Z}[C_2])=0$, $NK_0(\mathbb{Z}[C_2])=0$. Note that the similar results are true for any cyclic group of prime order.

\[\ker(\mathbb{Z}[C_2]\overset{\sigma \mapsto -1}\longrightarrow \mathbb{Z}) =(\sigma +1)\]
By relative exact sequence,
\[0=NK_3(\mathbb{Z})\longrightarrow NK_2(\mathbb{Z}[C_2],(\sigma+1))\overset{\cong}\longrightarrow NK_2(\mathbb{Z}[C_2])\longrightarrow NK_2(\mathbb{Z})=0.\]
And from $(\mathbb{Z}[C_2],(\sigma+1))\longrightarrow (\mathbb{Z}[C_2]/(\sigma-1),(\sigma+1)+(\sigma-1)/(\sigma-1))=(\mathbb{Z},(2))$ one has double relative exact sequence
\[0=NK_3(\mathbb{Z},(2))\longrightarrow NK_2(\mathbb{Z}[C_2];(\sigma+1),(\sigma-1))\overset{\cong}\longrightarrow NK_2(\mathbb{Z}[C_2],(\sigma+1))\longrightarrow NK_2(\mathbb{Z},(2))=0.\]
Note that $0=NK_{i+1}(\mathbb{Z}/2)\longrightarrow NK_i(\mathbb{Z},(2))\longrightarrow NK_i(\mathbb{Z})=0$.

\[\begin{tikzcd}
	 & NK_3(\mathbb{Z},(2))=0 \ar[d] & & & \\
	 & NK_2(\mathbb{Z}[C_2];(\sigma+1),(\sigma-1)) \ar[d,"\cong"] \\
0=NK_3(\mathbb{Z})	\ar[r] & NK_2(\mathbb{Z}[C_2],(\sigma+1)) \ar[r,"\cong"] \ar[d] & NK_2(\mathbb{Z}[C_2]) \ar[r] &NK_2(\mathbb{Z})=0\\
	& NK_2(\mathbb{Z},(2))=0 \\
\end{tikzcd}
\]
We obtain $NK_2(\mathbb{Z}[C_2])\cong NK_2(\mathbb{Z}[C_2],(\sigma+1),(\sigma-1))$, from Guin-Loday-Keune\cite{Guin-Waléry1981}, $NK_2(\mathbb{Z}[C_2];(\sigma+1),(\sigma-1))$ is isomorphic to $V=x \mathbb{F}_2[x]$, with the Dennis-Stein symbol $\langle x^n(\sigma-1),\sigma+1 \rangle$ corresponding to $x^n\in V$. Note that $1-x^n(\sigma-1)(\sigma+1)=1$ is invertible in $\mathbb{Z}[C_2][x]$ and $\sigma+1 \in (\sigma+1), x^n(\sigma-1) \in (\sigma-1)$.

\begin{theorem}
	$NK_2(\mathbb{Z}[C_2])\cong V$, $NK_1(\mathbb{Z}[C_2])=0$, $NK_0(\mathbb{Z}[C_2])=0$.
\end{theorem}

In fact, we have $NK_2(\mathbb{Z}[C_p])\cong x \mathbb{F}_p[x]$, $NK_1(\mathbb{Z}[C_p])=0$, $NK_0(\mathbb{Z}[C_p])=0$.



\section{\texorpdfstring{$NK_i$}{NKi} of the group \texorpdfstring{$C_4$}{C4}}
\label{sec:C4}
\section{\texorpdfstring{$NK_i$}{NKi} of the group \texorpdfstring{$D_4$}{D4}}
\label{sec:D4}