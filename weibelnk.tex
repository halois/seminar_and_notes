\chapter{Notes on $NK_0$ and $NK_1$ of the groups $C_4$ and $D_4$}
\section{Outline}
\begin{definition}[Bass $Nil$-groups]
$NK_n(\mathbb{Z}G)=\ker(K_n(\mathbb{Z}G[x])\overset{x\mapsto 0}\longrightarrow K_n(\mathbb{Z}G))$
\end{definition}
\begin{equation*}
	\begin{array}{c|c|c|c}
	G& NK_0(\mathbb{Z}G) & NK_1(\mathbb{Z}G) &NK_2(\mathbb{Z}G) \\
	\hline
	C_2 & 0 & 0&V \\
	\hline
	D_2=C_2\times C_2&V&\Omega_{\mathbb{F}_2[x]} & \\
	\hline
	C_4 & V&\Omega_{\mathbb{F}_2[x]} & \\
	\hline
	D_4=C_4\rtimes C_2& \\
	\end{array}
\end{equation*}

Note that $D_4=\langle \sigma, \tau|\sigma^4=1,\tau^2=1,\tau \sigma \tau=\sigma^{-1} \rangle$.\\
$V=x \mathbb{F}_2[x]=\oplus_{i=1}^\infty \mathbb{F}_2 x^i = \oplus_{i=1}^\infty \mathbb{Z}/2 x^i$: continuous $W(\mathbb{F}_2)$-module. As an abelian group, it is countable direct sum of copies of $\mathbb{F}_2=\mathbb{Z}/2$ on generators $x^i,i>0$.\\
$\Omega_{\mathbb{F}_2[x]}= \mathbb{F}_2[x]\,dx = \oplus_{i=1}^\infty \mathbb{F}_2 e^i $, often write $e^i$ stands for $x^{i-1}\, dx$. As an abelian group, $\Omega_{\mathbb{F}_2[x]}\cong V$. But it has different $W(\mathbb{F}_2)$-module structure.

\section{Preliminaries}
As additive group $W(\mathbb{Z})=(1+x \mathbb{Z}[[x]])^{\times}$, it is a module over the Cartier algebra consisting of row-and-column finite sums $\sum V_m [a_{mn}]F_n$, where $[a]$ are homothety operators for $a\in \mathbb{Z}$.

\paragraph{additional structure}
Verschiebung operators $V_m$, Frobenius operators $F_m$ (ring endomorphism), homothety operators $[a]$.

\begin{align*}
[a]\colon & \alpha(x) \mapsto \alpha(ax)\\
V_m\colon & \alpha(x) \mapsto \alpha(x^m)\\
F_m\colon & \alpha(x) \mapsto \sum_{\zeta^m=1} \alpha(\zeta x^{\frac{1}{m}})\\
F_m\colon & 1-rx \mapsto 1-r^mx
\end{align*}




\begin{remark}
	$W(R)\subset Cart(R)$, $\prod_{m=1}^\infty(1-r_mx^m)=\sum_{m=1}^\infty V_m[a_m]F_m$. See Dayton\& Weibel.
\end{remark}
\begin{prop}
	$[1]=V_1=F_1$: multiplicative identity. There are some identities:
	\begin{align*}
	 V_mV_n&=V_{mn}\\
	 F_mF_n&=F_{mn}\\
	 F_mV_n&=m\\
	 [a]V_m&=V_{m}[a^m]\\
	 F_m[a]&=[a^m] F_m\\
	 [a][b]&=[ab]\\
	 \mbox{if } (k,m)=1,& V_mF_k=F_kV_m\\
	\end{align*}
	
\end{prop}

We call a $W(R)$-module $M$ continuous if $\forall v \in M$, $\textrm{ann}_{W(R)}(v)$ is an open ideal in $W(R)$, that is $\exists k$ s.t. $(1-rx)^m *v =0$ for all $r\in R$ and $m\geqslant k$. Note that if $A$ is an $R$-module, $xA[x]$ is a continuous $W(R)$-module but that $xA[[x]]$ is not.

\section{$W(R)$-module structure} % (fold)
\label{sec:module_structure}

% section module_structure (end)
\paragraph{$W(\mathbb{F}_2)$-module structure on $V=x \mathbb{F}_2[x]$} See Dayton\& Weibel example 2.6, 2.9.

\begin{align*}
 V_m(x^n)&=x^{mn}\\
 F_d(x^n)&=\begin{cases}
 	dx^{n/d},& \mbox{ if $d|n$}\\
 	0,& \mbox{otherwise}
 \end{cases}\\
 [a]x^n&=a^nx^n
 \end{align*}
\paragraph{$W(\mathbb{F}_2)$-module structure on $\Omega_{\mathbb{F}_2[x]}=\mathbb{F}_2[x]\,dx = \oplus_{i=1}^\infty \mathbb{F}_2 e^i$} Dayton\& Weibel example 2.10
\begin{align*}
 V_m(x^{n-1}\,dx)&=mx^{mn-1}\,dx\\
 F_d(x^{n-1}\,dx)&=\begin{cases}
 	x^{n/d-1}\,dx,& \mbox{ if $d|n$}\\
 	0,& \mbox{otherwise}
 \end{cases}\\
 [a]x^{n-1}\,dx&=a^nx^{n-1}\,dx
 \end{align*}
 \begin{remark}
 	$\Omega_{\mathbb{F}_2[x]}$ is {\color{red}not} finitely generated as a module over the $\mathbb{F}_2$-Cartier algebra or over the subalgebra $W(\mathbb{F}_2)$.
 \end{remark}