\chapter{代数几何}
\section{Cheatsheet for Sheaves}
\paragraph{回顾}
首先从presheaves(预层)开始介绍,预层实际上是从$\mathfrak{Top}(X)$到$\mathfrak{Ab}$的反变函子,从而预层之间的态射就是一个自然变换。

接着是介绍层和层之间的态射(自然变换),层实际上就是预层满足一个正合列
\[0\longrightarrow \mathcal{F}(U) \longrightarrow \prod_i \mathcal{F}(U_i) \longrightarrow \prod_{i,j} \mathcal{F}(U_i \cap U_j) \]

紧接着介绍stalks, germs,a stalk是一个正极限(colimit),它的重要作用是说明层之间的同构等价于stalks之间的同构。

接下来定义层之间态射的核kernels,余核cokernels,像images,这个在后面是要说明层可以作成一个Abel范畴。首先预层态射的kernels, cokernels, images是容易定义的,可以证明预层的kernels是一个层,另外两个不是,于是引入了层化``sheafification'',并且还需要说明(sheafification, forgetful) 是伴随函子,并且预层在一点的stalk和层化后是一样的。介绍Quotient sheaf并介绍单射,满射和如何判定(local pointview)。

上面都是在同一个拓扑空间$X$上讨论的,最后用两个拓扑空间之间的连续映射定义了direct image, inverse image, restriction.


\subsection*{Definitions} 
	\begin{itemize}
	\item Presheaves $\mathcal{F}$, sections, $\mathcal{F}(U)=\Gamma(U,\mathcal{F})$, morphisms(natural transformations), isomorphisms
	
	Sheaves, morphisms, isomorphisms 
	\item Stalks {\color{red} $\mathcal{F}_P =\varinjlim_{P\in U}\mathcal{F}(U)$}, germs of sections of $\mathcal{F}$ at the point $P$(i.e. elements of the stalk)
	\item Kernels, cokernels, images of morphisms of presheaves $\phi \colon \mathcal{F} \longrightarrow \mathcal{G}$

    Kernels, cokernels, images of morphisms of sheaves (SHEAFIFICATION, {\color{red}kernels自然是层},后两者需要层化) 
	\item Subsheaves 是$X$不变,$\mathcal{F}(U)$变成子集,与下文的restriction区别。Quotient sheaves (预层层化后), {\color{red}$(\mathcal{F}/\mathcal{F'})_P=\mathcal{F}_P/\mathcal{F'}_P$} 
	\item Injective $\ker(\phi)=0$; Surjective $\ima(\phi)=\mathcal{G}$; exact sequences 
	\item $$f\colon X\longrightarrow Y$$
\[f_*\colon \mathrm{Sh}(X)\longrightarrow \mathrm{Sh}(Y)\]
$\mathcal{F}$: a sheaf on $X$, the direct image sheaf $f_*(\mathcal{F})$ on $Y$: {\color{red}$f_*(\mathcal{F})(V) = \mathcal{F}(f^{-1}(V))$} for any open set $V \subseteq Y$. 

\[f^{-1}\colon \mathrm{Sh}(Y)\longrightarrow \mathrm{Sh}(X)\]
$\mathcal{F}$: a sheaf on $Y$, the inverse image sheaf $f^{-1}(\mathcal{G})$ on $X$: the sheafification of the presheaf {\color{red}$U\mapsto \varinjlim_{V\supseteq f(U)}\mathcal{G}(V)$}.


$i\colon Z\longrightarrow X$ inclusion, $Z$ is a subset of $X$, $i^{-1}\mathcal{F}$ is called the restriction of $\mathcal{F}$ to $Z$, often denoted by $\mathcal{F}|_Z$. Note that $(\mathcal{F}|_Z)_P =\mathcal{F}|_P.$
\end{itemize}

\subsection*{Examples} 
\begin{itemize}
	\item $\mathcal{O}$: the sheaf of regular functions on $X$. The stalk $\mathcal{O}_P$ is the {\color{red}local ring} of $P$ on $X$.
	\item the sheaf of continuous real-valued functions on any topological space

the sheaf of differentiable functions on a differentiable manifold

the sheaf of holomorphic functions on a complex manifold
	\item constant sheaf $\mathcal{A}$, $\mathcal{A}(U)=\{\mbox{continuous maps of $U$ into $A$}\}$ ({\color{red}For every connected open set $U$, $\mathcal{A}(U)\cong A$}.)
\end{itemize}






\subsection*{Propostions} 
\begin{itemize}
	\item ``{\color{red}$+$ $\dashv$ $pre$}'': $\Hom_{Sh}(\mathcal{F}^+,\mathcal{G}) \cong \Hom_{PreSh}(\mathcal{F},pre(\mathcal{G}))$. Note that for any point $P$, $\mathcal{F}^+_P\cong \mathcal{F}_P$, and if $\mathcal{F}$ was a sheaf then $\mathcal{F}^+$ is isomorphic to $\mathcal{F}$.
	\item $\phi \colon \mathcal{F} \longrightarrow \mathcal{G}$ is an isomorphism $\Longleftrightarrow$ ${\color{red}\phi_P} \colon \mathcal{F}_P \longrightarrow \mathcal{G}_P$ is an isomorphism for every $P\in X$. 
	\item $\ker(\phi)$ is a subsheaf of $\mathcal{F}$. $\ima(\phi)$ can be identified with a subsheaf of $\mathcal{G}$.
	\item $\phi$ is injective $\Longleftrightarrow$ {\color{red}$\phi(U)$ is injective} for every open set of $X$.(Caution: {\color{red}Not true for surjective})

$\phi$ is surjective $\Longleftrightarrow$ {\color{red}$\phi_P$ on stalks are surjective} for each $P$. 
	\item 
$\cdots \longrightarrow \mathcal{F}^{i-1} \longrightarrow \mathcal{F}^{i} \longrightarrow \mathcal{F}^{i+1} \longrightarrow \cdots$ is exact $\Longleftrightarrow$ $\cdots \longrightarrow \mathcal{F}_{\color{red}P}^{i-1} \longrightarrow \mathcal{F}_{\color{red}P}^{i} \longrightarrow \mathcal{F}_{\color{red}P}^{i+1} \longrightarrow \cdots$ is exact for every $P$.
	\item $\mathrm{Sh(X)}$ is actually an abelian category, so many results in homological algebra like
the long exact (co)homology sequence, the snake lemma, the five lemma etc. work for
sheaves. \\
	 a kernel of a sheaf morphism is a sheaf, arbitrary product of sheaves is also a sheaf,  finite direct sum of sheaves is a sheaf, zero presheaf is a sheaf.

	 $\mathrm{Sh(X)}$ is closed under arbitrary products and kernels, therefore $\mathrm{Sh(X)}$ is a complete category; it is also cocomplete and additive
\end{itemize}

\subsection*{Exercises}
First, we give some useful results.
\begin{lemma*}
	1. Left adjoint functors are right exact and commute with(preserve) colimits(cocontinuous).\\
	2. Right adjoint functors are left exact and commute with(preserve) limits(continuous).\\
	3. Finite limits commute with filtered colimits in the category of sets.\\
	4. Arbitrary limits commute.
\end{lemma*}
\begin{example}
	1. left adjoint functors: ``sheafification'', $\otimes$, $f^{-1}$, they are right exact.\\
	2. right adjoint functors: ``forgetful functor'', $\Hom$, $f_*$, they are left exact.\\
	3. adjoint pairs: (sheafification,forgetful functor), $(\otimes,\Hom)$, $(f^{-1},f_*)$\\
	4. colimits: $\varinjlim$(like stalks), $\coker$, pushout, coproduct(like $\oplus$)\\
	5. limits: $\varprojlim$, $\ker$, pullback, product(like $\prod$ or $\times$)
\end{example}
\begin{corollary}
	1. Sheafification preserves stalks, surjections, colimts, $\oplus$. It is cocontinuous, that is, preserves colimits.\\
	2. The Forgetful functor preserves injections(kernels), limits, $\prod$.
\end{corollary}
\begin{remark*}
	In fact, the sheafification functor is exact. We will prove it in Exercise 1.4. It also preserves finite limits because it preserves finite products and kernels.

	We summarize all the facts about sheafification here:\\
   The sheafification is a exact functor (see \url{http://stacks.math.columbia.edu/tag/00WJ}), it preserves colimts(stalks, coker, sujections, coproduct $\oplus$) and finite limits(ker, injections, finite product).
\end{remark*}

Now we can go through the exercises
\begin{itemize}
	\item 1 
\end{itemize}



