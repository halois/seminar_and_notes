\chapter{Witt rings and $NK$-groups}
References:
\begin{itemize}
	\item C.\,A.\ Weibel, Mayer-Vietoris sequences and module structures on $NK_*$, pp. 466–493 in Lecture Notes
in Math. 854, Springer-Verlag, 1981.
	\item D.\, R.\ Grayson, Grothendieck rings and witt vectors.
	\item C.\,A.\ Weibel, The $K$-Book: An Introduction to Algebraic $K$-theory.
\end{itemize}
\section{Typical Witt rings} % (fold)
\label{sec:typical_witt_rings}
%介绍p-Witt ring
% section typical_witt_rings (end)
\section{Big Witt rings} % (fold)
\label{sec:big_witt_rings}

% section big_witt_rings (end)
\section{Module structure on $NK_*$} % (fold)
\label{sec:module_structure_on_}
\paragraph{Notations} $\Lambda$: a ring with $1$\\
$R$: commutative ring \\
$W(R)$: Witt ring of $R$\\
$\mathbf{End}(\Lambda)$: the exact category of endomorphisms of finitely generated projective right $\Lambda$-modules.\\
$\mathbf{Nil}(\Lambda)$: the full exact subcategory of nilpotent endomorphisms.\\
$\mathbf{P}(\Lambda)$: the exact category of finitely generated projective right $\Lambda$-modules.

Goals:
\begin{itemize}
	\item Define the $\End_0(R)$-module structure on $NK_*(\Lambda)$ 
	\item (Stienstra's observation) this can extend to a $W(R)$-module structure.
	\item Computations in $W(R)$ with Grothendieck rings.
\end{itemize}
\subsection{$\End_0(\Lambda)$}
Let $\mathbf{End}(\Lambda)$ denote the exact category of endomorphisms of finitely generated projective right $\Lambda$-modules.\\
Objects: pairs $(M,f)$ with $M$ finitely generated projective and $f\in \End(M)$.\\
Morphisms: $(M_1,f_1) \overset{\alpha}\longrightarrow (M_2,f_2)$ with $f
_2\circ \alpha =\alpha \circ f_1$, i.e. such $\alpha$ make the following diagram commutes
\[
\begin{tikzcd}
	M_1 \arrow[r,"f_1"] \arrow[d,"\alpha"] & M_1 \arrow[d,"\alpha"]\\
	M_2 \arrow[r,"f_2"]  & M_2
\end{tikzcd}
\]
There are two interesting subcategories of $\mathbf{End}(\Lambda)$ --- \\
$\mathbf{Nil}(\Lambda)$: the full exact subcategory of nilpotent endomorphisms.\\
$\mathbf{P}(\Lambda)$: the exact category of finitely generated projective right $\Lambda$-modules. (Remark: the reflective subcategory of zero endomorphisms is natually equivalent to $\mathbf{P}(\Lambda)$. Note that a full subcategory $i\colon \mathcal{C} \longrightarrow \mathcal{D}$ is called reflective if the inclusion functor $i$ has a left adjoint $T$, $(T \dashv i) \colon \mathcal{C}  \rightleftarrows \mathcal{D}$.)

Since inclusions are split and all the functors below are exact, they induce homomorphisms between $K$-groups
\[\mathbf{P}(\Lambda)  \rightleftarrows \mathbf{Nil}(\Lambda)\]
\[\mathbf{P}(\Lambda)  \rightleftarrows \mathbf{End}(\Lambda)\]
\[M \mapsto (M,0)\]
\[M \mapsfrom (M,f)\]
\begin{definition}
	$K_n(\mathbf{End}(\Lambda))=K_n(\Lambda) \oplus \End_n(\Lambda)$, $K_n(\mathbf{Nil}(\Lambda))=K_n(\Lambda) \oplus \Nil_n(\Lambda)$
\end{definition}
Now suppose $\Lambda$ is an $R$-algebra for some commutative ring $R$, then there are exact pairings (i.e. bifunctors):
\begin{align*}
	\otimes\colon &\mathbf{End}(R) \times \mathbf{End}(\Lambda) \longrightarrow \mathbf{End}(\Lambda) \\
	\otimes\colon &\mathbf{End}(R) \times \mathbf{Nil}(\Lambda) \longrightarrow \mathbf{Nil}(\Lambda) \\
 				  & (M,f) \otimes (N,g)=(M\otimes_R N, f\otimes g)
\end{align*}
These induce (use ``generators-and-relations'' tricks on $K_0$)
\begin{align*}
	K_0(\mathbf{End}(R)) \otimes K_*(\mathbf{End}(\Lambda)) \longrightarrow K_*(\mathbf{End}(\Lambda)) \\
	K_0(\mathbf{End}(R)) \otimes K_*(\mathbf{Nil}(\Lambda)) \longrightarrow K_*(\mathbf{Nil}(\Lambda)) \\
\end{align*}
$[(0,0)], [(R,1)]\in K_0(\mathbf{End}(R))$ act as the zero and identity maps.

I think we can fix an element $(M,f)\in \mathbf{End}(R)$, then $(M,f)\otimes$ induces an endofunctor of $\mathbf{End}(\Lambda)$. We can get endomorphisms of $K$-groups, then we check that this does not depent on the isomorphism classes and the bilinear property. (Can also see Weibel The $K$-book chapter2, chapter3 Cor 1.6.1, Ex 5.4, chapter4 Ex 1.14.)

If we take $R = \Lambda$, we see that $K_0(\mathbf{End}(R))$ is a commutative ring with unit $[(R,1)]$. $K_0(R)$ is an
ideal, generated by the idempotent $[(R,0)]$, and the quotient
ring is $\End_0(R)$.  Since $(R,0)\otimes$ reflects $\mathbf{End} (\Lambda)$ into $\mathbf{P} (\Lambda)$,
\[i\colon \mathbf{P} (\Lambda) \longrightarrow \mathbf{End} (\Lambda);\quad
 (R,0)\otimes\colon \mathbf{End} (\Lambda) \longrightarrow \mathbf{P} (\Lambda)\]
$K_0(R)$ acts as zero on $\End_*(\Lambda)$ and $\Nil_*(\Lambda)$. (Consider $P\in \mathbf{P}(R)$ acts on $\mathbf{End}(\Lambda)$, $(P,0)\otimes (N,g)=(P\otimes_R N,0) \in \mathbf{P}(\Lambda)$. )

The following is immediate (and well-known):
\begin{prop}
	If  $\Lambda$ is an $R$-algebra with $1$, $\End_*(\Lambda)$ and
$\Nil_*(\Lambda)$ are graded modules over the ring $\End_0 (R)$.
\end{prop}
Now we focus on $*=0$ and $\Lambda =R$:\\

The inclusion of $\mathbf{P}(R)$ in $\mathbf{End}(R)$ by $f = 0$ is split by the forgetful functor, and the kernel $\End_0 (R)$ of $K_0\mathbf{End}(R) \longrightarrow K_0 (R)$ is not only an ideal but a commutative ring with unit $1 = [(R,1)] - [(R, 0)]$.

\begin{theorem}[Almkvist]
The homomorphism (in fact it is a ring homomorphism)
\begin{align*}
	\chi \colon&  \End_0(R)\longrightarrow W(R)=(1+TR[[T]])^{\times}\\
     & (M,f) \mapsto \det(1-fT)
\end{align*}
	is injective and $\End_0(R)\cong \ima \chi =\left\{\frac{g(T)}{h(T)}\in W(R)\mid g(T),h(T) \in 1+TR[T]\right\}$
\end{theorem}
The map $\chi$ (taking characteristic polynomial) is well-difined, and we have 
\[\chi([(R,0)])=1, \quad \chi([(R,1)])=1-T\]
$\chi$ is a ring homomorphism, and $\ima \chi =$ the set of all rational functions in $W(R)$.

\begin{definition}[$NK_*$]
	s above, we define $NK_n(\Lambda)=\ker(K_n(\Lambda[y])\longrightarrow K_n(\Lambda))$. Grayson proved that $NK_n(\Lambda)\cong \Nil_{n-1}(\Lambda)$ in ``Higher algebraic $K$-theory II''.
\end{definition}

% section module_structure_on_ (end)