%!TEX root = ../testmain.tex
\chapter{$NK_2(\mathbb{F}_2[C_4])$}

$NK_i(R)=\ker(K_i(R[x])\overset{x\mapsto 0}\longrightarrow K_i(R)). $

\paragraph{目标} % (fold)
\label{par:目标}
\begin{enumerate}
	\item 证明 $NK_2(\mathbb{F}_2[C_4])$ 中有四阶元, 
	\item 确定它的结构和生成元. 
	\item 推广到$NK_2(\mathbb{F}_{2^f}[C_{2^n}])$和$NK_p(\mathbb{F}_{p^f}[C_{p^n}])$. 
	\item 未来推广到$NK_2(\mathbb{F}_{p^f}[G])$, 其中$G$是有限交换群, $G=C_{p^n} \times H$. 
\end{enumerate}
% paragraph 目标 (end)
































 
\section{思路}
已经有$NK_2(\mathbb{F}_2[C_2])$和$NK_2(\mathbb{F}_p[C_p])$的结果\cite{Guin-Waléry1981}. 这种方法详细见下文\ref{sec:C2Cp}. 

考虑$\mathbb{F}_{p^f}[C_{p^n}]$时, 可以将它{\color{blue}写成截断多项式$\mathbb{F}_{p^f}[t_1]/(t_1^{p^n})\cong \mathbb{F}_{p^f}[C_{p^n}]$}, $t_1\mapsto 1-\sigma$. 

\begin{align*}
\mathbb{F}_{p^f}[C_{p^n}]=\mathbb{F}_{p^f}[\sigma]/(1-\sigma^{p^n}) = \mathbb{F}_{p^f}[\sigma]/((1-\sigma)^{p^n}) &\cong \mathbb{F}_{p^f}[t_1]/(t_1^{p^n}) \\
 1-\sigma & \mapsto t_1
\end{align*}




考虑它的$NK_2$时, 可以{\color{blue}转化成相对$K_2$群}:
\[NK_2(\mathbb{F}_{p^f}[C_{p^n}]) \cong K_2(\mathbb{F}_{p^f}[t_1, t_2]/(t_1^{p^n}), (t_1))\cong K_2(\mathbb{F}_{p^f}[t_1, t_2]/(t_1^{p^n})). \]

利用van der Kallen\cite{MR86f:18017}中的方法对于这种情形的相对$K_2$群, 有这样的结论(符号说明见后文):
\begin{theorem*}
	$\Gamma_{\alpha, i}$诱导了{\color{red}同构}
\[ K_2(A, M)\cong \bigoplus_{(\alpha, i)\in\Lambda^{00}}\big(1+xk[x]/(x^{[\alpha, i]})\big)^{\times}. \]
\end{theorem*}

接下来的任务就是确定两件事情:
\begin{itemize}
	\item 确定集合{\color{blue} $\Lambda^{00}$}, 确定数值{\color{blue} $[\alpha, i]$}. 
	\item 确定右边这个{\color{blue}乘法群}的结构. 
\end{itemize}
实际上对于第一件事情, 我们只需要考虑这个集合的一部分, 因为$(1+xk[x]/(x))^{\times}$是平凡的, 所以只要考虑$[\alpha, i]>1$所对应的$(\alpha, i)$全体(后文详细解释). 

对于第二件事情, 有一些例子是可以直接计算的, 如$(1+x\mathbb{F}_2[x]/(x^{4}))^{\times}\cong \mathbb{Z}/4 \mathbb{Z} \oplus\mathbb{Z}/2 \mathbb{Z}$. 对于一般的情况引入big Witt vectors, 
\[BigWitt_n(\mathbb{F}_q):=(1+x \mathbb{F}_q\llbracket x\rrbracket )^{\times}/(1+x^{n+1} \mathbb{F}_q\llbracket x\rrbracket )^{\times} \cong (1+x\mathbb{F}_q[x]/(x^{n+1}))^{\times}\]
由big Witt vectors{\color{blue} 分解成typical Witt vectors}有
\begin{align*}
(1+x\mathbb{F}_{p^f}[x]/(x^{n+1}))^{\times}&\cong BigWitt_n(\mathbb{F}_{p^f}) \\
&\cong \bigoplus_{\scriptsize\substack{1\leq m\leq n \\ gcd(m, p)=1}}W_{1+ \left \lfloor\log_p \frac{n}{m}  \right \rfloor}(\mathbb{F}_{p^f}) \\
& = \bigoplus_{\scriptsize\substack{1\leq m\leq n \\ gcd(m, p)=1}}(\mathbb{Z}/p^{1+ \left \lfloor\log_p \frac{n}{m}  \right \rfloor}\mathbb{Z})^f, 
\end{align*}
其中$ \left \lfloor x \right \rfloor$表示不超过$x$的最大整数. 

结合这个同构\ref{K2(A, M)}就可以得到结论. 

\paragraph{结论} % (fold)
\label{par:结论}
已知的结果{\color{gray}
\begin{theorem*}
	(1)$NK_2(\mathbb{F}_2[C_2])\cong \bigoplus_{\infty} \mathbb{Z}/2 \mathbb{Z}$, \\
	(2)$NK_2(\mathbb{F}_2[C_2])\cong K_2(\mathbb{F}_2[t, x]/(t^2), (t))$是由Dennis-Stein符号$\{\langle tx^i, x \rangle \mid i\geq 0\}$与$\{\langle tx^i, t \rangle \mid i\geq 1\text{为奇数}\}$生成的, 这样的符号均为$2$阶元. 
\end{theorem*}
}
$NK_2(\mathbb{F}_2[C_4])$的结果
{\color{red}\begin{theorem*}
	(1)$NK_2(\mathbb{F}_2[C_4])\cong \bigoplus_{\infty} \mathbb{Z}/2 \mathbb{Z}\oplus \bigoplus_{\infty}\mathbb{Z}/4 \mathbb{Z}$, \\
	(2)$NK_2(\mathbb{F}_2[C_4])$是由Dennis-Stein符号
	\[\{\langle tx^{i-1}, x \rangle \mid i\geq 1\}, \{\langle tx^i, t \rangle \mid i\geq 1\text{为奇数}\}, \{\langle t^3x^{i-1}, x \rangle \mid i\geq 1\}, %\{\langle t^3x^{i-1}, x \rangle \mid i\geq 1, gcd(i, 3)=1\}, 
	\{\langle t^3x^i, t \rangle \mid i\geq 1\text{为奇数}\}\]
	生成的. 
\end{theorem*}}

$NK_2(\mathbb{F}_q[C_{2^n}])$的结果
{\color{red}\[NK_2(\mathbb{F}_q[C_{2^n}])\cong \bigoplus_\infty \bigoplus_{k=1}^n \mathbb{Z}/2^k\mathbb{Z}. \]}

% paragraph 结论 (end)


























\section{预备知识和引理}
这一节主要是介绍前文提到的{\color{blue} 如何将$NK_2$转化成相对$K_2$群, \cite{MR86f:18017}中符号说明与相对$K_2$群, 以及Witt向量的分解}. 

%!TEX root = ../../testmain.tex
%  复制于原来给老师看时做的笔记的部分,和最早的部分有一些小修改
%  未来直接两边都引用这个部分即可
令$k$是特征为$p>0$的有限域, 考虑两个变元的多项式环$k[t_1, t_2]$, $I=(t_1^n)$是$k[t_1, t_2]$的一个真理想,
% , 满足以下条件
% \begin{enumerate}
% 	\item $I$是由$k[t_1]$中的单项式生成的, 
% 	\item 对于某个$n$, $t_1^n\in I$. 
% \end{enumerate}
% 实际上这样的$I$具有形式$(t_1^n)$, 
令
\[A=k[t_1, t_2]/I, \]
设$M$是$A$的nil根(小根), 即$M=(t_1)$, 则有$A/M=k[t_2]$. 
\begin{prop}
	$K_2(k[t_1, t_2]/(t_1^n), (t_1, t_2))\cong K_2(A, M)=K_2(k[t_1, t_2]/(t_1^n), (t_1))$. 
\end{prop}
% 这个property参考Kallen文章P278
\begin{proof}
	由于$k[t_2]\overset{i_1}\hookrightarrow k[t_1,t_2]/(t_1^n)$, $k[t_1,t_2]/(t_1^n)\overset{p_1}\twoheadrightarrow k[t_2]$与$k\overset{i_2}\hookrightarrow k[t_1,t_2]/(t_1^n)$, $k[t_1,t_2]/(t_1^n)\overset{p_2}\twoheadrightarrow k$满足$p_1i_1=\id$, $p_2i_2=\id$, 故由$K_n$的函子性有$K_n(p_1)K_n(i_1)=\id$与$K_n(p_2)K_n(i_2)=\id$, 从而有以下相对$K$群的可裂正合列
	\[
	0\longrightarrow K_2(k[t_1, t_2]/(t_1^n), (t_1)) \longrightarrow K_2(k[t_1, t_2]/(t_1^n)) \longrightarrow K_2(k[t_2]) \longrightarrow 0
	\]
	\[
	0\longrightarrow K_2(k[t_1, t_2]/(t_1^n), (t_1, t_2)) \longrightarrow K_2(k[t_1, t_2]/(t_1^n)) \longrightarrow K_2(k) \longrightarrow 0
	\]

	由于$k$是有限域, 故$K_2(k)=0$, 又由于有限域都是正则环, 故$NK_2(k)=0$, 于是$K_2(k[t_2])=K_2(k)\oplus NK_2(k)=0$. 从而得
	\[K_2(k[t_1, t_2]/(t_1^n), (t_1))\cong K_2(k[t_1, t_2]/(t_1^n)) \cong K_2(k[t_1, t_2]/(t_1^n), (t_1, t_2)). \]
\end{proof}

当$k=\mathbb{F}_{p^f}$时, $k[t_1]/(t_1^{p^n})\cong \mathbb{F}_{p^f}[C_{p^n}]$, 其中$C_{p^n}$是$p^n$阶循环群. 有以下可裂正合列
\[0\longrightarrow NK_2(\mathbb{F}_{p^f}[C_{p^n}]) \longrightarrow K_2(\mathbb{F}_{p^f}[C_{p^n}][x])\longrightarrow K_2(\mathbb{F}_{p^f}[C_{p^n}]) \longrightarrow 0, \]
由于$K_2(\mathbb{F}_{p^f}[C_{p^n}][x])\cong K_2(\mathbb{F}_{p^f}[t_1, t_2]/(t_1^{p^n}))$, 并且$K_2(\mathbb{F}_{p^f}[C_{p^n}])=0$\cite{MORRIS198091}, 从而
\[NK_2(\mathbb{F}_{p^f}[C_{p^n}])\cong K_2(\mathbb{F}_{p^f}[t_1, t_2]/(t_1^{p^n})) \cong K_2(\mathbb{F}_{p^f}[t_1, t_2]/(t_1^{p^n}), (t_1)). \]


\subsection{Dennis-Stein符号}
\label{sub:dennis_stein_symbols}
一般地, 通过Dennis-Stein符号可以给出$K_2(A, M)=K_2(k[t_1, t_2]/(t_1^n), (t_1))$的一个表现
\begin{itemize}
	\item[生成元]   $\langle a, b \rangle $, 其中$(a, b)\in A\times M \cup M \times A$;
	\item[关系] (DS1) $\langle a, b\rangle = -\langle b, a \rangle$, \\
				(DS2) $\langle a, b\rangle +\langle c, b \rangle=\langle a+c-abc, b\rangle$, \\
	 			(DS3) $\langle a, bc\rangle =\langle ab, c\rangle +\langle ac, b\rangle$, 其中$(a, b, c)\in  M\times A\times A \cup A\times M \times A \cup A\times A\times M$. 
\end{itemize}

\begin{prop}
	对任意环$R$, 任意自然数$q>1$, $K_2(R[t]/(t^q), (t))$由满足上述关系的Dennis-Stein符号$\langle at^i, t\rangle$和$\langle at^i, b\rangle$生成, 其中$a, b\in R, 1\leq i<q$. 
\end{prop}
\begin{proof}
	参见\cite{MR82k:13016} Proposition 1.7. 
\end{proof}

\subsection{符号说明} % (fold)
\label{subsec:符号}
为了陈述定理, 将文中常用的记号详述如下(参考\cite{MR86f:18017})
\begin{itemize}
	\item $\mathbb{Z}_+$表示非负整数全体. 
	\item $\varepsilon^1 = (1, 0)\in \mathbb{Z}_+^2$, $\varepsilon^2 = (0, 1)\in \mathbb{Z}_+^2$.
	\item 对于$\alpha = (\alpha_1,\alpha_2) \in \mathbb{Z}_+^2$, 记$t^{\alpha}=t_1^{\alpha_1}t_2^{\alpha_2}$, 于是有$t^{\varepsilon^1}=t_1$, $t^{\varepsilon^2}=t_2$. 
	\item $\Delta=\{\alpha\in\mathbb{Z}_+^2\mid  t^{\alpha}\in I\}$, 若$\delta \in \Delta$, 则$\delta+\varepsilon^i \in \Delta, i=1, 2$. 
	\item $\Lambda=\{(\alpha, i)\in\mathbb{Z}_+^2 \times \{1, 2\}\mid  \alpha_i\geq 1, t^{\alpha}\in M\}$. 
	\item 对于$(\alpha, i)\in\Lambda$, 令{\color{blue}$[\alpha, i]=\min\{m\in \mathbb{Z}\mid m\alpha - \varepsilon^i\in \Delta\}$}. 
	若$(\alpha, i), (\alpha, j)\in \Lambda$, 有$[\alpha, i]\leq [\alpha, j]+1$. 
	\item 若$gcd(p, \alpha_1, \alpha_2)=1$, 令$[\alpha]=\max\{[\alpha, i]\mid  \alpha_i  \not\equiv 0 \bmod p\}$, 其中$p$是域$k$的特征.
	\item {\color{blue} $\Lambda^{00}= \big\{(\alpha, i)\in \Lambda\mid  gcd(\alpha_1, \alpha_2)=1, i\neq \min\{j\mid \alpha_j\not\equiv 0 \bmod p, [\alpha, j]=[\alpha]\} \big\}$}. 
\end{itemize}
% subsection 符号 (end)
% $\Delta$中对应的是t_1^{\alpha_1}t_2^{\alpha_2}在$I$里的,也就是如果t^\alpha在I里,\langle t^\alpha, t_i\rangle=0,
% 而$\Lambda$意思是说这样的符号是相对K2群里D-S符号, (t^\alpha,i)对应过去就是$\langle t^{\alpha-\varepsilon_i}, t_i$,保证了是相对K2群中的元素
% [\alpha,i]保证了1-x^{[\alpha,i]}f(x)映过去恰好是0,并且是这样子的最小,也就是说1-x^{[\alpha,i]+1}f(x)肯定是0,1-x^{[\alpha,i]-1}f(x)未必是0。

若$(\alpha, i)\in \Lambda$, $f(x)\in k[x]$, 令
\begin{align*}
\Gamma_{\alpha, i} \colon & (1+xk[x])^{\times} \longrightarrow K_2(A,M)\\
				& 1-xf(x)  \mapsto \langle f(t^\alpha)t^{\alpha-\varepsilon^i}, t_i \rangle.
\end{align*}
$\Gamma_{\alpha, i}$是群同态: $(1-xf_1(x))(1-xf_2(x)) = 1-x(f_1(x)+f_2(x)-xf_1(x)f_2(x))$, 
\begin{align*}
	\Gamma_{\alpha, i}((1-xf_1(x))(1-xf_2(x)))&=\Gamma_{\alpha, i}(1-x(f_1(x)+f_2(x)-xf_1(x)f_2(x)))\\
	&=\langle f_1(t^\alpha)t^{\alpha-\varepsilon^i}+f_2(t^\alpha)t^{\alpha-\varepsilon^i}-t^\alpha f_1(t^\alpha)f_2(t^\alpha)t^{\alpha-\varepsilon^i}, t_i \rangle \\
	\Gamma_{\alpha, i}(1-xf_1(x))\Gamma_{\alpha, i}(1-xf_2(x)) & = \langle f_1(t^\alpha)t^{\alpha-\varepsilon^i}, t_i \rangle +\langle f_2(t^\alpha)t^{\alpha-\varepsilon^i}, t_i \rangle\\
	& = \langle f_1(t^\alpha)t^{\alpha-\varepsilon^i}+f_2(t^\alpha)t^{\alpha-\varepsilon^i}-f_1(t^\alpha)t^{\alpha-\varepsilon^i}f_2(t^\alpha)t^{\alpha-\varepsilon^i}t_i, t_i \rangle
\end{align*}

% 这里写1-x^{[\alpha,i]}g(x) 映过去为0

% 注释掉这一部分,需要重新写
% 
% 若$g(t_1, t_2)=t_ih(t_1, t_2)\in \sqrt{I}=(t_1)$, 令
% \[\Gamma_i(1-g(t_1, t_2))=\langle h(t_1, t_2), t_i \rangle, \]
% 且有
% \[\Gamma_{\alpha, i}(1-xf(x))=\Gamma_i(1-t^{\alpha} f(t^{\alpha})). \]

% 由于$t_1\in \sqrt{I}$, $\Gamma_1$诱导了同态
% \begin{align*}
% (1+t_1k[t_1, t_2]/t_1 I)^{\times} &\longrightarrow K_2(A, M)\\
% 1-g(t_1, t_2) & \mapsto \langle h(t_1, t_2), t_1 \rangle
% \end{align*}
% $\Gamma_2$诱导了同态
% \begin{align*}
% (1+t_2\sqrt{I}/t_2 I)^{\times} &\longrightarrow K_2(A, M)\\
% 1-g(t_1, t_2) & \mapsto \langle h(t_1, t_2), t_2 \rangle
% \end{align*}
% 
% 上面需要重写
若$(\alpha, i)\in\Lambda$, $\Gamma_{\alpha, i}$诱导了同态
\[\big(1+xk[x]/(x^{[\alpha, i]})\big)^{\times} \longrightarrow K_2(A, M). \]
\begin{theorem}
\label{K2(A, M)}
	$\Gamma_{\alpha, i}$诱导了{\color{red}同构}
\[ K_2(A, M)\cong \bigoplus_{(\alpha, i)\in\Lambda^{00}}\big(1+xk[x]/(x^{[\alpha, i]})\big)^{\times}. \]
\end{theorem}
\begin{proof}
	参见\cite{MR86f:18017} Corollary 2.6.. 
\end{proof}

\subsection{Witt向量}
令$R$是交换环, $R$上的泛Witt向量环(the ring of universal/big Witt vectors over $R$)$BigWitt(R)$作为交换群同构于$(1+xR\llbracket x\rrbracket )^{\times}$, 即常数项为$1$的形式幂级数全体在乘法运算下形成的交换群, 
\begin{align*}
BigWitt(R) &\overset{\sim}\longrightarrow (1+xR\llbracket x\rrbracket )^{\times}\\
(r_1, r_2, \cdots) & \mapsto \prod_{n=1}^{\infty}(1-r_n x^n). 
\end{align*}



考虑$(1+xR\llbracket x\rrbracket )^{\times}$的子群$(1+x^{n+1}R\llbracket x\rrbracket )^{\times}$, 记交换群%$n$次截断泛Witt向量环(这个定义不太好说,underlying group不知怎么翻译)
$BigWitt_n(R)=(1+xR\llbracket x\rrbracket )^{\times}/(1+x^{n+1}R\llbracket x\rrbracket )^{\times}$. 显然$BigWitt_1(R)=R$, 并且注意当$n\geq 3$时, $BigWitt_n(\mathbb{F}_2)$不是循环群. 

% 
% 下方有修改, 				|^^^^|				请对比来选择. 
% 下方有修改, 				|^^^^|				请对比来选择. 
% 下方有修改, 				|^^^|				请对比来选择. 
% 下方有修改, 				|^^^|				请对比来选择. 
% 下方有修改, 				|^^^|				请对比来选择. 
% 下方有修  				|^^|				 对比来选择. 
% 下方有修  				|^^|				 对比来选择. 
% 下方有修  				|^^|				 对比来选择. 
% 下方有修  				|^^|				 对比来选择
% 下方有修  				|^|					 对比来选择
% 下方有     				|^|					  比来选泽
% 下方有    				||					   比来选 
% 							
% 									
%   修改 				    口					    修改
%   修改 				    口					    修改
%
{\color{blue}\begin{lemma}
	 $BigWitt_n(\mathbb{F}_q)\cong (1+x\mathbb{F}_q[x]/(x^{n+1}))^{\times}$. 
\end{lemma}}
\begin{proof}
	% 由定义$BigWitt_n(\mathbb{F}_q):=(1+x \mathbb{F}_q\llbracket x\rrbracket )^{\times}/(1+x^{n+1} \mathbb{F}_q\llbracket x\rrbracket )^{\times}$, 
	考虑群的满同态
	\begin{align*}
	(1+x \mathbb{F}_q\llbracket x\rrbracket )^{\times} &\longrightarrow (1+x \mathbb{F}_q[x]/(x^{n+1}))^{\times}\\
	1+\sum_{i\geq 1}a_i x^i &\mapsto 1+\sum_{i = 1}^{n}a_i x^i +(x^{n+1})
	\end{align*}
	其中$a_i\in \mathbb{F}_q$, 易知同态核为$(1+x^{n+1} \mathbb{F}_q\llbracket x\rrbracket )^{\times}$, 
	从而$BigWitt_n(\mathbb{F}_q)=(1+x \mathbb{F}_q\llbracket x\rrbracket )^{\times}/(1+x^{n+1} \mathbb{F}_q\llbracket x\rrbracket )^{\times} \cong (1+x\mathbb{F}_q[x]/(x^{n+1}))^{\times}$. 
\end{proof}


% 上面为新加
\begin{example} 
\label{ex:W3(F2)}
	$BigWitt_3(\mathbb{F}_2)\cong (1+x\mathbb{F}_2[x]/(x^{4}))^{\times}\cong \mathbb{Z}/4 \mathbb{Z} \oplus\mathbb{Z}/2 \mathbb{Z}$.
\end{example}
\begin{proof}
	$1+x\in (1+x \mathbb{F}_2[x]/(x^4))^{\times}$是$4$阶元, 由它生成的子群$\langle 1+x \rangle = \{1, 1+x, 1+x^2, 1+x+x^2+x^3\}$, 且$1+x^3$是二阶元, $\langle 1+x^3 \rangle = \{1, 1+x^3\}$. 令$\sigma, \tau$分别是$\mathbb{Z}/4 \mathbb{Z}$和$\mathbb{Z}/2 \mathbb{Z}$的生成元, 则有同构
		\begin{align*}
		\mathbb{Z}/4 \mathbb{Z} \oplus \mathbb{Z}/2 \mathbb{Z} &\longrightarrow BigWitt_4(\mathbb{F}_2) \\
		(\sigma, \tau) & \mapsto (1+x)(1+x^3)=1+x+x^3. 
		\end{align*}
\end{proof}

\begin{example}
	$BigWitt_4(\mathbb{F}_2) \cong \mathbb{Z}/8 \mathbb{Z} \oplus \mathbb{Z}/2 \mathbb{Z}. $
\end{example}
\begin{proof}
	$1+x \in BigWitt_5(\mathbb{F}_2)$是$8$阶元, 由它生成的子群$\langle 1+x \rangle = \{1, 1+x, 1+x^2, 1+x+x^2+x^3, 1+x^4, 1+x+x^4, 1+x^2+x^4, 1+x+x^2+x^3+x^4\}$, 另外$1+x^3$是二阶元, $\langle 1+x^3 \rangle = \{1, 1+x^3\}$. 令$\sigma, \tau$分别是$\mathbb{Z}/8 \mathbb{Z}$和$\mathbb{Z}/2 \mathbb{Z}$的生成元, 则有同构
	\begin{align*}
	\mathbb{Z}/8 \mathbb{Z} \oplus \mathbb{Z}/2 \mathbb{Z} &\longrightarrow BigWitt_4(\mathbb{F}_2) \\
	(\sigma, \tau) & \mapsto (1+x)(1+x^3)=1+x+x^3+x^4 
	\end{align*}
	于是$(\sigma^i, \tau^j), 0\leq i <8, 0\leq j<2$对应于$(1+x)^i(1+x^3)^j$, 详细的对应如下
	\begin{align*}
	(1, \tau) & \mapsto 1+x^3, & (\sigma, \tau) & \mapsto 1+x+x^3+x^4, \\
	 (\sigma^2, \tau) & \mapsto 1+x^2+x^3, & (\sigma^3, \tau) & \mapsto 1+x+x^2+x^4, \\
	(\sigma^4, \tau) & \mapsto 1+x^3+x^4, & (\sigma^5, \tau) & \mapsto 1+x+x^3, \\
	 (\sigma^6, \tau) & \mapsto 1+x^2+x^3+x^4, & (\sigma^7, \tau) & \mapsto 1+x+x^2, \\
	(1, 1)& \mapsto 1, & (\sigma, 1) & \mapsto 1+x, \\
	(\sigma^2, 1) & \mapsto 1+x^2, & (\sigma^3, 1) & \mapsto 1+x+x^2+x^3, \\
	(\sigma^4, 1) & \mapsto 1+x^4, &
	(\sigma^5, 1) & \mapsto 1+x+x^4, \\
	(\sigma^6, 1) & \mapsto 1+x^2+x^4, & (\sigma^7, 1) & \mapsto 1+x+x^2+x^3+x^4. 
	\end{align*}
\end{proof}



固定素数$p$, 考虑局部环$\mathbb{Z}_{(p)}=\mathbb{Z}[1/\ell \mid \text{所有素数}\ell\neq p]$, 即$\mathbb{Z}$在非零素理想$(p)=p \mathbb{Z}$处的局部化, 于是$\mathbb{Z}_{(p)}$-代数$R$就是所有除$p$外的素数均在其中可逆的交换环, 如$\mathbb{F}_{p^n}$是一个$\mathbb{Z}_{(p)}$-代数. 

下面考虑$p$-Witt向量环$W(A)$与截断$p$-Witt向量环$W_n(A)$, $p$-Witt向量为$(a_0, a_1, \cdots)$, 加法用Witt多项式定义, 本文仅考虑用加法定义的交换群结构, 例如
% $W(\mathbb{F}_p)=\mathbb{Z}_{p}$,
作为交换群$W_n(\mathbb{F}_{p^f})$同构于$(\mathbb{Z}/p^n\mathbb{Z})^f$. %这里可以加上Galois环 并且注意F_p^f是F_p上f维向量空间,因此底层群是(C_p)^f

Artin-Hasse级数定义为
\[AH(x)= \exp(\sum_{n\geq 0}\frac{x^{p^n}}{p^n})=1+x+\cdots \in 1+x \mathbb{Q}\llbracket x\rrbracket , \]
实际上$AH(x)\in 1+x \mathbb{Z}_{(p)}\llbracket x\rrbracket $(参考\cite{rabinoff2014theory} Theorem 7.2). 对于$BigWitt(R)=(1+xR\llbracket x\rrbracket)^{\times}$中的任一元素$\alpha$可以写成无穷乘积
\[\alpha = \prod_{n=1}^{\infty}(1-r_nx^n),  \]
其中$\ r_n\in R$是唯一的. 若$A$是$\mathbb{Z}_{(p)}$-代数, $BigWitt(A)=(1+xA\llbracket x\rrbracket)^{\times}$中的任一元素$\alpha$还有如下表法 \cite{katz2013witt}
\[\alpha = \prod_{n\geq 1}AH(a_n x^n), \ a_n\in A. \]
将整数$n$写成$n=mp^a$, 使得$gcd(m, p)=1, a\geq 0$, 由于$A$是$\mathbb{Z}_{(p)}$-代数, $m$可逆, 从而$[x\mapsto x^{1/m}]\in \End(BigWitt(A))$是双射, 于是我们可以将$\alpha\in BigWitt(A)$以如下的形式表出
\[\prod_{\scriptsize\substack{m\geq 1 \\ gcd(m, p)=1  \\ a\geq 0}}AH(a_{mp^a} x^{mp^a})^{1/m}. \]

另一方面对于$\mathbb{Z}_{(p)}$-代数$A$, 下列映射是群同态
\begin{align*}
W(A)&\longrightarrow BigWitt(A)\\
(a_0, a_1, \cdots) &\mapsto \prod_{i\geq 0}AH(a_i x^i). 
\end{align*}

作为交换群, $BigWitt_n(A)$可以分解为$p$-Witt向量环的直和, 实际上有以下同构\cite{Lauter1999A} % 这个引用还待查,再找更精确的表述的
\[
BigWitt(A) \cong \prod_{\scriptsize\substack{m\geq 1 \\ gcd(m, p)=1}} W(A), 
\]
元素$\prod\limits_{\scriptsize\substack{m\geq 1 \\ gcd(m, p)=1  \\ a\geq 0}}AH(a_{mp^a} x^{mp^a})^{1/m}$对应于一个$m$-分量为$(a_m, a_{mp}, a_{mp^2}, \cdots)\in W(A)$的Witt向量. 
对于截断的Witt向量环, 有同构\[
BigWitt_n(A) \cong \bigoplus_{\scriptsize\substack{1\leq m\leq n \\ gcd(m, p)=1}} W_{\ell(m, n)}(A), 
\]
其中$\ell(m, n)$是一个整数, 定义为
\[\ell(m, n)=1+\left \lfloor\log_p \frac{n}{m}  \right \rfloor, \]
即$\ell(m, n)=1+\text{使得$mp^k\leq n$成立的最大整数$k$}.$

{\color{blue}考虑特征为$p$的有限域$\mathbb{F}_q$, 有同构\cite{Lauter1999A}
\[BigWitt_n(\mathbb{F}_{q}) \cong \bigoplus_{\scriptsize\substack{1\leq m\leq n \\ gcd(m, p)=1}} W_{\ell(m, n)}(\mathbb{F}_{q}), \]}
注意到$\sum\limits_{\scriptsize\substack{1\leq m\leq n \\ gcd(m, p)=1}} \ell(m, n) = n$, 因此两边都是$q^n$阶交换群. 
\begin{corollary}
\label{cor:BW}
	若有限域$\mathbb{F}_{p^f}$的特征$ch(\mathbb{F}_{p^f})=p$, 则作为交换群有
	\[
	BigWitt_n(\mathbb{F}_{p^f})\cong \bigoplus_{\scriptsize\substack{1\leq m\leq n \\ gcd(m, p)=1}}W_{1+ \left \lfloor\log_p \frac{n}{m}  \right \rfloor}(\mathbb{F}_{p^f}) = \bigoplus_{\scriptsize\substack{1\leq m\leq n \\ gcd(m, p)=1}}(\mathbb{Z}/p^{1+ \left \lfloor\log_p \frac{n}{m}  \right \rfloor}\mathbb{Z})^f, 
	\]
	其中$ \left \lfloor x \right \rfloor$表示不超过$x$
	的最大整数. 
\end{corollary}
\begin{example}
	作为交换群, $BigWitt_3(\mathbb{F}_2)= W_{\ell(1, 3)}(\mathbb{F}_2)\oplus W_{\ell(3, 3)}(\mathbb{F}_2)=W_2(\mathbb{F}_2)\oplus W_1(\mathbb{F}_2)=\mathbb{Z}/4 \mathbb{Z}\oplus	\mathbb{Z}/2 \mathbb{Z}$, 

	$BigWitt_4(\mathbb{F}_2)= W_{\ell(1, 4)}(\mathbb{F}_2)\oplus W_{\ell(3, 4)}(\mathbb{F}_2)=W_3(\mathbb{F}_2)\oplus W_1(\mathbb{F}_2)=\mathbb{Z}/8 \mathbb{Z}\oplus	\mathbb{Z}/2 \mathbb{Z}$, 

	$BigWitt_2(\mathbb{F}_3)= W_{\ell(1, 2)}(\mathbb{F}_3)\oplus W_{\ell(2, 2)}(\mathbb{F}_3)=W_1(\mathbb{F}_3)\oplus W_1(\mathbb{F}_3)=\mathbb{Z}/3 \mathbb{Z}\oplus	\mathbb{Z}/3 \mathbb{Z}$. 

\end{example}






% %粘贴自NKFinAlg. tex, 后来有修改
% % todo:把这里修改的地方和原文对应起来
% 令$k$是特征为$p>0$的有限域, 考虑两个变元的多项式环$k[t_1, t_2]$, $I=(t_1^n)$是$k[t_1, t_2]$的一个真理想, 
% % , 满足以下条件
% % \begin{enumerate}
% % 	\item $I$是由$k[t_1]$中的单项式生成的, 
% % 	\item 对于某个$n$, $t_1^n\in I$. 
% % \end{enumerate}
% % 实际上这样的$I$具有形式$(t_1^n)$, 
% 令
% \[A=k[t_1, t_2]/I, \]
% 设$M$是$A$的nil根(小根), 即$M=(t_1)$, 则有$A/M=k[t_2]$. 
% \begin{prop}
% 	$K_2(k[t_1, t_2]/(t_1^n), (t_1, t_2))\cong K_2(A, M)=K_2(k[t_1, t_2]/(t_1^n), (t_1))$. 
% \end{prop}
% % 这个property参考Kallen文章P278
% \begin{proof}
% 	由于$k[t_2]\overset{i_1}\hookrightarrow k[t_1,t_2]/(t_1^n)$, $k[t_1,t_2]/(t_1^n)\overset{p_1}\twoheadrightarrow k[t_2]$与$k\overset{i_2}\hookrightarrow k[t_1,t_2]/(t_1^n)$, $k[t_1,t_2]/(t_1^n)\overset{p_2}\twoheadrightarrow k$满足$p_1i_1=\id$, $p_2i_2=\id$, 故由$K_n$的函子性有$K_n(p_1)K_n(i_1)=\id$与$K_n(p_2)K_n(i_2)=\id$, 从而有以下相对$K$群的可裂正合列
% 	\[
% 	0\longrightarrow K_2(k[t_1, t_2]/(t_1^n), (t_1)) \longrightarrow K_2(k[t_1, t_2]/(t_1^n)) \longrightarrow K_2(k[t_2]) \longrightarrow 0
% 	\]
% 	\[
% 	0\longrightarrow K_2(k[t_1, t_2]/(t_1^n), (t_1, t_2)) \longrightarrow K_2(k[t_1, t_2]/(t_1^n)) \longrightarrow K_2(k) \longrightarrow 0
% 	\]

% 	由于$k$是有限域, 故$K_2(k)=0$, 又由于有限域都是正则环, 故$NK_2(k)=0$, 于是$K_2(k[t_2])=K_2(k)\oplus NK_2(k)=0$. 从而得
% 	\[K_2(k[t_1, t_2]/(t_1^n), (t_1))\cong K_2(k[t_1, t_2]/(t_1^n)) \cong K_2(k[t_1, t_2]/(t_1^n), (t_1, t_2)). \]
% \end{proof}

% 当$k=\mathbb{F}_{p^f}$时, $k[t_1]/(t_1^{p^n})\cong \mathbb{F}_{p^f}[C_{p^n}]$, 其中$C_{p^n}$是$p^n$阶循环群. 有以下可裂正合列
% \[0\longrightarrow NK_2(\mathbb{F}_{p^f}[C_{p^n}]) \longrightarrow K_2(\mathbb{F}_{p^f}[C_{p^n}][x])\longrightarrow K_2(\mathbb{F}_{p^f}[C_{p^n}]) \longrightarrow 0, \]
% 由于$K_2(\mathbb{F}_{p^f}[C_{p^n}][x])\cong K_2(\mathbb{F}_{p^f}[t_1, t_2]/(t_1^{p^n}))$, 并且$K_2(\mathbb{F}_{p^f}[C_{p^n}])=0$\cite{MORRIS198091}, 从而
% \[NK_2(\mathbb{F}_{p^f}[C_{p^n}])\cong K_2(\mathbb{F}_{p^f}[t_1, t_2]/(t_1^{p^n})) \cong K_2(\mathbb{F}_{p^f}[t_1, t_2]/(t_1^{p^n}), (t_1)). \]


% \subsection{Dennis-Stein符号}
% \label{sub:dennis_stein_symbols}
% 一般地, 通过Dennis-Stein符号可以给出$K_2(A, M)=K_2(k[t_1, t_2]/(t_1^n), (t_1))$的一个表现
% \begin{itemize}
% 	\item[生成元]   $\langle a, b \rangle $, 其中$(a, b)\in A\times M \cup M \times A$;
% 	\item[关系] (DS1) $\langle a, b\rangle = -\langle b, a \rangle$, \\
% 				(DS2) $\langle a, b\rangle +\langle c, b \rangle=\langle a+c-abc, b\rangle$, \\
% 	 			(DS3) $\langle a, bc\rangle =\langle ab, c\rangle +\langle ac, b\rangle$, 其中$(a, b, c)\in A\times M \times A \cup M\times A\times M$. 
% \end{itemize}

% \begin{prop}
% 	对任意环$R$, 任意自然数$q>1$, $K_2(R[t]/(t^q), (t))$由满足上述关系的Dennis-Stein符号$\langle at^i, t\rangle$和$\langle at^i, b\rangle$生成, 其中$a, b\in R, 1\leq i<q$. 
% \end{prop}
% \begin{proof}
% 	参见\cite{MR82k:13016} Proposition 1.7. 
% \end{proof}

% \subsection{符号说明} % (fold)
% \label{subsec:符号}
% 为了陈述定理, 将文中常用的记号详述如下(参考\cite{MR86f:18017})
% \begin{itemize}
% 	\item $\mathbb{Z}_+$表示非负整数全体. 
% 	\item $\varepsilon^1 = (1, 0)\in \mathbb{Z}_+^2$, $\varepsilon^2 = (0, 1)\in \mathbb{Z}_+^2$.
% 	\item 对于$\alpha = (\alpha_1,\alpha_2) \in \mathbb{Z}_+^2$, 记$t^{\alpha}=t_1^{\alpha_1}t_2^{\alpha_2}$, 于是有$t^{\varepsilon^1}=t_1$, $t^{\varepsilon^2}=t_2$. 
% 	\item $\Delta=\{\alpha\in\mathbb{Z}_+^2\mid  t^{\alpha}\in I\}$, 若$\delta \in \Delta$, 则$\delta+\varepsilon^i \in \Delta, i=1, 2$. 
% 	\item $\Lambda=\{(\alpha, i)\in\mathbb{Z}_+^2 \times \{1, 2\}\mid  \alpha_i\geq 1, t^{\alpha}\in M\}$. 
% 	\item 对于$(\alpha, i)\in\Lambda$, 令{\color{blue}$[\alpha, i]=\min\{m\in \mathbb{Z}\mid m\alpha - \varepsilon^i\in \Delta\}$}. 
% 	若$(\alpha, i), (\alpha, j)\in \Lambda$, 有$[\alpha, i]\leq [\alpha, j]+1$. 
% 	\item 若$gcd(p, \alpha_1, \alpha_2)=1$, 令$[\alpha]=\max\{[\alpha, i]\mid  \alpha_i  \not\equiv 0 \bmod p\}$, 其中$p$是域$k$的特征.
% 	\item {\color{blue} $\Lambda^{00}= \big\{(\alpha, i)\in \Lambda\mid  gcd(\alpha_1, \alpha_2)=1, i\neq \min\{j\mid \alpha_j\not\equiv 0 \bmod p, [\alpha, j]=[\alpha]\} \big\}$}. 
% \end{itemize}
% % subsection 符号 (end)
% 若$(\alpha, i)\in \Lambda$, $f(x)\in k[x]$, 令
% \[\Gamma_{\alpha, i}(1-xf(x))= \langle f(t^\alpha)t^{\alpha-\varepsilon^i}, t_i \rangle, \]
% % 注释掉这一部分,需要重新写
% % 
% % 若$g(t_1, t_2)=t_ih(t_1, t_2)\in \sqrt{I}=(t_1)$, 令
% % \[\Gamma_i(1-g(t_1, t_2))=\langle h(t_1, t_2), t_i \rangle, \]
% % 且有
% % \[\Gamma_{\alpha, i}(1-xf(x))=\Gamma_i(1-t^{\alpha} f(t^{\alpha})). \]

% % 由于$t_1\in \sqrt{I}$, $\Gamma_1$诱导了同态
% % \begin{align*}
% % (1+t_1k[t_1, t_2]/t_1 I)^{\times} &\longrightarrow K_2(A, M)\\
% % 1-g(t_1, t_2) & \mapsto \langle h(t_1, t_2), t_1 \rangle
% % \end{align*}
% % $\Gamma_2$诱导了同态
% % \begin{align*}
% % (1+t_2\sqrt{I}/t_2 I)^{\times} &\longrightarrow K_2(A, M)\\
% % 1-g(t_1, t_2) & \mapsto \langle h(t_1, t_2), t_2 \rangle
% % \end{align*}
% % 
% % 上面需要重写
% 若$(\alpha, i)\in\Lambda$, $\Gamma_{\alpha, i}$诱导了同态
% \[\big(1+xk[x]/(x^{[\alpha, i]})\big)^{\times} \longrightarrow K_2(A, M). \]
% \begin{theorem}
% \label{K2(A, M)}
% 	$\Gamma_{\alpha, i}$诱导了{\color{red}同构}
% \[ K_2(A, M)\cong \bigoplus_{(\alpha, i)\in\Lambda^{00}}\big(1+xk[x]/(x^{[\alpha, i]})\big)^{\times}. \]
% \end{theorem}
% \begin{proof}
% 	参见\cite{MR86f:18017} Corollary 2.6.. 
% \end{proof}

% \subsection{Witt向量}
% 令$R$是交换环, $R$上的泛Witt向量环(the ring of universal/big Witt vectors over $R$)$BigWitt(R)$作为交换群同构于$(1+xR\llbracket x\rrbracket )^{\times}$, 即常数项为$1$的形式幂级数全体在乘法运算下形成的交换群, 
% \begin{align*}
% BigWitt(R) &\overset{\sim}\longrightarrow (1+xR\llbracket x\rrbracket )^{\times}\\
% (r_1, r_2, \cdots) & \mapsto \prod_{n=1}^{\infty}(1-r_n x^n). 
% \end{align*}



% 考虑$(1+xR\llbracket x\rrbracket )^{\times}$的子群$(1+x^{n+1}R\llbracket x\rrbracket )^{\times}$, 记交换群%$n$次截断泛Witt向量环(这个定义不太好说,underlying group不知怎么翻译)
% $BigWitt_n(R)=(1+xR\llbracket x\rrbracket )^{\times}/(1+x^{n+1}R\llbracket x\rrbracket )^{\times}$. 显然$BigWitt_1(R)=R$, 并且注意当$n\geq 3$时, $BigWitt_n(\mathbb{F}_2)$不是循环群. 

% % 
% % 下方有修改, 				|^^^^|				请对比来选择. 
% % 下方有修改, 				|^^^^|				请对比来选择. 
% % 下方有修改, 				|^^^|				请对比来选择. 
% % 下方有修改, 				|^^^|				请对比来选择. 
% % 下方有修改, 				|^^^|				请对比来选择. 
% % 下方有修  				|^^|				 对比来选择. 
% % 下方有修  				|^^|				 对比来选择. 
% % 下方有修  				|^^|				 对比来选择. 
% % 下方有修  				|^^|				 对比来选择
% % 下方有修  				|^|					 对比来选择
% % 下方有     				|^|					  比来选泽
% % 下方有    				||					   比来选 
% % 							
% % 									
% %   修改 				    口					    修改
% %   修改 				    口					    修改
% %
% {\color{blue}\begin{lemma}
% 	 $BigWitt_n(\mathbb{F}_q)\cong (1+x\mathbb{F}_q[x]/(x^{n+1}))^{\times}$. 
% \end{lemma}}
% \begin{proof}
% 	% 由定义$BigWitt_n(\mathbb{F}_q):=(1+x \mathbb{F}_q\llbracket x\rrbracket )^{\times}/(1+x^{n+1} \mathbb{F}_q\llbracket x\rrbracket )^{\times}$, 
% 	考虑群的满同态
% 	\begin{align*}
% 	(1+x \mathbb{F}_q\llbracket x\rrbracket )^{\times} &\longrightarrow (1+x \mathbb{F}_q[x]/(x^{n+1}))^{\times}\\
% 	1+\sum_{i\geq 1}a_i x^i &\mapsto 1+\sum_{i = 1}^{n}a_i x^i +(x^{n+1})
% 	\end{align*}
% 	其中$a_i\in \mathbb{F}_q$, 易知同态核为$(1+x^{n+1} \mathbb{F}_q\llbracket x\rrbracket )^{\times}$, 
% 	从而$BigWitt_n(\mathbb{F}_q)=(1+x \mathbb{F}_q\llbracket x\rrbracket )^{\times}/(1+x^{n+1} \mathbb{F}_q\llbracket x\rrbracket )^{\times} \cong (1+x\mathbb{F}_q[x]/(x^{n+1}))^{\times}$. 
% \end{proof}


% % 上面为新加
% \begin{example} 
% \label{ex:W3(F2)}
% 	$BigWitt_3(\mathbb{F}_2)\cong (1+x\mathbb{F}_2[x]/(x^{4}))^{\times}\cong \mathbb{Z}/4 \mathbb{Z} \oplus\mathbb{Z}/2 \mathbb{Z}$.
% \end{example}
% \begin{proof}
% 	$1+x\in (1+x \mathbb{F}_2[x]/(x^4))^{\times}$是$4$阶元, 由它生成的子群$\langle 1+x \rangle = \{1, 1+x, 1+x^2, 1+x+x^2+x^3\}$, 且$1+x^3$是二阶元, $\langle 1+x^3 \rangle = \{1, 1+x^3\}$. 令$\sigma, \tau$分别是$\mathbb{Z}/4 \mathbb{Z}$和$\mathbb{Z}/2 \mathbb{Z}$的生成元, 则有同构
% 		\begin{align*}
% 		\mathbb{Z}/4 \mathbb{Z} \oplus \mathbb{Z}/2 \mathbb{Z} &\longrightarrow BigWitt_4(\mathbb{F}_2) \\
% 		(\sigma, \tau) & \mapsto (1+x)(1+x^3)=1+x+x^3. 
% 		\end{align*}
% \end{proof}

% \begin{example}
% 	$BigWitt_4(\mathbb{F}_2) \cong \mathbb{Z}/8 \mathbb{Z} \oplus \mathbb{Z}/2 \mathbb{Z}. $
% \end{example}
% \begin{proof}
% 	$1+x \in BigWitt_5(\mathbb{F}_2)$是$8$阶元, 由它生成的子群$\langle 1+x \rangle = \{1, 1+x, 1+x^2, 1+x+x^2+x^3, 1+x^4, 1+x+x^4, 1+x^2+x^4, 1+x+x^2+x^3+x^4\}$, 另外$1+x^3$是二阶元, $\langle 1+x^3 \rangle = \{1, 1+x^3\}$. 令$\sigma, \tau$分别是$\mathbb{Z}/8 \mathbb{Z}$和$\mathbb{Z}/2 \mathbb{Z}$的生成元, 则有同构
% 	\begin{align*}
% 	\mathbb{Z}/8 \mathbb{Z} \oplus \mathbb{Z}/2 \mathbb{Z} &\longrightarrow BigWitt_4(\mathbb{F}_2) \\
% 	(\sigma, \tau) & \mapsto (1+x)(1+x^3)=1+x+x^3+x^4 
% 	\end{align*}
% 	于是$(\sigma^i, \tau^j), 0\leq i <8, 0\leq j<2$对应于$(1+x)^i(1+x^3)^j$, 详细的对应如下
% 	\begin{align*}
% 	(1, \tau) & \mapsto 1+x^3, & (\sigma, \tau) & \mapsto 1+x+x^3+x^4, \\
% 	 (\sigma^2, \tau) & \mapsto 1+x^2+x^3, & (\sigma^3, \tau) & \mapsto 1+x+x^2+x^4, \\
% 	(\sigma^4, \tau) & \mapsto 1+x^3+x^4, & (\sigma^5, \tau) & \mapsto 1+x+x^3, \\
% 	 (\sigma^6, \tau) & \mapsto 1+x^2+x^3+x^4, & (\sigma^7, \tau) & \mapsto 1+x+x^2, \\
% 	(1, 1)& \mapsto 1, & (\sigma, 1) & \mapsto 1+x, \\
% 	(\sigma^2, 1) & \mapsto 1+x^2, & (\sigma^3, 1) & \mapsto 1+x+x^2+x^3, \\
% 	(\sigma^4, 1) & \mapsto 1+x^4, &
% 	(\sigma^5, 1) & \mapsto 1+x+x^4, \\
% 	(\sigma^6, 1) & \mapsto 1+x^2+x^4, & (\sigma^7, 1) & \mapsto 1+x+x^2+x^3+x^4. 
% 	\end{align*}
% \end{proof}



% 固定素数$p$, 考虑局部环$\mathbb{Z}_{(p)}=\mathbb{Z}[1/\ell \mid \text{所有素数}\ell\neq p]$, 即$\mathbb{Z}$在非零素理想$(p)=p \mathbb{Z}$处的局部化, 于是$\mathbb{Z}_{(p)}$-代数$R$就是所有除$p$外的素数均在其中可逆的交换环, 如$\mathbb{F}_{p^n}$是一个$\mathbb{Z}_{(p)}$-代数. 

% 下面考虑$p$-Witt向量环$W(A)$与截断$p$-Witt向量环$W_n(A)$, $p$-Witt向量为$(a_0, a_1, \cdots)$, 加法用Witt多项式定义, 本文仅考虑用加法定义的交换群结构, 例如
% % $W(\mathbb{F}_p)=\mathbb{Z}_{p}$,
% 作为交换群$W_n(\mathbb{F}_{p^f})$同构于$(\mathbb{Z}/p^n\mathbb{Z})^f$. %这里可以加上Galois环 并且注意F_p^f是F_p上f维向量空间,因此底层群是(C_p)^f

% Artin-Hasse级数定义为
% \[AH(x)= \exp(\sum_{n\geq 0}\frac{x^{p^n}}{p^n})=1+x+\cdots \in 1+x \mathbb{Q}\llbracket x\rrbracket , \]
% 实际上$AH(x)\in 1+x \mathbb{Z}_{(p)}\llbracket x\rrbracket $(参考\cite{rabinoff2014theory} Theorem 7.2). 对于$BigWitt(R)=(1+xR\llbracket x\rrbracket)^{\times}$中的任一元素$\alpha$可以写成无穷乘积
% \[\alpha = \prod_{n=1}^{\infty}(1-r_nx^n),  \]
% 其中$\ r_n\in R$是唯一的. 若$A$是$\mathbb{Z}_{(p)}$-代数, $BigWitt(A)=(1+xA\llbracket x\rrbracket)^{\times}$中的任一元素$\alpha$还有如下表法 \cite{katz2013witt}
% \[\alpha = \prod_{n\geq 1}AH(a_n x^n), \ a_n\in A. \]
% 将整数$n$写成$n=mp^a$, 使得$gcd(m, p)=1, a\geq 0$, 由于$A$是$\mathbb{Z}_{(p)}$-代数, $m$可逆, 从而$[x\mapsto x^{1/m}]\in \End(BigWitt(A))$是双射, 于是我们可以将$\alpha\in BigWitt(A)$以如下的形式表出
% \[\prod_{\scriptsize\substack{m\geq 1 \\ gcd(m, p)=1  \\ a\geq 0}}AH(a_{mp^a} x^{mp^a})^{1/m}. \]

% 另一方面对于$\mathbb{Z}_{(p)}$-代数$A$, 下列映射是群同态
% \begin{align*}
% W(A)&\longrightarrow BigWitt(A)\\
% (a_0, a_1, \cdots) &\mapsto \prod_{i\geq 0}AH(a_i x^i). 
% \end{align*}

% 作为交换群, $BigWitt_n(A)$可以分解为$p$-Witt向量环的直和, 实际上有以下同构\cite{Lauter1999A} % 这个引用还待查,再找更精确的表述的
% \[
% BigWitt(A) \cong \prod_{\scriptsize\substack{m\geq 1 \\ gcd(m, p)=1}} W(A), 
% \]
% 元素$\prod\limits_{\scriptsize\substack{m\geq 1 \\ gcd(m, p)=1  \\ a\geq 0}}AH(a_{mp^a} x^{mp^a})^{1/m}$对应于一个$m$-分量为$(a_m, a_{mp}, a_{mp^2}, \cdots)\in W(A)$的Witt向量. 
% 对于截断的Witt向量环, 有同构\[
% BigWitt_n(A) \cong \bigoplus_{\scriptsize\substack{1\leq m\leq n \\ gcd(m, p)=1}} W_{\ell(m, n)}(A), 
% \]
% 其中$\ell(m, n)$是一个整数, 定义为
% \[\ell(m, n)=1+\left \lfloor\log_p \frac{n}{m}  \right \rfloor, \]
% 即$\ell(m, n)=1+\text{使得$mp^k\leq n$成立的最大整数$k$}.$

% {\color{blue}考虑特征为$p$的有限域$\mathbb{F}_q$, 有同构\cite{Lauter1999A}
% \[BigWitt_n(\mathbb{F}_{q}) \cong \bigoplus_{\scriptsize\substack{1\leq m\leq n \\ gcd(m, p)=1}} W_{\ell(m, n)}(\mathbb{F}_{q}), \]}
% 注意到$\sum\limits_{\scriptsize\substack{1\leq m\leq n \\ gcd(m, p)=1}} \ell(m, n) = n$, 因此两边都是$q^n$阶交换群. 
% \begin{corollary}
% \label{cor:BW}
% 	若有限域$\mathbb{F}_{p^f}$的特征$ch(\mathbb{F}_{p^f})=p$, 则作为交换群有
% 	\[
% 	BigWitt_n(\mathbb{F}_{p^f})\cong \bigoplus_{\scriptsize\substack{1\leq m\leq n \\ gcd(m, p)=1}}W_{1+ \left \lfloor\log_p \frac{n}{m}  \right \rfloor}(\mathbb{F}_{p^f}) = \bigoplus_{\scriptsize\substack{1\leq m\leq n \\ gcd(m, p)=1}}(\mathbb{Z}/p^{1+ \left \lfloor\log_p \frac{n}{m}  \right \rfloor}\mathbb{Z})^f, 
% 	\]
% 	其中$ \left \lfloor x \right \rfloor$表示不超过$x$
% 	的最大整数. 
% \end{corollary}
% \begin{example}
% 	作为交换群, $BigWitt_3(\mathbb{F}_2)= W_{\ell(1, 3)}(\mathbb{F}_2)\oplus W_{\ell(3, 3)}(\mathbb{F}_2)=W_2(\mathbb{F}_2)\oplus W_1(\mathbb{F}_2)=\mathbb{Z}/4 \mathbb{Z}\oplus	\mathbb{Z}/2 \mathbb{Z}$, 

% 	$BigWitt_4(\mathbb{F}_2)= W_{\ell(1, 4)}(\mathbb{F}_2)\oplus W_{\ell(3, 4)}(\mathbb{F}_2)=W_3(\mathbb{F}_2)\oplus W_1(\mathbb{F}_2)=\mathbb{Z}/8 \mathbb{Z}\oplus	\mathbb{Z}/2 \mathbb{Z}$, 

% 	$BigWitt_2(\mathbb{F}_3)= W_{\ell(1, 2)}(\mathbb{F}_3)\oplus W_{\ell(2, 2)}(\mathbb{F}_3)=W_1(\mathbb{F}_3)\oplus W_1(\mathbb{F}_3)=\mathbb{Z}/3 \mathbb{Z}\oplus	\mathbb{Z}/3 \mathbb{Z}$. 

% \end{example}
%粘贴自NKFinAlg. tex



























\section{$NK_2(\mathbb{F}_2[C_2])$的计算}
方法一是在讲Weibel文章\cite{weibel2009nk0}时讲过的. 方法二是基于上面的思路给出来的详细证明. 
{\normalsize \color{gray}\subsection{方法一}
% 粘贴自weibelnk. tex
\label{sec:C2Cp}
First, consider the simplest example $G=C_2=\langle \sigma \rangle=\{1, \sigma\}$. There is a Rim square
\begin{equation}
\label{rimsqC2}
	\begin{tikzcd}
		\mathbb{Z}[C_2] \ar[r, "\sigma\mapsto 1"] \ar[d, "\sigma\mapsto -1"']& \mathbb{Z}\ar[d, "q"]\\
		 \mathbb{Z}\ar[r, "q"] & \mathbb{F}_2\\
	\end{tikzcd}
\end{equation}
	
Since $\mathbb{F}_2$ (field) and $\mathbb{Z}$ (PID) are regular rings, $NK_i(\mathbb{F}_2)=0=NK_i(\mathbb{Z})$ for all $i$. 

By Mayer–Vietoris sequence, one can get $NK_1(\mathbb{Z}[C_2])=0$, $NK_0(\mathbb{Z}[C_2])=0$. Note that the similar results are true for any cyclic group of prime order. 
	\[\begin{tikzcd}[column sep=small]
		NK_2 \mathbb{F}_2 \ar[r] \ar[d, equal]& NK_1 \mathbb{Z}[C_2] \ar[r] & NK_1 \mathbb{Z} \oplus NK_1 \mathbb{Z} \ar[r] \ar[d, equal]& NK_1 \mathbb{F}_2 \ar[r] \ar[d, equal] & NK_0 \mathbb{Z}[C_2] \ar[r]&NK_0 \mathbb{Z} \oplus NK_0 \mathbb{Z}  \ar[d, equal] \\
		0 & & 0 &0 & & 0\\
	\end{tikzcd}\]

\[\ker(\mathbb{Z}[C_2]\overset{\sigma \mapsto -1}\longrightarrow \mathbb{Z}) =(\sigma +1)\]
By relative exact sequence, 
\[0=NK_3(\mathbb{Z})\longrightarrow NK_2(\mathbb{Z}[C_2], (\sigma+1))\overset{\cong}\longrightarrow NK_2(\mathbb{Z}[C_2])\longrightarrow NK_2(\mathbb{Z})=0. \]
And from $(\mathbb{Z}[C_2], (\sigma+1))\longrightarrow (\mathbb{Z}[C_2]/(\sigma-1), (\sigma+1)+(\sigma-1)/(\sigma-1))=(\mathbb{Z}, (2))$ one has double relative exact sequence
\[0=NK_3(\mathbb{Z}, (2))\longrightarrow NK_2(\mathbb{Z}[C_2];(\sigma+1), (\sigma-1))\overset{\cong}\longrightarrow NK_2(\mathbb{Z}[C_2], (\sigma+1))\longrightarrow NK_2(\mathbb{Z}, (2))=0. \]
Note that $0=NK_{i+1}(\mathbb{Z}/2)\longrightarrow NK_i(\mathbb{Z}, (2))\longrightarrow NK_i(\mathbb{Z})=0$. 

\[\begin{tikzcd}
	 & NK_3(\mathbb{Z}, (2))=0 \ar[d, red] & & & \\
	 & NK_2(\mathbb{Z}[C_2];(\sigma+1), (\sigma-1)) \ar[d, red, "\cong", red] \\
0=NK_3(\mathbb{Z})	\ar[r, blue] & NK_2(\mathbb{Z}[C_2], (\sigma+1)) \ar[r, blue, "\cong", blue] \ar[d, red] & NK_2(\mathbb{Z}[C_2]) \ar[r, blue] &NK_2(\mathbb{Z})=0\\
	& NK_2(\mathbb{Z}, (2))=0 \\
\end{tikzcd}
\]
We obtain $NK_2(\mathbb{Z}[C_2])\cong NK_2(\mathbb{Z}[C_2], (\sigma+1), (\sigma-1))$, from Guin-Loday-Keune\cite{Guin-Waléry1981}, $NK_2(\mathbb{Z}[C_2];(\sigma+1), (\sigma-1))$ is isomorphic to $V=x \mathbb{F}_2[x]$, with the Dennis-Stein symbol $\langle x^n(\sigma-1), \sigma+1 \rangle$ corresponding to $x^n\in V$. Note that $1-x^n(\sigma-1)(\sigma+1)=1$ is invertible in $\mathbb{Z}[C_2][x]$ and $\sigma+1 \in (\sigma+1), x^n(\sigma-1) \in (\sigma-1)$. 

\begin{theorem}
	$NK_2(\mathbb{Z}[C_2])\cong V$, $NK_1(\mathbb{Z}[C_2])=0$, $NK_0(\mathbb{Z}[C_2])=0$. 
\end{theorem}

In fact, when $p$ is a prime number, we have $NK_2(\mathbb{Z}[C_p])\cong x \mathbb{F}_p[x]$, $NK_1(\mathbb{Z}[C_p])=0$, $NK_0(\mathbb{Z}[C_p])=0$. 

\begin{example}[{$\mathbb{Z}[C_p]$}]
	$R=\mathbb{Z}[C_p]$, $I=(\sigma-1)$, $J=(1+\sigma+\cdots+\sigma^{p-1})$ such that $I\cap J =0$. There is a Rim square
		\[\begin{tikzcd}
			\mathbb{Z}[C_p] \ar[r, "\sigma \mapsto \zeta"] \ar[d, "f", "\sigma\mapsto 1"']& \mathbb{Z}[\zeta]\ar[d, "g"]\\
			\mathbb{Z} \ar[r] & \mathbb{F}_p\\
		\end{tikzcd}\]
$I/I^2\otimes_{\mathbb{Z}[C_p]^{op}}J/J^2 \cong \mathbb{Z}_p$ is cyclic of order $p$ and generated by $(\sigma-1)\otimes(1+\sigma+\cdots+\sigma^{p-1})$. Note that $p(\sigma-1)\otimes(1+\sigma+\cdots+\sigma^{p-1})=0$ since $(1+\sigma+\cdots+\sigma^{p-1})^2=p(1+\sigma+\cdots+\sigma^{p-1})$. 

And the map 
\begin{align*}
I/I^2\otimes_{\mathbb{Z}[C_p]^{op}}J/J^2 &\longrightarrow K_2(R, I)\\
(\sigma-1)\otimes(1+\sigma+\cdots+\sigma^{p-1}) & \mapsto \langle \sigma-1, 1+\sigma+\cdots+\sigma^{p-1} \rangle =\langle \sigma-1, 1\rangle^p=1
\end{align*}
Also see \cite{STEIN1980213}. 

\end{example}
\begin{example}[{$\mathbb{Z}[C_p][x]$}]
	There is a Rim square
		\[\begin{tikzcd}
			\mathbb{Z}[C_p][x] \ar[r] \ar[d]& \mathbb{Z}[\zeta][x]\ar[d]\\
			\mathbb{Z}[x] \ar[r] & \mathbb{F}_p[x]\\
		\end{tikzcd}\]
$K_2(\mathbb{Z}[C_p][x];I[x], J[x])\cong I[x]\otimes_{\mathbb{Z}[C_p][x]} J[x] =I\otimes_{\mathbb{Z}[C_p]} J[x]\cong \mathbb{Z}_p[x]$. 

Since $\Lambda=\mathbb{Z}, \mathbb{F}_p, \mathbb{Z}[\zeta]$ are regular, $K_i(\Lambda[x]) = K_i(\Lambda)$, i. e. $NK_i(\Lambda)=0$. Hence 
\[K_2(\mathbb{Z}[C_p][x], I[x], J[x])/K_2(\mathbb{Z}[C_p], I, J)\cong K_2(\mathbb{Z}[C_p][x])/K_2(\mathbb{Z}[C_p]), \]
finally $NK_2(\mathbb{Z}[C_p]) = K_2(\mathbb{Z}[C_p][x])/K_2(\mathbb{Z}[C_p]) \cong \mathbb{Z}/p[x]/\mathbb{Z}/p=x \mathbb{Z}/p [x]= x \mathbb{F}_p[x]$. 
\end{example}
}
% 粘贴自weibelnk. tex 结束









\subsection{{\color{red}方法二}}
%!TEX root = ../../testmain.tex
% 粘贴自NKFinAlg.tex 
% 有修改,注意原稿要校对
计算$k=\mathbb{F}_2$, $p=2$, $n=2$的情形, 即$NK_2(\mathbb{F}_2[C_2])\cong K_2(\mathbb{F}_2[t_1, t_2]/(t_1^2), (t_1))$. 
\begin{theorem}
	(1)$NK_2(\mathbb{F}_2[C_2])\cong \bigoplus_{\infty} \mathbb{Z}/2 \mathbb{Z}$, \\
	(2)$NK_2(\mathbb{F}_2[C_2])\cong K_2(\mathbb{F}_2[t, x]/(t^2), (t))$是由Dennis-Stein符号$\{\langle tx^i, x \rangle \mid i\geq 0\}$与$\{\langle tx^i, t \rangle \mid i\geq 1\text{为奇数}\}$生成的, 这样的符号均为$2$阶元. 
\end{theorem}
\begin{proof}
	(1)令$A=\mathbb{F}_2[t_1, t_2]/(t_1^2) \cong \mathbb{F}_2[C_{2}][x]$, 此时$I=(t_1^2)$, $M=(t_1)$, $A/M \cong \mathbb{F}_2[x]$. 记号如\ref{subsec:符号}所述.
\begin{align*}
\Delta &=\{(\alpha_1, \alpha_2)\in\mathbb{Z}_+^2\mid  t_1^{\alpha_1}t_2^{\alpha_2}\in (t_1^2)\}\\
	&=\{(\alpha_1, \alpha_2)\mid \alpha_1\geq 2, \alpha_2 \geq 0\}, \\
\Lambda &=\{((\alpha_1, \alpha_2), i)\in\mathbb{Z}_+^2 \times \{1, 2\}\mid \alpha_i\geq 1, \text{且} t_1^{\alpha_1}t_2^{\alpha_2}\in (t_1)\} \\
	&=\{((\alpha_1, \alpha_2), i)\in\mathbb{Z}_+^2 \times \{1, 2\}\mid \alpha_i\geq 1, \alpha_1\geq 1, \alpha_2\geq 0\} \\
	&=\{((\alpha_1, \alpha_2), 1) \mid \alpha_1\geq 1, \alpha_2\geq 0\}\cup \{((\alpha_1, \alpha_2), 2) \mid \alpha_1\geq 1, \alpha_2\geq 1\}. 
\end{align*}

若$(\alpha, i)\in \Lambda$, $[\alpha, i] = \min \{m\in \mathbb{Z}\mid m \alpha -\varepsilon^i \in \Delta\}$, 于是有
\begin{align*}
[\alpha, 1] & = \min \{m\in \mathbb{Z} \mid m \alpha -\varepsilon^1 \in \Delta\} \\
& =\min \{m\in \mathbb{Z} \mid (m \alpha_1-1, m \alpha_2)\in \Delta\} \\
& =\min \{m\in \mathbb{Z} \mid m \alpha_1\geq 3\}. \\
[\alpha, 2] & = \min \{m\in \mathbb{Z} \mid m \alpha -\varepsilon^2 \in \Delta\} \\
& =\min \{m\in \mathbb{Z} \mid (m \alpha_1, m \alpha_2-1)\in \Delta\} \\
& =\min \{m\in \mathbb{Z} \mid m \alpha_1\geq 2\}. 
\end{align*}
此时
\begin{align*}
[(1, \alpha_2), 1] & = 3, \ \alpha_2\geq 0, \\
[(2, \alpha_2), 1] & = 2, \ \alpha_2\geq 0, \\
[(\alpha_1, \alpha_2), 1] & = 1, \ \alpha_1\geq 3, \alpha_2\geq 0, \\
[(1, \alpha_2), 2] & = 2, \ \alpha_2\geq 1, \\
[(\alpha_1, \alpha_2), 2] & = 1, \ \alpha_1\geq 2, \alpha_2\geq 1. 
\end{align*}



若$gcd(2, \alpha_1, \alpha_2)=1$, 即$\alpha_1, \alpha_2$中至少一个是奇数, 令$[\alpha]=\max\{[\alpha, i]\mid  \alpha_i  \not\equiv 0 \bmod 2\}$. 若仅$\alpha_1$是奇数, $[\alpha]=[\alpha, 1]$, 若仅$\alpha_2$是奇数, $[\alpha]=[\alpha, 2]$, 若两者均为奇数, 则$[\alpha]=\max\{[\alpha, 1], [\alpha, 2]\}$, 即有
\begin{align*}
[(1, \alpha_2)]&=\max\{[(1, \alpha_2), 1], [(1, \alpha_2), 2]\}=3, \text{$\alpha_2 \geq 1$是奇数}\\
[(1, \alpha_2)]&=[(1, \alpha_2), 1]=3, \text{$\alpha_2 \geq 0$是偶数}\\
[(3, \alpha_2)]&=\max\{[(3, \alpha_2), 1], [(3, \alpha_2), 2]\}=1, \text{$\alpha_2\geq 1$是奇数}\\
[(3, \alpha_2)]&=[(3, \alpha_2), 1]=1, \text{$\alpha_2\geq 0$是偶数}\\
[(2, 1)]&=[(2, 1), 2]=1, \\
[\alpha]&=1, \text{其它符合条件的$\alpha$. }
\end{align*}
为了方便我们把上面的计算结果列表如下
\[\begin{array}{|c|c|c|c|}
\hline
(\alpha_1, \alpha_2) & [(\alpha_1, \alpha_2), 1] &[(\alpha_1, \alpha_2), 2] &[(\alpha_1, \alpha_2)]  \\
\hline
(1, \alpha_2)  & 3, \alpha_2 \geq 0& 2 , \alpha_2\geq 1 & 3 \\
\hline
(2, \alpha_2)  & 2, \alpha_2 \geq 0& 1 , \alpha_2\geq 1 & 1, \text{当$\alpha_2$是奇数时} \\
\hline
(3, \alpha_2)  & 1, \alpha_2 \geq 0& 1 , \alpha_2\geq 1 & 1 \\
\hline
(\alpha_1, 0), \alpha_1 \geq 3 & 1 & \text{无定义} &1, \text{当$\alpha_1$是奇数时} \\
\hline
(\alpha_1, \alpha_2), \alpha_1 \geq 3, \alpha_2 \geq 1& 1 & 1& 1, \text{当$gcd(2,\alpha_1, \alpha_2)=1$时} \\ % 注意这里(3,9)还在,后面要求alpha1,2互素时才能去掉
\hline
\end{array}\]


下面计算$\Lambda^{00}=\big\{(\alpha, i)\in \Lambda\mid  gcd(\alpha_1, \alpha_2)=1, i\neq \min\{j\mid \alpha_j\not\equiv 0 \bmod 2, [\alpha, j]=[\alpha]\} \big\}$.
\begin{enumerate}
	\item 对于任何的$\alpha_2\geq 0$, $((1, \alpha_2), 1) \not \in \Lambda^{00}$, 这是因为$1\not\equiv 0 \bmod 2$且$[(1, \alpha_2), 1]=3=[(1, \alpha_2)]$, 从而$\min\{j\mid \alpha_j\not\equiv 0 \bmod 2, [(1, \alpha_2), j]=[(1, \alpha_2)]\}=1$.
	\item 对于任何的$\alpha_2\geq 1$, $((1, \alpha_2), 2) \in \Lambda^{00}$, 且$[(1, \alpha_2), 2]=2$. 这是因为此时$[(1, \alpha_2), 1]=3=[(1, \alpha_2)]$, $\min\{j\mid \alpha_j\not\equiv 0 \bmod 2, [\alpha, j]=[\alpha]\}=1$, .
	\item 对于任何的奇数$\alpha_2\geq 1$, $((2, \alpha_2), 1) \in \Lambda^{00}$, 且$[(2, \alpha_2), 1]=2$. 因为$\alpha_2\not\equiv 0 \bmod 2$并且$[(2, \alpha_2), 2]=1=[(2, \alpha_2)]$, 故$\{j\mid \alpha_j\not\equiv 0 \bmod 2, [(2, \alpha_2), j]=[(2, \alpha_2)]\}=2\neq 1$. 注意若$\alpha_2\geq 0$为偶数时, $[2,\alpha_2]$无定义, 因此$((2, \alpha_2), 1) \not \in \Lambda^{00}$, 同理$((2, \alpha_2), 2) \not \in \Lambda^{00}$.
	\item 对于任何的奇数$\alpha_2\geq 1$, $((2, \alpha_2), 2) \not \in \Lambda^{00}$. 由于$[(2, \alpha_2), 2]=1=[(2, \alpha_2)]$, 与$2\neq \min\{j\mid \alpha_j\not\equiv 0 \bmod 2, [\alpha, j]=[\alpha]\}$矛盾.
	\item 对于偶数$\alpha_1\geq 3$和奇数$\alpha_2\geq 1$, $((\alpha_1, \alpha_2), 1)  \in \Lambda^{00}$, 且$[(\alpha_1, \alpha_2), 1]=1$. 而当$\alpha_1\geq 3$为奇数$\alpha_2\geq 1$时, 或$\alpha_1, \alpha_2$均为偶数时, $((\alpha_1, \alpha_2), 1) \not  \in \Lambda^{00}$. 由于$((\alpha_1, \alpha_2), 1)  \in \Lambda^{00}$要求$1\neq \min\{j\mid \alpha_j\not\equiv 0 \bmod 2, [(\alpha_1, \alpha_2), j]=[(\alpha_1, \alpha_2)]\}$, 当$\alpha_1\geq 3$为奇数时上式不成立, $2= \min\{j\mid \alpha_j\not\equiv 0 \bmod 2, [(\alpha_1, \alpha_2), j]=[(\alpha_1, \alpha_2)]\}$当且仅当$\alpha_1\geq 3$为偶数且$\alpha_2\geq 1$为奇数.
	\item 对于奇数$\alpha_1\geq 3$和任意$\alpha_2\geq 1 $, 且$[(\alpha_1, \alpha_2), 2]=1$. $((\alpha_1, \alpha_2), 2)  \in \Lambda^{00}$, 其余情况只要当$\alpha_1\geq 3$为偶数时$((\alpha_1, \alpha_2), 2) \not  \in \Lambda^{00}$. 由于$((\alpha_1, \alpha_2), 2)  \in \Lambda^{00}$要求$2\neq \min\{j\mid \alpha_j\not\equiv 0 \bmod 2, [(\alpha_1, \alpha_2), j]=[(\alpha_1, \alpha_2)]\}$, 当$\alpha_1$为偶数时上式不成立, 而当$\alpha_1$为奇数时, 任意$\alpha_2\geq 1$, $[(\alpha_1, \alpha_2), 1]=1=[(\alpha_1, \alpha_2)]$. 
\end{enumerate}



从而
\begin{align*}
\Lambda^{00}=&\{((1, \alpha_2), 2)\mid  \alpha_2\geq 1\} \\
	&\cup \{((2, \alpha_2), 1)\mid  \alpha_2\geq 1\text{为奇数}\} \\
	&\cup \{((\alpha_1, \alpha_2), 1) \mid \alpha_1\geq 3\text{为偶数}, \alpha_2\geq 1\text{为奇数}\} \\
	&\cup \{((\alpha_1, \alpha_2), 2) \mid \alpha_1\geq 3\text{为奇数}, \alpha_2\geq 1\}. 
\end{align*}
记$\Lambda^{00}_d=\{(\alpha, i)\in \Lambda^{00}\mid  [(\alpha, i)]=d\}$, $\Lambda^{00}_{> 1}=\{(\alpha, i)\in \Lambda^{00}\mid  [(\alpha, i)]>1\}$, 则有
$\Lambda^{00}_1=\{(\alpha, i)\in \Lambda^{00}\mid  [(\alpha, i)]=1\}$, $\Lambda^{00}_{>1}=\Lambda^{00}_2=\{(\alpha, i)\in \Lambda^{00}\mid  [(\alpha, i)]=2\}$, 于是有
\begin{align*}
\Lambda^{00}_1 &= \{((\alpha_1, \alpha_2), 1) \mid  \alpha_1\geq 3\text{为偶数}, \alpha_2\geq 1\text{为奇数}\} \cup \{((\alpha_1, \alpha_2), 2) \mid  \alpha_1\geq 3\text{为奇数}, \alpha_2\geq 1\}   \\
{\color{blue}\Lambda^{00}_2} &=\{((2, \alpha_2), 1)\mid  \alpha_2\geq 1\text{为奇数}\} \cup \{((1, \alpha_2), 2)\mid  \alpha_2\geq 1\}
\end{align*}
\[\Lambda^{00}=\Lambda^{00}_1 \sqcup \Lambda^{00}_2.\]
若$[\alpha, i]=1$时, $(1+x\mathbb{F}_{2}[x]/(x))^{\times}$是平凡的, $[\alpha, i]=2$时, $(1+x\mathbb{F}_{2}[x]/(x^{2}))^{\times}\cong \mathbb{Z}/2 \mathbb{Z}$, 从而由定理\ref{K2(A, M)}得

\begin{align*}
NK_2(\mathbb{F}_2[C_2])\cong K_2(A, M) &\cong \bigoplus_{(\alpha, i)\in\Lambda^{00}}(1+x\mathbb{F}_{2}[x]/(x^{[\alpha, i]}))^{\times}\\
& = \bigoplus_{(\alpha, i)\in \Lambda^{00}_2}(1+x\mathbb{F}_{2}[x]/(x^{2}))^{\times}\\
& = \bigoplus_{\scriptsize\substack{((1, \alpha_2), 2) \\ \alpha_2 \geq 1}}(1+x\mathbb{F}_{2}[x]/(x^{2}))^{\times} \oplus \bigoplus_{\scriptsize\substack{((2, \alpha_2), 1) \\ \alpha_2\geq 1\text{为奇数}}}(1+x\mathbb{F}_{2}[x]/(x^{2}))^{\times} \\
& = \bigoplus_{\alpha_2 \geq 1}\mathbb{Z}/2 \mathbb{Z} \oplus \bigoplus_{\alpha_2\geq 1\text{为奇数}}\mathbb{Z}/2 \mathbb{Z}, 
\end{align*}
作为交换群, 
\[NK_2(\mathbb{F}_2[C_2]) \cong \bigoplus_{\infty} \mathbb{Z}/2 \mathbb{Z}. \]
% 
% 定理第二部分
% 
% 

	(2)由\ref{K2(A, M)}, 对于任意$(\alpha, i)\in \Lambda^{00}$, $\Gamma_{\alpha, i}$诱导了同态
 \begin{align*}
 \Gamma_{\alpha, i} \colon \big(1+xk[x]/(x^{[\alpha, i]})\big)^{\times} &\longrightarrow K_2(A, M)\\
 1-xf(x) &\mapsto \langle f(t^\alpha)t^{\alpha-\varepsilon^i}, t_i \rangle. 
 \end{align*}
 此时只需考虑$\Lambda^{00}_2=\{((2, \alpha_2), 1)\mid  \alpha_2\geq 1\text{为奇数}\} \cup \{((1, \alpha_2), 2)\mid  \alpha_2\geq 1\}$, 对于任意$(\alpha, i)\in \Lambda^{00}_2$, $\Gamma_{\alpha, i}$均诱导了单射, 对任意$\alpha_2\geq 1$, 
  \begin{align*}
 \Gamma_{(1, \alpha_2), 2} \colon (1+x \mathbb{F}_2[x]/(x^{2}))^{\times} &\rightarrowtail K_2(A, M)\\
 1+x &\mapsto 
 \langle t_1t_2^{\alpha_2-1}, t_2 \rangle, 
 \end{align*}
对任意$\alpha_2\geq 1$为奇数, 
 \begin{align*}
 \Gamma_{(2, \alpha_2), 1} \colon (1+x \mathbb{F}_2[x]/(x^{2}))^{\times} &\rightarrowtail K_2(A, M)\\
 1+x &\mapsto \langle t_1t_2^{\alpha_2}, t_1 \rangle, 
 \end{align*}

我们作简单的替换令$t=t_1, x=t_2$, 于是$\langle t_1t_2^{\alpha_2-1}, t_2 \rangle = \langle tx^{\alpha_2-1}, x \rangle$, $\langle t_1t_2^{\alpha_2}, t_1 \rangle=\langle t x^{\alpha_2}, t  \rangle$. 由同构\ref{K2(A, M)}可知$NK_2(\mathbb{F}_2[C_2])$是由Dennis-Stein符号$\{\langle tx^i, x \rangle \mid i\geq 0\}$与$\{\langle tx^i, t \rangle \mid i\geq 1\text{为奇数}\}$生成的, 由于$t^2=0$故$\langle tx^i, x \rangle+\langle tx^i, x \rangle=\langle tx^i+tx^i-t^2x^{2i+1}, x \rangle=0$, $\langle tx^i, t \rangle+\langle tx^i, t \rangle=\langle tx^i+tx^i-t^3x^{2i}, t \rangle=0$. 
\end{proof}
\begin{remark}
	对于$i\geq 1\text{为偶数}$, $\langle tx^i, t \rangle=\langle x^{i/2}, t \rangle+\langle x^{i/2}, t \rangle=\langle x^{i/2}+x^{i/2}+tx^i, t \rangle=0$. 
\end{remark}

Weibel在\cite{weibel2009nk0}中给出了以下可裂正合列
	\[0\longrightarrow V/\Phi(V) \overset{F}\longrightarrow NK_2(\mathbb{F}_2[C_2])\overset{D}\longrightarrow \Omega_{\mathbb{F}_2[x]}\longrightarrow 0, \]
其中$V=x \mathbb{F}_2[x]$, $\Phi(V)=x^2 \mathbb{F}_2[x^2]$是$V$的子群, $\Omega_{\mathbb{F}_2[x]}\cong \mathbb{F}_2[x] \, \mathrm{d} x$是绝对K\"{a}hler微分模, $F(x^n)=\langle tx^n, t \rangle$, $D(\langle ft, g+g't \rangle)=f\, dg$. 显然$D(\langle tx^i, t \rangle)=0$, $D(\langle tx^i, x \rangle)=x^i\, \mathrm{d} x$, 可以看出$NK_2(\mathbb{F}_2[C_2])$的直和项
$$\bigoplus_{((2, \alpha_2), 1), \alpha_2\geq 1\text{为奇数}} \mathbb{Z}/2\mathbb{Z} \cong V/\Phi(V),$$ 
直和项
$$\bigoplus_{((1, \alpha_2), 2), \alpha_2\geq 1} \mathbb{Z}/2\mathbb{Z} \cong \mathbb{F}_2[x]\, \mathrm{d} x.$$ 

$V$和$\Omega_{\mathbb{F}_2[x]}$作为交换群是同构的, 但作为$W(\mathbb{F}_2)$-模是不同的. $V=x \mathbb{F}_2[x]$上的$W(\mathbb{F}_2)$-模结构(见\cite{MR96j:16008})为 
\begin{align*}
 V_m(x^n)&=x^{mn}, \\
 F_d(x^n)&=\begin{cases}
 	\, \mathrm{d} x^{n/d}, & \mbox{ 若 $d|n$}\\
 	0, & \mbox{其它}
 \end{cases}, \\
 [a]x^n&=a^nx^n. 
 \end{align*}
$\Omega_{\mathbb{F}_2[x]}=\mathbb{F}_2[x]\, \mathrm{d} x $上的$W(\mathbb{F}_2)$-模结构(见\cite{MR96j:16008})为
\begin{align*}
 V_m(x^{n-1}\, \mathrm{d} x)&=mx^{mn-1}\, \mathrm{d} x, \\
 F_d(x^{n-1}\, \mathrm{d} x)&=\begin{cases}
 	x^{n/d-1}\, \mathrm{d} x, & \mbox{ 若 $d|n$}\\
 	0, & \mbox{其它}
 \end{cases}, \\
 [a]x^{n-1}\, \mathrm{d} x&=a^nx^{n-1}\, \mathrm{d} x. 
 \end{align*}
结合两者我们可以得到$NK_2(\mathbb{F}_2[C_2])$的$W(\mathbb{F}_2)$-模结构为
\begin{align*}
 V_m(\langle tx^n, t \rangle)&=\begin{cases}
 	\langle tx^{mn}, t \rangle, & \mbox{若$m$是奇数}\\
 	0, & \mbox{若$m$是偶数}
 \end{cases}, \quad \mbox{$n\geq 1$为奇数} \\
  V_m(\langle tx^{n-1}, x \rangle)&=\begin{cases}
 	\langle tx^{mn-1}, x \rangle, & \mbox{若$m$是奇数}\\
 	0, & \mbox{若$m$是偶数}
 \end{cases}
 , \quad \mbox{$n\geq 1$} \\
 F_d(\langle tx^n, t \rangle)&=\begin{cases}
 	\langle tx^{n/d}, t \rangle, & \mbox{ 若$d|n$}\\
 	0, & \mbox{其它}
 \end{cases}, \quad \mbox{$n\geq 1$为奇数} \\
 F_d(\langle tx^{n-1}, x \rangle)&=\begin{cases}
 	\langle tx^{n/d-1}, x \rangle, & \mbox{若$d|n$}\\
 	0, & \mbox{其它}
 \end{cases}
 , \quad \mbox{$n\geq 1$} \\
 [1]\langle tx^n, t \rangle&=\langle tx^n, t \rangle, \quad \mbox{$n\geq 1$为奇数} \\
 [1]\langle tx^{n-1}, x \rangle&=\langle tx^{n-1}, x \rangle, \quad \mbox{$n\geq 1$}. 
 \end{align*}



% 粘贴自NKFinAlg. tex 结束
% % 粘贴自NKFinAlg.tex 
% % 有修改,注意原稿要校对
% 计算$k=\mathbb{F}_2$, $p=2$, $n=2$的情形, 即$NK_2(\mathbb{F}_2[C_2])\cong K_2(\mathbb{F}_2[t_1, t_2]/(t_1^2), (t_1))$. 
% \begin{theorem}
% 	(1)$NK_2(\mathbb{F}_2[C_2])\cong \bigoplus_{\infty} \mathbb{Z}/2 \mathbb{Z}$, \\
% 	(2)$NK_2(\mathbb{F}_2[C_2])\cong K_2(\mathbb{F}_2[t, x]/(t^2), (t))$是由Dennis-Stein符号$\{\langle tx^i, x \rangle \mid i\geq 0\}$与$\{\langle tx^i, t \rangle \mid i\geq 1\text{为奇数}\}$生成的, 这样的符号均为$2$阶元. 
% \end{theorem}
% \begin{proof}
% 	(1)令$A=\mathbb{F}_2[t_1, t_2]/(t_1^2) \cong \mathbb{F}_2[C_{2}][x]$, 此时$I=(t_1^2)$, $M=(t_1)$, $A/M \cong \mathbb{F}_2[x]$. 记号如\ref{subsec:符号}所述.
% \begin{align*}
% \Delta &=\{(\alpha_1, \alpha_2)\in\mathbb{Z}_+^2\mid  t_1^{\alpha_1}t_2^{\alpha_2}\in (t_1^2)\}\\
% 	&=\{(\alpha_1, \alpha_2)\mid \alpha_1\geq 2, \alpha_2 \geq 0\}, \\
% \Lambda &=\{((\alpha_1, \alpha_2), i)\in\mathbb{Z}_+^2 \times \{1, 2\}\mid \alpha_i\geq 1, \text{且} t_1^{\alpha_1}t_2^{\alpha_2}\in (t_1)\} \\
% 	&=\{((\alpha_1, \alpha_2), i)\in\mathbb{Z}_+^2 \times \{1, 2\}\mid \alpha_i\geq 1, \alpha_1\geq 1, \alpha_2\geq 0\} \\
% 	&=\{((\alpha_1, \alpha_2), 1) \mid \alpha_1\geq 1, \alpha_2\geq 0\}\cup \{((\alpha_1, \alpha_2), 2) \mid \alpha_1\geq 1, \alpha_2\geq 1\}. 
% \end{align*}

% 若$(\alpha, i)\in \Lambda$, $[\alpha, i] = \min \{m\in \mathbb{Z}\mid m \alpha -\varepsilon^i \in \Delta\}$, 于是有
% \begin{align*}
% [\alpha, 1] & = \min \{m\in \mathbb{Z} \mid m \alpha -\varepsilon^1 \in \Delta\} \\
% & =\min \{m\in \mathbb{Z} \mid (m \alpha_1-1, m \alpha_2)\in \Delta\} \\
% & =\min \{m\in \mathbb{Z} \mid m \alpha_1\geq 3\}. \\
% [\alpha, 2] & = \min \{m\in \mathbb{Z} \mid m \alpha -\varepsilon^2 \in \Delta\} \\
% & =\min \{m\in \mathbb{Z} \mid (m \alpha_1, m \alpha_2-1)\in \Delta\} \\
% & =\min \{m\in \mathbb{Z} \mid m \alpha_1\geq 2\}. 
% \end{align*}
% 此时
% \begin{align*}
% [(1, \alpha_2), 1] & = 3, \ \alpha_2\geq 0, \\
% [(2, \alpha_2), 1] & = 2, \ \alpha_2\geq 0, \\
% [(\alpha_1, \alpha_2), 1] & = 1, \ \alpha_1\geq 3, \alpha_2\geq 0, \\
% [(1, \alpha_2), 2] & = 2, \ \alpha_2\geq 1, \\
% [(\alpha_1, \alpha_2), 2] & = 1, \ \alpha_1\geq 2, \alpha_2\geq 1. 
% \end{align*}



% 若$gcd(2, \alpha_1, \alpha_2)=1$, 即$\alpha_1, \alpha_2$中至少一个是奇数, 令$[\alpha]=\max\{[\alpha, i]\mid  \alpha_i  \not\equiv 0 \bmod 2\}$. 若仅$\alpha_1$是奇数, $[\alpha]=[\alpha, 1]$, 若仅$\alpha_2$是奇数, $[\alpha]=[\alpha, 2]$, 若两者均为奇数, 则$[\alpha]=\max\{[\alpha, 1], [\alpha, 2]\}$, 即有
% \begin{align*}
% [(1, \alpha_2)]&=\max\{[(1, \alpha_2), 1], [(1, \alpha_2), 2]\}=3, \text{$\alpha_2 \geq 1$是奇数}\\
% [(1, \alpha_2)]&=[(1, \alpha_2), 1]=3, \text{$\alpha_2 \geq 0$是偶数}\\
% [(3, \alpha_2)]&=\max\{[(3, \alpha_2), 1], [(3, \alpha_2), 2]\}=1, \text{$\alpha_2\geq 1$是奇数}\\
% [(3, \alpha_2)]&=[(3, \alpha_2), 1]=1, \text{$\alpha_2\geq 0$是偶数}\\
% [(2, 1)]&=[(2, 1), 2]=1, \\
% [\alpha]&=1, \text{其它符合条件的$\alpha$. }
% \end{align*}
% 为了方便我们把上面的计算结果列表如下
% \[\begin{array}{|c|c|c|c|}
% \hline
% (\alpha_1, \alpha_2) & [(\alpha_1, \alpha_2), 1] &[(\alpha_1, \alpha_2), 2] &[(\alpha_1, \alpha_2)]  \\
% \hline
% (1, \alpha_2)  & 3, \alpha_2 \geq 0& 2 , \alpha_2\geq 1 & 3 \\
% \hline
% (2, \alpha_2)  & 2, \alpha_2 \geq 0& 1 , \alpha_2\geq 1 & 1, \text{当$\alpha_2$是奇数时} \\
% \hline
% (3, \alpha_2)  & 1, \alpha_2 \geq 0& 1 , \alpha_2\geq 1 & 1 \\
% \hline
% (\alpha_1, 0), \alpha_1 \geq 3 & 1 & \text{无定义} &1, \text{当$\alpha_1$是奇数时} \\
% \hline
% (\alpha_1, \alpha_2), \alpha_1 \geq 3, \alpha_2 \geq 1& 1 & 1& 1, \text{当$gcd(2,\alpha_1, \alpha_2)=1$时} \\ % 注意这里(3,9)还在,后面要求alpha1,2互素时才能去掉
% \hline
% \end{array}\]


% 下面计算$\Lambda^{00}=\big\{(\alpha, i)\in \Lambda\mid  gcd(\alpha_1, \alpha_2)=1, i\neq \min\{j\mid \alpha_j\not\equiv 0 \bmod 2, [\alpha, j]=[\alpha]\} \big\}$.
% \begin{enumerate}
% 	\item 对于任何的$\alpha_2\geq 0$, $((1, \alpha_2), 1) \not \in \Lambda^{00}$, 这是因为$1\not\equiv 0 \bmod 2$且$[(1, \alpha_2), 1]=3=[(1, \alpha_2)]$, 从而$\min\{j\mid \alpha_j\not\equiv 0 \bmod 2, [(1, \alpha_2), j]=[(1, \alpha_2)]\}=1$.
% 	\item 对于任何的$\alpha_2\geq 1$, $((1, \alpha_2), 2) \in \Lambda^{00}$, 且$[(1, \alpha_2), 2]=2$. 这是因为此时$[(1, \alpha_2), 1]=3=[(1, \alpha_2)]$, $\min\{j\mid \alpha_j\not\equiv 0 \bmod 2, [\alpha, j]=[\alpha]\}=1$, .
% 	\item 对于任何的奇数$\alpha_2\geq 1$, $((2, \alpha_2), 1) \in \Lambda^{00}$, 且$[(2, \alpha_2), 1]=2$. 因为$\alpha_2\not\equiv 0 \bmod 2$并且$[(2, \alpha_2), 2]=1=[(2, \alpha_2)]$, 故$\{j\mid \alpha_j\not\equiv 0 \bmod 2, [(2, \alpha_2), j]=[(2, \alpha_2)]\}=2\neq 1$. 注意若$\alpha_2\geq 0$为偶数时, $[2,\alpha_2]$无定义, 因此$((2, \alpha_2), 1) \not \in \Lambda^{00}$, 同理$((2, \alpha_2), 2) \not \in \Lambda^{00}$.
% 	\item 对于任何的奇数$\alpha_2\geq 1$, $((2, \alpha_2), 2) \not \in \Lambda^{00}$. 由于$[(2, \alpha_2), 2]=1=[(2, \alpha_2)]$, 与$2\neq \min\{j\mid \alpha_j\not\equiv 0 \bmod 2, [\alpha, j]=[\alpha]\}$矛盾.
% 	\item 对于偶数$\alpha_1\geq 3$和奇数$\alpha_2\geq 1$, $((\alpha_1, \alpha_2), 1)  \in \Lambda^{00}$, 且$[(\alpha_1, \alpha_2), 1]=1$. 而当$\alpha_1\geq 3$为奇数$\alpha_2\geq 1$时, 或$\alpha_1, \alpha_2$均为偶数时, $((\alpha_1, \alpha_2), 1) \not  \in \Lambda^{00}$. 由于$((\alpha_1, \alpha_2), 1)  \in \Lambda^{00}$要求$1\neq \min\{j\mid \alpha_j\not\equiv 0 \bmod 2, [(\alpha_1, \alpha_2), j]=[(\alpha_1, \alpha_2)]\}$, 当$\alpha_1\geq 3$为奇数时上式不成立, $2= \min\{j\mid \alpha_j\not\equiv 0 \bmod 2, [(\alpha_1, \alpha_2), j]=[(\alpha_1, \alpha_2)]\}$当且仅当$\alpha_1\geq 3$为偶数且$\alpha_2\geq 1$为奇数.
% 	\item 对于奇数$\alpha_1\geq 3$和任意$\alpha_2\geq 1 $, 且$[(\alpha_1, \alpha_2), 2]=1$. $((\alpha_1, \alpha_2), 2)  \in \Lambda^{00}$, 其余情况只要当$\alpha_1\geq 3$为偶数时$((\alpha_1, \alpha_2), 2) \not  \in \Lambda^{00}$. 由于$((\alpha_1, \alpha_2), 2)  \in \Lambda^{00}$要求$2\neq \min\{j\mid \alpha_j\not\equiv 0 \bmod 2, [(\alpha_1, \alpha_2), j]=[(\alpha_1, \alpha_2)]\}$, 当$\alpha_1$为偶数时上式不成立, 而当$\alpha_1$为奇数时, 任意$\alpha_2\geq 1$, $[(\alpha_1, \alpha_2), 1]=1=[(\alpha_1, \alpha_2)]$. 
% \end{enumerate}



% 从而
% \begin{align*}
% \Lambda^{00}=&\{((1, \alpha_2), 2)\mid  \alpha_2\geq 1\} \\
% 	&\cup \{((2, \alpha_2), 1)\mid  \alpha_2\geq 1\text{为奇数}\} \\
% 	&\cup \{((\alpha_1, \alpha_2), 1) \mid \alpha_1\geq 3\text{为偶数}, \alpha_2\geq 1\text{为奇数}\} \\
% 	&\cup \{((\alpha_1, \alpha_2), 2) \mid \alpha_1\geq 3\text{为奇数}, \alpha_2\geq 1\}. 
% \end{align*}
% 记$\Lambda^{00}_d=\{(\alpha, i)\in \Lambda^{00}\mid  [(\alpha, i)]=d\}$, $\Lambda^{00}_{> 1}=\{(\alpha, i)\in \Lambda^{00}\mid  [(\alpha, i)]>1\}$, 则有
% $\Lambda^{00}_1=\{(\alpha, i)\in \Lambda^{00}\mid  [(\alpha, i)]=1\}$, $\Lambda^{00}_{>1}=\Lambda^{00}_2=\{(\alpha, i)\in \Lambda^{00}\mid  [(\alpha, i)]=2\}$, 于是有
% \begin{align*}
% \Lambda^{00}_1 &= \{((\alpha_1, \alpha_2), 1) \mid  \alpha_1\geq 3\text{为偶数}, \alpha_2\geq 1\text{为奇数}\} \cup \{((\alpha_1, \alpha_2), 2) \mid  \alpha_1\geq 3\text{为奇数}, \alpha_2\geq 1\}   \\
% {\color{blue}\Lambda^{00}_2} &=\{((2, \alpha_2), 1)\mid  \alpha_2\geq 1\text{为奇数}\} \cup \{((1, \alpha_2), 2)\mid  \alpha_2\geq 1\}
% \end{align*}
% \[\Lambda^{00}=\Lambda^{00}_1 \sqcup \Lambda^{00}_2.\]
% 若$[\alpha, i]=1$时, $(1+x\mathbb{F}_{2}[x]/(x))^{\times}$是平凡的, $[\alpha, i]=2$时, $(1+x\mathbb{F}_{2}[x]/(x^{2}))^{\times}\cong \mathbb{Z}/2 \mathbb{Z}$, 从而由定理\ref{K2(A, M)}得

% \begin{align*}
% NK_2(\mathbb{F}_2[C_2])\cong K_2(A, M) &\cong \bigoplus_{(\alpha, i)\in\Lambda^{00}}(1+x\mathbb{F}_{2}[x]/(x^{[\alpha, i]}))^{\times}\\
% & = \bigoplus_{(\alpha, i)\in \Lambda^{00}_2}(1+x\mathbb{F}_{2}[x]/(x^{2}))^{\times}\\
% & = \bigoplus_{\scriptsize\substack{((1, \alpha_2), 2) \\ \alpha_2 \geq 1}}(1+x\mathbb{F}_{2}[x]/(x^{2}))^{\times} \oplus \bigoplus_{\scriptsize\substack{((2, \alpha_2), 1) \\ \alpha_2\geq 1\text{为奇数}}}(1+x\mathbb{F}_{2}[x]/(x^{2}))^{\times} \\
% & = \bigoplus_{\alpha_2 \geq 1}\mathbb{Z}/2 \mathbb{Z} \oplus \bigoplus_{\alpha_2\geq 1\text{为奇数}}\mathbb{Z}/2 \mathbb{Z}, 
% \end{align*}
% 作为交换群, 
% \[NK_2(\mathbb{F}_2[C_2]) \cong \bigoplus_{\infty} \mathbb{Z}/2 \mathbb{Z}. \]
% % 
% % 定理第二部分
% % 
% % 

% 	(2)由\ref{K2(A, M)}, 对于任意$(\alpha, i)\in \Lambda^{00}$, $\Gamma_{\alpha, i}$诱导了同态
%  \begin{align*}
%  \Gamma_{\alpha, i} \colon \big(1+xk[x]/(x^{[\alpha, i]})\big)^{\times} &\longrightarrow K_2(A, M)\\
%  1-xf(x) &\mapsto \langle f(t^\alpha)t^{\alpha-\varepsilon^i}, t_i \rangle. 
%  \end{align*}
%  此时只需考虑$\Lambda^{00}_2=\{((2, \alpha_2), 1)\mid  \alpha_2\geq 1\text{为奇数}\} \cup \{((1, \alpha_2), 2)\mid  \alpha_2\geq 1\}$, 对于任意$(\alpha, i)\in \Lambda^{00}_2$, $\Gamma_{\alpha, i}$均诱导了单射, 对任意$\alpha_2\geq 1$, 
%   \begin{align*}
%  \Gamma_{(1, \alpha_2), 2} \colon (1+x \mathbb{F}_2[x]/(x^{2}))^{\times} &\rightarrowtail K_2(A, M)\\
%  1+x &\mapsto 
%  \langle t_1t_2^{\alpha_2-1}, t_2 \rangle, 
%  \end{align*}
% 对任意$\alpha_2\geq 1$为奇数, 
%  \begin{align*}
%  \Gamma_{(2, \alpha_2), 1} \colon (1+x \mathbb{F}_2[x]/(x^{2}))^{\times} &\rightarrowtail K_2(A, M)\\
%  1+x &\mapsto \langle t_1t_2^{\alpha_2}, t_1 \rangle, 
%  \end{align*}

% 我们作简单的替换令$t=t_1, x=t_2$, 于是$\langle t_1t_2^{\alpha_2-1}, t_2 \rangle = \langle tx^{\alpha_2-1}, x \rangle$, $\langle t_1t_2^{\alpha_2}, t_1 \rangle=\langle t x^{\alpha_2}, t  \rangle$. 由同构\ref{K2(A, M)}可知$NK_2(\mathbb{F}_2[C_2])$是由Dennis-Stein符号$\{\langle tx^i, x \rangle \mid i\geq 0\}$与$\{\langle tx^i, t \rangle \mid i\geq 1\text{为奇数}\}$生成的, 由于$t^2=0$故$\langle tx^i, x \rangle+\langle tx^i, x \rangle=\langle tx^i+tx^i-t^2x^{2i+1}, x \rangle=0$, $\langle tx^i, t \rangle+\langle tx^i, t \rangle=\langle tx^i+tx^i-t^3x^{2i}, t \rangle=0$. 
% \end{proof}
% \begin{remark}
% 	对于$i\geq 1\text{为偶数}$, $\langle tx^i, t \rangle=\langle x^{i/2}, t \rangle+\langle x^{i/2}, t \rangle=\langle x^{i/2}+x^{i/2}+tx^i, t \rangle=0$. 
% \end{remark}

% Weibel在\cite{weibel2009nk0}中给出了以下可裂正合列
% 	\[0\longrightarrow V/\Phi(V) \overset{F}\longrightarrow NK_2(\mathbb{F}_2[C_2])\overset{D}\longrightarrow \Omega_{\mathbb{F}_2[x]}\longrightarrow 0, \]
% 其中$V=x \mathbb{F}_2[x]$, $\Phi(V)=x^2 \mathbb{F}_2[x^2]$是$V$的子群, $\Omega_{\mathbb{F}_2[x]}\cong \mathbb{F}_2[x]\, d x$是绝对K\"{a}hler微分模, $F(x^n)=\langle tx^n, t \rangle$, $D(\langle ft, g+g't \rangle)=f\, dg$. 显然$D(\langle tx^i, t \rangle)=0$, $D(\langle tx^i, x \rangle)=x^i\, dx$, 可以看出$NK_2(\mathbb{F}_2[C_2])$的直和项
% $$\bigoplus_{((2, \alpha_2), 1), \alpha_2\geq 1\text{为奇数}} \mathbb{Z}/2\mathbb{Z} \cong V/\Phi(V),$$ 
% 直和项
% $$\bigoplus_{((1, \alpha_2), 2), \alpha_2\geq 1} \mathbb{Z}/2\mathbb{Z} \cong \mathbb{F}_2[x]\, d x.$$ 

% $V$和$\Omega_{\mathbb{F}_2[x]}$作为交换群是同构的, 但作为$W(\mathbb{F}_2)$-模是不同的. $V=x \mathbb{F}_2[x]$上的$W(\mathbb{F}_2)$-模结构(见\cite{MR96j:16008})为 
% \begin{align*}
%  V_m(x^n)&=x^{mn}, \\
%  F_d(x^n)&=\begin{cases}
%  	dx^{n/d}, & \mbox{ 若 $d|n$}\\
%  	0, & \mbox{其它}
%  \end{cases}, \\
%  [a]x^n&=a^nx^n. 
%  \end{align*}
% $\Omega_{\mathbb{F}_2[x]}=\mathbb{F}_2[x]\, dx $上的$W(\mathbb{F}_2)$-模结构(见\cite{MR96j:16008})为
% \begin{align*}
%  V_m(x^{n-1}\, dx)&=mx^{mn-1}\, dx, \\
%  F_d(x^{n-1}\, dx)&=\begin{cases}
%  	x^{n/d-1}\, dx, & \mbox{ 若 $d|n$}\\
%  	0, & \mbox{其它}
%  \end{cases}, \\
%  [a]x^{n-1}\, dx&=a^nx^{n-1}\, dx. 
%  \end{align*}
% 结合两者我们可以得到$NK_2(\mathbb{F}_2[C_2])$的$W(\mathbb{F}_2)$-模结构为
% \begin{align*}
%  V_m(\langle tx^n, t \rangle)&=\begin{cases}
%  	\langle tx^{mn}, t \rangle, & \mbox{若$m$是奇数}\\
%  	0, & \mbox{若$m$是偶数}
%  \end{cases}, \quad \mbox{$n\geq 1$为奇数} \\
%   V_m(\langle tx^{n-1}, x \rangle)&=\begin{cases}
%  	\langle tx^{mn-1}, x \rangle, & \mbox{若$m$是奇数}\\
%  	0, & \mbox{若$m$是偶数}
%  \end{cases}
%  , \quad \mbox{$n\geq 1$} \\
%  F_d(\langle tx^n, t \rangle)&=\begin{cases}
%  	\langle tx^{n/d}, t \rangle, & \mbox{ 若$d|n$}\\
%  	0, & \mbox{其它}
%  \end{cases}, \quad \mbox{$n\geq 1$为奇数} \\
%  F_d(\langle tx^{n-1}, x \rangle)&=\begin{cases}
%  	\langle tx^{n/d-1}, x \rangle, & \mbox{若$d|n$}\\
%  	0, & \mbox{其它}
%  \end{cases}
%  , \quad \mbox{$n\geq 1$} \\
%  [1]\langle tx^n, t \rangle&=\langle tx^n, t \rangle, \quad \mbox{$n\geq 1$为奇数} \\
%  [1]\langle tx^{n-1}, x \rangle&=\langle tx^{n-1}, x \rangle, \quad \mbox{$n\geq 1$}. 
%  \end{align*}



% % 粘贴自NKFinAlg. tex 结束












\section{$NK_2(\mathbb{F}_2[C_4])$的结构}
%!TEX root = ../../testmain.tex
用同样的方法计算$NK_2(\mathbb{F}_2[C_{2^2}])$, 继而对于任意$n$可以得到类似的结果. 

\begin{theorem}
	$NK_2(\mathbb{F}_2[C_4])\cong \bigoplus_{\infty} \mathbb{Z}/2 \mathbb{Z}\oplus \bigoplus_{\infty}\mathbb{Z}/4 \mathbb{Z}$. 
\end{theorem}
\begin{proof}
	$\mathbb{F}_2[t_1, t_2]/(t_1^4)=\mathbb{F}_2[C_{4}][t_2]$, 此时$I=(t_1^4)$, $M=(t_1)$不变, 我们直接写出以下集合

\begin{align*}
\Delta &=\{(\alpha_1, \alpha_2)\mid \alpha_1\geq 4, \alpha_2 \geq 0\}, \\
\Lambda &=\{((\alpha_1, \alpha_2), 1) \mid \alpha_1\geq 1\}\cup \{((\alpha_1, \alpha_2), 2) \mid \alpha_1\geq 1, \alpha_2\geq 1\}, 
\end{align*}
用$\left \lceil x \right \rceil=\min \{m\in \mathbb{Z}\mid m\geq x\}$表示不小于$x$的最小整数, 
\begin{align*}
[\alpha, 1] & =\min \{m\in \mathbb{Z} \mid m \alpha_1\geq 5\}=\left \lceil 5/\alpha_1 \right \rceil, \\
[\alpha, 2] & =\min \{m\in \mathbb{Z} \mid m \alpha_1\geq 4\}=\left \lceil 4/\alpha_1 \right \rceil. 
\end{align*}

例如
\begin{align*}
[(1, \alpha_2), 1] & = 5, \ \alpha_2\geq 0, \\
[(2, \alpha_2), 1] & = 3, \ \alpha_2\geq 0, \\
[(3, \alpha_2), 1] & = 2, \ \alpha_2\geq 0, \\
[(4, \alpha_2), 1] & = 2, \ \alpha_2\geq 0, \\
[(\alpha_1, \alpha_2), 1] & = 1, \ \alpha_1\geq 5, \alpha_2\geq 0, \\
[(1, \alpha_2), 2] & = 4, \ \alpha_2\geq 1, \\
[(2, \alpha_2), 2] & = 2, \ \alpha_2\geq 1, \\
[(3, \alpha_2), 2] & = 2, \ \alpha_2\geq 1, \\
[(\alpha_1, \alpha_2), 2] & = 1, \ \alpha_1\geq 4, \alpha_2\geq 1. 
\end{align*}


\[\begin{array}{|c|c|c|c|}
\hline
(\alpha_1, \alpha_2) & [(\alpha_1, \alpha_2), 1] &[(\alpha_1, \alpha_2), 2] &[(\alpha_1, \alpha_2)]  \\
\hline
(1, \alpha_2)  & 5, \alpha_2 \geq 0& 4 , \alpha_2\geq 1 & 5 \\
\hline
(2, \alpha_2)  & 3, \alpha_2 \geq 0& 2 , \alpha_2\geq 1 & 2, \text{当$\alpha_2$是奇数时} \\
\hline
(3, \alpha_2)  & 2, \alpha_2 \geq 0& 2 , \alpha_2\geq 1 & 2 \\
\hline
(4, \alpha_2)  & 2, \alpha_2 \geq 0& 1 , \alpha_2\geq 1 & 1 \text{当$\alpha_2$是奇数时}\\
\hline
(\alpha_1, 0), \alpha_1 \geq 5 & 1 & \text{无定义} &1, \text{当$\alpha_1$是奇数时} \\
\hline
(\alpha_1, \alpha_2), \alpha_1 \geq 5, \alpha_2 \geq 1& 1 & 1& 1, \text{当$(\alpha_1, \alpha_2)=1$时} \\
\hline
\end{array}\]


记$\Lambda^{00}_d=\{(\alpha, i)\in \Lambda^{00}\mid   [(\alpha, i)]=d\}$, $\Lambda^{00}_{>1}=\{(\alpha, i)\in \Lambda^{00}\mid   [(\alpha, i)]>1\}$

由于$(\alpha, i)\in \Lambda^{00}_1$均有$[(\alpha, i)]=1$, 实际上要计算$(1+x\mathbb{F}_2[x]/(x^{[\alpha, i]}))^{\times}$只需确定$\Lambda^{00}_{>1}$. 由同样的方法可得
$\Lambda^{00}_4=\{((1, \alpha_2), 2)\mid  \alpha_2\geq 1\}$, $\Lambda^{00}_3=\{((2, \alpha_2), 1)\mid  \alpha_2\geq 1\text{为奇数}\}$, $\Lambda^{00}_2=\{((3, \alpha_2), 2)\mid  gcd(3, \alpha_2)=1, \alpha_2\geq 1\}\cup \{((4, \alpha_2), 1)\mid  \alpha_2\geq 1\text{为奇数}\}$, 
{\color{blue}\begin{align*}
\Lambda^{00}_{>1}&= \{((1, \alpha_2), 2)\mid  \alpha_2\geq 1\}\cup \{((3, \alpha_2), 2)\mid gcd(3, \alpha_2)=1, \alpha_2\geq 1\}\\
	& \cup \{((2, \alpha_2), 1)\mid  \alpha_2\geq 1\text{为奇数}\}  \cup \{((4, \alpha_2), 1)\mid  \alpha_2\geq 1\text{为奇数}\}. 
\end{align*}}




由定理\ref{K2(A, M)}, 

\begin{align*}
NK_2(\mathbb{F}_2[C_4])\cong K_2(A, M) &\cong \bigoplus_{(\alpha, i)\in\Lambda^{00}}(1+x\mathbb{F}_2[x]/(x^{[\alpha, i]}))^{\times}\\
& = \bigoplus_{(\alpha, i)\in \Lambda^{00}_{>1}}(1+x\mathbb{F}_2[x]/(x^{[\alpha, i]}))^{\times}\\
& = \bigoplus_{\scriptsize\substack{((3, \alpha_2), 2) \\ gcd(3, \alpha_2)=1 \\ \alpha_2\geq 1}}(1+x\mathbb{F}_2[x]/(x^{2}))^{\times}\oplus \bigoplus_{\scriptsize\substack{((4, \alpha_2), 1) \\ \alpha_2\geq 1\text{为奇数}}}(1+x\mathbb{F}_2[x]/(x^{2}))^{\times} \\
& \oplus \bigoplus_{\scriptsize\substack{((2, \alpha_2), 1) \\ \alpha_2\geq 1\text{为奇数}}}(1+x\mathbb{F}_2[x]/(x^{3}))^{\times} \oplus \bigoplus_{\scriptsize\substack{((1, \alpha_2), 2) \\ \alpha_2\geq 1}}(1+x\mathbb{F}_2[x]/(x^{4}))^{\times}. 
\end{align*}

由\ref{ex:W3(F2)}有$(1+x\mathbb{F}_2[x]/(x^{4}))^{\times}\cong \mathbb{Z}/2 \mathbb{Z}\times {\color{red}\mathbb{Z}/4 \mathbb{Z}}$, $(1+x\mathbb{F}_2[x]/(x^{3}))^{\times}\cong {\color{red}\mathbb{Z}/4 \mathbb{Z}}$, 于是$NK_2(\mathbb{F}_2[C_4])$作为交换群有
\[NK_2(\mathbb{F}_2[C_4]) \cong \bigoplus_{\infty} \mathbb{Z}/2 \mathbb{Z}\oplus \bigoplus_{\infty}{\color{red}\mathbb{Z}/4 \mathbb{Z}}. \]




对于任意$(\alpha, i)\in \Lambda^{00}_{>1}$, $\Gamma_{\alpha, i}$均诱导了单射, 对任意$\alpha_2\geq 1$, $gcd(3, \alpha_2)=1$
  \begin{align*}
 \Gamma_{(3, \alpha_2), 2} \colon (1+x \mathbb{F}_2[x]/(x^{2}))^{\times} &\rightarrowtail K_2(A, M)\\
 1+x &\mapsto  \langle t_1^3t_2^{\alpha_2-1}, t_2 \rangle, 
 \end{align*}
对任意$\alpha_2\geq 1$, 
  \begin{align*}
 \Gamma_{(1, \alpha_2), 2} \colon (1+x \mathbb{F}_2[x]/(x^{4}))^{\times} &\rightarrowtail K_2(A, M)\\
 {\color{red}1+x \text{(四阶元)}} &\mapsto {\color{red}\langle t_1t_2^{\alpha_2-1}, t_2 \rangle}, \\
 1+x^3 \text{(二阶元)} &\mapsto \langle t_1^3t_2^{3\alpha_2-1}, t_2 \rangle, 
 \end{align*}

对任意$\alpha_2\geq 1$为奇数, 
 \begin{align*}
 \Gamma_{(4, \alpha_2), 1} \colon (1+x \mathbb{F}_2[x]/(x^{2}))^{\times} &\rightarrowtail K_2(A, M)\\
 1+x &\mapsto \langle t_1^3t_2^{\alpha_2}, t_1 \rangle, \\
 \Gamma_{(2, \alpha_1), 1} \colon (1+x \mathbb{F}_2[x]/(x^{3}))^{\times} &\rightarrowtail K_2(A, M)\\
 {\color{red}1+x\text{(四阶元)} }&\mapsto {\color{red}\langle t_1t_2^{\alpha_2}, t_1 \rangle}. 
 \end{align*}

我们作简单的替换令$t=t_1, x=t_2$, 由同构\ref{K2(A, M)}可知$NK_2(\mathbb{F}_2[C_4])$是由Dennis-Stein符号
${\color{red}\{\langle tx^{i-1}, x \rangle \mid i\geq 1\}}$, 
${\color{red}\{\langle tx^i, t \rangle \mid i\geq 1\text{为奇数}\}}$, 
$\{\langle t^3x^{3i-1}, x \rangle \mid i\geq 1\}$, 
$\{\langle t^3x^{i-1}, x \rangle \mid i\geq 1, gcd(i, 3)=1\}$, 
$\{\langle t^3x^i, t \rangle \mid i\geq 1\text{为奇数}\}$
生成的. 

\end{proof}

\begin{remark}

	$\langle t^3x^{2i}, t \rangle =\langle tx^{i}, t \rangle+\langle tx^{i}, t \rangle$是二阶元. 
	根据\cite{MR80k:13005}, 存在同态
	\begin{align*}
	\rho_1 \colon \mathbb{F}_2[x]\, \mathrm{d} x &\longrightarrow NK_2(\mathbb{F}_2[C_4])\\
				x^i\, \mathrm{d} x &\mapsto \langle t^3x^i, x\rangle \\
	\rho_2 \colon x\mathbb{F}_2[x]/x^4\mathbb{F}_2[x^4] &\longrightarrow NK_2(\mathbb{F}_2[C_4])\\
				x^i &\mapsto \langle t^3x^i, t\rangle 
	\end{align*}

	$\{\langle t^3x^{i-1}, x \rangle \mid i\geq 1\}=\{\langle t^3x^{3i-1}, x \rangle \mid i\geq 1\}\cup\{\langle t^3x^{i-1}, x \rangle \mid i\geq 1, gcd(i, 3)=1\}$. %, 从而$\Omega_{\mathbb{F}_2[x]}\oplus x\mathbb{F}_2[x]/x^4\mathbb{F}_2[x^4]$是$NK_2(\mathbb{F}_2[C_4])$的直和项. 
\end{remark}
% 用同样的方法计算$NK_2(\mathbb{F}_2[C_{2^2}])$, 继而对于任意$n$可以得到类似的结果. 

% \begin{theorem}
% 	$NK_2(\mathbb{F}_2[C_4])\cong \bigoplus_{\infty} \mathbb{Z}/2 \mathbb{Z}\oplus \bigoplus_{\infty}\mathbb{Z}/4 \mathbb{Z}$. 
% \end{theorem}
% \begin{proof}
% 	$\mathbb{F}_2[t_1, t_2]/(t_1^4)=\mathbb{F}_2[C_{4}][t_2]$, 此时$I=(t_1^4)$, $M=(t_1)$不变, 我们直接写出以下集合

% \begin{align*}
% \Delta &=\{(\alpha_1, \alpha_2)\mid \alpha_1\geq 4, \alpha_2 \geq 0\}, \\
% \Lambda &=\{((\alpha_1, \alpha_2), 1) \mid \alpha_1\geq 1\}\cup \{((\alpha_1, \alpha_2), 2) \mid \alpha_1\geq 1, \alpha_2\geq 1\}, 
% \end{align*}
% 用$\left \lceil x \right \rceil=\min \{m\in \mathbb{Z}\mid m\geq x\}$表示不小于$x$的最小整数, 
% \begin{align*}
% [\alpha, 1] & =\min \{m\in \mathbb{Z} \mid m \alpha_1\geq 5\}=\left \lceil 5/\alpha_1 \right \rceil, \\
% [\alpha, 2] & =\min \{m\in \mathbb{Z} \mid m \alpha_1\geq 4\}=\left \lceil 4/\alpha_1 \right \rceil. 
% \end{align*}

% 例如
% \begin{align*}
% [(1, \alpha_2), 1] & = 5, \ \alpha_2\geq 0, \\
% [(2, \alpha_2), 1] & = 3, \ \alpha_2\geq 0, \\
% [(3, \alpha_2), 1] & = 2, \ \alpha_2\geq 0, \\
% [(4, \alpha_2), 1] & = 2, \ \alpha_2\geq 0, \\
% [(\alpha_1, \alpha_2), 1] & = 1, \ \alpha_1\geq 5, \alpha_2\geq 0, \\
% [(1, \alpha_2), 2] & = 4, \ \alpha_2\geq 1, \\
% [(2, \alpha_2), 2] & = 2, \ \alpha_2\geq 1, \\
% [(3, \alpha_2), 2] & = 2, \ \alpha_2\geq 1, \\
% [(\alpha_1, \alpha_2), 2] & = 1, \ \alpha_1\geq 4, \alpha_2\geq 1. 
% \end{align*}


% \[\begin{array}{|c|c|c|c|}
% \hline
% (\alpha_1, \alpha_2) & [(\alpha_1, \alpha_2), 1] &[(\alpha_1, \alpha_2), 2] &[(\alpha_1, \alpha_2)]  \\
% \hline
% (1, \alpha_2)  & 5, \alpha_2 \geq 0& 4 , \alpha_2\geq 1 & 5 \\
% \hline
% (2, \alpha_2)  & 3, \alpha_2 \geq 0& 2 , \alpha_2\geq 1 & 2, \text{当$\alpha_2$是奇数时} \\
% \hline
% (3, \alpha_2)  & 2, \alpha_2 \geq 0& 2 , \alpha_2\geq 1 & 2 \\
% \hline
% (4, \alpha_2)  & 2, \alpha_2 \geq 0& 1 , \alpha_2\geq 1 & 1 \text{当$\alpha_2$是奇数时}\\
% \hline
% (\alpha_1, 0), \alpha_1 \geq 5 & 1 & \text{无定义} &1, \text{当$\alpha_1$是奇数时} \\
% \hline
% (\alpha_1, \alpha_2), \alpha_1 \geq 5, \alpha_2 \geq 1& 1 & 1& 1, \text{当$(\alpha_1, \alpha_2)=1$时} \\
% \hline
% \end{array}\]


% 记$\Lambda^{00}_d=\{(\alpha, i)\in \Lambda^{00}\mid   [(\alpha, i)]=d\}$, $\Lambda^{00}_{>1}=\{(\alpha, i)\in \Lambda^{00}\mid   [(\alpha, i)]>1\}$

% 由于$(\alpha, i)\in \Lambda^{00}_1$均有$[(\alpha, i)]=1$, 实际上要计算$(1+x\mathbb{F}_2[x]/(x^{[\alpha, i]}))^{\times}$只需确定$\Lambda^{00}_{>1}$. 由同样的方法可得
% $\Lambda^{00}_4=\{((1, \alpha_2), 2)\mid  \alpha_2\geq 1\}$, $\Lambda^{00}_3=\{((2, \alpha_2), 1)\mid  \alpha_2\geq 1\text{为奇数}\}$, $\Lambda^{00}_2=\{((3, \alpha_2), 2)\mid  gcd(3, \alpha_2)=1, \alpha_2\geq 1\}\cup \{((4, \alpha_2), 1)\mid  \alpha_2\geq 1\text{为奇数}\}$, 
% {\color{blue}\begin{align*}
% \Lambda^{00}_{>1}=& \{((1, \alpha_2), 2)\mid  \alpha_2\geq 1\}\cup \{((3, \alpha_2), 2)\mid gcd(3, \alpha_2)=1, \alpha_2\geq 1\}\\
% 	& \cup \{((2, \alpha_2), 1)\mid  \alpha_2\geq 1\text{为奇数}\}  \cup \{((4, \alpha_2), 1)\mid  \alpha_2\geq 1\text{为奇数}\}. 
% \end{align*}}




% 由定理\ref{K2(A, M)}, 

% \begin{align*}
% NK_2(\mathbb{F}_2[C_4])\cong K_2(A, M) &\cong \bigoplus_{(\alpha, i)\in\Lambda^{00}}(1+x\mathbb{F}_2[x]/(x^{[\alpha, i]}))^{\times}\\
% & = \bigoplus_{(\alpha, i)\in \Lambda^{00}_{>1}}(1+x\mathbb{F}_2[x]/(x^{[\alpha, i]}))^{\times}\\
% & = \bigoplus_{\scriptsize\substack{((3, \alpha_2), 2) \\ gcd(3, \alpha_2)=1 \\ \alpha_2\geq 1}}(1+x\mathbb{F}_2[x]/(x^{2}))^{\times}\oplus \bigoplus_{\scriptsize\substack{((4, \alpha_2), 1) \\ \alpha_2\geq 1\text{为奇数}}}(1+x\mathbb{F}_2[x]/(x^{2}))^{\times} \\
% & \oplus \bigoplus_{\scriptsize\substack{((2, \alpha_2), 1) \\ \alpha_2\geq 1\text{为奇数}}}(1+x\mathbb{F}_2[x]/(x^{3}))^{\times} \oplus \bigoplus_{\scriptsize\substack{((1, \alpha_2), 2) \\ \alpha_2\geq 1}}(1+x\mathbb{F}_2[x]/(x^{4}))^{\times}. 
% \end{align*}

% 由\ref{ex:W3(F2)}有$(1+x\mathbb{F}_2[x]/(x^{4}))^{\times}\cong \mathbb{Z}/2 \mathbb{Z}\times {\color{red}\mathbb{Z}/4 \mathbb{Z}}$, $(1+x\mathbb{F}_2[x]/(x^{3}))^{\times}\cong {\color{red}\mathbb{Z}/4 \mathbb{Z}}$, 于是$NK_2(\mathbb{F}_2[C_4])$作为交换群有
% \[NK_2(\mathbb{F}_2[C_4]) \cong \bigoplus_{\infty} \mathbb{Z}/2 \mathbb{Z}\oplus \bigoplus_{\infty}{\color{red}\mathbb{Z}/4 \mathbb{Z}}. \]




% 对于任意$(\alpha, i)\in \Lambda^{00}_{>1}$, $\Gamma_{\alpha, i}$均诱导了单射, 对任意$\alpha_2\geq 1$, $gcd(3, \alpha_2)=1$
%   \begin{align*}
%  \Gamma_{(3, \alpha_2), 2} \colon (1+x \mathbb{F}_2[x]/(x^{2}))^{\times} &\rightarrowtail K_2(A, M)\\
%  1+x &\mapsto  \langle t_1^3t_2^{\alpha_2-1}, t_2 \rangle, 
%  \end{align*}
% 对任意$\alpha_2\geq 1$, 
%   \begin{align*}
%  \Gamma_{(1, \alpha_2), 2} \colon (1+x \mathbb{F}_2[x]/(x^{4}))^{\times} &\rightarrowtail K_2(A, M)\\
%  {\color{red}1+x \text{(四阶元)}} &\mapsto {\color{red}\langle t_1t_2^{\alpha_2-1}, t_2 \rangle}, \\
%  1+x^3 \text{(二阶元)} &\mapsto \langle t_1^3t_2^{3\alpha_2-1}, t_2 \rangle, 
%  \end{align*}

% 对任意$\alpha_2\geq 1$为奇数, 
%  \begin{align*}
%  \Gamma_{(4, \alpha_2), 1} \colon (1+x \mathbb{F}_2[x]/(x^{2}))^{\times} &\rightarrowtail K_2(A, M)\\
%  1+x &\mapsto \langle t_1^3t_2^{\alpha_2}, t_1 \rangle, \\
%  \Gamma_{(2, \alpha_1), 1} \colon (1+x \mathbb{F}_2[x]/(x^{3}))^{\times} &\rightarrowtail K_2(A, M)\\
%  {\color{red}1+x\text{(四阶元)} }&\mapsto {\color{red}\langle t_1t_2^{\alpha_2}, t_1 \rangle}. 
%  \end{align*}

% 我们作简单的替换令$t=t_1, x=t_2$, 由同构\ref{K2(A, M)}可知$NK_2(\mathbb{F}_2[C_4])$是由Dennis-Stein符号
% ${\color{red}\{\langle tx^{i-1}, x \rangle \mid i\geq 1\}}$, 
% ${\color{red}\{\langle tx^i, t \rangle \mid i\geq 1\text{为奇数}\}}$, 
% $\{\langle t^3x^{3i-1}, x \rangle \mid i\geq 1\}$, 
% $\{\langle t^3x^{i-1}, x \rangle \mid i\geq 1, gcd(i, 3)=1\}$, 
% $\{\langle t^3x^i, t \rangle \mid i\geq 1\text{为奇数}\}$
% 生成的. 

% \end{proof}

% \begin{remark}

% 	$\langle t^3x^{2i}, t \rangle =\langle tx^{i}, t \rangle+\langle tx^{i}, t \rangle$是二阶元. 
% 	根据\cite{MR80k:13005}, 存在同态
% 	\begin{align*}
% 	\rho_1 \colon \mathbb{F}_2[x]dx &\longrightarrow NK_2(\mathbb{F}_2[C_4])\\
% 				x^idx &\mapsto \langle t^3x^i, x\rangle \\
% 	\rho_2 \colon x\mathbb{F}_2[x]/x^2\mathbb{F}_2[x^2] &\longrightarrow NK_2(\mathbb{F}_2[C_4])\\
% 				x^i &\mapsto \langle t^3x^i, t\rangle 
% 	\end{align*}

% 	$\{\langle t^3x^{i-1}, x \rangle \mid i\geq 1\}=\{\langle t^3x^{3i-1}, x \rangle \mid i\geq 1\}\cup\{\langle t^3x^{i-1}, x \rangle \mid i\geq 1, gcd(i, 3)=1\}$, 从而$\Omega_{\mathbb{F}_2[x]}\oplus x\mathbb{F}_2[x]/x^2\mathbb{F}_2[x^2]$是$NK_2(\mathbb{F}_2[C_4])$的直和项. 
% \end{remark}




\section{$NK_2(\mathbb{F}_q[C_{2^n}])$} % (fold)
\label{sec:NK_2(F_q[C_{2^n}])}
%!TEX root = ../../testmain.tex
设$\mathbb{F}_q$是特征为$2$的有限域, $q=2^f$, $C_{2^n}$是$2^n$阶循环群, 这一节计算$NK_2(\mathbb{F}_q[C_{2^n}])$. 假设$A=\mathbb{F}_q[t_1, t_2]/(t_1^{2^n})=\mathbb{F}_q[C_{2^n}][x]$, 此时$I=(t_1^{2^n})$, $M=(t_1)$, $A/M=\mathbb{F}_q[x]$. 

\begin{lemma}
	$\Delta =\{(\alpha_1, \alpha_2)\mid \alpha_1\geq 2^n, \alpha_2 \geq 0\}$, $\Lambda = \{((\alpha_1, \alpha_2), 1) \mid \alpha_1\geq 1, \alpha_2\geq 0\}\cup \{((\alpha_1, \alpha_2), 2) \mid \alpha_1\geq 1, \alpha_2\geq 1\}$, 对任意$(\alpha, i)\in \Lambda$, $[\alpha, 1]=\left \lceil (2^n+1)/\alpha_1 \right \rceil$, $[\alpha, 2]=\left \lceil 2^n/\alpha_1 \right \rceil$, 其中$\left \lceil x \right \rceil=\min \{m\in \mathbb{Z}\mid  m\geq x\}$表示不小于$x$的最小整数. 
\end{lemma}

\begin{lemma}
令$I_1 =\{((\alpha_1, \alpha_2), 1)\mid gcd(\alpha_1, \alpha_2)=1, 1< \alpha_1\leq 2^n\text{为偶数}, \alpha_2\geq 1\text{为奇数}\}$, $I_2=\{((\alpha_1, \alpha_2), 2)\mid gcd(\alpha_1, \alpha_2)=1, 1\leq \alpha_1<2^n\text{为奇数}, \alpha_2\geq 1\}$, 则$\Lambda^{00}_{>1}=I_1\sqcup I_2$. 
\end{lemma}
由定理\ref{K2(A, M)}, 

\begin{align*}
NK_2(\mathbb{F}_q[C_{2^n}])\cong K_2(A, M) &\cong \bigoplus_{(\alpha, i)\in\Lambda^{00}}(1+x\mathbb{F}_q[x]/(x^{[\alpha, i]}))^{\times}\\
& = \bigoplus_{(\alpha, i)\in \Lambda^{00}_{>1}}(1+x\mathbb{F}_q[x]/(x^{[\alpha, i]}))^{\times}\\
& = \bigoplus_{(\alpha, 1)\in I_1}(1+x\mathbb{F}_q[x]/(x^{\left \lceil (2^n+1)/\alpha_1 \right \rceil}))^{\times} \\
& \oplus \bigoplus_{(\alpha, 2)\in I_2}(1+x\mathbb{F}_q[x]/(x^{\left \lceil 2^n/\alpha_1 \right \rceil}))^{\times}. 
\end{align*}
注意到$BigWitt_{k}(R)=(1+x R\llbracket x\rrbracket )^{\times}/(1+x^{k+1} R\llbracket x\rrbracket )^{\times} \cong (1+x R[x]/(x^{k+1}))^{\times}$, 
根据公式\ref{cor:BW}, 

\begin{align*}
NK_2(\mathbb{F}_q[C_{2^n}])\cong & \bigoplus_{(\alpha, 1)\in I_1}\bigoplus_{\scriptsize\substack{1\leq m\leq \left \lceil (2^n+1)/\alpha_1 \right \rceil-1 \\ gcd(m, 2)=1}}(\mathbb{Z}/2^{1+ \left \lfloor\log_2 \frac{\left \lceil (2^n+1)/\alpha_1 \right \rceil-1}{m}  \right \rfloor}\mathbb{Z})^f \\
& \oplus \bigoplus_{(\alpha, 2)\in I_2}\bigoplus_{\scriptsize\substack{ 1 \leq m\leq \left \lceil 2^n/\alpha_1 \right \rceil-1 \\ gcd(m, 2)=1}}(\mathbb{Z}/2^{1+ \left \lfloor\log_2 \frac{\left \lceil 2^n/\alpha_1 \right \rceil-1}{m}  \right \rfloor}\mathbb{Z})^f. 
\end{align*}

接下来我们证明对于任意$1\leq k\leq n$, $\mathbb{Z}/2^k \mathbb{Z}$都在$NK_2(\mathbb{F}_q[C_{p^n}])$出现无限多次

\begin{lemma}
\label{lem:log2}
	对于任意的$1\leq k < n$, $1+\left \lfloor \log_2(\frac{2^n-1}{2^k+1}) \right \rfloor = n-k$. 
\end{lemma}
\begin{proof}
	当$1\leq k < n$时, $2^k-1\geq 1 \geq \frac{1}{2^{n-k-1}}$, 即
	\[2^{n-1}-2^{n-k-1}\geq 1, \]
	上式等价于$2^n-1\geq 2^{n-k-1}(2^k+1)$, 且$2^n-1<2^{n-k}(2^k+1)$, 于是
	\[2^{n-k}> \frac{2^n-1}{2^k+1} \geq 2^{n-k-1}, \]
	取对数得$\left \lfloor \log_2(\frac{2^n-1}{2^k+1}) \right \rfloor = n-k-1$. 
\end{proof}
考虑$((1, \alpha_2), 2)\in I_2$, 
$$\bigoplus_{(\alpha, 2)\in I_2}\bigoplus_{\scriptsize\substack{1\leq m\leq  2^n-1 \\ gcd(m, 2)=1 }}(\mathbb{Z}/2^{1+ \left \lfloor\log_2 \frac{2^n-1}{m}  \right \rfloor}\mathbb{Z})^f$$
是$NK_2(\mathbb{F}_{2^f}[C_{2^n}])$的直和项, 当$m=1$时$1+ \left \lfloor\log_2 (2^n-1)\right \rfloor=n$, 当$m=2^k+1 (1\leq k < n)$为奇数时, 由\ref{lem:log2}, $1+ \left \lfloor\log_2 \frac{2^n-1}{m}\right \rfloor=n-k$, {\color{blue}于是对于任何的$1\leq k\leq n$, $\mathbb{Z}/2^k\mathbb{Z}$均出现在直和项中, 且对于任意$\alpha_2\geq 1$, 这样的项总会出现}, 于是
\[NK_2(\mathbb{F}_q[C_{2^n}])\cong \bigoplus_\infty \bigoplus_{k=1}^n \mathbb{Z}/2^k\mathbb{Z}. \]

接下来给出一些$NK_2(\mathbb{F}_q[C_{2^n}])$中的$2^k(1\leq k \leq n)$阶元素. 

对任意$\alpha_2\geq 1, a\in \mathbb{F}_q$, 
  \begin{align*}
 \Gamma_{(1, \alpha_2), 2} \colon (1+x \mathbb{F}_q[x]/(x^{2^n}))^{\times} &\rightarrowtail K_2(A, M)\\
 1+ax \text{($2^n$阶元)} &\mapsto \langle atx^{\alpha_2-1}, x \rangle, \\
 1+ax^3 \text{($2^{n-1}$阶元)} &\mapsto \langle at^3x^{3\alpha_2-1}, x \rangle, \\
 1+ax^{2^k+1} \text{($2^{n-k}$阶元)} &\mapsto \langle at^{2k+1}x^{(2k+1)\alpha_2-1}, x \rangle. 
 \end{align*}
% 设$\mathbb{F}_q$是特征为$2$的有限域, $q=2^f$, $C_{2^n}$是$2^n$阶循环群, 这一节计算$NK_2(\mathbb{F}_q[C_{2^n}])$. 假设$A=\mathbb{F}_q[t_1, t_2]/(t_1^{2^n})=\mathbb{F}_q[C_{2^n}][x]$, 此时$I=(t_1^{2^n})$, $M=(t_1)$, $A/M=\mathbb{F}_q[x]$. 

% \begin{lemma}
% 	$\Delta =\{(\alpha_1, \alpha_2)\mid \alpha_1\geq 2^n, \alpha_2 \geq 0\}$, $\Lambda = \{((\alpha_1, \alpha_2), 1) \mid \alpha_1\geq 1, \alpha_2\geq 0\}\cup \{((\alpha_1, \alpha_2), 2) \mid \alpha_1\geq 1, \alpha_2\geq 1\}$, 对任意$(\alpha, i)\in \Lambda$, $[\alpha, 1]=\left \lceil (2^n+1)/\alpha_1 \right \rceil$, $[\alpha, 2]=\left \lceil 2^n/\alpha_1 \right \rceil$, 其中$\left \lceil x \right \rceil=\min \{m\in \mathbb{Z}\mid  m\geq x\}$表示不小于$x$的最小整数. 
% \end{lemma}

% \begin{lemma}
% 令$I_1 =\{((\alpha_1, \alpha_2), 1)\mid gcd(\alpha_1, \alpha_2)=1, 1< \alpha_1\leq 2^n\text{为偶数}, \alpha_2\geq 1\text{为奇数}\}$, $I_2=\{((\alpha_1, \alpha_2), 2)\mid gcd(\alpha_1, \alpha_2)=1, 1\leq \alpha_1<2^n\text{为奇数}, \alpha_2\geq 1\}$, 则$\Lambda^{00}_{>1}=I_1\sqcup I_2$. 
% \end{lemma}
% 由定理\ref{K2(A, M)}, 

% \begin{align*}
% NK_2(\mathbb{F}_q[C_{2^n}])\cong K_2(A, M) &\cong \bigoplus_{(\alpha, i)\in\Lambda^{00}}(1+x\mathbb{F}_q[x]/(x^{[\alpha, i]}))^{\times}\\
% & = \bigoplus_{(\alpha, i)\in \Lambda^{00}_{>1}}(1+x\mathbb{F}_q[x]/(x^{[\alpha, i]}))^{\times}\\
% & = \bigoplus_{(\alpha, 1)\in I_1}(1+x\mathbb{F}_q[x]/(x^{\left \lceil (2^n+1)/\alpha_1 \right \rceil}))^{\times} \\
% & \oplus \bigoplus_{(\alpha, 2)\in I_2}(1+x\mathbb{F}_q[x]/(x^{\left \lceil 2^n/\alpha_1 \right \rceil}))^{\times}. 
% \end{align*}
% 注意到$BigWitt_{k}(R)=(1+x R\llbracket x\rrbracket )^{\times}/(1+x^{k+1} R\llbracket x\rrbracket )^{\times} \cong (1+x R[x]/(x^{k+1}))^{\times}$, 
% 根据公式\ref{cor:BW}, 

% \begin{align*}
% NK_2(\mathbb{F}_q[C_{2^n}])\cong & \bigoplus_{(\alpha, 1)\in I_1}\bigoplus_{\scriptsize\substack{1\leq m\leq \left \lceil (2^n+1)/\alpha_1 \right \rceil-1 \\ gcd(m, 2)=1}}(\mathbb{Z}/2^{1+ \left \lfloor\log_2 \frac{\left \lceil (2^n+1)/\alpha_1 \right \rceil-1}{m}  \right \rfloor}\mathbb{Z})^f \\
% & \oplus \bigoplus_{(\alpha, 2)\in I_2}\bigoplus_{\scriptsize\substack{ 1 \leq m\leq \left \lceil 2^n/\alpha_1 \right \rceil-1 \\ gcd(m, 2)=1}}(\mathbb{Z}/2^{1+ \left \lfloor\log_2 \frac{\left \lceil 2^n/\alpha_1 \right \rceil-1}{m}  \right \rfloor}\mathbb{Z})^f. 
% \end{align*}

% 接下来我们证明对于任意$1\leq k\leq n$, $\mathbb{Z}/2^k \mathbb{Z}$都在$NK_2(\mathbb{F}_q[C_{p^n}])$出现无限多次

% \begin{lemma}
% \label{lem:log2}
% 	对于任意的$1\leq k < n$, $1+\left \lfloor \log_2(\frac{2^n-1}{2^k+1}) \right \rfloor = n-k$. 
% \end{lemma}
% \begin{proof}
% 	当$1\leq k < n$时, $2^k-1\geq 1 \geq \frac{1}{2^{n-k-1}}$, 即
% 	\[2^{n-1}-2^{n-k-1}\geq 1, \]
% 	上式等价于$2^n-1\geq 2^{n-k-1}(2^k+1)$, 且$2^n-1<2^{n-k}(2^k+1)$, 于是
% 	\[2^{n-k}> \frac{2^n-1}{2^k+1} \geq 2^{n-k-1}, \]
% 	取对数得$\left \lfloor \log_2(\frac{2^n-1}{2^k+1}) \right \rfloor = n-k-1$. 
% \end{proof}
% 考虑$((1, \alpha_2), 2)\in I_2$, 
% $$\bigoplus_{(\alpha, 2)\in I_2}\bigoplus_{\scriptsize\substack{1\leq m\leq  2^n-1 \\ gcd(m, 2)=1 }}(\mathbb{Z}/2^{1+ \left \lfloor\log_2 \frac{2^n-1}{m}  \right \rfloor}\mathbb{Z})^f$$
% 是$NK_2(\mathbb{F}_{2^f}[C_{2^n}])$的直和项, 当$m=1$时$1+ \left \lfloor\log_2 (2^n-1)\right \rfloor=n$, 当$m=2^k+1 (1\leq k < n)$为奇数时, 由\ref{lem:log2}, $1+ \left \lfloor\log_2 \frac{2^n-1}{m}\right \rfloor=n-k$, {\color{blue}于是对于任何的$1\leq k\leq n$, $\mathbb{Z}/2^k\mathbb{Z}$均出现在直和项中, 且对于任意$\alpha_2\geq 1$, 这样的项总会出现}, 于是
% \[NK_2(\mathbb{F}_q[C_{2^n}])\cong \bigoplus_\infty \bigoplus_{k=1}^n \mathbb{Z}/2^k\mathbb{Z}. \]

% 接下来给出一些$NK_2(\mathbb{F}_q[C_{2^n}])$中的$2^k(1\leq k \leq n)$阶元素. 

% 对任意$\alpha_2\geq 1, a\in \mathbb{F}_q$, 
%   \begin{align*}
%  \Gamma_{(1, \alpha_2), 2} \colon (1+x \mathbb{F}_q[x]/(x^{2^n}))^{\times} &\rightarrowtail K_2(A, M)\\
%  1+ax \text{($2^n$阶元)} &\mapsto \langle atx^{\alpha_2-1}, x \rangle, \\
%  1+ax^3 \text{($2^{n-1}$阶元)} &\mapsto \langle at^3x^{3\alpha_2-1}, x \rangle, \\
%  1+ax^{2^k+1} \text{($2^{n-k}$阶元)} &\mapsto \langle at^{2k+1}x^{(2k+1)\alpha_2-1}, x \rangle. 
%  \end{align*}






% 



\section{其他问题和说明}

$NK_2(\mathbb{F}_{p^m}[C_{p^n}])=?$

$\mathbb{F}_2[C_2\times C_2] \cong\mathbb{F}_2[C_2]\otimes\mathbb{F}_2[C_2] \cong \mathbb{F}_2[x, y]/(x^2, y^2)$, 可以用同样的方法得到一些结果. 



\[
	0\longrightarrow K_2(k[t_1, t_2, t_3]/(t_1^n, t_2^n), (t_1, t_2)) \longrightarrow K_2(k[t_1, t_2, t_3]/(t_1^n, t_2^n)) \longrightarrow K_2(k[t_3]) \longrightarrow 0
	\]
对于有限域$k$来讲$K_2(k[t_3])=0$, 
\[0\longrightarrow NK_2(\mathbb{F}_2[C_{2}\times C_2]) \longrightarrow K_2(\mathbb{F}_2[C_{2}\times C_2][x])\longrightarrow K_2(\mathbb{F}_2[C_{2}\times C_2]) \longrightarrow 0, \]
中间那项可以用这篇文章里的方法确定, 又$K_2(\mathbb{F}_2[C_{2}\times C_2])=C_2^3$, 于是可以得到$NK_2(\mathbb{F}_2[C_{2}\times C_2])$, 是$\oplus_{\infty} \mathbb{Z}/2 \mathbb{Z}$. 

另外可以直接用这种方式重新计算$K_2(\mathbb{F}_2[C_4\times C_4])$, 见下一篇笔记. 


一个关于模结构的问题, 在Weibel的文章\cite{MR88f:18018}中5. 5和5. 7给出的模结构和本文上面的模结构并不一致, 用$V_m$作用差一个$t^m$. 