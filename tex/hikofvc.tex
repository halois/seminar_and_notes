\chapter{Notes on Higher $K$-theory of group-rings of virtually infinite cyclic groups}
\section{Introduction}
作者Aderemi O. Kuku and Guoping Tang
\subsection{Preliminaries}
\begin{definition}[Virtually cyclic groups]
A discrete group $V$ is called virtually cyclic if it contains a cyclic subgroup of finite index, i.e., if $V$ is finite or virtually infinite cyclic.
\end{definition}
Virtually infinite cyclic groups are of two types:
\begin{itemize}
 	\item[1]  $V = G \rtimes_{\alpha} T$ is a semi-direct product where $G$ is a finite group, $T = \langle t \rangle$ an infinite cyclic group generated by $t$, $\alpha \in Aut(G)$, and the action of $T$ is given by $\alpha(g )= tgt^{-1}$ for all $g\in G$.
 	\item[2] $V$ is a non-trivial amalgam of finite groups and has the form $V =G_0 *_H	G_1$ where $[G_0: H ]=2=[ G_1 :H]$.

 \end{itemize}
  We denote by $\mathcal{VCYC}$ the family of virtually cyclic subgroups of $G$.

\begin{equation*}
\text{virtually cyclic groups }
\begin{cases}
\text{finite groups}\\
\text{virtually infinite cyclic groups}
\begin{cases}
\text{I.}V=G\rtimes_\alpha T,G\text{ is finite},T=\langle t\rangle \cong \mathbb{Z} \\
\text{II.}V=G_0\ast_H G_1,H \text{is finite},[G_i:H]=2
\end{cases}
\end{cases}
\end{equation*}

若$G$是有限群,$V$满足$1\to G\to V\to T\to 1$,则$V$是类型I, $V=G\rtimes_\alpha T$, $\alpha:T\rightarrow Aut(G), \alpha (t)(g)=tgt^{-1}$. $V$中的乘法\footnote{这里的$\alpha$和文章中有些差异}为
\[(g_1,t_1)(g_2,t_2)=(g\alpha(t_1)g_2,t_1t_2)=(g_1t_1g_2t_1^{-1},t_1t_2).\]

若$G$是有限群,$V$满足$ V\to D_{\infty}\to 1$,$D_{\infty}=\Z_2*\Z_2$, 则$V$是类型II。
\[
\xymatrix{
  H \ar[r] \ar[d] & G_0 \ar[d]  \\
  G_1 \ar[r]& G_0*_H G_1  }
\]
is a push-out square.
\begin{definition}[Orders]
Let $R$ be a Dedekind domain with quotient field $F$. %, and $V$ a finite dimensional $F$-space, a full $R$-lattice in $V$ is a finitely generated $R$-submodule $M$ in $V$ s.t. $F\cdot M =V$
%\[F\cdot M=\{\sum_{\text{finite sum}}\alpha_i m_i |\alpha_i \in F, m_i \in M\}.\]
An $R$-order in a $F$-algebra $\Sigma$ is a subring $\Lambda$ of $\Sigma$, having the same unity as $\Sigma$ and s.t. $R$ is contained in the center of $\Lambda$, $\Lambda$ is finitely generated $R$-module and $F\otimes_R \Lambda =\Sigma$.

A $\Lambda$-lattice in $\Sigma$ is a $\Lambda$-bisubmodule of $\Sigma$ which generates $\Sigma$ as a $F$-space.

A maximal $R$-order $\Gamma$ in $\Sigma$ is an order that is not contained in any other $R$-order in $\Sigma$.
\end{definition}
\begin{example}
We give some examples:
\begin{itemize}
    \item[1.] $G$ is a finite group, then $RG$ is an $R$-order in $FG$ when $ch(F) \nmid |G|$.
	\item[2.] $R$ is a maximal $R$-order in $F$.
	\item[3.] $M_n(R)$ is a maximal $R$-order in $M_n(F)$.
\end{itemize}

\end{example}
\begin{remark}
Any $R$-order $\Lambda$ is contained in at least one maximal $R$-order in $\Sigma$. Any semisimple $F$-algebra $\Sigma$ contains at least one maximal $R$-order. However, if $\Sigma$ is commutative, then $\Sigma$ contains a unique maximal order, namely, the integral closure of $R$ in $\Sigma$.
\end{remark}
\begin{theorem}
$R, F, \Lambda, \Sigma$ as above, Then $K_0(\Lambda), G_0(\Lambda)$ are finitely generated abelian groups.
\end{theorem}

\subsection{The Farrell-Jones  conjecture}
Let $G$ be a discrete group and $\mathcal{F}$ a family of subgroups of $G$ closed under
conjugation and taking subgroups, e.g., $\mathcal{VCYC}$.

Let $Or_{\mathcal{F}}(G):= \{G/H| H \in F\} $, $R$ any ring with identity.

There exists a “Davis -L\"{u}ck” functor
\[\mathbb{K}R : Or_{\mathcal{F}}(G) \longrightarrow Spectra \]
\[G/H\mapsto \mathbb{K}R(G/H)=K(RH)\]
where $K(RH)$ is the $K$-theory spectrum such that $\pi_n(K(RH)) = K_n(RH)$.

There exists a homology theory 
\[H_n(-,\mathbb{K}R):G-CWcomplexes\longrightarrow \Z-Mod\]
\[X \mapsto H_n(X,\mathbb{K}R)\] 
Let $E_\mathcal{F}(G)$ be a $G$-CW-complex which is a model for the classifying space of $\mathcal{F}$. 
Note that $E_\mathcal{F}(G)^H$ is homotopic to the one point space (i.e., contractible) if $H\in F$ and $E_\mathcal{F}(G)^H=\emptyset$ if $H\notin F$ and $E_\mathcal{F}(G)$ is unique up to homotopy.

There exists an assembly map
\[A_{R,\mathcal{F}}:H_n(E_\mathcal{F}(G),\mathbb{K}R ) \longrightarrow K_n(RG) .\]
The Farrell-Jones isomorphism conjecture says that $A_{R,\mathcal{VCYC}}:H_n(E_\mathcal{VCYC}(G),\mathbb{K}R ) \cong K_n(RG)$ is an isomorphism for all $n\in Z$. Note that $KR$ is the non-connective $K$-theory spectrum such that $\pi_n(KR)$ is Quillen's $K_n(R),n\geq 0,$ and $\pi_n(KR)$ is Bass's negetive $K_n(R)$, for $n \leq 0$.
\subsection{Notations}
\begin{itemize}
 	\item $F$: number field, i.e, $\Q\subset F$ is a finite field extension.
 	\item $R$: the ring of integers in $F$.
 	\item $\Sigma$: a semisimple $F$-algebra.
 	\item $\Lambda$: an $R$-order in $\Sigma$, $\alpha: \Lambda \rightarrow \Lambda$: an $R$-automorphism.
 	\item $\Gamma \in \{ \alpha\text{-invariant } R\text{-orders in } \Sigma \text{ containing } \Lambda\}$ is a maximal element.
 	\item $\mathrm{max}(\Gamma)=\{\text{two-sided maximal ideals in } \Gamma\}$.
 	\item $\mathrm{max}_{\alpha}(\Gamma)=\{\text{two-sided maximal $\alpha$-invariant ideals in } \Gamma\}$.
 	\item $\mathcal{C}$: exact category, $K_n(\mathcal{C})=\pi_{n+1}(BQ\mathcal{C}), n\geq 0$. If $A$ is a unital ring, $K_n(A)=K_n(\mathcal{P}(A)), n\geq 0$. When $A$ is noetherian, $G_n(A)=K_n(\mathcal{M}(A))$.
 	\item $T=\langle t\rangle$: infinite cyclic group $\cong \Z$, $T^r$: free abelian group of rank $r$.
 	\item $A_{\alpha}[T]=A_{\alpha}[t,t^{-1}]$: $\alpha$-twisted Laurent series ring, $A_{\alpha}[T]=A[T]=A[t,t^{-1}]$ additively and multiplication given by $(rt^i)(st^j)=r\alpha^i(s)t^{i+j}$.(注:这里和文章有些区别)
 	\item $A_{\alpha}[t]$: the subgroup of $A_{\alpha}[T]$ generated by $A$ and $t$, that is, $A_{\alpha}[t]$ is the twisted polynomial ring.
 	\item $NK_n(A,\alpha):=\mathrm{ker} (K_n(A_{\alpha} [t])\to K_n(A)),n\in \mathbb{Z}$ where the homomorphism is induced by the augmentation $\epsilon : A_{\alpha} [t]\to A$. If $\alpha = \mathrm{id}$, $NK_n(A,\mbox{id})=NK_n(A)=\mathrm{ker} (K_n(A[t])\to K_n(A))$.
 \end{itemize}

\subsection{已知结果}
Next, we focus on higher $K$-theory of virtually cyclic groups
\begin{theorem}[A. Kuku]
For all $n \geq 1$,$ K_n(\Lambda)$ and $G_n(\Lambda)$ are finitely generated Abelian groups and hence that for any finite group $G$, $K_n(RG)$ and $G_n
(RG)$ are finitely generated. 
\end{theorem}
见Kuku, A.O.:$K_n, SK_n$ of integral group-ring and orders. Contemporary Mathematics Part
I, 55, 333-338 (1986)
和Kuku, A.O.:$K$-theory of group-rings of finite groups over maximal orders in division algebras. J. Algebra 91, 18-31 (1984).

Using the fundamental theorem for $G$-theory,
\[G_n(\Lambda[t])=G_n(\Lambda)\]
\[G_n(\Lambda[t,t^{-1}])=G_n(\Lambda)\oplus G_{n-1}(\Lambda)\] 
one gets that:
\begin{corollary}
For all $n \geq 1$, if $C$ is a finitely generated free Abelian group or monoid, then $G_n (\Lambda[C])$ are also finitely generated.
\end{corollary}

\begin{remark}
However we can not draw the same conclusion for $K_n(\Lambda[C])$ since for a ring $A$, it is known that {\color{red} all the $ NK_n(A)$ are not finitely generated unless they are zero}. 见Weibel, C.A.: Mayer Vietoris sequences and module structures on $NK_*$, Algebraic $K$-theory, Evanston 1980 (Proc. Conf., Northwestern Univ., Evanston, Ill., 1980), pp. 466-493,Lecture Notes in Math., 854, Springer, Berlin, 1981的Proposition 4.1
\end{remark} 
\subsection{这篇文章的结果}
\paragraph{第1节} 
\begin{theorem}[1.1]
The set of all two-sided, $\alpha$-invariant, $\Gamma$-lattices in $\Sigma$ is a free Abelian group under multiplication and has $\mathrm{max}_\alpha(\Gamma) $ as a basis.
\end{theorem}
\begin{theorem}[1.6]
Let $R$ be the ring of integers in a number field $F$, $\Lambda$ any $R$-order in a semi-simple $F$-algebra $\Sigma$. If $\alpha : \Lambda \rightarrow \Lambda $ is an $R$-automorphism, then there
exists an $R$-order $\Gamma \subset \Sigma $ such that\\
(1) $\Lambda \subset \Gamma $,\\
(2) $\Gamma$ is $\alpha$-invariant, and\\
(3) $\Gamma$ is a (right) {\color{red}regular} ring. In fact, $\Gamma$  is a (right) hereditary ring.
\end{theorem}
后面证明中反复用了这里的$\Gamma$是一个正则环。这两个定理推广了Farrell和Jones在文章The Lower Algebraic $K$-Theory of Virtually Infinite Cyclic
Groups.$K$-Theory 9, 13-30 (1995)中的定理1.5和定理1.2
\begin{theorem}[Farrell-Jones文章中的定理1.5]
The  set  of all two-sided,  $\alpha$-invariant,  $A$-lattices in $\Q G$  is  a  free 
Abelian  group  under  multiplication and  has  $\mathrm{max}_{\alpha}( A ) $ as a basis. 
\end{theorem}

\begin{theorem}[Farrell-Jones文章中的定理1.2]
Given a  finite  group  $G$  and  an automorphism  $\alpha: G \rightarrow  G$, then there  exists  a  $\Z$-order  $A \subset \Q G$  such that \\
(1) $\Z G\subset A$, \\
(2) $A$  is  $\alpha$-invariant,  and \\
(3) $A$  is  a  (right) {\color{red}regular}  ring,  in fact,  $A$  is  a  (right) hereditary  ring. 
\end{theorem}
第一节的结论来源于Farrell和Jones在其文章中的结论,将$\Z$和$\Q$的陈述推广到数域$F$和代数整数环$R$上,并且把之前的群环$\Q G$推广为任何半单$F$代数$\Sigma$。
\paragraph{第2节}
定理2.1中的方法是讲过的,关键一步是证两个范畴是自然等价。(文中有笔误:718页第一行$mt^n$应为$xt^n$, 后面所谓$m_i$应为$x_i$,另有一处$Hom$所在的范畴不在$\mathcal{B}$, 应在$\mathcal{M}(A_\alpha[T])$)
\begin{theorem}[2.2]
Let $R$ be the ring of integers in a number field $F$, $\Lambda$ any $R$-order
in a semi-simple $F$ -algebra $\Sigma$, $\alpha$ an automorphism of $\Lambda$. Then

(a) For all $n \geq 0$\\
(i) $NK_n(\Lambda,\alpha)$ is $s$-torsion for some positive integer $s$. Hence the torsion free
rank of $K_n(\Lambda_\alpha[t])$ is the torsion free rank of $K_n(\Lambda)$ and is finite.

 If $n \geq 2$, then the torsion free rank of $K_n(\Lambda_\alpha[t])$ is equal to the torsion free rank of $K_n(\Sigma)$.\\
(ii) If $G$ is a finite group of order $r$, then $NK_n(RG, \alpha)$ is $r$-torsion, where $\alpha$ is the automorphism of RG induced by that of $G$.

对第一类virtually infinite cyclic groups的结论: \\
(b) Let $V = G \rtimes_\alpha T$ be the semi-direct product of a finite group $G$ of order $r$ with an infinite cyclic group $T =\langle t\rangle$ with respect to the automorphism $\alpha : G\longrightarrow G , g \mapsto tgt^{-1}$. Then\\
(i) $K_n(RV) = 0$ for all $ n< -1$.\\
(ii) The inclusion $RG \hookrightarrow RV$ induces an epimorphism $K_{-1}(RG) \twoheadrightarrow K_{-1}
(RV)$. Hence $K_{-1}(RV)$ is finitely generated Abelian group.\\
(iii) For all $ n \geq 0$, $G_n(RV)$ is a finitely generated Abelian group.\\
(iv) $NK_n(RV ) $ is $r$-torsion for all $n \geq 0$ .
\end{theorem}

\paragraph{第3节}
对第二类virtually infinite cyclic groups的结论:
\begin{theorem}[3.2]
If $R$ is regular, then $NK_n(R; R^\alpha,R^\beta) = 0 $ for all $n \in \Z$. If $R$ is
quasi-regular then $NK_n(R; R^\alpha,R^\beta) = 0 $ for all $n \leq 0$.
\end{theorem}
\begin{theorem}[3.3]
Let $V$ be a virtually infinite cylic group in the second class having
the form $V = G_0*_H G_1$
where the groups $G_i, i = 0 , 1, $ and $H$ are finite and $[G_i: H ] = 2 $. Then the Nil-groups $NK_n(\Z H ; \Z[G_0 - H ], \Z[G_1 - H ])$ defined by the triple $(\Z H ; \Z[G_0 - H ], \Z[G_1 - H ])$ are $|H|$-torsion when $n \geq 0$
and $0 $ when $ n \leq -1$.
\end{theorem}
\section{$K$-theory for the first type of virtually infinite cyclic groups}
我们首先回顾Farrell和Jones在文章中的做法:
\paragraph{原型}
$G$: finite group, $|G|=q$, $\Z G$ is a $\Z$-order in $\Q G$, then there exists a regular ring $A \subset \Q G$ which is a $\Z$-order, and we have\footnote{参考Reiner, I.: Maximal Orders中定理41.1: $n=|G|$, $\Gamma$ is an $R$-order in $FG$ containing $RG$, then $RG\subset \Gamma \subset n^{-1}RG$ when ch$(F)$$\nmid n$.} $qA\subset \Z G$. 

Hence, we have the  following Cartesian square 
\[
\xymatrix{
  \Z G \ar[r] \ar[d] & A \ar[d]  \\
  \Z G/qA \ar[r]& A/qA  }
\]
Since  $\alpha$  induces  automorphisms  of all  $4$  rings  in  this  square,  we  have  another cartesian square
\[
\xymatrix{
  \Z (G\rtimes_{\alpha} T) \ar[r] \ar[d] & A_\alpha[T]\ar[d]  \\
  (\Z G/qA)_\alpha[T] \ar[r]& (A/qA)_\alpha[T]  }
\]
于是可以分别得到Mayer-Vietoris正合序列。
\begin{definition}
A  ring $R$ is quasi-regular if it  contains a  two-sided  nilpotent ideal  $N$ 
such that $R/N $ is right  regular.
\end{definition}
重要的结论是
\begin{itemize}
	\item[Prop1.1] If $R$ is a (right) regular, $\alpha: R\longrightarrow R$ an automorphism, then $R_\alpha[t], R_\alpha[T]=R_\alpha[t,t^{-1}]$ are also (right) regular.
	\item[Prop1.4] $\Z G/qA , A/qA,(\Z G/qA)_\alpha[T] , (A/qA)_\alpha[T] $ are all quasi-regular\footnote{If $S$ is a quasi-regular ring, then $K_{-n}(S)=0$.(正确不?)}.
\end{itemize}
{\color{green}即得到的方块右上角是regular ring,下方是quasi-regular ring}。
于是得到$K_n(\Z (G\rtimes_{\alpha} T))=0, n\leq 2$ 且有$K_{-1}(\Z G)\twoheadrightarrow K_{-1}(\Z (G\rtimes_{\alpha} T))$是满射。

\paragraph{推广到数域$F$和代数整数环$R$}
$G$: finite group, $|G|=s$, $\Lambda=RG$ is a $R$-order in $\Sigma=FG$, then there exists a regular ring $\Gamma \subset \Sigma=FG$ which is a $R$-order, and we have $s\Gamma\subset RG$. 

Hence, we have the  following Cartesian square 
\[
\xymatrix{
  RG \ar[r] \ar[d] & \Gamma \ar[d]  \\
  RG/s\Gamma \ar[r]& \Gamma/s\Gamma  }
\]
Since  $\alpha$  induces  automorphisms  of all  $4$  rings  in  this  square,  we  have  another cartesian square
\[
\xymatrix{
  R (G\rtimes_{\alpha} T) \ar[r] \ar[d] & \Gamma_\alpha[T]\ar[d]  \\
  (RG/s\Gamma)_\alpha[T] \ar[r]& (\Gamma/s\Gamma)_\alpha[T]  }
\]
于是可以分别得到Mayer-Vietoris正合序列。

对应到这里来{\color{green} $\Gamma, \Gamma_\alpha[T]$是正则环},{\color{green}$RG/s\Gamma, \Gamma/s\Gamma, (RG/s\Gamma)_\alpha[T], (\Gamma/s\Gamma)_\alpha[T]$是quasi-regular rings}.

\paragraph{群环推广到半单代数}
考虑$\Lambda \subset \Gamma \subset \Sigma$分别是$R$-order, 正则环,半单$F$-代数,则存在正整数$s$使得$\Lambda\subset \Gamma \subset \Lambda(1/s)$, 令$q=s\Gamma$

Hence, we have the  following Cartesian square 
\[
\xymatrix{
  \Lambda \ar[r] \ar[d] & \Gamma \ar[d]  \\
  \Lambda/q \ar[r]& \Gamma/q  }
\]
Since  $\alpha$  induces  automorphisms  of all  $4$  rings  in  this  square,  we  have  another cartesian square
\[
\xymatrix{
  \Lambda_\alpha[t] \ar[r] \ar[d] & \Gamma_\alpha[t]\ar[d]  \\
  (\Lambda/q)_\alpha[t] \ar[r]& (\Gamma/q)_\alpha[t]  }
\]
写出MV序列后每项均$\otimes \Z[1/s]$仍然正合\footnote{文献[16]中是对素数$p$的陈述,对于一般的整数是否成立?},再分别取核得到Nil群的长正合列。

$\Gamma, \Gamma_\alpha[t]$ regular $\Longrightarrow$ $NK_n(\Gamma,\alpha)=0$.\\
$\Lambda/q, \Gamma/q, (\Lambda/q)_\alpha[t], (\Gamma/q)_\alpha[t]$ are all quasi-regular.

\begin{remark}
Farrell,Jones文章中四个环是quasi-regular的结论证明中用到了Artinian性质,从而可以推广到这篇文章所讨论的情形。

一些注记: 1.$A$: finite, $J(A)$: its Jacobson radical, why is $A/J(A)$ regular?因为是有限环
2.720页第四行的文献应为[16], 引用的结论为“$I$ is a nilpotent ideal in a $\Z/p^m\Z $-algebra $\Lambda$ with unit, then $K_*(\Lambda, I)$ is a $p$-group”, 这个结论对一般的正整数$s$成立。同样地在719页得到序列(III)时同样参考[16]里的结论以及在721页倒数第8行所引用的[16]Cor 3.3(d)中的$p$对任何正整数成立。

原文中“By [9] the torsion
free rank of $K_n(\Lambda)$ is finite and if $n \geq 2$ the torsion free rank of $K_n(\Sigma)$ is the
torsion free rank of $K_n (\Lambda)$ (see [12])”引用的参考文献为

[9] van der Kallen, W.: Generators and relations in algebraic $K$-theory, Proceedings of the International Congress of Mathematicians (Helsinki, 1978), pp. 305-310, Acad. Sci.Fennica, Helsinki, 1980

[12] Kuku, A.O.: Ranks of $K_n$ and $G_n$ of orders and group rings of finite groups over integers in number fields. J. Pure Appl. Algebra 138, 39-44 (1999)\\
但在[9]中并未找到相应结论。

另外文献[10]在网上未找到电子文档。

Open problem: What is the rank of $K_{-1}(RV)$ ?

\end{remark}


\section{Nil-groups for the second type of virtually infinite cyclic groups}
范畴$\mathcal{T}$:对象为${\bf R}=(R;B,C)$,其中$R$是环, $B,C$是$R$-双模,态射为$(\phi,f,g):(R,B,C)\to (S,D,E) $,其中$\phi:R\to S $是环同态,$f: B \otimes_R S \to D  $与$g: C \otimes_R S \to E $是$R$-$S$双模同态。

 \[\rho:\mathcal{T}\longrightarrow Rings\]
 \begin{equation*}
  \rho({\bf R})=R_\rho = \left(
                     \begin{array}{cc}
                       T_R(C\otimes_R B) & C\otimes_R T_R(B\otimes_R C) \\
                       B\otimes_R T_R(C\otimes_R B) & T_R(B\otimes_R C) \\
                     \end{array}
                   \right)
\end{equation*}
If $M$ is an $R$-module, then its tensor algebra $T(M)=R\oplus M \oplus (M\otimes_R M)\oplus (M\otimes_R M\otimes_R M)\oplus \cdots$.
\[\varepsilon:R_\rho \longrightarrow \begin{pmatrix} R & 0 \\ 0 &  R \end{pmatrix}\]
\[NK_n({\bf R}):=\mbox{ker}(K_n(R_\rho)\to K_n\begin{pmatrix} R & 0 \\ 0 &  R \end{pmatrix})\]

Let $V$ be a group in the second class of the form $V = G_0*_H G_0$ where the groups $G_i,i = 0 , 1, $and $H$ are finite and $[G_i: H ] = 2$. Considering $G_i - H$ as the right coset of $H$ in $G_i$ which is different from $H$, the free $\Z$-module $\Z[Gi - H ]$ with basis $G_i - H $nis a $\Z H $-bimodule which is isomorphic to $\Z H$ as a left $\Z H$-module, but the right action is twisted by an automorphism of $\Z H$ induced by an automorphism of $H$. Then the Waldhausen's Nil-groups are defined to be $NK_n(\Z H ; \Z[G_0 - H ], \Z[G_0 - H ])$ using the triple $(\Z H ; \Z[G_0 - H ], \Z[G_0 - H ])$. This inspires us to consider the following general case. Let $R$ be a ring with identity and $\alpha : R \longrightarrow R$ a ring auto-morphism. We denote by $R^\alpha$ the $R$-bimodule which is $R$ as a left $R$-module but with right multiplication given by $a \cdot r = a\alpha(r)$. For any automorphisms $\alpha$ and $\beta$ of $ R$ , we consider the triple ${\bf R} = (R; R^\alpha,R^\beta)$. We will prove that $\rho({\bf R})$ is in fact a twisted polynomial ring and this is important for later use.

\begin{theorem}[3.1]
Suppose that $\alpha$ and $\beta$ are automorphisms of $R$. For the triple ${\bf R} = (R; R^\alpha,R^\beta)$, let $R_\rho$ be the ring $\rho({\bf R})$, and let $\gamma$ be a ring automorphism of
$\begin{pmatrix} R & 0 \\ 0 &  R \end{pmatrix}$
defined by
\[\gamma:\begin{pmatrix} a & 0 \\ 0 &  b \end{pmatrix} \mapsto \begin{pmatrix} \beta(b) & 0 \\ 0 &  \alpha(a) \end{pmatrix}.\]
Then there is a ring isomorphism
\[\mu:R_{\rho} \longrightarrow \begin{pmatrix} R & 0 \\ 0 &  R \end{pmatrix}_{\gamma}[x] .\]
\end{theorem}
加法群同构是显然的,只需验证乘法同态。

利用这个结论将这种形式的Nil群转化为Farrell Nil群,利用已知的命题来证明结论。如正则环的$NK_n$为$0$, 拟正则环的$NK_n$当$n\leq 0$时为$0$.

当我们接下来研究$R=\Z H$, $h=|H|$时,取一个regular order $\Gamma$,我们有相应的4 triples,于是得到4个twisted polynomial rings $R_\rho, \Gamma_\rho; (R/h\Gamma)_\rho, (\Gamma/h\Gamma)_\rho$.

之前第二节的方块
\[
\xymatrix{
  RG \ar[r] \ar[d] & \Gamma \ar[d]  \\
  RG/s\Gamma \ar[r]& \Gamma/s\Gamma  }
\]
在这里(之前的$R, G, s$换成$\Z, H, h$)变成了(注意这里$R=\Z H$)
\[
\xymatrix{
  R=\Z H \ar[r] \ar[d] & \Gamma \ar[d]  \\
  \Z H/h\Gamma \ar[r]& \Gamma/h\Gamma  }
\]
从而有
\[
\xymatrix{
  {\begin{pmatrix} R & 0 \\ 0 &  R \end{pmatrix}} \ar[r] \ar[d] & {\begin{pmatrix} \Gamma & 0 \\ 0 &  \Gamma \end{pmatrix} }\ar[d]  \\
  {\begin{pmatrix} R/h\Gamma & 0 \\ 0 &  R/h\Gamma \end{pmatrix}} \ar[r] & {\begin{pmatrix} \Gamma/h\Gamma & 0 \\ 0 &  \Gamma/h\Gamma \end{pmatrix}  }
  }
\]
接着有
\[
\xymatrix{
  {\begin{pmatrix} R & 0 \\ 0 &  R \end{pmatrix}_{\gamma}[x] } \ar[r] \ar[d] & {\begin{pmatrix} \Gamma & 0 \\ 0 &  \Gamma \end{pmatrix}_{\gamma}[x] } \ar[d]  \\
  {\begin{pmatrix} R/h\Gamma & 0 \\ 0 &  R/h\Gamma \end{pmatrix}_{\gamma}[x]} \ar[r] & 
  {\begin{pmatrix} \Gamma/h\Gamma & 0 \\ 0 &  \Gamma/h\Gamma \end{pmatrix}_{\gamma}[x]}  
  }
\]
而这个方块恰好是
\[
\xymatrix{
  R_{\rho} \ar[r] \ar[d] & \Gamma_{\rho} \ar[d]  \\
  (R/h\Gamma)_{\rho} \ar[r]& (\Gamma/h\Gamma)_{\rho}  }
\]
证明中使用了$n\leq -1$时quasi-regular ring 的$NK_n$为$0$.
\begin{remark}
722页中间参考文献[3]未找到augmentation map。另外这里把$f,g$是双模同态在原文基础上进行了修改。

726页第8行“(2) and (3)” 应为“(3) and (4)”。
\end{remark}
