%!TEX root = ../main.tex
\chapter{Lower Bounds for the Order of $K_2(\mathbb{Z}G)$ and $Wh_2(G)$} % (fold)
\label{cha:lower_bounds_for_the_order_of_}
2016.3.19 阅读这篇文章\cite{Stein1976} 1976年发表在{\em{Math. Ann.}}。

基本假设:$p$: rational prime, $G$: elementary abelian $p$-group.

用的方法:Bloch; van der Kallen  $K_2$ of truncated polynomial rings

结论:
the $p$-rank of $K_2(\mathbb{Z}G)$\footnote{this is a finite group} grows expotentially with the rank of $G$.

$Wh_2(G)$: ``pseudo-isotopy'' group is nontrivial if $G$ has rank at least $2$.


这篇文章之前已知的结论(exact computations)
Dunwoody, $G$ cyclic of order 2 or 3, {\color{green} $K_2(\mathbb{Z}G)$ is an elementary abelian $2$-group of rank 2 if $G$ has order $2$ and of rank $1$ if $G$ has order $3$}. 两者都有$Wh_2(G)$平凡。


一些记号和基本结论
$R$ commutative ring, $A$ a subring of $R$.
$\Omega_{R/A}^1$ the module of K\"{a}hler differentials of $R$ considerd as an algebra over $A$ and $R^*$ will denote the group of units of $R$.

the $p$-rank of an abelian group $G$ is $\dim_{\mathbb{F}_p}(G\otimes_\mathbb{Z} \mathbb{F}_p)$.

\paragraph{elementary abelian $p$-groups} % (fold)
\label{par:elementary_}An elementary abelian $p$-group is an abelian group in which every nontrivial element has order $p$. The number $p$ must be prime, and the elementary abelian groups are a particular kind of $p$-group. The case where $p = 2$, i.e., an elementary abelian $2$-group, is sometimes called a Boolean group.

结构:Every elementary abelian $p$-group is a vector space over the prime field $\mathbb{F}_p$ with $p$ elements, and conversely every such vector space is an elementary abelian group.

By the classification of finitely generated abelian groups, or by the fact that every vector space has a basis, every finite elementary abelian group must be of the form $(\mathbb{Z}/p \mathbb{Z})^n$ for $n$ a non-negative integer (sometimes called the group's rank). Here, $C_p=\mathbb{Z}/p \mathbb{Z}$ denotes the cyclic group of order $p$.

In general, a (possibly infinite) elementary abelian $p$-group is a direct sum of cyclic groups of order $p$. (Note that in the finite case the direct product and direct sum coincide, but this is not so in the infinite case.)


% paragraph elementary_ (end)

\section{第一部分} % (fold)
\label{sec:第一部分}
环是$\mathbb{F}_q$ 有限域的情况。

先说结论

首先是一个奇素数的结论
\begin{prop}
\label{prop:podd}
	Let $q=p^f$ be odd and let $G$ be an elementary abelian $p$-group of rank $n$.
	Then $K_2(\mathbb{F}_qG)$ is an elementary $p$-group of rank $f(n-1)(p^n-1)$.
\end{prop}
接着是素数$2$的结论
\begin{prop}
	Let $q=2^f$ be odd and let $G$ be an elementary abelian $2$-group of rank $n$.
	Then $K_2(\mathbb{F}_qG)$ is an elementary $2$-group of rank $f(n-1)(2^n-1)$.
\end{prop}

结论实际上是可以统一的,但是方法有些区别,因此原文中分开表述。

我们引进方法时借鉴了van der Kallen的方法和记号

Let $R$ be a commutative ring. The abelian group  $TD(R)$  is the universal $R$-module having generators  $Da, Fa, a \in R$,  subject to the relations
\begin{align*}
D(ab) &= aDb + bDa,\\
D(a + b) &= Da + Db + F(ab),\\
F(a + b) &= Fa + Fb,\\
Fa  &= D(1  + a)- Da.
\end{align*}
There is a natural surjective homomorphism of $R$-modules
\[TD(R)\twoheadrightarrow \Omega^1_{R/\mathbb{Z}}\longrightarrow 1\]
whose kernel is the submodule of  $TD(R)$  generated by the  $Fa, a\in R$.  Relations
imply
\[
F(c^2a)=cFa
\]
($F(c^2a)=F(ca\cdot c)=D(ca+c)-D(ac)-D(c)=D(c(a+1))-D(ac)-D(c)=cD(a+1)-(a+1)D(c)-aD(c)-cD(a)-D(c)=cF(a)$, $0=F(0)=F(a-a)=F(a)+F(-a)$, $\Rightarrow F(a)=-F(a)=F(-a)$, $\Rightarrow F(2a)=0$)

for all  $a, c \in R$  see \cite{MR45:252}p. 1204. \\
Hence $F(2a)=2F(a)=0$, if $2$ is a unit of $R$, $F(a)=0$, then the kernel is trivial and $\Omega^1_{R/\mathbb{Z}}\cong  TD(R)$,

\[1\longrightarrow TD(R)\overset{\cong}\longrightarrow \Omega^1_{R/\mathbb{Z}}\longrightarrow 1.\]  
\begin{example}
	$R=\mathbb{Z}$, then the kernel of the above surjection is $\mathbb{Z}/2\mathbb{Z}$.\\
	If $R$ is a field of characteristic $\neq 2$, then $TD(R)\cong \Omega^1_{R/\mathbb{Z}}$.\\
	If $R$ is a perfect field, then $TD(R)\cong \Omega^1_{R/\mathbb{Z}}$.
\end{example}
\begin{definition}
	We define groups  $\Phi_i(R)$, $i\geq 2$, by the exact sequence

	\begin{equation}
	\label{exact:phi}
	1 \longrightarrow \Phi_i(R) \longrightarrow K_2 (R[x]/(x^i))  \longrightarrow K_2(R[x]/(x^{i-1})) \longrightarrow 1.
	\end{equation}

\end{definition}
This sequence is exact at the right as  $SK_1(R[x]/(x^i), (x^{i-1})/(x^i))=  1$ (cf. \cite{MR50:2304} Theorem 6.2 and \cite{MR40:2736}9.2, p. 267).
\subsection{Remarks} % (fold)
\label{subsec:remarks}我们把Bass书\cite{MR40:2736}中相关的结论(p. 267)放在这里以备今后查询和使用

A semi-local ring is a ring for which $R/rad(R)$ is a semisimple ring, where $rad(R)$ is the Jacobson radical of $R$. In commutative algebra, semi-local means ``finitely many maximal ideals'', for instance, all rational numbers $r/s$ with $s$ prime to $30$ form a semi-local ring, with maximal ideals generated respectively by $2,3$, and $5$. This is a PID, as a matter of fact, any semi-local Dedekind domain is a PID. And if $R$ is a commutative noetherian ring, the set of zero-divisors is a union of finitely many prime ideals (namely, the ``associated primes'' of $(0)$), thus its classical ring of quotients (obtained from $R$ by inverting all of its non zero-divisors) is a semi-local ring. See \cite{book:975816} p.174. and \cite{MR40:2736} p. 86.



In studying the stable structure of general linear groups in algebraic $K$-theory, Bass proved the following basic result (ca. 1964) on  the  unit
structure  of semilocal rings.
\begin{theorem}
	 If $R$ is a semi-local ring, then $R$ has stable range $1$,
in the sense that, whenever $Ra + Rb = R$, there exists $r \in R$ such that
$a + rb \in R^*$.
\end{theorem}
\begin{example}
Some classes of semi-local rings: left(right) artinian rings, finite direct products of local rings, matrix rings over local rings, module-finite algebras over commutative semi-local rings.\\
The quotient $\mathbb{Z}/m\mathbb{Z}$ is a semi-local ring. In particular, if $m$ is a prime power, then $\mathbb{Z}/m\mathbb{Z}$ is a local ring.\\
A finite direct sum of fields $\bigoplus_{i=1}^n{F_i}$ is a semi-local ring.
In the case of commutative rings with unit, this example is prototypical in the following sense: the Chinese remainder theorem shows that for a semi-local commutative ring $R$ with unit and maximal ideals $m_1, \cdots, m_n$
\[R/\bigcap_{i=1}^n m_i\cong\bigoplus_{i=1}^n R/m_i\]
(The map is the natural projection). The right hand side is a direct sum of fields. Here we note that $\cap_i m_i=rad(R)$, and we see that $R/rad(R)$ is indeed a semisimple ring.\\
The classical ring of quotients for any commutative Noetherian ring is a semilocal ring.\\
The endomorphism ring of an Artinian module is a semilocal ring.\\
Semi-local rings occur for example in commutative algebra when a (commutative) ring R is localized with respect to the multiplicatively closed subset $S = \cap (R - p_i)$, where the $p_i$ are finitely many prime ideals.
\end{example}

\begin{theorem}
	Let $I$ be a two-sided ideal in a ring $R$. Assume either that $R$ is semi-local or that $I\subset rad(R)$. Then 
	\[GL_1(R,I)\longrightarrow K_1(R,I)\]
	is surjective, and, for all $m\geq 2$,
	\[GL_m(R,I)/E_m(R,I)\longrightarrow K_1(R,I)\]
	is an isomorphism. Moreover $[GL_m(R),GL_m(R,I)]\subset E_m(R,I)$, with equality for $m\geq 3$.
\end{theorem}
\begin{corollary}
	Suppose that $R$ above is commutative, then $E_n(R,I)\iso SL_n(R,I)$ is an isomorphism for all $n\geq 1$, and $SK_1(R,I)=0$.
\end{corollary}
\begin{proof}
	The determinant induces the inverse, 
	\[\det \colon K_1(R,I)\longrightarrow GL_1(R,I).\]
	In particular, if $\alpha\in GL_n(R,I)$ and $\det(\alpha)=1$ then $\alpha \in E_n(R,I)$, i.e. $SL_n(R,I)\subset E_n(R,I)$. The opposite inclusion is trivial. Finally $SK_1(R,I)=SL(R,I)/E(R,I)=0$.
\end{proof}


还有一个小插曲,当$k$是域时,$k[x]/(x^m)$是局部环的证明
\begin{prop}
	 Let $I$ be an ideal in the ring $R$.\\
a) If $rad(I)$ is maximal, then $R/I$ is a local ring.\\
b) In particular, if $m$ is a maximal ideal and $n \in \mathbb{Z}^+$ then $R/m^n$ is a local ring.
\end{prop}
\begin{proof}
a) We know that $rad(I)=\bigcap_{P\supset I}P$, so if $rad(I) = m$ is maximal it must be the only prime ideal containing I. Therefore, by correspondence $R/I$ is a local ring. (In fact it is a ring with a unique prime ideal.)\\
b)$rad(m^n) = rad(m) = m$, so part a) applies. 
\end{proof}

\begin{example}
	For instance, for any prime number $p$, $\mathbb{Z}/(p^k)$ is a local ring, whose maximal ideal is generated by $p$. It is easy to see (using the Chinese Remainder Theorem) that conversely, if $\mathbb{Z}/(n)$ is a local ring then $n$ is a prime power. 

	The ring $\mathbb{Z}_p$ of $p$-adic integers is a local ring. For any field $k$, the ring $k\llbracket t\rrbracket $ of formal power series with coefficients in $k$ is a local ring. Both of these rings are also PIDs. A ring which is a local PID is called a discrete valuation ring. Note that a local ring is connected, i.e., $e^2 = e \Rightarrow e \in \{0,1\}$.

令$R$是$k[x]$, $I$是$(x^m)$, 有$rad(x^m)=(x)$是极大理想(由于$0\rightarrow (x) \rightarrow k[x]\rightarrow k \rightarrow 0$正合),从而$k[x]/(x^i)$是局部环。
\end{example}
Remarks到此结束
% subsection remarks (end)
\subsection{Theorem} % (fold)
\label{subsec:theorem}
The first part of the following theorem is due to van der Kallen \cite{MR45:252} and the
second to Bloch \cite{MR81j:14011}.
\begin{theorem}
	Let $R$ be a commutative ring. Then\\
(1) $\Phi_2(R) \cong  TD(R)$;\\
(2) If $R$ is a local $\mathbb{F}_p$-algebra and $p$ is odd prime, then
\[\Phi_i(R)\cong \begin{cases}
	\Omega^1_{R/\mathbb{Z}}, i \not\equiv 0,1 \bmod p\\
	\Omega^1_{R/\mathbb{Z}}\oplus R/R^{p^r}, i=mp^r, (p,m)=1.
\end{cases}\]
\end{theorem}
当$p$是odd prime时,这一定理(2)可应用于$\mathbb{F}_p[C_p]$,因为$\mathbb{F}_p[C_p]\cong \mathbb{F}_p[t]/(t^p)$
\begin{lemma}
	Let $q=p^f$ and let $H$ be a finite abelian $p$-group. Then $\Omega^1_{\mathbb{F}_q H/\mathbb{Z}}$is a free $\mathbb{F}_q H$-module of rank equal to the $p$-rank of H.
\end{lemma}
\begin{proof}
	In terms of polynomials, we have
	\[\mathbb{F}_q H\cong \mathbb{F}_q[x_1,\cdots,x_n]/I\]
where $n$ is the $p$-rank of $H$ and $I$ is the ideal of $\mathbb{F}_q[x_1,\cdots,x_n]$ generated by polynomials of the form  $F_i=x_i^{q_i}-1$ where $q_i$ is a power of $p$. By [BoreI,A.: Linear algebraic groups. New York: W. A. Benjamin 1969, p. 61],  $\Omega^1_{\mathbb{F}_qH/\mathbb{Z}}$
is the $\mathbb{F}_qH$-module with generators  $dx_1,\cdots,dx_n$ subject to the relations
\[\sum_i \frac{d F_i}{d x_i}d x_i =0.\]
Since the ring has characteristic $p$, the relations are trivial and the module is free. 
As $\mathbb{F}_q$ is perfect, its module of differentials is trivial. Hence $\Omega^1_{\mathbb{F}_q H/\mathbb{F}_q}=\Omega^1_{\mathbb{F}_q H/\mathbb{Z}}$,
yielding the result.
\end{proof}

由这个引理得到了\ref{prop:podd}. 

下面是节选一些可能用到的陈述。
\begin{itemize}
	\item  {\color{green} $\mathbb{F}_q G$ is a local ring}, where $G$ is an elementary abelian $p$-group, for example $G=(\mathbb{Z}/p \mathbb{Z})^n$.
\end{itemize}


对odd prime的证明如下
\begin{proof}
	We begin by showing that  $K_2(\mathbb{F}_qG)$  is an elementary abelian $p$-group even in
case $p=2$. As {\color{green} $\mathbb{F}_q G$ is a local ring}, it follows that  $K_2(\mathbb{F}_qG)$  is
generated by the Steinberg symbols $\{u, v\}$, $u, v \in \mathbb{F}_q G^*$. Now $u^p, v^P \in \mathbb{F}^*$ as $G$
is an elementary abelian $p$-group ($p$次后$G$中的元就变成单位元了). Choose $w\in \mathbb{F}_q^*$ so that  $w^p= u^p$.(这里注意之前的$u$是群环里的,这里的$w$取在域里)  Then
\begin{align*}
\{u, v\}^p &= \{u^p, v\}\\
&= \{w^p, v\}\\
&= \{w, v^p\}.
\end{align*}
Thus $\{w, v^p\}$ is trivial as it lies in the image of $K_2(\mathbb{F}_q)= 1$(有限域的$K_2$是平凡的,并且这个符号是在$K_2$中). Hence  $K_2(\mathbb{F}_qG)$  has exponent $p$.

Let $H$ be generated by $x_1 ,\cdots, x_{n-1}$ where $x_1 ,\cdots, x_n$ are independent generators of $G$. Then (由于特征是$p$才有下面的最后一步,对于$\mathbb{Z}$是不对的)
\[\mathbb{F}_q G = \mathbb{F}_qH [x_n] /(x_n^p-  1) \cong \mathbb{F}_q H[x]/(x^p).\]

Exact sequence \ref{exact:phi} together with Theorem yield
\begin{align*}
\rank K_2 (\mathbb{F}_q G) & = \rank  K_2(\mathbb{F}_q H) + (p-  1) \rank \Omega^1_{\mathbb{F}_q H/\mathbb{Z}}+  \rank \mathbb{F}_qH/\mathbb{F}_q \\
& = \rank K_2(\mathbb{F}_q H)+f(p-1)(n-1)p^{n-1}  +f(p^{n- 1}- 1)
\end{align*}

and the result follows by induction.

上面的结论我们详细写出来是
\begin{align*}
1 \longrightarrow {\color{blue}{\color{blue} \Phi_p(\mathbb{F}_q H)}} \longrightarrow   &{\color{green}K_2( \mathbb{F}_q G)=K_2 (\mathbb{F}_q H[x]/(x^p))}  \longrightarrow K_2(\mathbb{F}_q H[x]/(x^{p-1})) \longrightarrow 1, \\
1 \longrightarrow {\color{red} \Phi_{p-1}(\mathbb{F}_q H)} \longrightarrow  &K_2 (\mathbb{F}_q H[x]/(x^{p-1}))  \longrightarrow K_2(\mathbb{F}_q H[x]/(x^{p-2})) \longrightarrow 1, \\
& \cdots \\
1 \longrightarrow {\color{red} \Phi_2(\mathbb{F}_q H)} \longrightarrow & K_2 (\mathbb{F}_q H[x]/(x^2))  \longrightarrow {\color{yellow} K_2(\mathbb{F}_q H[x]/(x))} \longrightarrow 1. \\
\end{align*}
Note that $\mathbb{F}_q H[x]/(x)=\mathbb{F}_q H$, $G=(\mathbb{Z}/p \mathbb{Z})^n$, $H=(\mathbb{Z}/p \mathbb{Z})^{n-1}$
then
\begin{align*}
\rank {\color{green}K_2 (\mathbb{F}_q G)} & = {\color{blue}\rank  \Phi_p(\mathbb{F}_q H)} + \rank K_2(\mathbb{F}_q H[x]/(x^{p-1})) \\
&= {\color{blue}\rank  \Phi_p(\mathbb{F}_q H)} + {\color{red} \rank \Phi_{p-1}(\mathbb{F}_q H) +\cdots +\rank \Phi_2(\mathbb{F}_q H)} +{\color{yellow} \rank K_2(\mathbb{F}_q H)}  \\
& ={\color{blue}\rank  \Omega^1_{\mathbb{F}_q H/\mathbb{Z}}} + {\color{blue}\rank  \mathbb{F}_q H/\mathbb{F}_q} + {\color{red} (p-2) \rank \Omega^1_{\mathbb{F}_q H/\mathbb{Z}}} + {\color{yellow} \rank K_2(\mathbb{F}_q H)}  \\
& = (p-1)\rank \Omega^1_{\mathbb{F}_q H/\mathbb{Z}} + \rank \mathbb{F}_q H/\mathbb{F}_q+ {\color{yellow} \rank K_2(\mathbb{F}_q H)}  \\
& = f(p-1)(n-1)p^{n-1} + f(p^{n-1}-1) +\rank  {\color{yellow} K_2(\mathbb{F}_q H)}
\end{align*}
since 
\[{\color{blue} \Phi_p(\mathbb{F}_q H)} = \Omega^1_{\mathbb{F}_q H/\mathbb{Z}} \oplus \mathbb{F}_q H/\mathbb{F}_q H^p,\]
\[ {\color{red}\Phi_i(\mathbb{F}_q H)} = \Omega^1_{\mathbb{F}_q H/\mathbb{Z}}= (\mathbb{F}_q H)^{n-1}, 2\leq i \leq p-1,\]
\[\mathbb{F}_q H/\mathbb{F}_q H^p= \mathbb{F}_q H/\mathbb{F}_q\] 
$\mathbb{F} H$是以$H$中元素为基的自由$F$模
并且
\[\rank \Omega^1_{\mathbb{F}_q H/\mathbb{Z}}= \rank (\mathbb{F}_{p^f} H)^{n-1} = (n-1)f|H|=(n-1)fp^{n-1}\]
\[\rank \mathbb{F}_q H/\mathbb{F}_q = \rank \mathbb{F}_q H- \rank \mathbb{F}_q = f(p^{n-1}-1).\]
接下来是归纳计算,首先我们看它截至到哪一步:最后一步应该是$\mathbb{F}_q[(\mathbb{Z}/p \mathbb{Z})^2]$,因为$K_2(\mathbb{F}_q[\mathbb{Z}/p \mathbb{Z}])=0$, 这时有
\begin{align*}
\rank K_2(\mathbb{F}_q[(\mathbb{Z}/p \mathbb{Z})^2])&=\rank K_2(\mathbb{F}_q[\mathbb{Z}/p \mathbb{Z}])+ (p- 1) \rank \Omega^1_{\mathbb{F}_q [\mathbb{Z}/p \mathbb{Z}]/\mathbb{Z}}+  \rank \mathbb{F}_q[\mathbb{Z}/p \mathbb{Z}]/\mathbb{F}_q\\
& = 0+f(p-1)(2-1)p^{2-1} +f(p^{2-1}-1)
\end{align*}

从而我们知道
\begin{align*}
\rank K_2 (\mathbb{F}_q G) & = f(p-1)(n-1)p^{n-1}  +f(p^{n- 1}-1) + \cdots + f(p-1)p^{1}  +f(p^{1}-1) \\
& = \sum_{i=1}^{n-1}(f(p-1)ip^i  +f(p^i-1))\\
& = -f \frac{p-p^n}{1-p}+f(n-1)p^n +f \frac{p-p^n}{1-p} -(n-1)f\\
&= f(n-1)(p^n-1)
\end{align*}
这里的计算用到等比数列求和,记$S=\sum_{i=1}^{n-1}ip^i$
\[pS=\sum_{i=1}^{n-1}ip^{i+1}=\sum_{i=2}^{n}(i-1)p^{i}\]
\[S-pS=\sum_{i=1}^{n-1} p^i - (n-1)p^n\]
因此
\[S=\frac{p-p^n}{(1-p)^2}-\frac{(n-1)p^n}{(1-p)}\]
\begin{align*}
\sum_{i=1}^{n-1}(f(p-1)ip^i  +f(p^i-1)) &= f(p-1)S+f \frac{p-p^n}{1-p}-(n-1)f \\
& = f(p-1)(\frac{p-p^n}{(1-p)^2}-\frac{(n-1)p^n}{(1-p)})+f \frac{p-p^n}{1-p}-(n-1)f \\
& = -f \frac{p-p^n}{1-p}+f(n-1)p^n +f \frac{p-p^n}{1-p} -(n-1)f\\
&= f(n-1)(p^n-1)
\end{align*}

\end{proof}
In case $p = 2$ the details become more complicated.(暂且略过这个情形)


% subsection theorem (end)





% section 第一部分 (end)
\section{第二部分} % (fold)
\label{sec:第二部分}
第二部分是考了系数环是$\mathbb{Z}$的情形,如何将上面的有限域和这里的整数环联系起来,就是用了一个相对$K$群的正合列。

We now exploit these computations of $K_2(\mathbb{F}_q G)$ to obtain lower bounds for $K_2(\mathbb{Z} G)$ and  $Wh_2(G)$.  There is an exact sequence
\begin{equation}
\label{exact:sk}
K_2(\mathbb{Z}G)\longrightarrow K_2(\mathbb{F}_p G)\longrightarrow SK_1(\mathbb{Z}G, p\mathbb{Z}G)\longrightarrow  SK_1(\mathbb{Z}G)\longrightarrow  1
\end{equation}

This sequence is exact on the right because $\mathbb{F}_p G$ is a local ring, which implies $SK_1(\mathbb{F}_p G)  = 1$ \cite{MR40:2736}, p. 267.

\begin{theorem}
(1)	Let $G$ be an elementary abelian $2$-group of rank $n$. Then  $K_2(\mathbb{Z} G)$  has $2$-rank at least $(n-1)2^n+2$ and $Wh_2(G)$ has $2$-rank at least
$(n-1)2^n- \frac{(n+2)(n-1)}{2}$. In particular, $Wh_2(G)$ is non-trivial if $n \geq 2$.\\
(2) Let $p$ be an odd prime and let $G$ be an elementary abelian $p$-group of rank $n$. Then  $K_2(\mathbb{Z} G)$  has $p$-rank at least $(n-1)(p^n-1)-\binom{p+n-1}{p}$ and $Wh_2(G)$ has $p$-rank at least
$(n-1)(p^n-1)-\binom{p+n-1}{p}-\frac{n(n-1)}{2}$. In particular, $Wh_2(G)$ is non-trivial if $n \geq 2$.
\end{theorem}
\begin{proof}
	(1) Since  $K_1(\mathbb{Z}G, 2\mathbb{Z}G)\longrightarrow  K_1(\mathbb{Z}G)$ is injective [Keating, M.E.: On the K-theory of the quaternion group. Mathematika 20, 59--62 (1973), Remark 2.4], we see that
$K_2(\mathbb{Z}G)\longrightarrow K_2(\mathbb{F}_2 G)$  is surjective. 

If $g_1, \cdots, g_n$, are the generators of $G$, then the $n + 1$
symbols $\{- 1, - 1 \}, \{ - 1, g_1 \}, \cdots, \{ - 1, g_n\}$ are independent [\cite{MR50:2304}, p. 65] and lie in the kernel of this map. Hence 
\[\rank K_2(\mathbb{Z} G) \geq (n-1)(2^n-1)+(n+1) =(n-1)2^n +2. \]
Recall that for $G$ abelian, $Wh_2(G)$ is the quotient of $K_2(\mathbb{Z}G)$ by the subgroup generated by all symbols of the form $\{\sigma, \tau\}$, $\sigma,\tau \in \pm G$ [Hatcher, A.E.: Pseudo-isotopy and $K_2$,  pp. 328-336. Lecture Notes in Mathematics 342. Berlin, Heidelberg, New York: Springer 1973]. It is easy to see from the bimuttiplicative and anti-symmetric properties of symbols that this subgroup has rank at most $\binom{n+1}{2}+1$. Moreover, by using the various maps $\mathbb{Z}G\longrightarrow \mathbb{Z}$ which send elements of $G$ to $\pm 1$, it can be shown that the rank of this subgroup is precisely $\binom{n+1}{2}+1$. $(n-1)2^n+2-\binom{n+1}{2}-1=(n-1)2^n- \frac{(n+2)(n-1)}{2}$.

(2) {\color{red}以下这一段没有完全读懂。}Let $B$ be the integral chosure of $\mathbb{Z}G$ in $\mathbb{Q}G$. Then $SK_1(B, p^{n+1}B)$ has $p$-rank $\frac{p^n-1}{p-1}$ [Bass, H., Milnor, J., Serre, J. P.: Solution of the congruence subgroup problem for  $SL_n$($n \geq  3$) and
$Sp_{2n}$($n\geq 2$).  Publ. Math. IHES 33, 59--137 (1967), Corollary 4.3, p. 95].

But $SK_1(B, p^{n+1} B) \cong SK_1 (\mathbb{Z}G, p^{n+1} B)$ [\cite{MR40:2736}, p. 484]
since $p^nB$ lies in the conductor of $B$ over $\mathbb{Z}G$, and $SK_1(\mathbb{Z}G, p^{n+1}B)$ maps onto $SK_1(\mathbb{Z}G, p\mathbb{Z}G)$ [\cite{MR40:2736}, 9.3, p. 267]. Hence $p$-rank $SK_1(\mathbb{Z}G, p\mathbb{Z}G)\leq \frac{p^n-1}{p-1}$. The $p$-rank of $SK_1(\mathbb{Z}G)$ is $\frac{p^n-1}{p-1} - \binom{p+n-1}{p}$ [Alperin, R.C., Dennis, R. K., Stein, M. R.: The non-triviality of  $SK_1(\mathbb{Z}\pi)$, pp. 1-7. Lecture Notes in Mathematics 353. Berlin, Heidelberg, New York: Springer t973, Theorem 2]. The result now follows from exact sequence \ref{exact:sk}.

And noting that the subgroup generated by the symbols
$\{\sigma, \tau\}$, $\sigma,\tau \in \pm G$ has $p$-rank at most $\frac{n(n-1)}{2}$.
\end{proof}

\begin{remark}
	The  subgroup of  $K_2(\mathbb{Z} G)$  generated by elements of the form  $ \langle a,b \rangle$,
$1 +ab \in (\mathbb{Z}G)^* $ maps onto $K_2(\mathbb{F}_2 G)$ for $G$ an elementary abelian $2$-group of
rank $\leq 2$. W. van der Kallen has shown that this subgroup maps onto in general. This follows from the rank $2$ case via

Lemma  (van der Kallen).  Let $I$ be a nilpotent ideal of the commutative ring $R$. Let $v_i \in R$ additively generate $R/I$ and let $w_j\in I$ additively generate $I$. Then $K_2(I)=\ker(K_2(R)\longrightarrow K_2(R/I))$ is generated by all elements of the form $\langle v_i , w_j \rangle$ and $\langle w_j, w_i^{2^k-1}w_j \rangle$.
\end{remark}




我的一些问题:$NK_2(\mathbb{F}_q G)$如何算,$NK_1(\mathbb{Z}G,p \mathbb{Z}G)=?$, 最简单的可以考虑$NK_2(\mathbb{F}_p C_p)$, 接着是$NK_2(\mathbb{F}_{p^2} C_p)$.


% section 第二部分 (end)
\section{推广和其它} % (fold)
\label{sub:推广和其它}
之前考虑的是$\mathbb{Z}G$, $G$ elementary. 可以推广到$G$ finite group, $\mathcal{O}$ be the ring of integers of an algebraic number field.

If $S$ is a Sylow $p$-subgroup of $G$, then $\mathcal{O}G$ is a free module over $\mathcal{O}S$ and the composition
\[K_2(\mathcal{O}S)\longrightarrow K_2(\mathcal{O}G)\longrightarrow K_2(\mathcal{O}S)\]
(where the second map is the transfer) is multiplication by $(G:S)$. Hence
$p$-rank  $K_2(\mathcal{O}G) \geq$ $p$-rank $K_2(\mathcal{O}S)$  and estimates may be obtained by restricting to the case of a $p$-group. 
\begin{theorem}
	 Let $\mathcal{O}$ be the ring of integers in an algebraic number field which is
Galois over $\mathbb{Q}$ and let $G$ be an elementary abelian $p$-group of rank $n$. If $p$ is unramfied in $\mathcal{O}$ with each residue field having degree $f$ over $\mathbb{F}_p$, then $K_2 (\mathcal{O}G)$ has $p$-rank at least\\
(i)  $f(n - 1) (2^n - 1)$ if $p = 2$  and $\mathcal{O}$ has a real embedding,\\
(ii)  $f(n-1)(2^n- l)-\binom{n+1}{2}$ if $p=2$ and $\mathcal{O}$ is totally imaginary,\\
(iii)  $f(n-1)(p^n-l)-  \binom{p+n-1}{p}$ if $p$ is odd.
\end{theorem}

\paragraph{abelian $p$-groups which are not elementary} % (fold)
\label{par:abelian_}
有以下一个结论
\begin{prop}
	Let $p$ be an odd prime and suppose $G=H \times C$ where $C$ is cyclic of
order $p^t$,  $|H|  =p^k$  and $s=p\mbox{-}\rank H$. Let $\mathcal{O}$ be the ring of integers in a number field. Choose a prime $\mathfrak{p}$ of $\mathcal{O}$ lying over $p$ and having residue degree $f$ over $\mathbb{F}_p$. Then
\begin{align*}
&\ord_p |K_2(\mathcal{O}G/\mathfrak{p}G)|-\ord_p |K_2(\mathcal{O}H/\mathfrak{p}H)|\\
\geq & f\Big(p^k\big(s(p-1)p^{t-1} + 1\big)-|H^{p^t}|\Big) +p^k(p^{t-1}-1)- (p-1) \sum_{r=1}^{t-1}|H^{p^r}|p^{t-r-1}.
\end{align*}


\end{prop}
% paragraph abelian_ (end)
% section 推广和其它 (end)

% cha lower_bounds_for_the_order_of_ (end)