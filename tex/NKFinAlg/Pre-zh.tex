%!TEX root = ../../testmain.tex
%  复制于原来给老师看时做的笔记的部分,和最早的部分有一些小修改
%  未来直接两边都引用这个部分即可
令$k$是特征为$p>0$的有限域, 考虑两个变元的多项式环$k[t_1, t_2]$, $I=(t_1^n)$是$k[t_1, t_2]$的一个真理想,
% , 满足以下条件
% \begin{enumerate}
% 	\item $I$是由$k[t_1]$中的单项式生成的, 
% 	\item 对于某个$n$, $t_1^n\in I$. 
% \end{enumerate}
% 实际上这样的$I$具有形式$(t_1^n)$, 
令
\[A=k[t_1, t_2]/I, \]
设$M$是$A$的nil根(小根), 即$M=(t_1)$, 则有$A/M=k[t_2]$. 
\begin{prop}
	$K_2(k[t_1, t_2]/(t_1^n), (t_1, t_2))\cong K_2(A, M)=K_2(k[t_1, t_2]/(t_1^n), (t_1))$. 
\end{prop}
% 这个property参考Kallen文章P278
\begin{proof}
	由于$k[t_2]\overset{i_1}\hookrightarrow k[t_1,t_2]/(t_1^n)$, $k[t_1,t_2]/(t_1^n)\overset{p_1}\twoheadrightarrow k[t_2]$与$k\overset{i_2}\hookrightarrow k[t_1,t_2]/(t_1^n)$, $k[t_1,t_2]/(t_1^n)\overset{p_2}\twoheadrightarrow k$满足$p_1i_1=\id$, $p_2i_2=\id$, 故由$K_n$的函子性有$K_n(p_1)K_n(i_1)=\id$与$K_n(p_2)K_n(i_2)=\id$, 从而有以下相对$K$群的可裂正合列
	\[
	0\longrightarrow K_2(k[t_1, t_2]/(t_1^n), (t_1)) \longrightarrow K_2(k[t_1, t_2]/(t_1^n)) \longrightarrow K_2(k[t_2]) \longrightarrow 0
	\]
	\[
	0\longrightarrow K_2(k[t_1, t_2]/(t_1^n), (t_1, t_2)) \longrightarrow K_2(k[t_1, t_2]/(t_1^n)) \longrightarrow K_2(k) \longrightarrow 0
	\]

	由于$k$是有限域, 故$K_2(k)=0$, 又由于有限域都是正则环, 故$NK_2(k)=0$, 于是$K_2(k[t_2])=K_2(k)\oplus NK_2(k)=0$. 从而得
	\[K_2(k[t_1, t_2]/(t_1^n), (t_1))\cong K_2(k[t_1, t_2]/(t_1^n)) \cong K_2(k[t_1, t_2]/(t_1^n), (t_1, t_2)). \]
\end{proof}

当$k=\mathbb{F}_{p^f}$时, $k[t_1]/(t_1^{p^n})\cong \mathbb{F}_{p^f}[C_{p^n}]$, 其中$C_{p^n}$是$p^n$阶循环群. 有以下可裂正合列
\[0\longrightarrow NK_2(\mathbb{F}_{p^f}[C_{p^n}]) \longrightarrow K_2(\mathbb{F}_{p^f}[C_{p^n}][x])\longrightarrow K_2(\mathbb{F}_{p^f}[C_{p^n}]) \longrightarrow 0, \]
由于$K_2(\mathbb{F}_{p^f}[C_{p^n}][x])\cong K_2(\mathbb{F}_{p^f}[t_1, t_2]/(t_1^{p^n}))$, 并且$K_2(\mathbb{F}_{p^f}[C_{p^n}])=0$\cite{MORRIS198091}, 从而
\[NK_2(\mathbb{F}_{p^f}[C_{p^n}])\cong K_2(\mathbb{F}_{p^f}[t_1, t_2]/(t_1^{p^n})) \cong K_2(\mathbb{F}_{p^f}[t_1, t_2]/(t_1^{p^n}), (t_1)). \]


\subsection{Dennis-Stein符号}
\label{sub:dennis_stein_symbols}
一般地, 通过Dennis-Stein符号可以给出$K_2(A, M)=K_2(k[t_1, t_2]/(t_1^n), (t_1))$的一个表现
\begin{itemize}
	\item[生成元]   $\langle a, b \rangle $, 其中$(a, b)\in A\times M \cup M \times A$;
	\item[关系] (DS1) $\langle a, b\rangle = -\langle b, a \rangle$, \\
				(DS2) $\langle a, b\rangle +\langle c, b \rangle=\langle a+c-abc, b\rangle$, \\
	 			(DS3) $\langle a, bc\rangle =\langle ab, c\rangle +\langle ac, b\rangle$, 其中$(a, b, c)\in  M\times A\times A \cup A\times M \times A \cup A\times A\times M$. 
\end{itemize}

\begin{prop}
	对任意环$R$, 任意自然数$q>1$, $K_2(R[t]/(t^q), (t))$由满足上述关系的Dennis-Stein符号$\langle at^i, t\rangle$和$\langle at^i, b\rangle$生成, 其中$a, b\in R, 1\leq i<q$. 
\end{prop}
\begin{proof}
	参见\cite{MR82k:13016} Proposition 1.7. 
\end{proof}

\subsection{符号说明} % (fold)
\label{subsec:符号}
为了陈述定理, 将文中常用的记号详述如下(参考\cite{MR86f:18017})
\begin{itemize}
	\item $\mathbb{Z}_+$表示非负整数全体. 
	\item $\varepsilon^1 = (1, 0)\in \mathbb{Z}_+^2$, $\varepsilon^2 = (0, 1)\in \mathbb{Z}_+^2$.
	\item 对于$\alpha = (\alpha_1,\alpha_2) \in \mathbb{Z}_+^2$, 记$t^{\alpha}=t_1^{\alpha_1}t_2^{\alpha_2}$, 于是有$t^{\varepsilon^1}=t_1$, $t^{\varepsilon^2}=t_2$. 
	\item $\Delta=\{\alpha\in\mathbb{Z}_+^2\mid  t^{\alpha}\in I\}$, 若$\delta \in \Delta$, 则$\delta+\varepsilon^i \in \Delta, i=1, 2$. 
	\item $\Lambda=\{(\alpha, i)\in\mathbb{Z}_+^2 \times \{1, 2\}\mid  \alpha_i\geq 1, t^{\alpha}\in M\}$. 
	\item 对于$(\alpha, i)\in\Lambda$, 令{\color{blue}$[\alpha, i]=\min\{m\in \mathbb{Z}\mid m\alpha - \varepsilon^i\in \Delta\}$}. 
	若$(\alpha, i), (\alpha, j)\in \Lambda$, 有$[\alpha, i]\leq [\alpha, j]+1$. 
	\item 若$gcd(p, \alpha_1, \alpha_2)=1$, 令$[\alpha]=\max\{[\alpha, i]\mid  \alpha_i  \not\equiv 0 \bmod p\}$, 其中$p$是域$k$的特征.
	\item {\color{blue} $\Lambda^{00}= \big\{(\alpha, i)\in \Lambda\mid  gcd(\alpha_1, \alpha_2)=1, i\neq \min\{j\mid \alpha_j\not\equiv 0 \bmod p, [\alpha, j]=[\alpha]\} \big\}$}. 
\end{itemize}
% subsection 符号 (end)
% $\Delta$中对应的是t_1^{\alpha_1}t_2^{\alpha_2}在$I$里的,也就是如果t^\alpha在I里,\langle t^\alpha, t_i\rangle=0,
% 而$\Lambda$意思是说这样的符号是相对K2群里D-S符号, (t^\alpha,i)对应过去就是$\langle t^{\alpha-\varepsilon_i}, t_i$,保证了是相对K2群中的元素
% [\alpha,i]保证了1-x^{[\alpha,i]}f(x)映过去恰好是0,并且是这样子的最小,也就是说1-x^{[\alpha,i]+1}f(x)肯定是0,1-x^{[\alpha,i]-1}f(x)未必是0。

若$(\alpha, i)\in \Lambda$, $f(x)\in k[x]$, 令
\begin{align*}
\Gamma_{\alpha, i} \colon & (1+xk[x])^{\times} \longrightarrow K_2(A,M)\\
				& 1-xf(x)  \mapsto \langle f(t^\alpha)t^{\alpha-\varepsilon^i}, t_i \rangle.
\end{align*}
$\Gamma_{\alpha, i}$是群同态: $(1-xf_1(x))(1-xf_2(x)) = 1-x(f_1(x)+f_2(x)-xf_1(x)f_2(x))$, 
\begin{align*}
	\Gamma_{\alpha, i}((1-xf_1(x))(1-xf_2(x)))&=\Gamma_{\alpha, i}(1-x(f_1(x)+f_2(x)-xf_1(x)f_2(x)))\\
	&=\langle f_1(t^\alpha)t^{\alpha-\varepsilon^i}+f_2(t^\alpha)t^{\alpha-\varepsilon^i}-t^\alpha f_1(t^\alpha)f_2(t^\alpha)t^{\alpha-\varepsilon^i}, t_i \rangle \\
	\Gamma_{\alpha, i}(1-xf_1(x))\Gamma_{\alpha, i}(1-xf_2(x)) & = \langle f_1(t^\alpha)t^{\alpha-\varepsilon^i}, t_i \rangle +\langle f_2(t^\alpha)t^{\alpha-\varepsilon^i}, t_i \rangle\\
	& = \langle f_1(t^\alpha)t^{\alpha-\varepsilon^i}+f_2(t^\alpha)t^{\alpha-\varepsilon^i}-f_1(t^\alpha)t^{\alpha-\varepsilon^i}f_2(t^\alpha)t^{\alpha-\varepsilon^i}t_i, t_i \rangle
\end{align*}

% 这里写1-x^{[\alpha,i]}g(x) 映过去为0

% 注释掉这一部分,需要重新写
% 
% 若$g(t_1, t_2)=t_ih(t_1, t_2)\in \sqrt{I}=(t_1)$, 令
% \[\Gamma_i(1-g(t_1, t_2))=\langle h(t_1, t_2), t_i \rangle, \]
% 且有
% \[\Gamma_{\alpha, i}(1-xf(x))=\Gamma_i(1-t^{\alpha} f(t^{\alpha})). \]

% 由于$t_1\in \sqrt{I}$, $\Gamma_1$诱导了同态
% \begin{align*}
% (1+t_1k[t_1, t_2]/t_1 I)^{\times} &\longrightarrow K_2(A, M)\\
% 1-g(t_1, t_2) & \mapsto \langle h(t_1, t_2), t_1 \rangle
% \end{align*}
% $\Gamma_2$诱导了同态
% \begin{align*}
% (1+t_2\sqrt{I}/t_2 I)^{\times} &\longrightarrow K_2(A, M)\\
% 1-g(t_1, t_2) & \mapsto \langle h(t_1, t_2), t_2 \rangle
% \end{align*}
% 
% 上面需要重写
若$(\alpha, i)\in\Lambda$, $\Gamma_{\alpha, i}$诱导了同态
\[\big(1+xk[x]/(x^{[\alpha, i]})\big)^{\times} \longrightarrow K_2(A, M). \]
\begin{theorem}
\label{K2(A, M)}
	$\Gamma_{\alpha, i}$诱导了{\color{red}同构}
\[ K_2(A, M)\cong \bigoplus_{(\alpha, i)\in\Lambda^{00}}\big(1+xk[x]/(x^{[\alpha, i]})\big)^{\times}. \]
\end{theorem}
\begin{proof}
	参见\cite{MR86f:18017} Corollary 2.6.. 
\end{proof}

\subsection{Witt向量}
令$R$是交换环, $R$上的泛Witt向量环(the ring of universal/big Witt vectors over $R$)$BigWitt(R)$作为交换群同构于$(1+xR\llbracket x\rrbracket )^{\times}$, 即常数项为$1$的形式幂级数全体在乘法运算下形成的交换群, 
\begin{align*}
BigWitt(R) &\overset{\sim}\longrightarrow (1+xR\llbracket x\rrbracket )^{\times}\\
(r_1, r_2, \cdots) & \mapsto \prod_{n=1}^{\infty}(1-r_n x^n). 
\end{align*}



考虑$(1+xR\llbracket x\rrbracket )^{\times}$的子群$(1+x^{n+1}R\llbracket x\rrbracket )^{\times}$, 记交换群%$n$次截断泛Witt向量环(这个定义不太好说,underlying group不知怎么翻译)
$BigWitt_n(R)=(1+xR\llbracket x\rrbracket )^{\times}/(1+x^{n+1}R\llbracket x\rrbracket )^{\times}$. 显然$BigWitt_1(R)=R$, 并且注意当$n\geq 3$时, $BigWitt_n(\mathbb{F}_2)$不是循环群. 

% 
% 下方有修改, 				|^^^^|				请对比来选择. 
% 下方有修改, 				|^^^^|				请对比来选择. 
% 下方有修改, 				|^^^|				请对比来选择. 
% 下方有修改, 				|^^^|				请对比来选择. 
% 下方有修改, 				|^^^|				请对比来选择. 
% 下方有修  				|^^|				 对比来选择. 
% 下方有修  				|^^|				 对比来选择. 
% 下方有修  				|^^|				 对比来选择. 
% 下方有修  				|^^|				 对比来选择
% 下方有修  				|^|					 对比来选择
% 下方有     				|^|					  比来选泽
% 下方有    				||					   比来选 
% 							
% 									
%   修改 				    口					    修改
%   修改 				    口					    修改
%
{\color{blue}\begin{lemma}
	 $BigWitt_n(\mathbb{F}_q)\cong (1+x\mathbb{F}_q[x]/(x^{n+1}))^{\times}$. 
\end{lemma}}
\begin{proof}
	% 由定义$BigWitt_n(\mathbb{F}_q):=(1+x \mathbb{F}_q\llbracket x\rrbracket )^{\times}/(1+x^{n+1} \mathbb{F}_q\llbracket x\rrbracket )^{\times}$, 
	考虑群的满同态
	\begin{align*}
	(1+x \mathbb{F}_q\llbracket x\rrbracket )^{\times} &\longrightarrow (1+x \mathbb{F}_q[x]/(x^{n+1}))^{\times}\\
	1+\sum_{i\geq 1}a_i x^i &\mapsto 1+\sum_{i = 1}^{n}a_i x^i +(x^{n+1})
	\end{align*}
	其中$a_i\in \mathbb{F}_q$, 易知同态核为$(1+x^{n+1} \mathbb{F}_q\llbracket x\rrbracket )^{\times}$, 
	从而$BigWitt_n(\mathbb{F}_q)=(1+x \mathbb{F}_q\llbracket x\rrbracket )^{\times}/(1+x^{n+1} \mathbb{F}_q\llbracket x\rrbracket )^{\times} \cong (1+x\mathbb{F}_q[x]/(x^{n+1}))^{\times}$. 
\end{proof}


% 上面为新加
\begin{example} 
\label{ex:W3(F2)}
	$BigWitt_3(\mathbb{F}_2)\cong (1+x\mathbb{F}_2[x]/(x^{4}))^{\times}\cong \mathbb{Z}/4 \mathbb{Z} \oplus\mathbb{Z}/2 \mathbb{Z}$.
\end{example}
\begin{proof}
	$1+x\in (1+x \mathbb{F}_2[x]/(x^4))^{\times}$是$4$阶元, 由它生成的子群$\langle 1+x \rangle = \{1, 1+x, 1+x^2, 1+x+x^2+x^3\}$, 且$1+x^3$是二阶元, $\langle 1+x^3 \rangle = \{1, 1+x^3\}$. 令$\sigma, \tau$分别是$\mathbb{Z}/4 \mathbb{Z}$和$\mathbb{Z}/2 \mathbb{Z}$的生成元, 则有同构
		\begin{align*}
		\mathbb{Z}/4 \mathbb{Z} \oplus \mathbb{Z}/2 \mathbb{Z} &\longrightarrow BigWitt_4(\mathbb{F}_2) \\
		(\sigma, \tau) & \mapsto (1+x)(1+x^3)=1+x+x^3. 
		\end{align*}
\end{proof}

\begin{example}
	$BigWitt_4(\mathbb{F}_2) \cong \mathbb{Z}/8 \mathbb{Z} \oplus \mathbb{Z}/2 \mathbb{Z}. $
\end{example}
\begin{proof}
	$1+x \in BigWitt_5(\mathbb{F}_2)$是$8$阶元, 由它生成的子群$\langle 1+x \rangle = \{1, 1+x, 1+x^2, 1+x+x^2+x^3, 1+x^4, 1+x+x^4, 1+x^2+x^4, 1+x+x^2+x^3+x^4\}$, 另外$1+x^3$是二阶元, $\langle 1+x^3 \rangle = \{1, 1+x^3\}$. 令$\sigma, \tau$分别是$\mathbb{Z}/8 \mathbb{Z}$和$\mathbb{Z}/2 \mathbb{Z}$的生成元, 则有同构
	\begin{align*}
	\mathbb{Z}/8 \mathbb{Z} \oplus \mathbb{Z}/2 \mathbb{Z} &\longrightarrow BigWitt_4(\mathbb{F}_2) \\
	(\sigma, \tau) & \mapsto (1+x)(1+x^3)=1+x+x^3+x^4 
	\end{align*}
	于是$(\sigma^i, \tau^j), 0\leq i <8, 0\leq j<2$对应于$(1+x)^i(1+x^3)^j$, 详细的对应如下
	\begin{align*}
	(1, \tau) & \mapsto 1+x^3, & (\sigma, \tau) & \mapsto 1+x+x^3+x^4, \\
	 (\sigma^2, \tau) & \mapsto 1+x^2+x^3, & (\sigma^3, \tau) & \mapsto 1+x+x^2+x^4, \\
	(\sigma^4, \tau) & \mapsto 1+x^3+x^4, & (\sigma^5, \tau) & \mapsto 1+x+x^3, \\
	 (\sigma^6, \tau) & \mapsto 1+x^2+x^3+x^4, & (\sigma^7, \tau) & \mapsto 1+x+x^2, \\
	(1, 1)& \mapsto 1, & (\sigma, 1) & \mapsto 1+x, \\
	(\sigma^2, 1) & \mapsto 1+x^2, & (\sigma^3, 1) & \mapsto 1+x+x^2+x^3, \\
	(\sigma^4, 1) & \mapsto 1+x^4, &
	(\sigma^5, 1) & \mapsto 1+x+x^4, \\
	(\sigma^6, 1) & \mapsto 1+x^2+x^4, & (\sigma^7, 1) & \mapsto 1+x+x^2+x^3+x^4. 
	\end{align*}
\end{proof}



固定素数$p$, 考虑局部环$\mathbb{Z}_{(p)}=\mathbb{Z}[1/\ell \mid \text{所有素数}\ell\neq p]$, 即$\mathbb{Z}$在非零素理想$(p)=p \mathbb{Z}$处的局部化, 于是$\mathbb{Z}_{(p)}$-代数$R$就是所有除$p$外的素数均在其中可逆的交换环, 如$\mathbb{F}_{p^n}$是一个$\mathbb{Z}_{(p)}$-代数. 

下面考虑$p$-Witt向量环$W(A)$与截断$p$-Witt向量环$W_n(A)$, $p$-Witt向量为$(a_0, a_1, \cdots)$, 加法用Witt多项式定义, 本文仅考虑用加法定义的交换群结构, 例如
% $W(\mathbb{F}_p)=\mathbb{Z}_{p}$,
作为交换群$W_n(\mathbb{F}_{p^f})$同构于$(\mathbb{Z}/p^n\mathbb{Z})^f$. %这里可以加上Galois环 并且注意F_p^f是F_p上f维向量空间,因此底层群是(C_p)^f

Artin-Hasse级数定义为
\[AH(x)= \exp(\sum_{n\geq 0}\frac{x^{p^n}}{p^n})=1+x+\cdots \in 1+x \mathbb{Q}\llbracket x\rrbracket , \]
实际上$AH(x)\in 1+x \mathbb{Z}_{(p)}\llbracket x\rrbracket $(参考\cite{rabinoff2014theory} Theorem 7.2). 对于$BigWitt(R)=(1+xR\llbracket x\rrbracket)^{\times}$中的任一元素$\alpha$可以写成无穷乘积
\[\alpha = \prod_{n=1}^{\infty}(1-r_nx^n),  \]
其中$\ r_n\in R$是唯一的. 若$A$是$\mathbb{Z}_{(p)}$-代数, $BigWitt(A)=(1+xA\llbracket x\rrbracket)^{\times}$中的任一元素$\alpha$还有如下表法 \cite{katz2013witt}
\[\alpha = \prod_{n\geq 1}AH(a_n x^n), \ a_n\in A. \]
将整数$n$写成$n=mp^a$, 使得$gcd(m, p)=1, a\geq 0$, 由于$A$是$\mathbb{Z}_{(p)}$-代数, $m$可逆, 从而$[x\mapsto x^{1/m}]\in \End(BigWitt(A))$是双射, 于是我们可以将$\alpha\in BigWitt(A)$以如下的形式表出
\[\prod_{\scriptsize\substack{m\geq 1 \\ gcd(m, p)=1  \\ a\geq 0}}AH(a_{mp^a} x^{mp^a})^{1/m}. \]

另一方面对于$\mathbb{Z}_{(p)}$-代数$A$, 下列映射是群同态
\begin{align*}
W(A)&\longrightarrow BigWitt(A)\\
(a_0, a_1, \cdots) &\mapsto \prod_{i\geq 0}AH(a_i x^i). 
\end{align*}

作为交换群, $BigWitt_n(A)$可以分解为$p$-Witt向量环的直和, 实际上有以下同构\cite{Lauter1999A} % 这个引用还待查,再找更精确的表述的
\[
BigWitt(A) \cong \prod_{\scriptsize\substack{m\geq 1 \\ gcd(m, p)=1}} W(A), 
\]
元素$\prod\limits_{\scriptsize\substack{m\geq 1 \\ gcd(m, p)=1  \\ a\geq 0}}AH(a_{mp^a} x^{mp^a})^{1/m}$对应于一个$m$-分量为$(a_m, a_{mp}, a_{mp^2}, \cdots)\in W(A)$的Witt向量. 
对于截断的Witt向量环, 有同构\[
BigWitt_n(A) \cong \bigoplus_{\scriptsize\substack{1\leq m\leq n \\ gcd(m, p)=1}} W_{\ell(m, n)}(A), 
\]
其中$\ell(m, n)$是一个整数, 定义为
\[\ell(m, n)=1+\left \lfloor\log_p \frac{n}{m}  \right \rfloor, \]
即$\ell(m, n)=1+\text{使得$mp^k\leq n$成立的最大整数$k$}.$

{\color{blue}考虑特征为$p$的有限域$\mathbb{F}_q$, 有同构\cite{Lauter1999A}
\[BigWitt_n(\mathbb{F}_{q}) \cong \bigoplus_{\scriptsize\substack{1\leq m\leq n \\ gcd(m, p)=1}} W_{\ell(m, n)}(\mathbb{F}_{q}), \]}
注意到$\sum\limits_{\scriptsize\substack{1\leq m\leq n \\ gcd(m, p)=1}} \ell(m, n) = n$, 因此两边都是$q^n$阶交换群. 
\begin{corollary}
\label{cor:BW}
	若有限域$\mathbb{F}_{p^f}$的特征$ch(\mathbb{F}_{p^f})=p$, 则作为交换群有
	\[
	BigWitt_n(\mathbb{F}_{p^f})\cong \bigoplus_{\scriptsize\substack{1\leq m\leq n \\ gcd(m, p)=1}}W_{1+ \left \lfloor\log_p \frac{n}{m}  \right \rfloor}(\mathbb{F}_{p^f}) = \bigoplus_{\scriptsize\substack{1\leq m\leq n \\ gcd(m, p)=1}}(\mathbb{Z}/p^{1+ \left \lfloor\log_p \frac{n}{m}  \right \rfloor}\mathbb{Z})^f, 
	\]
	其中$ \left \lfloor x \right \rfloor$表示不超过$x$
	的最大整数. 
\end{corollary}
\begin{example}
	作为交换群, $BigWitt_3(\mathbb{F}_2)= W_{\ell(1, 3)}(\mathbb{F}_2)\oplus W_{\ell(3, 3)}(\mathbb{F}_2)=W_2(\mathbb{F}_2)\oplus W_1(\mathbb{F}_2)=\mathbb{Z}/4 \mathbb{Z}\oplus	\mathbb{Z}/2 \mathbb{Z}$, 

	$BigWitt_4(\mathbb{F}_2)= W_{\ell(1, 4)}(\mathbb{F}_2)\oplus W_{\ell(3, 4)}(\mathbb{F}_2)=W_3(\mathbb{F}_2)\oplus W_1(\mathbb{F}_2)=\mathbb{Z}/8 \mathbb{Z}\oplus	\mathbb{Z}/2 \mathbb{Z}$, 

	$BigWitt_2(\mathbb{F}_3)= W_{\ell(1, 2)}(\mathbb{F}_3)\oplus W_{\ell(2, 2)}(\mathbb{F}_3)=W_1(\mathbb{F}_3)\oplus W_1(\mathbb{F}_3)=\mathbb{Z}/3 \mathbb{Z}\oplus	\mathbb{Z}/3 \mathbb{Z}$. 

\end{example}