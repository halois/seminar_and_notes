%!TEX root = ../main.tex
\chapter{On the calucation of $K_2(\mathbb{F}_2[C_4\times C_4])$}

\section{Abstract}
We calulate $K_2(\mathbb{F}_2[C_4\times C_4])$ by using relative $K_2$-group $K_2(\mathbb{F}_2[t_1,t_2]/(t_1^4,t_2^4),(t_1,t_2))$.

% 本文利用相对$K_2$群$K_2(\mathbb{F}_2[t_1,t_2]/(t_1^4,t_2^4),(t_1,t_2))$计算$K_2(\mathbb{F}_2[C_4\times C_4])$.

% keywords: relative $K_2$-group, Dennis-Stein symbols,  truncated polynomial ring

% 关键词 相对$K_2$群, Dennis-Stein符号, 截断多项式环

\section{Introduction}
Let $C_n$ denote the cyclic group of order $n$. Chen et al.\cite{陈虹:419} calculated $K_2(\mathbb{F}_2[C_4\times C_4])$ by the relative $K_2$-group $K_2(\mathbb{F}_2C_4[t]/(t^4),(t))$ of the truncated polynomial ring $\mathbb{F}_2C_4[t]/(t^4)$. In this short notes, we use another method to calculate $K_2(\mathbb{F}_2[C_4\times C_4])$ directly.
\section{Preliminaries}
Let $k$ be a finite field of characteristic $p>0$. Let $I=(t_1^m,t_2^n)$ be a proper ideal in the polynomial ring $k[t_1,t_2]$. Put $A=k[t_1,t_2]/I$. We will write the image of $t_i$ in $A$ also as $t_i$. Let $M=(t_1,t_2)$ be the nilradical of $A$. Note that $A/M=k$. One has a presentation for $K_2(A,M)$ in terms of Dennis-Stein symbols:
\begin{itemize}
	\item[generators:]   $\langle a,b \rangle $, $(a,b)\in A\times M \cup M \times A$;
	\item[relations:] $\langle a,b\rangle = -\langle b,a \rangle$, \\
	$\langle a,b\rangle +\langle c,b \rangle=\langle a+c-abc,b\rangle$, \\
	 $\langle a,bc\rangle =\langle ab,c\rangle +\langle ac,b\rangle$ for $(a,b,c)\in A\times M \times A \cup M\times A\times M$.
\end{itemize}

Now we introduce some notations followed \cite{MR86f:18017}
\begin{itemize}
	\item $\mathbb{N}$: the monoid of non-negative integers, 
	\item $\epsilon^1 = (1,0)\in \mathbb{N}^2$, $\epsilon^2 = (0,1)\in \mathbb{N}^2$,
	\item for $\alpha \in \mathbb{N}^2$, one writes $t^{\alpha}=t_1^{\alpha_1}t_2^{\alpha_2}$, so $t^{\epsilon^1}=t_1$, $t^{\epsilon^2}=t_2$,
	\item $\Delta=\{\alpha\in\mathbb{N}^2\mid  t^{\alpha}\in I\}$,
	\item $\Lambda=\{(\alpha,i)\in\mathbb{N}^2 \times \{1,2\}\mid  \alpha_i\geq 1, t^{\alpha}\in M\}$,
	\item for $(\alpha,i)\in\Lambda$, set $[\alpha,i]=\min\{m\in \mathbb{Z}\mid m\alpha - \epsilon^i\in \Delta\}$,
	\item if $gcd(p,\alpha_1,\alpha_2)=1$, let $[\alpha]=\max\{[\alpha,i]\mid  \alpha_i  \not\equiv 0 \bmod p\}$
	\item $\Lambda^{00}= \big\{(\alpha,i)\in \Lambda\mid  gcd(\alpha_1,\alpha_2)=1, i\neq \min\{j\mid \alpha_j\not\equiv 0 \bmod p, [\alpha,j]=[\alpha]\} \big\}$,
\end{itemize}

If $(\alpha,i)\in \Lambda$, $f(x)\in k[x]$, put
\[\Gamma_{\alpha,i}(1-xf(x))= \langle f(t^\alpha)t^{\alpha-\epsilon^i},t_i \rangle,\]
then $\Gamma_{\alpha,i}$ induces a homomorphism
\[(1+xk[x]/(x^{[\alpha,i]}))^{\times} \longrightarrow K_2(A,M).\]
\begin{lemma}
\label{K2(A,M)}
	The $\Gamma_{\alpha,i}$ induce an isomorphism
\[ K_2(A,M)\cong \bigoplus_{(\alpha,i)\in\Lambda^{00}}(1+xk[x]/(x^{[\alpha,i]}))^{\times}.\]
\end{lemma}
\begin{proof}
	See Corollary 2.6 in \cite{MR86f:18017}。
\end{proof}

\begin{lemma} 
\label{lem:BW}
	$(1+x\mathbb{F}_2[x]/(x^{3}))^{\times}\cong \mathbb{Z}/4 \mathbb{Z}$, $(1+x\mathbb{F}_2[x]/(x^{4}))^{\times}\cong \mathbb{Z}/4 \mathbb{Z} \oplus\mathbb{Z}/2 \mathbb{Z}$.
\end{lemma}
\begin{proof}
	It is easy to see that $(1+x\mathbb{F}_2[x]/(x^{3}))^{\times}$ is generated by $1+x$, and the order of $ 1+x $ is $4$, we conclude that $(1+x\mathbb{F}_2[x]/(x^{3}))^{\times}\cong \mathbb{Z}/4 \mathbb{Z}$.

	Obeserve that the orders of the elements $1+x,1+x^3\in (1+x \mathbb{F}_2[x]/(x^4))^{\times}$ are $4$ and $2$ respectively. The subgroups $\langle 1+x \rangle = \{1,1+x,1+x^2,1+x+x^2+x^3\}$, $\langle 1+x^3 \rangle = \{1,1+x^3\}$. Let $\sigma,\tau$ be the generators of $\mathbb{Z}/4 \mathbb{Z}$ and $\mathbb{Z}/2 \mathbb{Z}$ respectively, then the homomorphism 
		\begin{align*}
		\mathbb{Z}/4 \mathbb{Z} \oplus \mathbb{Z}/2 \mathbb{Z} &\longrightarrow (1+x\mathbb{F}_2[x]/(x^{4}))^{\times} \\
		(\sigma,\tau) & \mapsto (1+x)(1+x^3)=1+x+x^3. \\
		\end{align*}
	is an isomorphism.
\end{proof}
\section{Main result}
Let $C_4\times C_4$ be the direct product of two cyclic groups of order $4$, then we have $\mathbb{F}_2[C_4\times C_4]  \cong \mathbb{F}_2[t_1,t_2]/(t_1^4,t_2^4)$ since the characteristic of $\mathbb{F}_2$ is $2$.
\begin{lemma}
	$K_2(\mathbb{F}_2[C_4\times C_4] ) \cong K_2(\mathbb{F}_2[t_1,t_2]/(t_1^4,t_2^4),(t_1,t_2))$.
\end{lemma}
\begin{proof}
	The following sequence is split exact
	\[
	0\longrightarrow K_2(\mathbb{F}_2[t_1,t_2]/(t_1^4,t_2^4),(t_1,t_2)) \overset{f}\longrightarrow K_2(\mathbb{F}_2[t_1,t_2]/(t_1^4,t_2^4)) \overset{t_i\mapsto 0}\longrightarrow K_2(\mathbb{F}_2) \longrightarrow 0.
	\]
	The homomorphism $f$ is an isomorphism since $K_2$-group of any finite field is trivial.
\end{proof}

\begin{theorem}
	Let $C_4\times C_4$ be the direct product of two cyclic groups of order $4$, then $K_2(\mathbb{F}_2[C_4\times C_4])  \cong (\mathbb{Z}/4 \mathbb{Z})^3 \oplus (\mathbb{Z}/2 \mathbb{Z})^9$.
\end{theorem}
\begin{proof}
	Set $A=\mathbb{F}_2[t_1,t_2]/(t_1^4,t_2^4)$, then $I=(t_1^4,t_2^4)$, $M=(t_1,t_2)$, $A/M=\mathbb{F}_2$. Thus 
	\[\Delta =\{(\alpha_1,\alpha_2)\in\mathbb{N}^2\mid  \alpha_1 \geq 4 \text{ or } \alpha_2 \geq 4\},\] 
	\[\Lambda =\{(\alpha,i)\mid \alpha_i \geq 1\}.\]

	For $(\alpha,i)\in \Lambda$, 
	\[ [\alpha,1]=\min\{\left \lceil \frac{5}{\alpha_1} \right \rceil, \left \lceil \frac{4}{\alpha_2} \right \rceil \}, \]
	\[ [\alpha,2]=\min\{\left \lceil \frac{4}{\alpha_1} \right \rceil, \left \lceil \frac{5}{\alpha_2} \right \rceil \},\]
	where $\left \lceil x \right \rceil=\min \{m\in \mathbb{Z}\mid m\geq x\}$.

	Next we want to compute the set $\Lambda^{00}$. Since $(1+x\mathbb{F}_2[x]/(x))^{\times}$ is trivial, it is sufficient to consider the subset $\Lambda^{00}_{>1}:=\{(\alpha,i)\in \Lambda^{00}\mid [(\alpha,i)]>1\}$, and then 
	\[K_2(A,M)\cong \bigoplus_{(\alpha,i)\in\Lambda^{00}}(1+x \mathbb{F}_2[x]/(x^{[\alpha,i]}))^{\times} = \bigoplus_{(\alpha,i)\in\Lambda^{00}_{>1}}(1+x \mathbb{F}_2[x]/(x^{[\alpha,i]}))^{\times}.\]

	(1) If $1\leq \alpha_1 \leq 4$ is even and $1\leq \alpha_2 \leq 4$ is odd, then $(\alpha,1)\in \Lambda^{00}_{>1}$ and $[\alpha,1]=\min\{\left \lceil \frac{5}{\alpha_1} \right \rceil, \left \lceil \frac{4}{\alpha_2} \right \rceil \}$.

	(2) If $1\leq \alpha_1 \leq 4$ is odd and $1\leq \alpha_2 \leq 4$ is even, then $(\alpha,2)\in \Lambda^{00}_{>1}$ and $[\alpha,2]=\min\{\left \lceil \frac{4}{\alpha_1} \right \rceil, \left \lceil \frac{5}{\alpha_2} \right \rceil \}$. 

	(3) If $1\leq \alpha_1,\alpha_2 \leq 4$ are both odd and $gcd(\alpha_1,\alpha_2)=1$, then $(\alpha,2)\in \Lambda^{00}_{>1}$ only when $[\alpha]=[\alpha,1]$.

	By the computation \ref{lem:BW}, we can get the following table
	\[\begin{array}{|c|c|c|c|}
\hline
(\alpha,i)\in \Lambda^{00}_{>1} & [\alpha,i] & (1+x \mathbb{F}_2[x]/(x^{[\alpha,i]}))^{\times} \\
\hline
((2,1),1)  & 3 & \mathbb{Z}/4 \mathbb{Z}\\
\hline
((2,3),1)  & 2& \mathbb{Z}/2 \mathbb{Z} \\
\hline
((4,1),1)  & 2& \mathbb{Z}/2 \mathbb{Z} \\
\hline
((4,3),1)  & 2& \mathbb{Z}/2 \mathbb{Z} \\
\hline
((1,2),2)  & 3& \mathbb{Z}/4 \mathbb{Z} \\
\hline
((1,4),2)  & 2& \mathbb{Z}/2 \mathbb{Z} \\
\hline
((1,1),2)  & 4& \mathbb{Z}/2 \mathbb{Z}\oplus \mathbb{Z}/4 \mathbb{Z} \\
\hline
((1,3),2)  & 2& \mathbb{Z}/2 \mathbb{Z} \\
\hline
((3,2),2)  & 2& \mathbb{Z}/2 \mathbb{Z} \\
\hline
((3,4),2)  & 2& \mathbb{Z}/2 \mathbb{Z} \\
\hline
((3,1),2)  & 2& \mathbb{Z}/2 \mathbb{Z}\\
\hline
\end{array}\]
Hence $K_2(\mathbb{F}_2[C_4\times C_4])  \cong (\mathbb{Z}/4 \mathbb{Z})^3 \oplus (\mathbb{Z}/2 \mathbb{Z})^9$.

Furthermore, one can use the homomorphism $\Gamma_{\alpha,i}$ to determine the generators as below,\\
the generators of order $4$:
\[\langle t_1t_2,t_1 \rangle, \langle t_1t_2,t_2 \rangle, \langle t_1,t_2 \rangle,\] 
the generators of order $2$:
\[\langle t_1t_2^3,t_1 \rangle, \langle t_1^3t_2,t_1 \rangle, \langle t_1^3t_2^3,t_1 \rangle,\langle t_1t_2^3,t_2 \rangle, \langle t_1^3t_2^2,t_2 \rangle, \langle t_1t_2^2,t_2 \rangle,\langle t_1^3t_2,t_2 \rangle, \langle t_1^3t_2^3,t_2 \rangle, \langle t_1^3,t_2 \rangle.\]
\end{proof}
\begin{remark}
	Compared with \cite{陈虹:419}, note that $\langle t_1^3,t_2 \rangle=\langle t_1^2t_2,t_1\rangle$, because
	\begin{align*}
	\langle t_1^3,t_2 \rangle &= \langle t_1^2,t_1t_2 \rangle-\langle t_1^2t_2,t_1 \rangle \\
		&= \langle t_1,t_1^2t_2 \rangle -\langle t_1^2t_2,t_1 \rangle-\langle t_1^2t_2,t_1 \rangle\\
		&= -3\langle t_1^2t_2,t_1 \rangle\\
		&= -\langle t_1^2t_2,t_1 \rangle \\
		&=\langle t_1^2t_2,t_1 \rangle,
	\end{align*}
	since $\langle t_1^2t_2,t_1 \rangle+\langle t_1^2t_2,t_1 \rangle=\langle 0,t_1\rangle =0$ and $\langle t_1^3,t_2 \rangle=-\langle t_1^3,t_2 \rangle$.
\end{remark}






