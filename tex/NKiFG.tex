%!TEX root = ../main.tex
\chapter{$NK_i(FG)$ } % (fold)

\section{Preliminaries} % (fold)
\label{sec:preliminaries}

\begin{definition}
	If $F$ is any functor from the category of rings to the category of abelian groups,we write $NF(R)$ for the cokernel of the natural map $F(R)\overset{x \mapsto 1} \longrightarrow F(R[x])$; $NF$ is also a functor on rings. Moreover, the ring map $R[x] \longrightarrow R$ provides a splitting $F(R[x]) \longrightarrow  F(R)$ of the natural map, so we have a decomposition $F(R[x])\cong F(R) \oplus NF(R)$.
\end{definition}
In particular, when $F$ is $K_i$ we have functors $NK_i$ and a decomposition $K_i(R[x]) = K_i(R) \oplus NK_i(R)$. Since the ring maps $R[x] \overset{x\mapsto r}\longrightarrow R$ are split surjections for every $r \in R$, hence for every $r$ we also have $NK_0(R)\cong K_0(R[x],(x-r))$ and $NK_1(R)\cong K_1(R[x],(x-r))$.

Let $U$ be a functor from the category of commutative rings to the category of abelian groups,
\begin{align*}
U\colon & \mathbf{CRing} \longrightarrow \mathbf{Ab}\\
		& R \mapsto U(R)=R^{\times}.
\end{align*}
We can also define the functor $NU$.

\begin{lemma}
	Let $R$ be a commutative ring with nilradical $\mathfrak{N}$. If $r_0 + r_1x + \cdots + r_nx^n$ is a unit of $R[x]$ then $r_0 \in R^{\times}$ and $r_1, \cdots, r_n$ are nilpotent. We have 
\begin{enumerate}
	\item  $NU(R)$ is the subgroup $1 + x\mathfrak{N}[x]$ of $R[x]^{\times}$;
	\item $R[x]^{\times} = R^{\times} \oplus NU(R)$;
	\item $R$ is reduced if and only if $R^{\times} = R[x]^{\times}$;
	\item Suppose that $R$ is an algebra over a field $k$. If $char(k)=p$, $NU(R)$ is a $p$-group. If $char(k) = 0$, $NU(R)$ is a uniquely divisible abelian group (= a $\mathbb{Q}$-module).
\end{enumerate}
\end{lemma}

\begin{definition}
	Let $R$ be a commutative ring, the determinant of a matrix provides a group homomorphism from $GL(R)$ onto the group $R^{\times}$ of units of $R$. Define $SK_1(R)$ to be the kernel of the induced surjection $\det\colon K_1(R) \longrightarrow R^{\times}$. Moreover, there is a direct sum decomposition $K_1(R) = R^{\times} \oplus SK_1(R)$.
\end{definition}
\begin{example}
	If $F$ is a field then $SK_1(F) = 0$. Similarly, if $R$ is a Euclidean domain such as $\mathbb{Z}$ or $F[x]$ then $SK_1(R) = 0$ and hence $K_1(R)$ = $R^{\times}$. In particular, $K_1(\mathbb{Z}) = \mathbb{Z}^{\times} = \{ \pm 1\}$ and $K_1(F[x]) = F^{\times}$. If $F$ is a finite field extension of $\mathbb{Q}$ (a number field) and $R$ is an integrally closed subring of $F$ , then Bass, Milnor and Serre proved that $SK_1(R) = 0$.
\end{example}
\begin{lemma}
	If $R$ is a commutative semilocal ring, then $SK_1(R) = 0$ and $K_1(R) = R^{\times}$. In particular, if $(R,\mathfrak{m})$ is a commutative local ring, then $SK_1(R) = 0$.
\end{lemma}
\begin{proof}
	See \cite{weibel2013k} lemma 3.1.4.
\end{proof}
\begin{lemma}
\label{lem:nk1truncomp}
	Let $R$ ba a commutative regular ring, $A=R[t]/(t^N)$, then
\[\Nil_0(A) \rightarrowtail \End_0(A)\]
is an injection, and 
\begin{align*}
NK_1(A)\cong\Nil_0(A)&\cong (1+txA[x])^{\times} = (1+xA[x])^{\times}\\
[(A,t)] &\mapsto 1-tx \\
[(P,\nu)] & \mapsto \det(1-\nu x)
\end{align*}
\end{lemma}
\begin{proof}
	See \cite{weibel2013k} chapter 2 example 7.4.5 and chapter 3 example 3.8.1.
\end{proof}
% section preliminaries (end)


\section{Main Theorem} % (fold)
\label{sec:main_theorem}
\begin{theorem}
\label{thm:1}
	(1) Let $R$ be a commutative semilocal ring, if $NK_1(R)=0$, then $SK_1(R[x])=0$ and $R[x]^{\times}=R^{\times}$ (which means $R$ is reduced);\\
	(2) If $R$ is a reduced commutative semilocal ring, then $NK_1(R)=SK_1(R[x])$;\\
	(3) If $R$ is a commutative semilocal ring but not reduced, then $NK_1(R)\neq 0$, this is equivalent to say that $NK_1(R)$ is not finitely generated.\\
\end{theorem}
\begin{proof}
	Let $R$ be a commutative semilocal ring. There are two split exact sequences
\[0\longrightarrow SK_1(R[x])\longrightarrow K_1(R[x])\longrightarrow R[x]^{\times}\longrightarrow 0, \]
\[0\longrightarrow NK_1(R)\longrightarrow K_1(R[x])\longrightarrow K_1(R)\longrightarrow 0. \]
Recall that $SK_1(R)=0$ for any commutative semilocal ring $R$, hence $K_1(R)\cong R^{\times}$.
We have
\begin{align*}
K_1(R[x]) & = NK_1(R)\oplus R^{\times} \\
		& = SK_1(R[x]) \oplus R[x]^{\times}\\
		& = SK_1(R[x]) \oplus R^{\times} \oplus NU(R)\\
		& = SK_1(R[x]) \oplus R^{\times} \oplus (1 + x\mathfrak{N}[x]),
\end{align*}
hence $NK_1(R)\cong SK_1(R[x])\oplus (1 + x\mathfrak{N}[x])$.

(1) If $NK_1(R)=0$, one can obtain $SK_1(R[x])=0$ and $NU(R)=0$;

(2) Since $NU(R)=0$;

(3) In this case $NK_1(R)\supset 1 + x\mathfrak{N}[x]\neq 0$, and it is known that $NK_i$ is either $0$ or is not finitely generated.
\end{proof}

\begin{theorem}
	If $F$ is a field of characteristic $p>0$ and $G$ is a finite abelian $p$-group, then $NK_1(FG)\neq 0$. In particular $NK_1(\mathbb{F}_{p^m}[C_{p^n}]) \neq 0$.
\end{theorem}
\begin{proof}
	First we claim that $R=FG$ is a commutative local ring. In fact suppose $F$ is a field of characteristic $p$ and $G$ is a finite group, then $FG$ is a local ring if and only if $G$ is a finite $p$-group, in this case the maximal ideal is the augmentation ideal $\mathfrak{m}=IG$ which is also nilpotent, see \cite{Iyengar2010Modules}. And it is easy to see that $R$ is not reduced: take an element $a$ of the maximal order $o(a)$, $o(a)= p^m \leq |G|$ for some $m$, then $a^{o(a)/p}-1$ is a nilpotent element since $(a^{o(a)/p}-1)^p=0$.

	Since local rings are semilocal, we conclude that $NK_1(R)\neq 0$ by Theorem \ref{thm:1}(3).
\end{proof}

\begin{theorem}
	Let $\mathbb{F}$ be a finite field of characteristic $p>0$ and $H$ is a finite group such that $(p, |H|)=1$, then $NK_i(\mathbb{F}H)=0$  for all $i$.
\end{theorem}
\begin{proof}
	By Maschke's theorem, $\mathbb{F}H$ is semisimple. Since it is finite, by Wedderburn-Artin theorem $\mathbb{F}H$ is a direct product of matrix rings over finite fields $F$ of characteristic $p$, $\mathbb{F}H \cong \prod_m M_m(F)$ and 
	$\mathbb{F}H[x] \cong \prod_m M_m(F[x])$. Hence $K_i(\mathbb{F}H)$ is isomorphic to a direct product of groups $K_i(F)$, and $NK_i(\mathbb{F}H)$ is isomorphic to a direct product of groups $NK_i(F)$for all $i$. Since finite fields are regular, we obtain $NK_i(\mathbb{F}H)=0$.
\end{proof}




Since $\mathbb{F}_{p^m}[C_{p^n}]=\mathbb{F}_{p^m}[t]/(t^{p^n})$ and $\mathbb{F}_{p^m}$ is regular, we can use Lemma \ref{lem:nk1truncomp} to describe $NK_1(\mathbb{F}_{p^m}[C_{p^n}])$ as follows.
\begin{theorem}
	$NK_1(\mathbb{F}_{p^m}[C_{p^n}])\cong $. 还没算完 %\bigoplus_{\infty} \mathbb{Z}/p \mathbb{Z}
\end{theorem}
{\color{red} 这个证明还有问题

\begin{proof}
	By lemma \ref{lem:nk1truncomp}, $NK_1(\mathbb{F}_{p^m}[C_{p^n}])\cong  (1+tx\mathbb{F}_{p^m}[t]/(t^{p^n})[x])^{\times}$. As an abelian group, we have
		\begin{align*}
			(1+tx\mathbb{F}_{p^m}[t]/(t^{p^n})[x])^{\times}=  (1+tx\mathbb{F}_{p^m}[x]\llbracket t\rrbracket )^{\times}/(1+t^{p^n}x\mathbb{F}_{p^m}[x]\llbracket t\rrbracket )^{\times} = W_{p^n-1}(x\mathbb{F}_{p^m}[x]).
			% x (\underbrace{\mathbb{F}_{p^m}\oplus \mathbb{F}_{p^m} \oplus \cdots \oplus \mathbb{F}_{p^m}}_{p^n})[x] \\
			%    =& x (\underbrace{\mathbb{Z}/p \mathbb{Z}\oplus  \cdots \oplus \mathbb{Z}/p \mathbb{Z}}_{mp^n})[x]\\
			%    =& \bigoplus_{\infty} \mathbb{Z}/p \mathbb{Z}.
		\end{align*}

\end{proof}}



% section main_theorem (end)

由同态
\[(1+t \mathbb{F}_3\llbracket t\rrbracket )^{\times} \longrightarrow (1+t \mathbb{F}_3[t]/(t^3))^{\times}\]
它的核是$(1+t^3 \mathbb{F}_3\llbracket t\rrbracket )^{\times}$.
从而$(1+t \mathbb{F}_3[t]/(t^3))^{\times} \cong (1+t \mathbb{F}_3\llbracket t\rrbracket )^{\times}/(1+t^3 \mathbb{F}_3\llbracket t\rrbracket )^{\times}$, 又有$(1+t \mathbb{F}_3\llbracket t\rrbracket )^{\times}/(1+t^3 \mathbb{F}_3\llbracket t\rrbracket )^{\times}= W_2(\mathbb{F}_3)$. 注意这里的$W(R)$是big Witt向量,$W_2(\mathbb{F}_3) \neq \mathbb{Z}/\mathbb{Z}^9$


算$(1+t \mathbb{F}_3[t]/(t^3))^{\times}$中的元素,只有$3$阶元:

$(1+t(a_0+a_1t))(1+t(b_0+b_1t))=1+t(a_0+b_0+(a_1+b_1+a_0b_0)t)$,
考虑pair$(a_0,a_1)$,想找到它和$\mathbb{Z}/9 \mathbb{Z}$之间的对应,但是经过计算
这样的pair没有$9$阶元,比如$(2,1)(2,1)(2,1)=(1,0)(2,1)=(0,0)$.

最终算出来$(1+t \mathbb{F}_3[t]/(t^3))^{\times} = \mathbb{Z}/3 \oplus \mathbb{Z}/3$.


\subsection*{Acknowledgement.}
致谢