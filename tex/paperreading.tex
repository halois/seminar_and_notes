%!TEX root = ../main.tex
\chapter{Some useful results}
\section{Lower bounds for the order of $K_2(\mathbb{Z}G)$ and $Wh_2(G)$} % (fold)
\label{sec:lower_bounds_for_the_order_of_}
2016.3.19 阅读这篇文章\cite{Stein1976} 1976年发表在{\em{Math. Ann.}}。

基本假设:$p$: rational prime, $G$: elementary abelian $p$-group.

用的方法:Bloch; van der Kallen  $K_2$ of truncated polynomial rings

结论:
the $p$-rank of $K_2(\mathbb{Z}G)$\footnote{this is a finite group} grows expotentially with the rank of $G$.

$Wh_2(G)$: ``pseudo-isotopy'' group is nontrivial if $G$ has rank at least $2$.


这篇文章之前已知的结论(exact computations)
Dunwoody, $G$ cyclic of order 2 or 3, {\color{green} $K_2(\mathbb{Z}G)$ is an elementary abelian $2$-group of rank 2 if $G$ has order $2$ and of rank $1$ if $G$ has order $3$}. 两者都有$Wh_2(G)$平凡。


一些记号和基本结论
$R$ commutative ring, $A$ a subring of $R$.
$\Omega_{R/A}^1$ the module of K\"{a}hler differentials of $R$ considerd as an algebra over $A$ and $R^*$ will denote the group of units of $R$.

the $p$-rank of an abelian group $G$ is $\dim_{\mathbb{F}_p}(G\otimes_\mathbb{Z} \mathbb{F}_p)$.

\paragraph{第一部分} % (fold)
\label{par:第一部分}
环是$\mathbb{F}_q$ 有限域的情况。

先说结论

首先是一个奇素数的结论
\begin{prop}
	Let $q=p^f$ be odd and let $G$ be an elementary abelian $p$-group of rank $n$.
	Then $K_2(\mathbb{F}_qG)$ is an elementary $p$-group of rank $f(n-1)(p^n-1)$.
\end{prop}
接着是素数$2$的结论
\begin{prop}
	Let $q=2^f$ be odd and let $G$ be an elementary abelian $2$-group of rank $n$.
	Then $K_2(\mathbb{F}_qG)$ is an elementary $2$-group of rank $f(n-1)(2^n-1)$.
\end{prop}

结论实际上是可以统一的,但是方法有些区别,因此原文中分开表述。

我们引进方法时借鉴了van der Kallen的方法和记号

Let $R$ be a commutative ring. The abelian group  $TD(R)$  is the universal $R$-module having generators  $Da, Fa, a \in R$,  subject to the relations
\begin{align*}
D(ab) &= aDb + bDa,\\
D(a + b) &= Da + Db + F(ab),\\
F(a + b) &= Fa + Fb,\\
Fa  &= D(1  + a)- Da.
\end{align*}
There is a natural surjective homomorphism of $R$-modules
\[TD(R)\twoheadrightarrow \Omega^1_{R/\mathbb{Z}}\longrightarrow 1\]
whose kernel is the submodule of  $TD(R)$  generated by the  $Fa, a\in R$.  Relations
imply
\[
F(c^2a)=cFa
\]
($F(c^2a)=F(ca\cdot c)=D(ca+c)-D(ac)-D(c)=D(c(a+1))-D(ac)-D(c)=cD(a+1)-(a+1)D(c)-aD(c)-cD(a)-D(c)=cF(a)$, $0=F(0)=F(a-a)=F(a)+F(-a)$, $\Rightarrow F(a)=-F(a)=F(-a)$, $\Rightarrow F(2a)=0$)

for all  $a, c \in R$  [ vander Kallen, W.: Le $K_2$ des nombres duaux. C.R. Ac. Sc. Paris, t. 273, 1204--1207 (1971), p. 1204]. \\
Hence $F(2a)=2F(a)=0$, if $2$ is a unit of $R$, $F(a)=0$, then the kernel is trivial and $\Omega^1_{R/\mathbb{Z}}\cong  TD(R)$,
\[1\longrightarrow TD(R)\overset{\cong}\longrightarrow \Omega^1_{R/\mathbb{Z}}\longrightarrow 1.\]  
\begin{example}
	$R=\mathbb{Z}$, then the kernel of the above surjection is $\mathbb{Z}/2\mathbb{Z}$.\\
	If $R$ is a field of characteristic $\neq 2$, then $TD(R)\cong \Omega^1_{R/\mathbb{Z}}$.\\
	If $R$ is a perfect field, then $TD(R)\cong \Omega^1_{R/\mathbb{Z}}$.
\end{example}
\begin{definition}
	We define groups  $\Phi_i(R)$, $i\geq 2$, by the exact sequence
	\[1 \longrightarrow \Phi_i(R) \longrightarrow K_2 (R[x]/(x^i))  \longrightarrow K_2(R[x]/(x^{i-1})) \longrightarrow 1. \]
\end{definition}
This sequence is exact at the right as  $SK_1(R[x]/(x^i), (x^{i-1})/(x^i))=  1$ (cf. \cite{MR50:2304} Theorem 6.2 and \cite{MR40:2736}9.2, p. 267).
\paragraph{Remarks} % (fold)
\label{par:remarks}我们把Bass书\cite{MR40:2736}中相关的结论(p. 267)放在这里以备今后查询和使用

A semi-local ring is a ring for which $R/rad(R)$ is a semisimple ring, where $rad(R)$ is the Jacobson radical of $R$. In commutative algebra, semi-local means ``finitely many maximal ideals'', for instance, all rational numbers $r/s$ with $s$ prime to $30$ form a semi-local ring, with maximal ideals generated respectively by $2,3$, and $5$. This is a PID, as a matter of fact, any semi-local Dedekind domain is a PID. And if $R$ is a commutative noetherian ring, the set of zero-divisors is a union of finitely many prime ideals (namely, the ``associated primes'' of $(0)$), thus its classical ring of quotients (obtained from $R$ by inverting all of its non zero-divisors) is a semi-local ring. See \cite{book:975816} p.174. and \cite{MR40:2736} p. 86.



In studying the stable structure of general linear groups in algebraic $K$-theory, Bass proved the following basic result (ca. 1964) on  the  unit
structure  of semilocal rings.
\begin{theorem}
	 If $R$ is a semi-local ring, then $R$ has stable range $1$,
in the sense that, whenever $Ra + Rb = R$, there exists $r \in R$ such that
$a + rb \in R^*$.
\end{theorem}
\begin{example}
Some classes of semi-local rings: left(right) artinian rings, finite direct products of local rings, matrix rings over local rings, module-finite algebras over commutative semi-local rings.\\
The quotient $\mathbb{Z}/m\mathbb{Z}$ is a semi-local ring. In particular, if $m$ is a prime power, then $\mathbb{Z}/m\mathbb{Z}$ is a local ring.\\
A finite direct sum of fields $\bigoplus_{i=1}^n{F_i}$ is a semi-local ring.
In the case of commutative rings with unit, this example is prototypical in the following sense: the Chinese remainder theorem shows that for a semi-local commutative ring $R$ with unit and maximal ideals $m_1, \cdots, m_n$
\[R/\bigcap_{i=1}^n m_i\cong\bigoplus_{i=1}^n R/m_i\]
(The map is the natural projection). The right hand side is a direct sum of fields. Here we note that $\cap_i m_i=rad(R)$, and we see that $R/rad(R)$ is indeed a semisimple ring.\\
The classical ring of quotients for any commutative Noetherian ring is a semilocal ring.\\
The endomorphism ring of an Artinian module is a semilocal ring.\\
Semi-local rings occur for example in commutative algebra when a (commutative) ring R is localized with respect to the multiplicatively closed subset $S = \cap (R - p_i)$, where the $p_i$ are finitely many prime ideals.
\end{example}

\begin{theorem}
	Let $I$ be a two-sided ideal in a ring $R$. Assume either that $R$ is semi-local or that $I\subset rad(R)$. Then 
	\[GL_1(R,I)\longrightarrow K_1(R,I)\]
	is surjective, and, for all $m\geq 2$,
	\[GL_m(R,I)/E_m(R,I)\longrightarrow K_1(R,I)\]
	is an isomorphism. Moreover $[GL_m(R),GL_m(R,I)]\subset E_m(R,I)$, with equality for $m\geq 3$.
\end{theorem}
\begin{corollary}
	Suppose that $R$ above is commutative, then $E_n(R,I)\iso SL_n(R,I)$ is an isomorphism for all $n\geq 1$, and $SK_1(R,I)=0$.
\end{corollary}
\begin{proof}
	The determinant induces the inverse, 
	\[\det \colon K_1(R,I)\longrightarrow GL_1(R,I).\]
	In particular, if $\alpha\in GL_n(R,I)$ and $\det(\alpha)=1$ then $\alpha \in E_n(R,I)$, i.e. $SL_n(R,I)\subset E_n(R,I)$. The opposite inclusion is trivial. Finally $SK_1(R,I)=SL(R,I)/E(R,I)=0$.
\end{proof}
% paragraph remarks (end)

还有一个小插曲,当$k$是域时,$k[x]/(x^m)$是局部环的证明
\begin{prop}
	 Let $I$ be an ideal in the ring $R$.\\
a) If $rad(I)$ is maximal, then $R/I$ is a local ring.\\
b) In particular, if $m$ is a maximal ideal and $n \in \mathbb{Z}^+$ then $R/m^n$ is a local ring.
\end{prop}
\begin{proof}
a) We know that $rad(I)=\bigcap_{P\supset I}P$, so if $rad(I) = m$ is maximal it must be the only prime ideal containing I. Therefore, by correspondence $R/I$ is a local ring. (In fact it is a ring with a unique prime ideal.)\\
b)$rad(m^n) = rad(m) = m$, so part a) applies. 
\end{proof}


\begin{example}
	For instance, for any prime number $p$, $\mathbb{Z}/(p^k)$ is a local ring, whose maximal ideal is generated by $p$. It is easy to see (using the Chinese Remainder Theorem) that conversely, if $\mathbb{Z}/(n)$ is a local ring then $n$ is a prime power. 

	The ring $\mathbb{Z}_p$ of $p$-adic integers is a local ring. For any field $k$, the ring $k[[t]]$ of formal power series with coefficients in $k$ is a local ring. Both of these rings are also PIDs. A ring which is a local PID is called a discrete valuation ring. Note that a local ring is connected, i.e., $e^2 = e \Rightarrow e \in \{0,1\}$.

令$R$是$k[x]$, $I$是$(x^m)$, 有$rad(x^m)=(x)$是极大理想(由于$0\rightarrow (x) \rightarrow k[x]\rightarrow k \rightarrow 0$正合),从而$k[x]/(x^i)$是局部环。
\end{example}


% paragraph 第一部分 (end)
\paragraph{第二部分} % (fold)
\label{par:第二部分}
第二部分是考了系数环是$\mathbb{Z}$的情形,如何将上面的有限域和这里的整数环联系起来,就是用了一个相对$K$群的正合列。
% paragraph 第二部分 (end)




\section{Excision} % (fold)
\label{sec:excision}
excision失效就是说if $A \longrightarrow B$ is a morphism of rings and $I$ is an ideal of $A$ mapped isomorphically to an ideal of $B$,
then $K_n(A, I) \longrightarrow K_n(B, I)$ need not be an isomorphism. 由于这个不是同构,没法有Mayer-Vietoris序列
	\[\begin{tikzcd}
	\cdots \ar[r]&K_{i+1}(A/I) \ar[r] \ar[d]&	K_i(A,I) \ar[r,green] \ar[d]& K_i(A)\ar[r]\ar[d] & K_i(A/I)\ar[r] \ar[d] & \cdots\\
 \cdots \ar[r]&K_{i+1}(B/I) \ar[r,green] &	K_i(B,I) \ar[r] \ar[u, dashrightarrow, red, xshift=1ex] & K_i(B)\ar[r] & K_i(B/I)\ar[r] & \cdots
	\end{tikzcd}\]
要连接$K_n(A, I) \longrightarrow K_n(B, I)$就要考虑birelative $K$-groups(也称double relative $K$-groups),$K(A,B,I)$定义为homotpy fiber of the map $K(A, I) \longrightarrow K(B, I)$。
% section excision (end)














% section lower_bounds_for_the_order_of_ (end)