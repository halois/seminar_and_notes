%!TEX root = ../main.tex
\chapter{有限域总结}

基本的结果:
\begin{itemize}
	\item  Every finite field has prime power order. Every finite field must have characteristic $p$ for some prime $p$.
	\item For every prime power $q = p^n$, there is a finite field of that order. Any finite field with $q = p^n$ elements is isomorphic to the splitting field of $x^q - x$ over $\mathbb{F}_p$.
	\item Any two finite fields of the same size are isomorphic (usually not in just one way).
	\item A subfield of $\mathbb{F}_{p^n}$ has order $p^d$ where $d|n$, and there is one such subfield for each $d$.
	\item Let $F$ be a finite field containing a subfield $K$ with $q$ elements. Then $F$ has $q^m$ elements, where $m = [F : K]$.
	\item Let $F$ be a finite field. Then $F$ has $p^n$ elements, where the prime $p$ is the characteristic of $F$ and $n$ is the degree of $F$ over its prime subfield.
	\item For a prime $p$ and positive integer $n$, there is an irreducible $g(x)$ of degree $n$ in $\mathbb{F}_p[x]$, and $\mathbb{F}_p[x]/(g(x))$ is a field of order $p^n$. 这样构造的域也称为Galois域,the term Galois field lives on today among coding theorists in computer science and electrical engineering as a synonym for finite field and Moore’s notation $GF(q)$ is often used in place of $\mathbb{F}_q$.
	\item If $[\mathbb{F}_p(\alpha) : \mathbb{F}_p] = d$, the $\mathbb{F}_p$-conjugates of $\alpha$ are $\alpha,\alpha^p,\alpha^{p^2},\cdots,\alpha^{p^{d-1}}$.
	\item Every finite extension of $\mathbb{F}_p$ is a Galois extension whose Galois group over $\mathbb{F}_p$ is generated by the $p$th power map. The Galois group $Gal(\mathbb{F}_{p^n}/\mathbb{F}_p)$ is cyclic and a generator is the $p$-th power map $\phi_p\colon t \mapsto t^p$ on $\mathbb{F}_{p^n}$.
	\item If $F$ is a finite field with $q$ elements, then every $a \in F$ satisfies $a^q = a$.
	\item If $F$ is a finite field with $q$ elements and $K$ is a subfield of $F$, then the polynomial $x^q - x$ in $K[x]$ factors in $F[x]$ as $x^q -x= \prod_{a\in F}(x-a)$ and $F$ is a splitting field of $x^q -x$ over $K$.

	\item If $F$ is a finite field of order $q$, the group $F^\times$ is cyclic of order $q-1$.
\end{itemize}


例子:
\begin{itemize}
	\item $\mathbb{F}_4$ as $\mathbb{F}_2(\theta) = \{0, 1, \theta, \theta + 1\}$, where $\theta^2 + \theta + 1 = 0$, we find that both $\theta$ and $\theta + 1$ are primitive elements.
	\item  Two fields of order $8$ are $\mathbb{F}_2[x]/(x^3 + x + 1)$ and $\mathbb{F}_2[x]/(x^3 + x^2 + 1)$.
	\item Two fields of order $9$ are $\mathbb{F}_3[x]/(x^2 + 1)$ and $\mathbb{F}_3[x]/(x^2 + x + 2)$.
	\item The polynomial $x^3 - 2$ is irreducible in $\mathbb{F}_7[x]$, so $\mathbb{F}_7[x]/(x^3 - 2)$ is a field of order $7^3 = 343$.
\end{itemize}


警告:The ring $\mathbb{Z}/(m)$ is a field only when m is a prime number. In order to create fields of non-prime size we must do something other than look at $\mathbb{Z}/(m)$. Every finite field is isomorphic to a field of the form $\mathbb{F}_p[x]/(f(x))$



In the field $\mathbb{F}_3[x]/(x^2 +1)$, the nonzero numbers are a group of order $8$. The powers of $x$ only take on $4$ values, so $x$ is not a generator. The element $x + 1$ is a generator: its successive powers are exhaust all the nonzero elements
of $\mathbb{F}_3[x]/(x^2 +1)$.

\[
\begin{array}{c|cccccccc}
	k& 1&2&3&4&5&6&7&8 \\
	\hline
	x^k& x+1&2x&2x+1&2&2x+2&x&x+2&1
\end{array}
\]

 

计算 For every $f(x) \in \mathbb{F}_p[x]$, $f(x)^{p^m} = f(x^{p^m} )$ for $m \geq 0$.





Combinatorics. An important theme in combinatorics is $q$-analogues, which are algebraic expressions in a variable $q$ that become classical objects when $q = 1$, or when $q \mapsto 1$. For example, the $q$-binomial coefficient is
\[\binom{n}{k}_q = \frac{(q^n -1)(q^n -q)\cdots(q^n -q^{k-1})}{(q^k -1)(q^k -q)\cdots(q^k -q^{k-1})},\]
which for $n\geq k$ is a polynomial in $q$ with integer coefficients. When $q\mapsto 1$ this has
the value  $\binom{n}{k}$. While  $\binom{n}{k}$  counts the number of $k$-element subsets of a finite set, when $q$ is a prime power the number  $\binom{n}{k}_q$  counts the number of $k$-dimensional subspaces  of $\mathbb{F}_q^n$. Identities involving $q$-binomial coefficients can be proved by checking them when $q$ runs through prime powers, using linear algebra over the fields $\mathbb{F}_q$.




