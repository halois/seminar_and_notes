%!TEX root = ../main.tex
\chapter{$NK_{2}(\mathbb{F}_{2^f}[C_{2^n}])$}
English title:$NK_2$-group of $\mathbb{F}_{2^f}[C_{2^n}]$

 In this paper, we calculated $NK_2(\mathbb{F}_{2^f}[C_{2^n}])$ using the method from van der Kallen's paper. In particular, we also determined the explicit structure and $W(\mathbb{F}_2)$-module structure in Steinberg symbols of $NK_2(\mathbb{F}_{2}[C_{2}])$.

Keywords: $NK$-group, Dennis-Stein symbols, group algebra, Witt vectors

2010 Mathematics Subject Classification: 19C99, 16S34(Group rings)


\section{引言}
Bass和Murthy\cite{MR36:2671}首先给出了群$G$的Whitehead群$Wh(G)=K_1(\mathbb{Z}G)/\{\pm g | g\in G\}$不是有限生成的例子,他们的例子来源于计算某些环$R$的$NK$-群$NK_1(R)$。设$R$是含幺结合环,对于任意的$n\in \mathbb{Z}$,定义$NK_n(R)=\ker(K_n(R[x])\overset{x\mapsto 0}\longrightarrow K_n(R))$,其中$R[x]$是环$R$上的一元多项式环,当$n<0$时$K_{n}$是Bass定义的负$K$-理论,$n=0,1,2$时,$K_0,K_1,K_2$是由Grothendieck, Bass, Milnor等定义的经典$K$-理论,当$n\geq 2$时,$K_n$是由Quillen等定义的高阶$K$-理论。在计算环$R[\mathbb{Z}]=R[t,t^{-1}]$的$K$-理论中,$NK$-群作为$K_n(R[t,t^{-1}])$的直和项自然产生。1977年Farrell在文献\cite{MR56:8624}中证明了关于$NK$-群的重要性质,即若$NK_1(R)\neq 0$,则$NK_1(R)$不是有限生成的。实际上这一结论对于任何$NK_i$($i\leq 1$)均成立的,并且后来van der Kallen\cite{MR81g:18017}与Prasolov\cite{MR83m:18013}证明对任何$NK_i$($i> 1$)也是成立的。Weibel\cite{MR82k:18010}在Almkvist\cite{MR55:5721}与Grayson\cite{MR81e:13014}等的基础上将$NK$-群与Witt向量联系起来。对于交换正则环$R$,Weibel\cite{weibel2013k}给出了计算$NK_1(R[x]/(x^n))$的方法,而对于$R[x]/(x^n)$这样的截断多项式环,van der Kallen\cite{MR45:252}\cite{MR86f:18017}与Stienstra\cite{MR82k:13016}等对这类环的$K_2$群做了详细的研究。本文我们根据文献\cite{MR86f:18017}中的方法利用截断多项式的$K_2$群计算了$NK_2(\mathbb{F}_2[C_{2}])$,其中$\mathbb{F}_2[C_{2}]\cong \mathbb{F}_2[t]/(t^2)$,后者称为$\mathbb{F}_2$上的对偶数环,利用Dennis-Stein符号给出了$NK_2(\mathbb{F}_2[C_{2}])$的一组生成元,并给出了$W(\mathbb{F}_2)$-模结构。在不引起歧义的情况下,记$x$在$R[x]/(x^n)$下的像仍为$x$.


% 考虑把这一段加上
% local structure of algebraic k-theory 7.3.5
% Curves and Nil Terms
% If A is a ring, there is a close connection between finitely generated modules over
% A and over the polynomial ring A[t]. For instance, Serre’s problem asks whether
% finitely generated projective k[t 0 ,...,t n ]-modules are free when k is a field (that the
% answer is “yes” is the Quillen–Suslin theorem, [234, 275]). Consequently, there is a
% close connection between K(A) and K(A[t]), and the map K(A) → K(A[t]) is an
% equivalence if A is regular (finitely generated modules have finite projective dimen-
% sion) and A[s,t] is coherent (every finitely generated module is finitely presented),
% see e.g., [97] or [297] which cover a wide range of related situations.
% 7.3.5.1 The Algebraic K-Theory of the Polynomial Algebra
% In the general case, K(A) → K(A[t]) is not an equivalence, and one can ask ques-
% tions about the cofiber NK(A). By extending from the commutative case, we might
% think of A[t] as the affine line on A, and so NK(A) measures to what extent alge-
% braic K-theory fails to be “homotopy invariant” over A. In the regular Noetherian
% case we have that NK = 0, which is essential for the comparison with the motivic
% literature which is based on homotopy invariant definitions, as those of Karoubi–
% Villamayor [157, 158] or Weibel [307].
% The situation for topological Hochschild and cyclic homology is worse, in that
% TC(A) → TC(A[t]) and THH(A) → THH(A[t]) are rarely equivalences, regard-
% less of good regularity conditions on A (THH(A[t]) is accessible through the meth-
% ods of Sect. 7.3.7 below). That said, we still can get information about the K-theory
% nil-term NK(A). Let Nil A be the category of nilpotent endomorphisms of finitely
% generated projective A-modules. That is, an object of Nil A is a pair (P,f) where
% P is a finitely generated module and f : P → P is an A-module homomorphism
% for which there exist an n such that the nth iterate is trivial, f n = 0. The zero en-
% domorphisms split off, giving an equivalence K(Nil A ) ? K(A)∨Nil(A). By [108,
% p. 236] there is a natural equivalence NK(A)
% ∼
% → ?Nil(A), and it is the latter spec-
% trum which is accessible through trace methods.
% In particular, if A is a regular Noetherian F p -algebra, Hesselholt and Madsen
% [130] give a description of Nil(A[t]/t n ) in terms of the big de Rham–Witt complex
% of Sect. 7.3.4.




% english version
% 
% In \cite{MR36:2671}, Bass and Murthy gave the first examples of groups $G$ such that
% $Wh_1(G)$ is not finitely generated result from calculating $NK_1(R)$ for certain rings $R$. Let $R$ be a unital ring, define $NK_n(R)=\ker(K_n(R[x])\overset{x\mapsto 0}\longrightarrow K_n(R))$ for any $n\in \mathbb{Z}$ where $R[x]$ is the polynomial ring over $R$. When $n<0$, $K_n$ is the negative $K$-theory defined by Bass, while $K_n$ is defined by Quillen's $Q$-construction for $n\geq 0$. In \cite{MR56:8624}, Farrell proved an important property of $NK$-group, that is, if $NK_1(R)\neq 0$, then $NK_1(R)$ is not finitely generated. 









\section{预备知识}
令$k$是特征为$p>0$的有限域, 考虑两个变元的多项式环$k[t_1, t_2]$, 设$I=(t_1^n)$是$k[t_1, t_2]$的一个真理想
% , 满足以下条件
% \begin{enumerate}
% 	\item $I$是由$k[t_1]$中的单项式生成的, 
% 	\item 对于某个$n$, $t_1^n\in I$. 
% \end{enumerate}

% 实际上这样的$I$具有形式$(t_1^n)$, 
令
\[A=k[t_1,t_2]/I,\]
设$M$是$A$的nil根(小根), 即$M=(t_1)$, 则有$A/M=k[t_2]$. 
\begin{prop}
	$K_2(k[t_1,t_2]/(t_1^n),(t_1,t_2))=K_2(A,(t_1,t_2))\cong K_2(A,M)=K_2(k[t_1,t_2]/(t_1^n),(t_1))$
\end{prop}
\begin{proof}
	首先有下面两个相对$K$群的正合列
	\[
	0\longrightarrow K_2(k[t_1,t_2]/(t_1^n),(t_1)) \longrightarrow K_2(k[t_1,t_2]/(t_1^n)) \longrightarrow K_2(k[t_2]) \longrightarrow 0
	\]
	\[
	0\longrightarrow K_2(k[t_1,t_2]/(t_1^n),(t_1,t_2)) \longrightarrow K_2(k[t_1,t_2]/(t_1^n)) \longrightarrow K_2(k) \longrightarrow 0
	\]

	由于$k$是有限域,它是正则环,于是有$K_2(k)=0$, $K_2(k[t_2])=K_2(k)\oplus NK_2(k)=0$. 从而可以得到
	\[K_2(k[t_1,t_2]/(t_1^n),(t_1))\cong K_2(k[t_1,t_2]/(t_1^n)) \cong K_2(k[t_1,t_2]/(t_1^n),(t_1,t_2)).\]
\end{proof}

当$k=\mathbb{F}_{p^f}$时,$k[t_1]/(t_1^{p^n})\cong \mathbb{F}_{p^f}[C_{p^n}]$,其中$C_{p^n}$是$p^n$阶循环群。定义$NK_i(R)=\ker(NK_i(R[x])\overset{x\mapsto 0}\longrightarrow K_i(R))$,于是有
\[0\longrightarrow NK_2(\mathbb{F}_{p^f}[C_{p^n}]) \longrightarrow K_2(\mathbb{F}_{p^f}[C_{p^n}][x])\longrightarrow K_2(\mathbb{F}_{p^f}[C_{p^n}]) \longrightarrow 0,\]
由于$K_2(\mathbb{F}_{p^f}[C_{p^n}][x])=K_2(\mathbb{F}_{p^f}[t_1,t_2]/(t_1^{p^n}))$,并且$K_2(\mathbb{F}_{p^f}[C_{p^n}])=0$,从而
\[NK_2(\mathbb{F}_{p^f}[C_{p^n}])\cong K_2(\mathbb{F}_{p^f}[t_1,t_2]/(t_1^{p^n})) \cong K_2(\mathbb{F}_{p^f}[t_1,t_2]/(t_1^{p^n}),(t_1)).\]


\subsection{Dennis-Stein符号}
\label{sub:dennis_stein_symbols}
回到一般情形,$K_2(A,M)=K_2(k[t_1,t_2]/(t_1^n),(t_1))$可以用Dennis-Stein符号表示

\begin{itemize}
	\item[生成元]   $\langle a,b \rangle $, $(a,b)\in A\times M \cup M \times A$;
	\item[关系] $\langle a,b\rangle = -\langle b,a \rangle$, \\
	$\langle a,b\rangle +\langle c,b \rangle=\langle a+c-abc,b\rangle$, \\
	 $\langle a,bc\rangle =\langle ab,c\rangle +\langle ac,b\rangle$ 其中$(a,b,c)\in A\times M \times A \cup M\times A\times M$.
\end{itemize}

\begin{prop}
	对任意环$R$,$q\geq 1$,$K_2(R[t]/(t^q),(t))$由Dennis-Stein符号$\langle at^i,t\rangle$和$\langle at^i,b\rangle$生成,其中$a,b\in R, 1\leq i<q$。
\end{prop}
\begin{proof}
	参见文献\cite{MR82k:13016}。
\end{proof}
	% 接下来给出一个生成元和关系更少的表现
	% \begin{theorem}
	% 	$K_2(A,M)$作为交换群有如下表现

	% 	\begin{itemize}
	% 	\item[生成元]   $\langle f,t_i \rangle $, $i=1,2$ $(f,t_i)\in A\times M \cup M \times A$;
	% 	\item[关系] $\langle f,t_i\rangle +\langle g,t_i \rangle=\langle f+g-fgt_i,t_i\rangle$, 当$t_i f,t_i g \in M$, \\
	% 	令$t^{\alpha}=t_1^{\alpha_1}t_2^{\alpha_2}\in M =(t_1)$, $f(x)\in k[x]$, 则$\alpha_1 \langle f(t^{\alpha})t_1^{\alpha_1-1}t_2^{\alpha_2},t_1\rangle +\alpha_2 \langle f(t^{\alpha}){t_1}^{\alpha_1}t_2^{\alpha_2-1},t_2\rangle =0$, $ \alpha_i\geq 1.$
	% \end{itemize}
	% \end{theorem}
% \begin{corollary}
% 	令$\mathcal{B}$是$k$作为$\mathbb{F}_p$上向量空间的一组基,则$K_2(A,M)$作为交换群有如下表现
% 	\begin{itemize}
% 	\item[生成元]   $\langle bt^{\alpha- \epsilon^i},t_i \rangle $, $b\in \mathcal{B}$, $(\alpha,i)\in 	\Lambda^0$;
% 	\item[关系] $p^{w(\alpha,i)}\langle bt^{\alpha- \epsilon^i},t_i \rangle =0$.
% \end{itemize}
% \end{corollary}

\subsection{符号说明} % (fold)
\label{subsec:符号}
为了表述方便,遵从\cite{MR86f:18017}的符号详述如下\\
\begin{itemize}
	\item $\mathbb{Z}_+$: 非负整数全体, 
	\item $\epsilon^1 = (1,0)\in \mathbb{Z}_+^2$, $\epsilon^2 = (0,1)\in \mathbb{Z}_+^2$,
	\item 对于$\alpha \in \mathbb{Z}_+^2$, 记$t^{\alpha}=t_1^{\alpha_1}t_2^{\alpha_2}$, 于是有$t^{\epsilon^1}=t_1$, $t^{\epsilon^2}=t_2$,
	\item $\Delta=\{\alpha\in\mathbb{Z}_+^2\mid  t^{\alpha}\in I\}$,
	\item $\Lambda=\{(\alpha,i)\in\mathbb{Z}_+^2 \times \{1,2\}\mid  \alpha_i\geq 1, t^{\alpha}\in M\}$, 若$\delta \in \Delta$, 则有$\delta+\epsilon^i \in \Delta, I=1,2$,
	\item 对于$(\alpha,i)\in\Lambda$, 令$[\alpha,i]=\min\{m\in \mathbb{Z}\mid m\alpha - \epsilon^i\in \Delta\}$, %$w(\alpha,i)=\min\{w\in \mathbb{Z}_+\mid  p^w \geq [\alpha,i]\}$, 
	若$(\alpha,i),(\alpha,j)\in \Lambda$, 有$[\alpha,i]\leq [\alpha,j]+1$,
	\item 若$gcd(p,\alpha_1,\alpha_2)=1$, 令$[\alpha]=\max\{[\alpha,i]\mid  \alpha_i  \not\equiv 0 \bmod p\}$
	\item $\Lambda^{00}= \big\{(\alpha,i)\in \Lambda\mid  gcd(\alpha_1,\alpha_2)=1, i\neq \min\{j\mid \alpha_j\not\equiv 0 \bmod p, [\alpha,j]=[\alpha]\} \big\}$.
	% \item $\Lambda^{0}= \{(m\alpha,i)\in \Lambda\mid  gcd(m,p)=1, (\alpha,i)\in \Lambda^{00}\}$.
\end{itemize}
% subsection 符号 (end)
若$(\alpha,i)\in \Lambda$, $f(x)\in k[x]$, 令
\[\Gamma_{\alpha,i}(1-xf(x))= \langle f(t^\alpha)t^{\alpha-\epsilon^i},t_i \rangle,\]
若$g(t_1,t_2)=t_ih(t_1,t_2)\in \sqrt{I}=(t_1)$, 令
\[\Gamma_i(1-g(t_1,t_2))=\langle h(t_1,t_2),t_i \rangle,\]
且有
\[\Gamma_{\alpha,i}(1-xf(x))=\Gamma_i(1-t^{\alpha} f(t^{\alpha})).\]

由于$t_1\in \sqrt{I}$, $\Gamma_1$诱导了同态
\begin{align*}
(1+t_1k[t_1,t_2]/t_1 I)^{\times} &\longrightarrow K_2(A,M)\\
1-g(t_1,t_2) & \mapsto \langle h(t_1,t_2),t_1 \rangle
\end{align*}
$\Gamma_2$诱导了同态
\begin{align*}
(1+t_2\sqrt{I}/t_2 I)^{\times} &\longrightarrow K_2(A,M)\\
1-g(t_1,t_2) & \mapsto \langle h(t_1,t_2),t_2 \rangle
\end{align*}
若$(\alpha,i)\in\Lambda$, $\Gamma_{\alpha,i}$诱导了同态
\[(1+xk[x]/(x^{[\alpha,i]}))^{\times} \longrightarrow K_2(A,M).\]
\begin{theorem}
\label{K2(A,M)}
	$\Gamma_{\alpha,i}$诱导了同构
\[ K_2(A,M)\cong \bigoplus_{(\alpha,i)\in\Lambda^{00}}(1+xk[x]/(x^{[\alpha,i]}))^{\times}.\]
\end{theorem}
\begin{proof}
	参见文献\cite{MR86f:18017}。
\end{proof}

\subsection{Witt向量}
令$R$是一个交换环,big Witt环(the ring of universal/big Witt vectors over $R$,泛Witt环)$BigWitt(R)$作为Abel群同构于$(1+xR\llbracket x\rrbracket )^{\times}$,即常数项为$1$的形式幂级数全体在乘法运算下形成的交换群,
\begin{align*}
BigWitt(R) &\longrightarrow (1+xR\llbracket x\rrbracket )^{\times}\\
(r_1,r_2,\cdots) & \mapsto \prod_i(1-r_i x^i)^{-1}.
\end{align*}



考虑子群$(1+x^{n+1}R\llbracket x\rrbracket )^{\times}$,定义$BigWitt_n(R)=(1+xR\llbracket x\rrbracket )^{\times}/(1+x^{n+1}R\llbracket x\rrbracket )^{\times}$。 显然$BigWitt_1(R)=R$, 并且当$n\geq 3$时,$BigWitt_n(\mathbb{F}_2)$不是循环群。


\begin{example} 
\label{ex:W3(F2)}
	$BigWitt_3(\mathbb{F}_2)\cong (1+x\mathbb{F}_2[x]/(x^{4}))^{\times}\cong \mathbb{Z}/4 \mathbb{Z} \oplus\mathbb{Z}/2 \mathbb{Z}$
\end{example}
\begin{proof}
	由定义
$BigWitt_3(\mathbb{F}_2)=(1+x \mathbb{F}_2\llbracket x\rrbracket )^{\times}/(1+x^4 \mathbb{F}_2\llbracket x\rrbracket )^{\times}$,且有同态
\begin{align*}
(1+x \mathbb{F}_2\llbracket x\rrbracket )^{\times} &\longrightarrow (1+x \mathbb{F}_2[x]/(x^4))^{\times}\\
1+\sum_{i\geq 1}a_i x^i &\mapsto 1+a_1x+a_2x^2+a_3x^3
\end{align*}
它的核是$(1+x^4 \mathbb{F}_2\llbracket x\rrbracket )^{\times}$.
从而$(1+x \mathbb{F}_2[x]/(x^4))^{\times} \cong BigWitt_3(\mathbb{F}_2)=(1+x \mathbb{F}_2\llbracket x\rrbracket )^{\times}/(1+x^4 \mathbb{F}_2\llbracket x\rrbracket )^{\times}$.

考虑$1+x\in (1+x \mathbb{F}_2[x]/(x^4))^{\times}$是$4$阶元,由它生成的子群$\langle 1+x \rangle = \{1,1+x,1+x^2,1+x+x^2+x^3\}$, 且$1+x^3$是二阶元,令$\sigma,\tau$分别是$\mathbb{Z}/4 \mathbb{Z}$和$\mathbb{Z}/2 \mathbb{Z}$的生成元,则有同构
	\begin{align*}
	\mathbb{Z}/4 \mathbb{Z} \oplus \mathbb{Z}/2 \mathbb{Z} &\longrightarrow BigWitt_4(\mathbb{F}_2) \\
	(\sigma,\tau) & \mapsto (1+x)(1+x^3)=1+x+x^3. \\
	\end{align*}
\end{proof}

\begin{example}
	$BigWitt_4(\mathbb{F}_2) \cong \mathbb{Z}/8 \mathbb{Z} \oplus \mathbb{Z}/2 \mathbb{Z}.$
\end{example}
\begin{proof}
由定义
$BigWitt_4(\mathbb{F}_2)=(1+x \mathbb{F}_2\llbracket x\rrbracket )^{\times}/(1+x^5 \mathbb{F}_2\llbracket x\rrbracket )^{\times}$,且有同态
\[(1+x \mathbb{F}_2\llbracket x\rrbracket )^{\times} \longrightarrow (1+x \mathbb{F}_2[x]/(x^5))^{\times}\]
它的核是$(1+x^5 \mathbb{F}_2\llbracket x\rrbracket )^{\times}$.
从而$(1+x \mathbb{F}_2[x]/(x^5))^{\times} \cong BigWitt_4(\mathbb{F}_2)=(1+x \mathbb{F}_2\llbracket x\rrbracket )^{\times}/(1+x^5 \mathbb{F}_2\llbracket x\rrbracket )^{\times}$.

	考虑$1+x \in BigWitt_5(\mathbb{F}_2)$, 它是$8$阶元,由它生成的子群$\langle 1+x \rangle = \{1,1+x,1+x^2,1+x+x^2+x^3,1+x^4,1+x+x^4,1+x^2+x^4,1+x+x^2+x^3+x^4\}$,另外$1+x^3$是二阶元,令$\sigma,\tau$分别是$\mathbb{Z}/8 \mathbb{Z}$和$\mathbb{Z}/2 \mathbb{Z}$的生成元,则有同构
	\begin{align*}
	\mathbb{Z}/8 \mathbb{Z} \oplus \mathbb{Z}/2 \mathbb{Z} &\longrightarrow BigWitt_4(\mathbb{F}_2) \\
	(\sigma,\tau) & \mapsto (1+x)(1+x^3)=1+x+x^3+x^4 \\
	\end{align*}
	于是$(\sigma^i,\tau^j), 0\leq i <8, 0\leq j<2$对应于$(1+x)^i(1+x^3)^j$, 详细的对应如下
	\begin{align*}
	(1,\tau) & \mapsto 1+x^3,& (\sigma,\tau) & \mapsto 1+x+x^3+x^4, \\
	 (\sigma^2,\tau) & \mapsto 1+x^2+x^3,& (\sigma^3,\tau) & \mapsto 1+x+x^2+x^4, \\
	(\sigma^4,\tau) & \mapsto 1+x^3+x^4, & (\sigma^5,\tau) & \mapsto 1+x+x^3, \\
	 (\sigma^6,\tau) & \mapsto 1+x^2+x^3+x^4, & (\sigma^7,\tau) & \mapsto 1+x+x^2, \\
	(1,1)& \mapsto 1,& (\sigma,1) & \mapsto 1+x, \\
	(\sigma^2,1) & \mapsto 1+x^2, & (\sigma^3,1) & \mapsto 1+x+x^2+x^3, \\
	(\sigma^4,1) & \mapsto 1+x^4, &
	(\sigma^5,1) & \mapsto 1+x+x^4, \\
	(\sigma^6,1) & \mapsto 1+x^2+x^4, & (\sigma^7,1) & \mapsto 1+x+x^2+x^3+x^4. \\
	\end{align*}

\end{proof}



固定素数$p$,考虑局部环$\mathbb{Z}_{(p)}=\mathbb{Z}[1/\ell \mid \text{所有素数}\ell\neq p]$,即$\mathbb{Z}$在素理想$(p)=p \mathbb{Z}$的局部化,于是一个$\mathbb{Z}_{(p)}$-代数$R$就是除$p$外的素数均可逆的交换环,如$\mathbb{F}_{p^n}$是$\mathbb{Z}_{(p)}$-代数。

考虑$p$-Witt环$W(A)$与截断$p$-Witt环$W_n(A)$,$p$-Witt向量为$(a_0,a_1,\cdots)$,加法用Witt多项式定义,以下仅考虑用加法定义的Abel群结构,例如$W(\mathbb{F}_p)=\mathbb{Z}_{p}$,作为Abel群$W_n(\mathbb{F}_{p^f})$同构于$(\mathbb{Z}/p^n\mathbb{Z})^f$。%这里可以加上Galoishuh

Artin-Hasse级数定义为
\[AH(x)= \exp(-\sum_{n\geq 0}\frac{x^{p^n}}{p^n})=1-x+\cdots \in 1+x \mathbb{Q}\llbracket x\rrbracket ,\]
实际上$AH(x)\in 1+x \mathbb{Z}_{(p)}\llbracket x\rrbracket $。对于$BigWitt(R)=1+xR\llbracket x\rrbracket $中的任一元素$\alpha$存在以下写成无穷乘积的表法
\[\alpha = \prod_{n\geq 1}(1-r_nx^n),\ r_n\in R,\]
若$A$是$\mathbb{Z}_{(p)}$-代数,$BigWitt(A)=1+xA\llbracket x\rrbracket $中的任一元素$\alpha$还有如下表法
\[\alpha = \prod_{n\geq 1}AH(a_n x^n),\ a_n\in A.\]
将整数$n$写成$n=mp^a$,使得$gcd(m,p)=1, a\geq 0$,由于$A$是$\mathbb{Z}_{(p)}$-代数,$m$可逆,从而$[x\mapsto x^{1/m}]\in \End(BigWitt(A))$是双射,于是我们可以将$\alpha\in BigWitt(A)$以如下的形式表出
\[\prod_{\scriptsize\substack{m\geq 1 \\ gcd(m,p)=1  \\ a\geq 0}}AH(a_{mp^a} x^{mp^a})^{1/m}.\]

另一方面对于$\mathbb{Z}_{(p)}$-代数$A$,下列映射是群同态
\begin{align*}
W(A)&\longrightarrow BigWitt(A)\\
(a_0,a_1,\cdots) &\mapsto \prod_{i\geq 0}AH(a_i x^i).
\end{align*}

$BigWitt_n(A)$可以分解为$p$-Witt环的直和,实际上有以下同构
\[
BigWitt(A) \cong \prod_{\scriptsize\substack{m\geq 1 \\ gcd(m,p)=1}} W(A),
\]
元素$\prod\limits_{\scriptsize\substack{m\geq 1 \\ gcd(m,p)=1  \\ a\geq 0}}AH(a_{mp^a} x^{mp^a})^{1/m}$对应于一个$m$-分量为$(a_m,a_{mp},a_{mp^2},\cdots)\in W(A)$的Witt向量。
对于截断的Witt环,有同构\[
BigWitt_n(A) \cong \bigoplus_{\scriptsize\substack{1\leq m\leq n \\ gcd(m,p)=1}} W_{\ell(m,n)}(A),
\]
其中$\ell(m,n)$是一个整数,定义为
\[\ell(m,n)=1+\text{使得$mp^k\leq n$成立的最大整数$k$}.\]

考虑特征为$p$的有限域$\mathbb{F}_q$,有同构\cite{Lauter1999A}
\[BigWitt_n(\mathbb{F}_{q}) \cong \bigoplus_{\scriptsize\substack{1\leq m\leq n \\ gcd(m,p)=1}} W_{\ell(m,n)}(\mathbb{F}_{q}),\]
注意到两边都是$q^n$阶的群,因为$\sum\limits_{\scriptsize\substack{1\leq m\leq n \\ gcd(m,p)=1}} \ell(m,n) = n$。
\begin{corollary}
\label{cor:BW}
	若有限域$\mathbb{F}_{p^f}$的特征$ch(\mathbb{F}_{p^f})=p$,则作为Abel群有
	\[
	BigWitt_n(\mathbb{F}_{p^f})\cong \bigoplus_{\scriptsize\substack{1\leq m\leq n \\ gcd(m,p)=1}}W_{1+ \left \lfloor\log_p \frac{n}{m}  \right \rfloor}(\mathbb{F}_{p^f}) = \bigoplus_{\scriptsize\substack{1\leq m\leq n \\ gcd(m,p)=1}}(\mathbb{Z}/p^{1+ \left \lfloor\log_p \frac{n}{m}  \right \rfloor}\mathbb{Z})^f,
	\]
	其中$ \left \lfloor x \right \rfloor$表示不超过$x$
	% $ \left \lfloor\log_p \frac{n}{m}  \right \rfloor$表示不超过$\log_p \frac{n}{m}$
	的最大整数。
\end{corollary}
\begin{example}
	$BigWitt_3(\mathbb{F}_2)= W_{\ell(1,3)}(\mathbb{F}_2)\oplus W_{\ell(3,3)}(\mathbb{F}_2)=W_2(\mathbb{F}_2)\oplus W_1(\mathbb{F}_2)=\mathbb{Z}/4 \mathbb{Z}\oplus	\mathbb{Z}/2 \mathbb{Z}$,$BigWitt_4(\mathbb{F}_2)= W_{\ell(1,4)}(\mathbb{F}_2)\oplus W_{\ell(3,4)}(\mathbb{F}_2)=W_3(\mathbb{F}_2)\oplus W_1(\mathbb{F}_2)=\mathbb{Z}/8 \mathbb{Z}\oplus	\mathbb{Z}/2 \mathbb{Z}$,$BigWitt_2(\mathbb{F}_3)= W_{\ell(1,2)}(\mathbb{F}_3)\oplus W_{\ell(2,2)}(\mathbb{F}_3)=W_1(\mathbb{F}_3)\oplus W_1(\mathbb{F}_3)=\mathbb{Z}/3 \mathbb{Z}\oplus	\mathbb{Z}/3 \mathbb{Z}$。

\end{example}










\section{$NK_2(\mathbb{F}_2[C_2])$} % (fold)
这一节计算$k=\mathbb{F}_2$, $p=2$, $n=2$的情形,即$NK_2(\mathbb{F}_2[C_2])\cong K_2(\mathbb{F}_2[t_1,t_2]/(t_1^2),(t_1))$.
\begin{theorem}
	(1)$NK_2(\mathbb{F}_2[C_2])\cong \bigoplus_{\infty} \mathbb{Z}/2 \mathbb{Z}$,\\
	(2)$NK_2(\mathbb{F}_2[C_2])\cong K_2(\mathbb{F}_2[t,x]/(t^2),(t))$是由Dennis-Stein符号$\{\langle tx^i,x \rangle \mid i\geq 0\}$与$\{\langle tx^i,t \rangle \mid i\geq 1\text{为奇数}\}$生成的,这样的符号均为$2$阶元。
\end{theorem}
\begin{proof}
	(1)令$A=\mathbb{F}_2[t_1,t_2]/(t_1^2)=\mathbb{F}_2[C_{2}][x]$, 此时$I=(t_1^2)$, $M=(t_1)$, $A/M=\mathbb{F}_2[x]$. 
\begin{align*}
\Delta &=\{(\alpha_1,\alpha_2)\in\mathbb{Z}_+^2\mid  t_1^{\alpha_1}t_2^{\alpha_2}\in (t_1^2)\}\\
	&=\{(\alpha_1,\alpha_2)\mid \alpha_1\geq 2, \alpha_2 \geq 0\},
\end{align*}

\begin{align*}
\Lambda &=\{((\alpha_1,\alpha_2),i)\in\mathbb{Z}_+^2 \times \{1,2\}\mid \alpha_i\geq 1, \text{且} t_1^{\alpha_1}t_2^{\alpha_2}\in (t_1)\} \\
	&=\{((\alpha_1,\alpha_2),i)\in\mathbb{Z}_+^2 \times \{1,2\}\mid \alpha_i\geq 1, \alpha_1\geq 1, \alpha_2\geq 0\} \\
	&=\{((\alpha_1,\alpha_2),1) \mid \alpha_1\geq 1, \alpha_2\geq 0\}\cup \{((\alpha_1,\alpha_2),2) \mid \alpha_1\geq 1, \alpha_2\geq 1\},
\end{align*}

若$(\alpha,i)\in \Lambda$, $[\alpha,i] = \min \{m\in \mathbb{Z}\mid m \alpha -\epsilon^i \in \Delta\}$, 于是有
\begin{align*}
[\alpha,1] & = \min \{m\in \mathbb{Z} \mid m \alpha -\epsilon^1 \in \Delta\} \\
& =\min \{m\in \mathbb{Z} \mid (m \alpha_1-1,m \alpha_2)\in \Delta\} \\
& =\min \{m\in \mathbb{Z} \mid m \alpha_1\geq 3\}.
\end{align*}
\begin{align*}
[\alpha,2] & = \min \{m\in \mathbb{Z} \mid m \alpha -\epsilon^2 \in \Delta\} \\
& =\min \{m\in \mathbb{Z} \mid (m \alpha_1,m \alpha_2-1)\in \Delta\} \\
& =\min \{m\in \mathbb{Z} \mid m \alpha_1\geq 2\}.
\end{align*}
此时
\begin{align*}
[(1,\alpha_2),1] & = 3, \ \alpha_2\geq 0, \\
[(2,\alpha_2),1] & = 2, \ \alpha_2\geq 0, \\
[(\alpha_1,\alpha_2),1] & = 1, \ \alpha_1\geq 3, \alpha_2\geq 0, \\
[(1,\alpha_2),2] & = 2, \ \alpha_2\geq 1, \\
[(\alpha_1,\alpha_2),2] & = 1, \ \alpha_1\geq 2, \alpha_2\geq 1.
\end{align*}

% ------------

% 由于此时$p=char(\mathbb{F}_2)=2$,
% \[w(\alpha,i)=\min \{w\in \mathbb{Z}_+\mid  2^w \geq [\alpha,i]\},\]
% 同理可以得到
% \begin{align*}
% w((1,\alpha_2),1) & = 2, \ \alpha_2\geq 0, \\
% w((2,\alpha_2),1) & = 1, \ \alpha_2\geq 0, \\
% w((\alpha_1,\alpha_2),1) & = 0, \ \alpha_1\geq 3, \alpha_2\geq 0, \\
% w((1,\alpha_2),2) & = 1, \ \alpha_2\geq 1, \\
% w((\alpha_1,\alpha_2),2) & = 0, \ \alpha_1\geq 2, \alpha_2\geq 1.
% \end{align*}

% ----------

若$gcd(2,\alpha_1,\alpha_2)=1$,即$\alpha_1,\alpha_2$中至少一个是奇数,令$[\alpha]=\max\{[\alpha,i]\mid  \alpha_i  \not\equiv 0 \bmod 2\}$, $\alpha=(\alpha_1,\alpha_2)$,若仅$\alpha_1$是奇数,$[\alpha]=[\alpha,1]$,若仅$\alpha_2$是奇数,$[\alpha]=[\alpha,2]$,若两者均为奇数,则$[\alpha]=\max\{[\alpha,1],[\alpha,2]\}$,有
\begin{align*}
[(1,\alpha_2)]&=\max\{[(1,\alpha_2),1],[(1,\alpha_2),2]\}=3, \text{$\alpha_2 \geq 1$是奇数}\\
[(1,\alpha_2)]&=[(1,\alpha_2),1]=3, \text{$\alpha_2 \geq 0$是偶数}\\
[(3,\alpha_2)]&=\max\{[(3,\alpha_2),1],[(3,\alpha_2),2]\}=1,\text{$\alpha_2\geq 1$是奇数}\\
[(3,\alpha_2)]&=[(3,\alpha_2),1]=1,\text{$\alpha_2\geq 0$是偶数}\\
[(2,1)]&=[(2,1),2]=1,\\
[\alpha]&=1,\text{其它符合条件的$\alpha$.}
\end{align*}
为了方便我们把上面的计算结果列表如下
\[\begin{array}{|c|c|c|c|}
\hline
(\alpha_1,\alpha_2) & [(\alpha_1,\alpha_2),1] &[(\alpha_1,\alpha_2),2] &[(\alpha_1,\alpha_2)]  \\
\hline
(1,\alpha_2)  & 3, \alpha_2 \geq 0& 2 , \alpha_2\geq 1 & 3 \\
\hline
(2,\alpha_2)  & 2, \alpha_2 \geq 0& 1 , \alpha_2\geq 1 & 1, \text{当$\alpha_2$是奇数时} \\
\hline
(3,\alpha_2)  & 1, \alpha_2 \geq 0& 1 , \alpha_2\geq 1 & 1 \\
\hline
(\alpha_1,0),\alpha_1 \geq 3 & 1 & \text{无定义} &1, \text{当$\alpha_1$是奇数时} \\
\hline
(\alpha_1,\alpha_2),\alpha_1 \geq 3, \alpha_2 \geq 1& 1 & 1& 1, \text{当$(\alpha_1,\alpha_2)=1$时} \\
\hline
\end{array}\]


下面我们计算$\Lambda^{00}=\big\{(\alpha,i)\in \Lambda\mid  gcd(\alpha_1,\alpha_2)=1, i\neq \min\{j\mid \alpha_j\not\equiv 0 \bmod 2,[\alpha,j]=[\alpha]\} \big\}$,

分情况来讨论
\begin{enumerate}
	\item 对于任何的$\alpha_2\geq 0$, $((1,\alpha_2),1) \not \in \Lambda^{00}$,这是因为$1\not\equiv 0 \bmod 2$且$[(1,\alpha_2),1]=3=[(1,\alpha_2)]$,从而$\min\{j\mid \alpha_j\not\equiv 0 \bmod 2,[(1,\alpha_2),j]=[(1,\alpha_2)]\}=1$;
	\item 对于任何的奇数$\alpha_2\geq 0$,$((2,\alpha_2),1) \in \Lambda^{00}$,偶数$\alpha_2\geq 0$,$((2,\alpha_2),1) \not \in \Lambda^{00}$, 因为$\alpha_2\not\equiv 0 \bmod 2$并且$[(2,\alpha_2),2]=1=[(2,\alpha_2)]$,故$\{j\mid \alpha_j\not\equiv 0 \bmod 2,[(2,\alpha_2),j]=[(2,\alpha_2)]\}=2\neq 1$,此时$[(2,\alpha_2),1]=2$;
	\item 对于偶数$\alpha_1\geq 3$和奇数$\alpha_2\geq 1$,$((\alpha_1,\alpha_2),1)  \in \Lambda^{00}$,其余情况当$\alpha_1\geq 3$为奇数或$\alpha_1,\alpha_2$均为偶数时$((\alpha_1,\alpha_2),1) \not  \in \Lambda^{00}$。由于要求$1\neq \min\{j\mid \alpha_j\not\equiv 0 \bmod 2,[(\alpha_1,\alpha_2),j]=[(\alpha_1,\alpha_2)]\}$,当$\alpha_1\geq 3$为奇数时上式不成立,$2= \min\{j\mid \alpha_j\not\equiv 0 \bmod 2,[(\alpha_1,\alpha_2),j]=[(\alpha_1,\alpha_2)]\}$当且仅当$\alpha_1\geq 3$为偶数且$\alpha_2\geq 1$为奇数,此时$[(\alpha_1,\alpha_2),1]=1$;
	\item 对于任何的$\alpha_2\geq 1$, $((1,\alpha_2),2) \in \Lambda^{00}$,由于此时$[(1,\alpha_2),1]=3=[(1,\alpha_2)]$,$\min\{j\mid \alpha_j\not\equiv 0 \bmod 2,[\alpha,j]=[\alpha]\}=1$,此时$[(1,\alpha_2),2]=2$;
	\item 对于任何的奇数$\alpha_2\geq 1$, $((2,\alpha_2),2) \not \in \Lambda^{00}$,由于$[(2,\alpha_2),2]=1=[(2,\alpha_2)]$,与$2\neq \min\{j\mid \alpha_j\not\equiv 0 \bmod 2,[\alpha,j]=[\alpha]\}$矛盾;
	\item 对于奇数$\alpha_1\geq 3$和任意$\alpha_2\geq 1 $,$((\alpha_1,\alpha_2),2)  \in \Lambda^{00}$,其余情况只要当$\alpha_1\geq 3$为偶数时$((\alpha_1,\alpha_2),2) \not  \in \Lambda^{00}$。要求$2\neq \min\{j\mid \alpha_j\not\equiv 0 \bmod 2,[(\alpha_1,\alpha_2),j]=[(\alpha_1,\alpha_2)]\}$,当$\alpha_1$为偶数时上式不成立,而当$\alpha_1$为奇数时,任意$\alpha_2\geq 1$,$[(\alpha_1,\alpha_2),1]=1=[(\alpha_1,\alpha_2)]$,此时$[(\alpha_1,\alpha_2),2]=1$。
\end{enumerate}



从而
\begin{align*}
\Lambda^{00}=&\{((2,\alpha_2),1)\mid  \alpha_2\geq 1\text{为奇数}\} \\
	&\cup \{((1,\alpha_2),2)\mid  \alpha_2\geq 1\} \\
	&\cup \{((\alpha_1,\alpha_2),1) | \alpha_1\geq 3\text{为偶数},\alpha_2\geq 1\text{为奇数}\} \\
	&\cup \{((\alpha_1,\alpha_2),2) | \alpha_1\geq 3\text{为奇数},\alpha_2\geq 1\}.
\end{align*}
记$\Lambda^{00}_1=\{(\alpha,i)\in \Lambda^{00}| [(\alpha,i)]=1\}$,$\Lambda^{00}_2=\{(\alpha,i)\in \Lambda^{00}| [(\alpha,i)]=2\}$,我们有
\[\Lambda^{00}_1= \{((\alpha_1,\alpha_2),1) | \alpha_1\geq 3\text{为偶数},\alpha_2\geq 1\text{为奇数}\} \cup \{((\alpha_1,\alpha_2),2) | \alpha_1\geq 3\text{为奇数},\alpha_2\geq 1\}\]
\[\Lambda^{00}_2=\{((2,\alpha_2),1)\mid  \alpha_2\geq 1\text{为奇数}\} \cup \{((1,\alpha_2),2)\mid  \alpha_2\geq 1\}\]
\[\Lambda^{00}=\Lambda^{00}_1 \sqcup \Lambda^{00}_2\]
% \[\Lambda^0=\{(m\alpha,i)\in \Lambda\mid  gcd(m,2)=1, (\alpha,i)\in \Lambda^{00}\}=\{(m\alpha,2)\mid m\text{奇数},gcd(\alpha_1,\alpha_2)=1,\alpha_1\geq 1, \alpha_2\geq 1\}.\]

若$[\alpha,i]=1$时,$(1+x\mathbb{F}_{2}[x]/(x))^{\times}$是平凡的,$[\alpha,i]=2$时,$(1+x\mathbb{F}_{2}[x]/(x^{2}))^{\times}\cong \mathbb{Z}/2 \mathbb{Z}$,从而由定理\ref{K2(A,M)}得

\begin{align*}
NK_2(\mathbb{F}_2[C_2])\cong K_2(A,M) &\cong \bigoplus_{(\alpha,i)\in\Lambda^{00}}(1+x\mathbb{F}_{2}[x]/(x^{[\alpha,i]}))^{\times}\\
& = \bigoplus_{(\alpha,i)\in \Lambda^{00}_2}(1+x\mathbb{F}_{2}[x]/(x^{2}))^{\times}\\
& = \bigoplus_{\scriptsize\substack{((1,\alpha_2),2) \\ \alpha_2 \geq 1}}(1+x\mathbb{F}_{2}[x]/(x^{2}))^{\times} \oplus \bigoplus_{\scriptsize\substack{((2,\alpha_2),1) \\ \alpha_2\geq 1\text{为奇数}}}(1+x\mathbb{F}_{2}[x]/(x^{2}))^{\times} \\
& = \bigoplus_{\alpha_2 \geq 1}\mathbb{Z}/2 \mathbb{Z} \oplus \bigoplus_{\alpha_2\geq 1\text{为奇数}}\mathbb{Z}/2 \mathbb{Z},
\end{align*}
作为Abel群,
\[NK_2(\mathbb{F}_2[C_2]) \cong \bigoplus_{\infty} \mathbb{Z}/2 \mathbb{Z}.\]
% 
% 定理第二部分
% 
% 

	(2)由\ref{K2(A,M)},对于任意$(\alpha,i)\in \Lambda^{00}$,$\Gamma_{\alpha,i}$诱导了同态
 \begin{align*}
 \Gamma_{\alpha,i} \colon (1+xk[x]/(x^{[\alpha,i]}))^{\times} &\longrightarrow K_2(A,M)\\
 1-xf(x) &\mapsto \langle f(t^\alpha)t^{\alpha-\epsilon^i},t_i \rangle.
 \end{align*}
 此时只需考虑$\Lambda^{00}_2=\{((2,\alpha_2),1)\mid  \alpha_2\geq 1\text{为奇数}\} \cup \{((1,\alpha_2),2)\mid  \alpha_2\geq 1\}$,对于任意$(\alpha,i)\in \Lambda^{00}_2$,$\Gamma_{\alpha,i}$均诱导了单射,对任意$\alpha_2\geq 1$,
  \begin{align*}
 \Gamma_{(1,\alpha_2),2} \colon (1+x \mathbb{F}_2[x]/(x^{2}))^{\times} &\rightarrowtail K_2(A,M)\\
 1+x &\mapsto %\langle t^{(1,\alpha_2)-(0,1)},t_2 \rangle=
 \langle t_1t_2^{\alpha_2-1},t_2 \rangle,
 \end{align*}
对任意$\alpha_2\geq 1$为奇数,
 \begin{align*}
 \Gamma_{(2,\alpha_2),1} \colon (1+x \mathbb{F}_2[x]/(x^{2}))^{\times} &\rightarrowtail K_2(A,M)\\
 1+x &\mapsto \langle t_1t_2^{\alpha_2},t_1 \rangle,
 \end{align*}

我们作简单的替换令$t=t_1, x=t_2$,于是$\langle t_1t_2^{\alpha_2-1},t_2 \rangle = \langle tx^{\alpha_2-1},x \rangle$,$\langle t_1t_2^{\alpha_2},t_1 \rangle=\langle t x^{\alpha_2},t  \rangle$。由同构\ref{K2(A,M)}可知$NK_2(\mathbb{F}_2[C_2])$是由Dennis-Stein符号$\{\langle tx^i,x \rangle \mid i\geq 0\}$与$\{\langle tx^i,t \rangle \mid i\geq 1\text{为奇数}\}$生成的,由于$t^2=0$故$\langle tx^i,x \rangle+\langle tx^i,x \rangle=\langle tx^i+tx^i-t^2x^{2i+1},x \rangle=0$,$\langle tx^i,t \rangle+\langle tx^i,t \rangle=\langle tx^i+tx^i-t^3x^{2i},t \rangle=0$。
\end{proof}
\begin{remark}
	对于$i\geq 1\text{为偶数}$,$\langle tx^i,t \rangle=\langle x^{i/2},t \rangle+\langle x^{i/2},t \rangle=\langle x^{i/2}+x^{i/2}+tx^i,t \rangle=0$。
\end{remark}

Weibel在文献\cite{weibel2009nk0}中给出了以下可裂正合列
	\[0\longrightarrow V/\Phi(V) \overset{F}\longrightarrow NK_2(\mathbb{F}_2[C_2])\overset{D}\longrightarrow \Omega_{\mathbb{F}_2[x]}\longrightarrow 0,\]
其中$V=x \mathbb{F}_2[x]$,$\Phi(V)=x^2 \mathbb{F}_2[x^2]$是$V$的子群,$\Omega_{\mathbb{F}_2[x]}\cong \mathbb{F}_2[x]\,d x$是绝对K\"{a}hler微分模,$F(x^n)=\langle tx^n,t \rangle$,$D(\langle ft,g+g't \rangle)=f\,dg$。显然$D(\langle tx^i,t \rangle)=0$,$D(\langle tx^i,x \rangle)=x^i\, dx$,可以看出$NK_2(\mathbb{F}_2[C_2])$的直和项$\bigoplus_{((2,\alpha_2),1), \alpha_2\geq 1\text{为奇数}} \mathbb{Z}/2\mathbb{Z} \cong V/\Phi(V)$,直和项$\bigoplus_{((1,\alpha_2),2), \alpha_2\geq 1} \mathbb{Z}/2\mathbb{Z} \cong \mathbb{F}_2[x]\,d x$。

$V$和$\Omega_{\mathbb{F}_2[x]}$作为Abel群是同构的,但作为$W(\mathbb{F}_2)$-模是不同的。$V=x \mathbb{F}_2[x]$上的$W(\mathbb{F}_2)$-模结构(见\cite{MR96j:16008})为 
\begin{align*}
 V_m(x^n)&=x^{mn}, \\
 F_d(x^n)&=\begin{cases}
 	dx^{n/d},& \mbox{ 若 $d|n$}\\
 	0,& \mbox{其它}
 \end{cases}, \\
 [a]x^n&=a^nx^n.
 \end{align*}
$\Omega_{\mathbb{F}_2[x]}=\mathbb{F}_2[x]\,dx $上的$W(\mathbb{F}_2)$-模结构(见\cite{MR96j:16008})为
\begin{align*}
 V_m(x^{n-1}\,dx)&=mx^{mn-1}\,dx, \\
 F_d(x^{n-1}\,dx)&=\begin{cases}
 	x^{n/d-1}\,dx,& \mbox{ 若 $d|n$}\\
 	0,& \mbox{其它}
 \end{cases}, \\
 [a]x^{n-1}\,dx&=a^nx^{n-1}\,dx.
 \end{align*}
结合两者我们可以得到$NK_2(\mathbb{F}_2[C_2])$的$W(\mathbb{F}_2)$-模结构为
\begin{align*}
 V_m(\langle tx^n,t \rangle)&=\begin{cases}
 	\langle tx^{mn},t \rangle,& \mbox{若$m$是奇数}\\
 	0,& \mbox{若$m$是偶数}
 \end{cases},\quad \mbox{$n\geq 1$为奇数} \\
  V_m(\langle tx^{n-1},x \rangle)&=\begin{cases}
 	\langle tx^{mn-1},x \rangle,& \mbox{若$m$是奇数}\\
 	0,& \mbox{若$m$是偶数}
 \end{cases}
 ,\quad \mbox{$n\geq 1$} \\
 F_d(\langle tx^n,t \rangle)&=\begin{cases}
 	\langle tx^{n/d},t \rangle,& \mbox{ 若$d|n$}\\
 	0,& \mbox{其它}
 \end{cases},\quad \mbox{$n\geq 1$为奇数} \\
 F_d(\langle tx^{n-1},x \rangle)&=\begin{cases}
 	\langle tx^{n/d-1},x \rangle,& \mbox{若$d|n$}\\
 	0,& \mbox{其它}
 \end{cases}
 ,\quad \mbox{$n\geq 1$} \\
 [1]\langle tx^n,t \rangle&=\langle tx^n,t \rangle,\quad \mbox{$n\geq 1$为奇数} \\
 [1]\langle tx^{n-1},x \rangle&=\langle tx^{n-1},x \rangle,\quad \mbox{$n\geq 1$}.
 \end{align*}



% 
% 
% 
\section{$NK_2(\mathbb{F}_2[C_{4}])$}
这一节首先用同样的方法计算$NK_2(\mathbb{F}_2[C_{2^2}])$,继而对于任意$n$可以得到类似的结果。

\begin{theorem}
	$NK_2(\mathbb{F}_2[C_4])\cong \bigoplus_{\infty} \mathbb{Z}/2 \mathbb{Z}\oplus \bigoplus_{\infty}\mathbb{Z}/4 \mathbb{Z}$。
\end{theorem}
\begin{proof}
	$\mathbb{F}_2[t_1,t_2]/(t_1^4)=\mathbb{F}_2[C_{4}][t_2]$,此时$I=(t_1^4)$,$M=(t_1)$不变,我们直接写出以下集合

\begin{align*}
\Delta &=\{(\alpha_1,\alpha_2)\mid \alpha_1\geq 4, \alpha_2 \geq 0\},\\
\Lambda &=\{((\alpha_1,\alpha_2),1) \mid \alpha_1\geq 1\}\cup \{((\alpha_1,\alpha_2),2) \mid \alpha_1\geq 1, \alpha_2\geq 1\},
\end{align*}
用$\left \lceil x \right \rceil=\min \{m\in \mathbb{Z}|m\geq x\}$表示不小于$x$的最小整数,
\begin{align*}
[\alpha,1] & =\min \{m\in \mathbb{Z} \mid m \alpha_1\geq 5\}=\left \lceil 5/\alpha_1 \right \rceil,\\
[\alpha,2] & =\min \{m\in \mathbb{Z} \mid m \alpha_1\geq 4\}=\left \lceil 4/\alpha_1 \right \rceil.
\end{align*}

例如
\begin{align*}
[(1,\alpha_2),1] & = 5, \ \alpha_2\geq 0, \\
[(2,\alpha_2),1] & = 3, \ \alpha_2\geq 0, \\
[(3,\alpha_2),1] & = 2, \ \alpha_2\geq 0, \\
[(4,\alpha_2),1] & = 2, \ \alpha_2\geq 0, \\
[(\alpha_1,\alpha_2),1] & = 1, \ \alpha_1\geq 5, \alpha_2\geq 0, \\
[(1,\alpha_2),2] & = 4, \ \alpha_2\geq 1, \\
[(2,\alpha_2),2] & = 2, \ \alpha_2\geq 1, \\
[(3,\alpha_2),2] & = 2, \ \alpha_2\geq 1, \\
[(\alpha_1,\alpha_2),2] & = 1, \ \alpha_1\geq 4, \alpha_2\geq 1.
\end{align*}


\[\begin{array}{|c|c|c|c|}
\hline
(\alpha_1,\alpha_2) & [(\alpha_1,\alpha_2),1] &[(\alpha_1,\alpha_2),2] &[(\alpha_1,\alpha_2)]  \\
\hline
(1,\alpha_2)  & 5, \alpha_2 \geq 0& 4 , \alpha_2\geq 1 & 5 \\
\hline
(2,\alpha_2)  & 3, \alpha_2 \geq 0& 2 , \alpha_2\geq 1 & 2, \text{当$\alpha_2$是奇数时} \\
\hline
(3,\alpha_2)  & 2, \alpha_2 \geq 0& 2 , \alpha_2\geq 1 & 2 \\
\hline
(4,\alpha_2)  & 2, \alpha_2 \geq 0& 1 , \alpha_2\geq 1 & 1 \text{当$\alpha_2$是奇数时}\\
\hline
(\alpha_1,0),\alpha_1 \geq 5 & 1 & \text{无定义} &1, \text{当$\alpha_1$是奇数时} \\
\hline
(\alpha_1,\alpha_2),\alpha_1 \geq 5, \alpha_2 \geq 1& 1 & 1& 1, \text{当$(\alpha_1,\alpha_2)=1$时} \\
\hline
\end{array}\]


记$\Lambda^{00}_d=\{(\alpha,i)\in \Lambda^{00}| [(\alpha,i)]=d\}$,$\Lambda^{00}_{>1}=\{(\alpha,i)\in \Lambda^{00}| [(\alpha,i)]>1\}$

由于$(\alpha,i)\in \Lambda^{00}_1$均有$[(\alpha,i)]=1$,实际上要计算$(1+x\mathbb{F}_2[x]/(x^{[\alpha,i]}))^{\times}$只需确定$\Lambda^{00}_{>1}$。由同样的方法可得
$\Lambda^{00}_4=\{((1,\alpha_2),2)\mid  \alpha_2\geq 1\}$,$\Lambda^{00}_3=\{((2,\alpha_2),1)\mid  \alpha_2\geq 1\text{为奇数}\}$,$\Lambda^{00}_2=\{((3,\alpha_2),2)\mid  gcd(3,\alpha_2)=1,\alpha_2\geq 1\}\cup \{((4,\alpha_2),1)\mid  \alpha_2\geq 1\text{为奇数}\}$,
% $[(1,\alpha_2),2]=4$,$[(2,\alpha_2),1]=3$,$[(3,\alpha_2),2]=2$,$[(4,\alpha_2),1]=2$,
\begin{align*}
\Lambda^{00}_{>1}=& \{((1,\alpha_2),2)\mid  \alpha_2\geq 1\}\cup \{((3,\alpha_2),2)\mid gcd(3,\alpha_2)=1, \alpha_2\geq 1\}\\
	& \cup \{((2,\alpha_2),1)\mid  \alpha_2\geq 1\text{为奇数}\}  \cup \{((4,\alpha_2),1)\mid  \alpha_2\geq 1\text{为奇数}\}
\end{align*}



% 由于此时$p=char(\mathbb{F}_2)=2$,
% \[w(\alpha,i)=\min \{w\in \mathbb{Z}_+\mid  2^w \geq [\alpha,i]\},\]
% 同理可以得到
% \begin{align*}
% w((1,\alpha_2),1) & = 2, \ \alpha_2\geq 0, \\
% w((2,\alpha_2),1) & = 1, \ \alpha_2\geq 0, \\
% w((\alpha_1,\alpha_2),1) & = 0, \ \alpha_1\geq 3, \alpha_2\geq 0, \\
% w((1,\alpha_2),2) & = 1, \ \alpha_2\geq 1, \\
% w((\alpha_1,\alpha_2),2) & = 0, \ \alpha_1\geq 2, \alpha_2\geq 1.
% \end{align*}

% 若$gcd(2,\alpha_1,\alpha_2)=1$,即$\alpha_1,\alpha_2$中至少一个是奇数,同样的道理
% \begin{align*}
% [(1,\alpha_2)]&=\max\{[(1,\alpha_2),1],[(1,\alpha_2),2]\}=5,\\
% [(3,\alpha_2)]&=\max\{[(3,\alpha_2),1],[(3,\alpha_2),2]\}=2,\\
% [(2,1)]&=2,\\
% [\alpha]&=1,\text{其它符合条件的$\alpha$.}
% \end{align*}
% 下面我们计算$\Lambda^{00}=\big\{(\alpha,i)\in \Lambda\mid  gcd(\alpha_1,\alpha_2)=1, i\neq \min\{j\mid \alpha_j\not\equiv 0 \bmod 2,[\alpha,j]=[\alpha]\} \big\}$,首先注意到$\min\{j\mid \alpha_j\not\equiv 0 \bmod 2,[\alpha,j]=[\alpha]\}=1$,因为$\alpha=(1,\alpha_2)$时,$[\alpha,1]=[\alpha]$。从而
% \[\Lambda^{00}=\{(\alpha,2)\mid gcd(\alpha_1,\alpha_2)=1,\alpha_1\geq 1, \alpha_2\geq 1\}.\]

% \[\Lambda^0=\{(m\alpha,i)\in \Lambda\mid  gcd(m,2)=1, (\alpha,i)\in \Lambda^{00}\}=\{(m\alpha,2)\mid m\text{奇数},gcd(\alpha_1,\alpha_2)=1,\alpha_1\geq 1, \alpha_2\geq 1\}.\]

由定理\ref{K2(A,M)}, 从而

\begin{align*}
NK_2(\mathbb{F}_2[C_4])\cong K_2(A,M) &\cong \bigoplus_{(\alpha,i)\in\Lambda^{00}}(1+x\mathbb{F}_2[x]/(x^{[\alpha,i]}))^{\times}\\
& = \bigoplus_{(\alpha,i)\in \Lambda^{00}_{>1}}(1+x\mathbb{F}_2[x]/(x^{[\alpha,i]}))^{\times}\\
& = \bigoplus_{\scriptsize\substack{((3,\alpha_2),2) \\ gcd(3,\alpha_2)=1 \\ \alpha_2\geq 1}}(1+x\mathbb{F}_2[x]/(x^{2}))^{\times}\oplus \bigoplus_{\scriptsize\substack{((4,\alpha_2),1) \\ \alpha_2\geq 1\text{为奇数}}}(1+x\mathbb{F}_2[x]/(x^{2}))^{\times} \\
& \oplus \bigoplus_{\scriptsize\substack{((2,\alpha_2),1) \\ \alpha_2\geq 1\text{为奇数}}}(1+x\mathbb{F}_2[x]/(x^{3}))^{\times} \oplus \bigoplus_{\scriptsize\substack{((1,\alpha_2),2) \\ \alpha_2\geq 1}}(1+x\mathbb{F}_2[x]/(x^{4}))^{\times}
\end{align*}

由\ref{ex:W3(F2)}有$(1+x\mathbb{F}_2[x]/(x^{4}))^{\times}\cong \mathbb{Z}/2 \mathbb{Z}\times \mathbb{Z}/4 \mathbb{Z}$,$(1+x\mathbb{F}_2[x]/(x^{3}))^{\times}\cong \mathbb{Z}/4 \mathbb{Z}$,于是$NK_2(\mathbb{F}_2[C_4])$作为Abel群有
\[NK_2(\mathbb{F}_2[C_4]) \cong \bigoplus_{\infty} \mathbb{Z}/2 \mathbb{Z}\oplus \bigoplus_{\infty}\mathbb{Z}/4 \mathbb{Z}.\]




对于任意$(\alpha,i)\in \Lambda^{00}_{>1}$,$\Gamma_{\alpha,i}$均诱导了单射,对任意$\alpha_2\geq 1$,$gcd(3,\alpha_2)=1$
  \begin{align*}
 \Gamma_{(3,\alpha_2),2} \colon (1+x \mathbb{F}_2[x]/(x^{2}))^{\times} &\rightarrowtail K_2(A,M)\\
 1+x &\mapsto  \langle t_1^3t_2^{\alpha_2-1},t_2 \rangle,
 \end{align*}
对任意$\alpha_2\geq 1$,
  \begin{align*}
 \Gamma_{(1,\alpha_2),2} \colon (1+x \mathbb{F}_2[x]/(x^{4}))^{\times} &\rightarrowtail K_2(A,M)\\
 1+x \text{(四阶元)} &\mapsto \langle t_1t_2^{\alpha_2-1},t_2 \rangle,\\
 1+x^3 \text{(二阶元)} &\mapsto \langle t_1^3t_2^{3\alpha_2-1},t_2 \rangle,
 \end{align*}

对任意$\alpha_2\geq 1$为奇数,
 \begin{align*}
 \Gamma_{(4,\alpha_2),1} \colon (1+x \mathbb{F}_2[x]/(x^{2}))^{\times} &\rightarrowtail K_2(A,M)\\
 1+x &\mapsto \langle t_1^3t_2^{\alpha_2},t_1 \rangle,
 \end{align*}
 \begin{align*}
 \Gamma_{(2,\alpha_1),1} \colon (1+x \mathbb{F}_2[x]/(x^{3}))^{\times} &\rightarrowtail K_2(A,M)\\
 1+x\text{(四阶元)} &\mapsto \langle t_1t_2^{\alpha_2},t_1 \rangle.
 \end{align*}

我们作简单的替换令$t=t_1, x=t_2$,由同构\ref{K2(A,M)}可知$NK_2(\mathbb{F}_2[C_4])$是由Dennis-Stein符号$\{\langle tx^{i-1},x \rangle \mid i\geq 1\},\{\langle t^3x^{3i-1},x \rangle \mid i\geq 1\},\{\langle t^3x^{i-1},x \rangle \mid i\geq 1,gcd(i,3)=1\},\{\langle tx^i,t \rangle \mid i\geq 1\text{为奇数}\},\{\langle t^3x^i,t \rangle \mid i\geq 1\text{为奇数}\}$生成的。

\end{proof}

\begin{remark}

	$\langle t^3x^{2i},t \rangle =\langle tx^{i},t \rangle+\langle tx^{i},t \rangle$是二阶元。
	根据\cite{MR80k:13005},存在同态
	\begin{align*}
	\rho_1 \colon \mathbb{F}_2[x]dx &\longrightarrow NK_2(\mathbb{F}_2[C_4])\\
				x^idx &\mapsto \langle t^3x^i,x\rangle \\
	\rho_2 \colon x\mathbb{F}_2[x]/x^2\mathbb{F}_2[x^2] &\longrightarrow NK_2(\mathbb{F}_2[C_4])\\
				x^i &\mapsto \langle t^3x^i,t\rangle 
	\end{align*}

	$\{\langle t^3x^{i-1},x \rangle \mid i\geq 1\}=\{\langle t^3x^{3i-1},x \rangle \mid i\geq 1\}\cup\{\langle t^3x^{i-1},x \rangle \mid i\geq 1,gcd(i,3)=1\}$,从而$\Omega_{\mathbb{F}_2[x]}\oplus x\mathbb{F}_2[x]/x^2\mathbb{F}_2[x^2]$是$NK_2(\mathbb{F}_2[C_4])$的直和项。
\end{remark}

\section{$NK_2(\mathbb{F}_q[C_{2^n}])$} % (fold)
\label{sec:NK_2(F_q[C_{2^n}])}

设$\mathbb{F}_q$是特征为$2$的有限域,$q=2^f$,$C_{2^n}$是$2^n$阶循环群,这一节计算$NK_2(\mathbb{F}_q[C_{2^n}])$。假设$A=\mathbb{F}_q[t_1,t_2]/(t_1^{2^n})=\mathbb{F}_q[C_{2^n}][x]$, 此时$I=(t_1^{2^n})$, $M=(t_1)$, $A/M=\mathbb{F}_q[x]$。

\begin{lemma}
	$\Delta =\{(\alpha_1,\alpha_2)\mid \alpha_1\geq 2^n, \alpha_2 \geq 0\}$,$\Lambda = \{((\alpha_1,\alpha_2),1) \mid \alpha_1\geq 1, \alpha_2\geq 0\}\cup \{((\alpha_1,\alpha_2),2) \mid \alpha_1\geq 1, \alpha_2\geq 1\}$,对任意$(\alpha,i)\in \Lambda$,$[\alpha,1]=\left \lceil (2^n+1)/\alpha_1 \right \rceil$,$[\alpha,2]=\left \lceil 2^n/\alpha_1 \right \rceil$,其中$\left \lceil x \right \rceil=\min \{m\in \mathbb{Z}|m\geq x\}$表示不小于$x$的最小整数,。
\end{lemma}

\begin{lemma}
令$I_1 =\{((\alpha_1,\alpha_2),1)\mid gcd(\alpha_1,\alpha_2)=1, 1< \alpha_1\leq 2^n\text{为偶数}, \alpha_2\geq 1\text{为奇数}\}$,$I_2=\{((\alpha_1,\alpha_2),2)\mid gcd(\alpha_1,\alpha_2)=1, 1\leq \alpha_1<2^n\text{为奇数}, \alpha_2\geq 1\}$,则$\Lambda^{00}_{>1}=I_1\sqcup I_2$。
% \begin{align*}
% 	\Lambda^{00}_{>1}=&\{((\alpha_1,\alpha_2),2)\mid gcd(\alpha_1,\alpha_2)=1, 1\leq \alpha_1<2^n\text{为奇数}, \alpha_2\geq 1\}\\
% 	&\cup \{((\alpha_1,\alpha_2),1)\mid gcd(\alpha_1,\alpha_2)=1, 1< \alpha_1\leq 2^n\text{为偶数}, \alpha_2\geq 1\text{为奇数}\}.
% \end{align*}
\end{lemma}
由定理\ref{K2(A,M)},

\begin{align*}
NK_2(\mathbb{F}_q[C_{2^n}])\cong K_2(A,M) &\cong \bigoplus_{(\alpha,i)\in\Lambda^{00}}(1+x\mathbb{F}_q[x]/(x^{[\alpha,i]}))^{\times}\\
& = \bigoplus_{(\alpha,i)\in \Lambda^{00}_{>1}}(1+x\mathbb{F}_q[x]/(x^{[\alpha,i]}))^{\times}\\
& = \bigoplus_{(\alpha,1)\in I_1}(1+x\mathbb{F}_q[x]/(x^{\left \lceil (2^n+1)/\alpha_1 \right \rceil}))^{\times} \\
& \oplus \bigoplus_{(\alpha,2)\in I_2}(1+x\mathbb{F}_q[x]/(x^{\left \lceil 2^n/\alpha_1 \right \rceil}))^{\times}.
\end{align*}
注意到$BigWitt_{k}(R)=(1+x R\llbracket x\rrbracket )^{\times}/(1+x^{k+1} R\llbracket x\rrbracket )^{\times} \cong (1+x R[x]/(x^{k+1}))^{\times}$,
根据公式\ref{cor:BW},
% 以下两段是旧的公式,应该不正确
% \[(1+x\mathbb{F}_q[x]/(x^{\left \lceil (2^n+1)/\alpha_1 \right \rceil}))^{\times}=\bigoplus_{\scriptsize\substack{1\leq m\leq \left \lceil (2^n+1)/\alpha_1 \right \rceil-1 \\ gcd(m,2)=1 }}\mathbb{Z}/q^{1+ \left \lfloor\log_2 \frac{\left \lceil (2^n+1)/\alpha_1 \right \rceil-1}{m}  \right \rfloor}\mathbb{Z}, \]
% \[(1+x\mathbb{F}_q[x]/(x^{\left \lceil 2^n/\alpha_1 \right \rceil}))^{\times}=\bigoplus_{\scriptsize\substack{1\leq m \leq \left \lceil 2^n/\alpha_1 \right \rceil-1 \\ gcd(m,2)=1}}\mathbb{Z}/q^{1+ \left \lfloor\log_2 \frac{\left \lceil 2^n/\alpha_1 \right \rceil-1}{m}  \right \rfloor}\mathbb{Z}, \]


\begin{align*}
NK_2(\mathbb{F}_q[C_{2^n}])\cong & \bigoplus_{(\alpha,1)\in I_1}\bigoplus_{\scriptsize\substack{1\leq m\leq \left \lceil (2^n+1)/\alpha_1 \right \rceil-1 \\ gcd(m,2)=1}}(\mathbb{Z}/2^{1+ \left \lfloor\log_2 \frac{\left \lceil (2^n+1)/\alpha_1 \right \rceil-1}{m}  \right \rfloor}\mathbb{Z})^f \\
& \oplus \bigoplus_{(\alpha,2)\in I_2}\bigoplus_{\scriptsize\substack{ 1 \leq m\leq \left \lceil 2^n/\alpha_1 \right \rceil-1 \\ gcd(m,2)=1}}(\mathbb{Z}/2^{1+ \left \lfloor\log_2 \frac{\left \lceil 2^n/\alpha_1 \right \rceil-1}{m}  \right \rfloor}\mathbb{Z})^f.
\end{align*}

接下来我们证明对于任意$1\leq k\leq n$,$\mathbb{Z}/2^k \mathbb{Z}$都在$NK_2(\mathbb{F}_q[C_{p^n}])$出现无限多次

\begin{lemma}
\label{lem:log2}
	对于任意的$1\leq k < n$,$1+\left \lfloor \log_2(\frac{2^n-1}{2^k+1}) \right \rfloor = n-k$。
\end{lemma}
\begin{proof}
	当$1\leq k < n$时,$2^k-1\geq 1 \geq \frac{1}{2^{n-k-1}}$,即
	\[2^{n-1}-2^{n-k-1}\geq 1 \]
	上式等价于$2^n-1\geq 2^{n-k-1}(2^k+1)$,且$2^n-1<2^{n-k}(2^k+1)$,于是
	\[2^{n-k}> \frac{2^n-1}{2^k+1} \geq 2^{n-k-1}\]
	取对数得$\left \lfloor \log_2(\frac{2^n-1}{2^k+1}) \right \rfloor = n-k-1$。
\end{proof}
考虑$((1,\alpha_2),2)\in I_2$,
$$\bigoplus_{(\alpha,2)\in I_2}\bigoplus_{\scriptsize\substack{1\leq m\leq  2^n-1 \\ gcd(m,2)=1 }}(\mathbb{Z}/2^{1+ \left \lfloor\log_2 \frac{2^n-1}{m}  \right \rfloor}\mathbb{Z})^f$$
是$NK_2(\mathbb{F}_{2^f}[C_{2^n}])$的直和项,当$m=1$时$1+ \left \lfloor\log_2 (2^n-1)\right \rfloor=n$,当$m=2^k+1 (1\leq k < n)$为奇数时,由\ref{lem:log2},$1+ \left \lfloor\log_2 \frac{2^n-1}{m}\right \rfloor=n-k$,于是对于任何的$1\leq k\leq n$,$\mathbb{Z}/2^k\mathbb{Z}$均出现在直和项中,且对于任意$\alpha_2\geq 1$,这样的项总会出现,于是
\[NK_2(\mathbb{F}_q[C_{2^n}])\cong \bigoplus_\infty \bigoplus_{k=1}^n \mathbb{Z}/2^k\mathbb{Z}.\]

接下来给出一些$NK_2(\mathbb{F}_q[C_{2^n}])$中的$2^k(1\leq k \leq n)$阶元素。

对任意$\alpha_2\geq 1, a\in \mathbb{F}_q$,
  \begin{align*}
 \Gamma_{(1,\alpha_2),2} \colon (1+x \mathbb{F}_q[x]/(x^{2^n}))^{\times} &\rightarrowtail K_2(A,M)\\
 1+ax \text{($2^n$阶元)} &\mapsto \langle atx^{\alpha_2-1},x \rangle,\\
 1+ax^3 \text{($2^{n-1}$阶元)} &\mapsto \langle at^3x^{3\alpha_2-1},x \rangle,\\
 1+ax^{2^k+1} \text{($2^{n-k}$阶元)} &\mapsto \langle at^{2k+1}x^{(2k+1)\alpha_2-1},x \rangle.
 \end{align*}






% 

\section{$NK_2$ of finite abelian $p$-groups}


\section{其他可以考虑的问题}

$NK_2(\mathbb{F}_{p^m}[C_{p^n}])=?$

$\mathbb{F}_2[C_2\times C_2] \cong\mathbb{F}_2[C_2]\otimes\mathbb{F}_2[C_2] \cong \mathbb{F}_2[x,y]/(x^2,y^2)$,看看能否用同样的方法得到一些结果。



\[
	0\longrightarrow K_2(k[t_1,t_2,t_3]/(t_1^n,t_2^n),(t_1,t_2)) \longrightarrow K_2(k[t_1,t_2,t_3]/(t_1^n,t_2^n)) \longrightarrow K_2(k[t_3]) \longrightarrow 0
	\]
对于有限域$k$来讲$K_2(k[t_3])=0$,
\[0\longrightarrow NK_2(\mathbb{F}_2[C_{2}\times C_2]) \longrightarrow K_2(\mathbb{F}_2[C_{2}\times C_2][x])\longrightarrow K_2(\mathbb{F}_2[C_{2}\times C_2]) \longrightarrow 0,\]
中间那项就是可以用这篇文章里的方法确定,又$K_2(\mathbb{F}_2[C_{2}\times C_2])$可以通过Gao Yubin等文章得到,应该是$C_2^3$,于是可以得到$NK_2(\mathbb{F}_2[C_{2}\times C_2])$,猜测也是$\oplus_{\infty} \mathbb{Z}/2 \mathbb{Z}$.

另外可以考虑直接用本章里的方式重新计算高玉彬师兄文章里的结果,看是否更简洁,或者是否更繁复,复杂在哪里,哪里可以进行简化,简化后是否可以用到算$NK$的内容中。


一个关于模结构的问题,在Weibel的文章\cite{MR88f:18018}中5.5和5.7给出的模结构和本文上面的模结构并不一致,用$V_m$作用差一个$t^m$。





