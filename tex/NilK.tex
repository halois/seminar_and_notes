%!TEX root = ../main.tex
\chapter{An introduction to $NK$-groups}
There are functors  
\begin{align*}
K_n \colon \mathbf{Ring} &\longrightarrow \mathbf{Ab}\\
R &\mapsto K_n(R)\\
[f:R\rightarrow S] &\mapsto [K_n(f):K_n(R)\rightarrow K_n(S)]
\end{align*}
\begin{itemize}
	\item $n<0$: negative $K$-theory.
	\item $K_0, K_1, K_2$: classical $K$-theory:
		\begin{itemize}
			\item (Grothendieck) $K_0(R)=G(\mathbf{P}(R))$. If $R$ is a field or the ring of integers $\mathbb{Z}$, then $K_0(R)=\mathbb{Z}$.
			\item (Bass) $K_1(R)=GL(R)/E(R)$. $K_1(\mathbb{Z})=\mathbb{Z}/2\mathbb{Z}=\{\pm 1\}$.
			\item (Milnor) $K_2(R)=\ker(St(R)\longrightarrow E(R))$.
		\end{itemize}
	\item $n\geq 3$: (Quillen) higher $K$-theory, $K_n(R)=\pi_n\big(BGL(R)^+\big) \cong \pi_n\big(\Omega BQ\mathbf{P}(R)\big)$.
\end{itemize}
\begin{definition}
	Whitehead group $Wh(G)=K_1(\mathbb{Z}G)/\{\pm g \mid g\in G\}$.
\end{definition}
A question arise $K_n(RG)=?$ where $RG$ is a group ring.

For example, one want to compute $K_n(R[\mathbb{Z}])$, first note that $R[\mathbb{Z}]=R[x,x^{-1}]$.
\section{Nil-groups}
\begin{definition}[Bass $NK$-groups]
	Let $R[x]$ be a polynomial ring over $R$, the map $i$ the injection $R \overset{i}\hookrightarrow R[x]$, then there is a surjection $\varepsilon$ split by $i$
	\[\varepsilon \colon R[x] \overset{x \mapsto 0}\longrightarrow R.\]
	The Bass $NK$-groups(Nil-groups) are defined by 
	\[NK_n(R):=\ker \big(K_n(R[x])\longrightarrow K_n(R) \big), \forall n\in \mathbb{Z}.\]
	Hence $K_n(R[x]) \cong K_n(R)\oplus NK_n(R)$.
\end{definition}
\begin{theorem}[Fundamental theorem of algebraic $K$-theory, Bass-Heller-Swan formula]
	The following exact sequence is split
	\[0\longrightarrow K_n(R) \overset{\Delta}\longrightarrow K_n(R[x])\oplus K_n(R[x^{-1}]) \overset{\pm}\longrightarrow K_n(R[x,x^{-1}]) \overset{\partial}\longrightarrow K_{n-1}(R) \longrightarrow 0.\]
	And we have 
	\[K_n(R[x,x^{-1}])\cong K_n(R)\oplus K_{n-1}(R) \oplus NK_n(R)\oplus NK_n(R).\]
\end{theorem}
\begin{remark}
	History: H. Bass defined negative $K$-theory
	\[K_n(R):= \coker\big( K_n(R[x])\oplus K_n(R[x^{-1}]) \longrightarrow K_n(R[x,x^{-1}]) \big)\]
	and $NK_i$ for $i\leq 1$ 
	\[NK_n(R):=\ker \big(K_n(R[x])\longrightarrow K_n(R) \big).\]
	D. Quillen defined higher $K$-theory in 1970s (especially the famous paper \emph{ Higher Algebraic $K$-theory I}\cite{MR49:2895} in 1972).

	The two copies of $NK_i(R)$ come from the embeddings $R[x]\hookrightarrow R[x,x^{-1}]$ and $R[x^{-1}]\hookrightarrow R[x,x^{-1}]$.
\end{remark}
{\color{red}注意这里有遗留的符号问题,就是$\alpha$的符号}
\begin{definition}[$\alpha$-twisted polynomial rings and $\alpha$-twisted Laurent polynomial rings]
	Let $\alpha \colon R \longrightarrow R$ be an automorphism, $R_\alpha[x,x^{-1}]$ is called the $\alpha$-twisted Laurent polynomial ring. Addtively $R_\alpha[x,x^{-1}]=R[x,x^{-1}]$. Multiplication in $R_\alpha[x]$is defined by $(rx^i)(sx^j)=r\alpha^{-i}(s)x^{i+j}$. There is an automorphism of $R_\alpha[x,x^{-1}]$  induced by $\alpha$ and which we also denote by $\alpha$, defined by the following condition
	\[\alpha(rx^i)=\alpha(r)x^i, r\in R.\] 
	The subrings $R_\alpha[x],R_\alpha[x^{-1}]$ of $R_\alpha[x,x^{-1}]$ generated by $R$ and $x$, and $R$ and $x^{-1}$ respectively are called $\alpha$-twisted polynomial rings.
\end{definition}
\paragraph{Some results about twisted plynomial rings and twisted Laurent polynomial rings}
According to \cite{MR41:5457}, we list some useful results here, all modules are considered as right $R$-modules:
\begin{itemize}
	\item \emph{Twisted Hilbert syzygy theorem:} If the right global dimension $r.gl.dim(R)=n$, then $r.gl.dim(R_{\alpha}[x])=n+1$ and $r.gl.dim(R_{\alpha}[x,x^{-1}])\leq n+1$.
	\item \emph{Twisted Hilbert basis theorem:} If $R$ is (right) noetheridan, then both $R_{\alpha}[x]$ and $R_{\alpha}[x,x^{-1}]$ are (right) noetherian.
	\item If $R$ is (right) regular, then $R_{\alpha}[x]$ and $R_{\alpha}[x,x^{-1}]$ are (right) regular.
	\item \emph{Twisted Grothendieck theorem:} If $R$ is (right) regular, then $K_0(R)\longrightarrow K_0(R_{\alpha}[x])$ is an isomorphism and $K_0(R)\longrightarrow K_0(R_{\alpha}[x,x^{-1}]) $ is an epimorphism. {\color{red}看下这个能不能推广到一般$n$.}
\end{itemize}



\begin{definition}[Farrell Nil-groups]
	Let $\varepsilon$ be the surjection $R_{\alpha}[x] \overset{x \mapsto 0}\longrightarrow R$, the Farrell twisted Nil-group $NK_n(R,\alpha)$ is defined as the kernel of $K_n(\varepsilon) \colon K_n(R_\alpha[x])\longrightarrow K_n(R)$ for any $n\in \mathbb{Z}$, then $K_n(R_\alpha[x])\cong K_n(R)\oplus NK_n(R,\alpha)$.
\end{definition}
\begin{remark}
	If $\alpha=\id$, then $NK_n(R,\id) = NK_n(R)$ are Bass Nil-groups. 
\end{remark}

Let $\mathbf{P}(R)$ be the exact category of finitely generated projective $R$-modules, define $\mathbf{Nil}(R)$ to be the exact category of nilpotent endomorphisms, objects in $\mathbf{Nil}(R)$ are pairs $(P,f)$ with $P$ finitely generated projective $R$-module and $f\in \End(P)$ nilpotent, morphisms are $(P_1,f_1) \overset{\alpha}\longrightarrow (P_2,f_2)$ with $f
_2\circ \alpha =\alpha \circ f_1$, i.e.\ such $\alpha$ make the following diagram commutes
\[
\begin{tikzcd}
	P_1 \arrow[r,"f_1"] \arrow[d,"\alpha"] & P_1 \arrow[d,"\alpha"]\\
	P_2 \arrow[r,"f_2"]  & P_2
\end{tikzcd}.
\]

D. Grayson defined $\mathbf{Nil}(R,\alpha)$ to be the exact category of $\alpha$-semilinear nilpotent endomorphisms \cite{MR89i:16021}, objects are pairs $(P,f)$ with $P$ finitely generated projective and $f\in \End(P)$ nilpotent and $\alpha$-semilinear (i.e.\ $f(mr)=f(m)\alpha(r)$).

Since the inclusion $i \colon \mathbf{P}(R)  \longrightarrow \mathbf{Nil}(R)$ and $i \colon \mathbf{P}(R)  \longrightarrow \mathbf{Nil}(R,\alpha)$ are split and the functors below are exact, they induce homomorphisms between $K$-groups
\[\mathbf{P}(R)  \rightleftarrows \mathbf{Nil}(R,\alpha)\]
\[\mathbf{P}(R)  \rightleftarrows \mathbf{Nil}(R)\]
\[P \mapsto (P,0)\]
\[P \mapsfrom (P,f)\]
\begin{definition}
	For $n\in \mathbb{Z}$, define $\Nil_n(R):= \ker\big(K_n(\mathbf{Nil}(R))\longrightarrow K_n(R)\big)$ and $\Nil_n(R,\alpha):= \ker\big(K_n(\mathbf{Nil}(R,\alpha))\longrightarrow K_n(R)\big)$ where $K_n(R)=K_n(\mathbf{P}(R))$ and $K_n$ is defined by Quillen's Q-construction, then $K_n(\mathbf{Nil}(R))=K_n(R) \oplus \Nil_n(R)$, $K_n(\mathbf{Nil}(R,\alpha))=K_n(R) \oplus \Nil_n(R,\alpha)$.
\end{definition}
{\color{red}注意这里有遗留的符号问题,就是$\alpha$的符号,因为\cite{MR41:5457}和\cite{MR89i:16021}记法不同,需要读这两篇文章确定符号!}
\begin{theorem}[\cite{MR89i:16021}]
	There is a localization exact sequence
	\[\cdots \longrightarrow K_{n+1}(R_{\alpha}[x,x^{-1}]) \longrightarrow K_n(\mathbf{Nil}(R,\alpha)) \longrightarrow K_n(R_{\alpha}[x])\longrightarrow K_{n}(R_{\alpha}[x,x^{-1}]) \longrightarrow \cdots,\]
	and $K_n(R_{\alpha}[x,x^{-1}]) \cong F_{n-1}(\alpha) \oplus \Nil_{n-1}(R,\alpha) \oplus \Nil_{n-1}^{\alpha^{-1}}(R)$.
\end{theorem}
\begin{prop}[Grayson \cite{MR58:28137}, \cite{MR89i:16021}]
	For any $n\geq 1$, $NK_n(R)\cong \Nil_{n-1}(R)$, $NK_n^{\alpha^{-1}}(R)\cong \Nil_{n-1}(R,\alpha)$.
\end{prop}
For $n=1$, F. T. Farrell and W.-C. Hsiang first got the result $NK_1^{\alpha^{-1}}(R)\cong \Nil_{0}(R,\alpha)$ in \cite{MR41:5457}.

Notice that $NK_1(R)\cong K_1(R[x],(x-r))$, $\forall r\in R$.
Since $[(P,\nu)]=[(P\oplus Q,\nu\oplus 0)]-[(Q,0)]\in K_0(\mathbf{Nil}(R))$, we see $\Nil_0(R)$ is generated by elements of the form $[(R^n,\nu)]-n[(R,0)]$ for some $n$ and some nilpotent matrix $\nu$, in fact one has
\begin{align*}
	NK_1(R) & \cong \Nil_0(R),\\
[1-\nu x] & \leftrightarrow [(R^n,\nu)]-n[(R,0)].
\end{align*}

\begin{definition}[Virtually cyclic groups]
A discrete group $V$ is called virtually cyclic if it contains a cyclic subgroup of finite index, i.e., if $V$ is finite or virtually infinite cyclic.
\end{definition}
Virtually infinite cyclic groups are of two types:
\begin{itemize}
 	\item[1]  $V = G \rtimes_{\alpha} T$ is a semi-direct product where $G$ is a finite group, $T = \langle x \rangle$ an infinite cyclic group generated by $x$, $\alpha \in Aut(G)$, and the multiplication of $V$ is given by $(g_1,x^i)(g_2,x^j)=(g_1\alpha^{-i}(g_2),x^{i+j})$. 
 	\item[2] $V$ is a non-trivial amalgam of finite groups and has the form $V =G_0 *_H	G_1$ where $[G_0: H ]=2=[ G_1 :H]$.
 \end{itemize}
What we are really interested in here is $G\rtimes_{\alpha} T$ --- the first type of infinite virtually cyclic groups. Note that $R[G\rtimes_{\alpha} T]\cong (R[G])_\alpha[x,x^{-1}]$.
 



\section{Regular rings}
\begin{definition}
	A noetherian ring $R$ is regular \index{regular ring} if every finitely generated $R$-module $M$ has a finite resolution by finite generated projective $R$-modules, i.e., an acyclic complex of finitely generated $R$-modules
	\[0\longrightarrow P_n\longrightarrow \cdots \longrightarrow P_0 \longrightarrow M \longrightarrow 0, \]
	with $P_i$ projective.
\end{definition}
\begin{theorem}[Fundamental theorem of regular rings]
	Assume $R$ is a (left, noetherian) regular ring, then 

	(1) (homotopy invariance) $K_i(R[x])\cong K_i(R)$,

	(2) $K_i(R[x,x^{-1}])\cong K_i(R)\oplus K_{i-1}(R)$, for $i \in \mathbb{Z}$.\\
	Hence $NK_i(R)=0$ for $i\in \mathbb{Z}$, and $K_n(R)=0$ for $n<0$. 
\end{theorem}
We list some examples here: PIDs (principal ideal domains) and Dedekind domains are regular, such as $\mathbb{Z}$, $\mathbb{Z}/n\mathbb{Z}$, $\mathbb{F}_q$, $\mathbb{Z}[\zeta_n]$, etc. Localization of a regular ring is again regular. Hilbert's Syzygy Theorem simply says that polynomial rings over a field are regular. Note that if a ring is (left) noetherian and has finite global dimension, then it is (left) regular. So noetherian hereditary rings are regular.

Set $R_n=\mathbb{C}[x_0,\cdots,x_n]/(\sum x_i^2=1)$. This is the complex coordinate ring of the $n$-sphere; it is a regular ring for every $n$ and $R_1\cong \mathbb{C}[z,z^{-1}]$.

However, the integral group rings of non-trivial finite groups are not regular. If $H$ is a non-trivial finite cyclic group, then $H_n(H,\mathbb{Z})\neq 0$ for all odd $n$. Therefore the finitely generated $\mathbb{Z}H$-module $\mathbb{Z}$ CANNOT have a finite projective resolution, hence $\mathbb{Z}H$ is not regular. 
Let $H$ be a cyclic group of order $n$, $t$ a generator of $H$, then $\mathbb{Z}H\cong \mathbb{Z}[t]/(t^n-1)$. Put $N=\sum_{i=0}^{n-1}t^i = 1+t+t^2+\cdots+t^{n-1}$ (norm element of $\mathbb{Z}H$), then $(t-1)N=0$ (in the polynomial ring $\mathbb{Z}[t]$, $t^n-1=(t-1)N$). There is a resolution $P_{\cdot}$ of $\mathbb{Z}$
\[\cdots \overset{N}\longrightarrow \mathbb{Z}H \overset{t-1}\longrightarrow \mathbb{Z}H \overset{N}\longrightarrow \mathbb{Z}H \overset{t-1}\longrightarrow \mathbb{Z}H\overset{\varepsilon}\longrightarrow \mathbb{Z} \longrightarrow 0,\]
then tensor $P_\cdot$ with $\mathbb{Z}$ over $\mathbb{Z}H$, one has
\[\cdots \overset{n}\longrightarrow \mathbb{Z} \overset{0}\longrightarrow \mathbb{Z} \overset{n}\longrightarrow \mathbb{Z} \overset{0}\longrightarrow \mathbb{Z} \longrightarrow 0.\]
So we have
\begin{equation*}
H_i(H,\mathbb{Z})=\tor_i^{\mathbb{Z}H}(\mathbb{Z},\mathbb{Z})=\begin{cases}
	\mathbb{Z},&i=0,\\
	\mathbb{Z}/n\mathbb{Z},& i \text{ odd,}\\
	0,& i \text{ even.}
\end{cases}
\end{equation*}
For a finite group $G$, consider $0\neq t \in G$, let $H$ be the subgroup generated by $t$, it follows from Shapiro's lemma that $H_n(H,\mathbb{Z})=H_n(G,\mathbb{Z}G\otimes_{\mathbb{Z}H}\mathbb{Z})$.
\begin{definition}[$F$-regular rings]
	If $F$ is any functor from rings to abelian groups, write $NF(R)$ for the cokernel of the natural map $F(R) \longrightarrow F(R[x])$; $NF$ is also a functor on rings. Moreover, the ring map $R[x] \overset{x\mapsto 1}\longrightarrow R$ provides a splitting $F(R[x]) \longrightarrow F(R)$ of the natural map, so we have a decomposition $F(R[x]) \cong F(R) \oplus NF(R)$.

	We say that a ring $R$ is $F$-regular if $F(R) = F(R[x_1,\cdots,x_n])$ for all $n$. Since $NF(R[t]) = NF(R) \oplus N^2 F(R)$, we see by induction on $p$ that R is $F$-regular if and only if $N^p F(R) = 0$ for all $p \geq 1$.
\end{definition}
For example,
\begin{itemize}
	\item Regular rings are $K_n$-regular for all $n\in \mathbb{Z}$.
	\item (Rosenberg) Commutative $C^*$-algebra are $K_n$-regular for all $n$. 
	\item If $R$ is $K_0$-regular, then so is $R[x,x^{-1}]$.
	\item If $R$ is $K_n$-regular for some $n\leq 0$, then $R$ is also $K_{n-1}$-regular.
	\item If $R$ is an artinian ring, then $R$ is $K_0$-regular and $K_{n}(R)=0$ for all $n<0$.
	\item If $R$ is regular, then the graded rings $R[t_0,\cdots,t_n]/(t_0\cdots t_n)$ are $K_i$-regular for $i\leq 1$.
\end{itemize}
\begin{lemma}[\cite{weibel2013k} chapter III, 3.4.1]
	 Let $R = R_0 \oplus R_1 \oplus \cdots$ be a graded ring. Then the kernel of $F(R) \longrightarrow F(R_0)$ embeds in $NF(R)$ and even in the kernel of $NF(R) \longrightarrow NF(R_0)$. In particular, if $NF(R) = 0$ then $F(R) \cong F(R_0)$.
\end{lemma}
It follows that for any functor $F$, $NF(R)$ is a summand of $N^2F(R)$ and hence $N^pF(R)$ for all $p > 0$. For every $F$, $F(R[t_1,\cdots,t_n])\cong (1 + N)^n F(R)$. If $F$ is a contracted functor, then $F(R[t_1,t_1^{-1},\cdots,t_n ,t_n^{-1}]) \cong (1 + 2N + L)^n F(R)$.
\begin{corollary}
	If $N^nF(R)=0$, then $F(R[t_1,\cdots,t_n])=F(R)$.
\end{corollary}
\begin{proof}
	Since $N^nF(R)=0$ implies $N^jF(R)=0$ for any $1\leq j\leq n$, then 
	\[F(R[t_1,\cdots,t_n])\cong (1 + N)^n F(R)=F(R)\oplus \bigoplus_{j=1}^n \binom{n}{j}N^jF(R)=F(R). \qedhere \]
\end{proof}
\begin{example}[\cite{weibel2013k} chapter III, 3.8.1]
	Let $R$ be a commutative regular ring and $A=R[x]/(x^n)$, then $NK_1(A)\cong (1+tA[t])^{\times}=(1+xtA[t])^{\times}$.
\end{example}
There is a conjecture for group rings whose coefficient	ring are regular: 
\begin{conjecture}[Isomorphism Conjecture for torsion-free groups]
	If $G$ is a torsion-free group, then the assembly map 
	\[H_n(BG;K(R))=\pi_n(BGL(R)^+\wedge BG_+)\longrightarrow K_n(R[G])\]
	should be an isomorphism for any regular ring $R$ where $H_n(BG;K(R))$ is the generalized homology of $BG$ with coefficients in $K(R)$.
\end{conjecture}


\section{The nonfiniteness of Nil} % (fold)
\label{sec:the_nonfiniteness_of_nil}
\paragraph{Bass Nil-groups}
	\begin{theorem}
		For $i\in \mathbb{Z}$, if $NK_i(R)\neq 0$, then $NK_i(R)$ is not finitely generated.
	\end{theorem}
	In 1977, F. T. Farrell\cite{MR56:8624} stated that $NK_1(R)$ is finitely generated only when it vanishes. In fact, for $i\leq 1$, $NK_i(R)$ are always either trivial or infinitely generated.
	This result was subsequently extended to the higher Bass Nil-groups $NK_i(R)$ with $i\geq 1$
	by  van der Kallen(1978) \cite{MR81g:18017}) and Prasolov(1982) \cite{Prasolov1982} . 

	证明步骤可见papers,GTM147习题,苏阳书中有详细的步骤。

	For Bass Nil-groups, the proof is based on the following lemmas
\begin{lemma}
\label{lem:multn}
	The composition $t_n\circ s_n$  is multiplication by $n$ in the group  $\Nil_i(R)$.
\end{lemma}
\begin{lemma}
\label{lem:existN}
	For any element  $x\in \Nil_i(R)$  there exists a natural number $N$ such that $t_n(x) = 0$ for all  $n\geq N$.
\end{lemma}
To prove the theorem we now assume that the group  $\Nil_i(R)\neq 0$ is finitely generated as an abelian group.  By Lemma \ref{lem:existN}, there exists a natural number $N$ such that $t_n = 0$ for  $n\geq N$.  

Since $\Nil_i(R)\neq 0$ is finitely generated, choose a prime $p$ such that $p$ does not appear in the decomposition of $\Nil_i(R)$, then multiplication by $p$ is a monomorphism of  $\Nil_i(R)$ into itself, and choose an integer $k$ such that $n = p^k\geq N$.  Then $t_n\circ s_n=0$, and by Lemma \ref{lem:multn}, $t_n\circ s_n$ is multiplication by $p^k$, i.e., a monomorphism.  Thus the group  $\Nil_i(R)$  cannot be finitely generated.

\paragraph{Farrell Nil-groups}
	\begin{theorem}
		Let $R$ be a ring, $\alpha \colon R \longrightarrow R$ a ring automorphism of finite order, and $i \in \mathbb{Z}$. Then $NK_i(R,\alpha)$ is either trivial, or infinitely generated as an abelian group.
	\end{theorem}
	For Farrell's twisted Nils, when the automorphism $\alpha$ has finite order, Grunewald(2007)\cite{Grunewald2007Non} and Ramos(2007)\cite{Ramos2007Non} independently proved the result for $i \leq 1$. And in 2014, Lafont, Prassidis and Wang\cite{Lafont2014Revisiting} proved the general case.
	
	Khoroshevskii\cite{Khoroshevskii1974On} proved that every automorphism $\phi$ has order bounded by $|G|-1$, provided $G$ is not the trivial group, i.e.\ $|\phi|\leq|G|-1$ and the equality is achieved only for elementary abelian groups. So we have the following
\begin{corollary}
		Let $R$ be a ring, $G$ a finite group, $\alpha \colon G \longrightarrow G$ a group automorphism, then $NK_n(RG,\alpha)$ is either trivial, or infinitely generated.
\end{corollary}



% section the_nonfiniteness_of_nil (end)
 
\section{Vanishing results} % (fold)
\label{sec:vanishing_results}
Since Nil-groups are either zero or infinitely generated, we firstly sumarize some vanishing results, and then give some non-vanishing results.
\begin{itemize}
	\item If $R$ is regular, then $NK_n(R)=0$ for all $n$ and $K_n(R)=0$ for $n<0$ (Quillen). Moreover for any ring automorphism $\alpha$, $NK_n(R,\alpha)=0$ for all $n$ (Farrell and Hsiang).
	\item (Connolly and Prassidis\cite{Connolly2002On}) Let $\alpha: R\longrightarrow R$ be an automorphism, if $R$ is $\alpha$-quasi-regular (that is, there exist a two-sided nilpotent ideal $J$ of $R$ which is $\alpha$-invariant such that $R/J$ is regular), then $NK_n(R,\alpha)=0$ for any $n\leq 0$.
	\item $NK_1(\mathbb{Z}F)=0$ for any free group $F$ of finite rank.
	\item Group rings of nontrivial finite groups:
		\begin{itemize}
			\item Let $G$ be a finite group , then $NK_n(\mathbb{Z}G)=0$ for $n\leq -1$.
			\item (Harmon\cite{MR88g:18012}) If $G$ is a finite group of square-free order, then $NK_n(\mathbb{Z}G)=0$ for $n\leq 1$. Moreover, if $n\leq 1$, then $N^jK_n(\mathbb{Z}G)=0$ for $j\geq 1$.
			\item (Juan-Pineda and Ramos\cite{Ramos2009On}) If $G$ is a non-trivial cyclic group of square-free order, then $NK_n(\mathbb{Z}G,\alpha)=0$ for $n\leq 1$ and any group automorphism $\alpha: G\longrightarrow G$.  
		\end{itemize}
\end{itemize}

% section vanishing_results (end)
\section{Non-vanishing results} % (fold)
\label{sec:non_vanishing_results}
Note that very few Bass Nil-groups are explicitly known and virtually nothing is known about the $NK_i$ for $i\geq 3$.

C. Weibel\cite{weibel2009nk0} computed some explicit examples and gave $W(\mathbb{F}_2)$-module structure on some of them (omitted here):
\begin{itemize}
	\item $NK_0(\mathbb{Z}[C_4])\cong \bigoplus_{i=1}^\infty \mathbb{Z}/2\mathbb{Z}$, $NK_1(\mathbb{Z}[C_4])\cong \bigoplus_{i=1}^\infty \mathbb{Z}/2\mathbb{Z}$.
	\item $NK_0(\mathbb{Z}[C_2\times C_2])\cong \bigoplus_{i=1}^\infty \mathbb{Z}/2\mathbb{Z}$, $NK_1(\mathbb{Z}[C_2\times C_2])\cong \bigoplus_{i=1}^\infty \mathbb{Z}/2\mathbb{Z}$.
	\item $NK_2(\mathbb{Z}[C_2])\cong \bigoplus_{i=1}^\infty \mathbb{Z}/2\mathbb{Z}$.
	\item $NK_i(\mathbb{Z}[D_4])\neq 0$ for $i=0,1$.
\end{itemize}

Guin-Walery and Loday\cite{Guin-Waléry1981} proved that $NK_2(\mathbb{Z}[C_p])\cong \bigoplus_{i=1}^\infty \mathbb{Z}/p\mathbb{Z}$, and is generated by Dennis-Stein symbols $\langle (1-\sigma)x^j,(1+\sigma+\cdots+\sigma^{p-1})\rangle$, where $C_p$ is the cyclic group of prime order $p$ and generated by $\sigma$.

D. J.-Pineda\cite{Juan2007On} gave some non-vanishing results:
\begin{itemize}
	\item Let $C_n$ be a nontrivial finite cyclic group, then $NK_2(\mathbb{Z}[C_n]) \neq 0$. C. Weibel proved (\cite{weibel2013k} chapter III, 3.4.2]) that for any ring if $N^sK_i(R)=0$, it follows that $N^jK_i(R) = 0$ for $j = 1,2,\cdots,s - 1$. Therefore $N^jK_2(\mathbb{Z}[C_n]) \neq 0$, for all $j \geq 1$.
	\item  Let $G$ be a nontrivial finite cyclic group or a split extension of a nontrivial finite cyclic group. Then $N^jK_n(\mathbb{Z}[G]) \neq 0$ for all $n \geq 2$  and $j \geq n - 1$.
	\item  By lemma \ref{lem:NjKn}, if $NK_2(R) \neq 0$, then $N^jK_i(R) \neq 0$ for all $i \geq 2$ and $j \geq i - 1$.
\end{itemize}
\begin{prop}[Vorst \cite{MR80k:18016}, Corollary 2.1]
\label{prop:vanderkallen}
For all $n\geq 1$ we have\\
(1)  $NK_n(R[x])=0$ implies $NK_{n-1}(R)=0$.\\
(2)  $K_n$-regularity implies $K_{n-1}$-regularity.
\end{prop}
\begin{lemma}
\label{lem:NjKn}
	Let $R$ be a commutative ring with identity.\\
	(1)If $N^2K_n(R) = 0$, then $NK_n(R) = 0$ and $NK_{n-1}(R) = 0$.\\
	(2)Moreover, let $j\geq 2$ be an integer, if $N^jK_n(R) = 0$, then $N^iK_n(R) = 0$ for $1\leq i\leq j$ and $N^{j-1}K_{n-1}(R) = 0$. Equivalently, $N^{j-1}K_{n-1}(R)\neq 0$ implies $N^jK_n(R) \neq 0$, and $N^iK_n(R)\neq 0$ implies $N^kK_n(R)\neq 0$ for $k\geq i\geq 1$.
\end{lemma}
\begin{proof}
	For any functor $F$, $NF(R)$ is a summand of $N^pF(R)$.  If $N^sK_i(R)=0$, it follows that $N^jK_i(R) = 0$ for $j = 1,2,\cdots,s - 1$ (\cite{weibel2013k} chapter III, 3.4.2]). So $N^2K_n(R) = 0$ implies $NK_n(R) = 0$, hence $NK_n(R[x])\cong NK_n(R)\oplus N^2K_n(R)=0$. By proposition \ref{prop:vanderkallen}, $NK_{n-1}(R) = 0$.

	Next, replace $R$ by $R[x]$, one has $N^2K_n(R[x]) = 0$ implies $NK_{n-1}(R[x]) = 0$. $N^2K_n(R[x])\cong N^2K_n(R)\oplus N^3K_n(R)=0$ and $NK_{n-1}(R[x]) =NK_{n-1}(R)\oplus N^2K_{n-1}(R)= 0$, so $N^3K_n(R)=0$ implies $N^2K_{n-1}(R)= 0$. Hence the second part is obtained by induction.
\end{proof}

% section non_vanishing_results (end)